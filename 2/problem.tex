%%%%%%%%%%%%%%%%%%%%%%%%%%
\chapter{Problem Formulation}
\label{chap::PROBLEM FORMULATION}
%%%%%%%%%%%%%%%%%%%%%%%%%%

\baselineskip=26pt

\hspace{5mm}Since long or timing
critical buses are normally assigned to a pair of high metal
layers, one horizontal and one vertical, we only consider the
two-layer bus routing problem. The horizontal and vertical buses
can be routed in either layer. In the bus-pin-aware bus-driven
floorplanning problem, we are given the following:

1. A set of $n$ modules $M =$ \{$m_1$, $m_2,..., m_n$\}, each module $m_i$
is associated with height $h_i$ and width $w_i$, where $h_i, w_i \in$
$R^+$.

2. A set of $m$ buses $B =$ \{$b_1$, $b_2,..., b_m$\}, each bus $b_j$ has a
width $t_j$ and goes through a set of modules. Moreover, the
position of each bus pin is also defined in each bus, where
$t_j \in R^+$.

Our goal is to determine the orientation and position of the bus pins
on each module and decide the routing path for each bus such
that no overlapping occurs between different horizontal (vertical)
bus components.
Moreover, no overlapping is allowed between different modules.
The objectives of the bus-pin-aware bus-driven floorplanning problem are:\\
(1) to {\bf minimize the chip area},\\
(2) to {\bf minimize the bus area},\\
(3) to {\bf minimize the number of infeasible buses},\\
(4) to {\bf minimize the deviation of each bus}.\\
