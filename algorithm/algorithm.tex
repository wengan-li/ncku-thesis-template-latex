% ------------------------------------------------
% Page start
% ------------------------------------------------
\chapter{Algorithm}
\label{chapter:algorithm}

\baselineskip=26pt
\thispagestyle{empty}
% ------------------------------------------------

\subsection{Design}

\subsubsection{Li's Hash}

Li's Hash is an algorithm that combines the concept of n-gram indexing \cite{web:wiki:n-gram} (Figure \ref{fig:algorithm:n-gram}), JSON (JavaScript Object Notation) \cite{web:wiki:json} (Figure \ref{fig:algorithm:json_example}) and hash table \cite{web:wiki:hash-table}.\\

JSON to store data which is exactly as same as hash table, but only different of the value's data type, which is because of the mapping design. And the n-gram indexing normally only target the data in string.\\

\begin{figure}[h]
\centering
\includegraphics[scale=0.7]{./algorithm/pic/json_example_v1.png}
%\includegraphics[width=0.4\textwidth]{./algorithm/pic/json_example_v1.png}
\caption{A example of JSON.}
\label{fig:algorithm:json_example}
\end{figure}

\begin{figure}[h]
\centering
\includegraphics[scale=0.6]{./algorithm/pic/n-gram_v1.png}
%\includegraphics[width=0.4\textwidth]{./algorithm/pic/n-gram_v1.png}
\caption{A example of n-gram indexing.}
\label{fig:algorithm:n-gram}
\end{figure}

Combining all of them can form a special indexing structure which can store the data and query different data type of data, which can very useful for all kind of query and storage system design.\\

Also because Li's Hash is just an algorithm, so it can easily to suit for all kind of key-value stores, or just swap the back-end database that can use another without change any front-end code.\\

%\clearpage

% Data type section
\subsubsection{Data type}

Li's Hash try to fellow KISS (Keep It Simple \& Stupid) principle for user, so Li's Hash will only provide few basic data type to replace the general data type in SQL database \cite{web:mysql:data-types,web:mysql:data-types-store-requirements,web:sqlite:data-types-3,web:transact-sql:data-types}, to decrease the time that user need to understand all kind of data type before writing the code, and can be ignore the range of data type, to let them focus on their system design.\\

The provided data type are \textit{STRING}, \textit{INTEGER}, \textit{REAL}, \textit{BOOLEAN} and \textit{BLOB}. Table \ref{table:algorithm:database-layer:code-example} shows part of data type comparison between Li's Hash and relational database.

Table \ref{table:algorithm:data_type_description} is the description that the storage size and range about each data type.\\

\begin{table}[h]
\centering
\caption{Data type comparison}
\label{table:algorithm:database-layer:code-example}
\begin{tabular}{|c|c|}

\hline
\multicolumn{1}{|c|}{\textbf{Li's Hash}} &
\multicolumn{1}{c|}{\textbf{Relational database}} \\

\hline
\multicolumn{1}{|c|}{STRING} &
\multicolumn{1}{c|}
{\tabincell{c}{
TEXT \\ CHAR \\ VARCHAR
}} \\

\hline
\multicolumn{1}{|c|}{BOOLEAN} &
\multicolumn{1}{c|}
{\tabincell{c}{
CHAR(1) \\ BOOLEAN
}} \\

\hline
\multicolumn{1}{|c|}{BLOB} &
\multicolumn{1}{c|}
{\tabincell{c}{
BLOB
}} \\

\hline
\multicolumn{1}{|c|}{INTEGER} &
\multicolumn{1}{c|}
{\tabincell{c}{
INT \\ INTEGER \\ BIGINT
}} \\

\hline
\multicolumn{1}{|c|}{REAL} &
\multicolumn{1}{c|}
{\tabincell{c}{
REAL \\ DOUBLE \\ FLOAT \\ DECIMAL
}} \\

\hline
\end{tabular}
\end{table}


\begin{table}[h]
\centering
\caption{Data type description}
\label{table:algorithm:data_type_description}
\begin{tabular}{|c|c|c|c|c|c|}

\hline
\multicolumn{1}{|c|}{Type name} &
\multicolumn{1}{c|}{STRING} &
\multicolumn{1}{c|}{BOOLEAN} &
\multicolumn{1}{c|}{INTEGER} &
\multicolumn{1}{c|}{REAL} &
\multicolumn{1}{c|}{BLOB} \\

\hline
\multicolumn{1}{|c|}{indexing} &
\multicolumn{1}{c|}{Y} &
\multicolumn{1}{c|}{Y} &
\multicolumn{1}{c|}{Y} &
\multicolumn{1}{c|}{Y} &
\multicolumn{1}{c|}{N} \\

\hline
\multicolumn{1}{|c|}{byte length (\textit{b})} &
\multicolumn{1}{c|}{dynamic} &
\multicolumn{1}{c|}{1} &
\multicolumn{1}{c|}{8} &
\multicolumn{1}{c|}{dynamic} &
\multicolumn{1}{c|}{dynamic} \\

\hline
\multicolumn{1}{|c|}{data range} &
\multicolumn{1}{c|}{-} &
\multicolumn{1}{c|}{0 $\thicksim$ 1} &
\multicolumn{1}{c|}{
\tabincell{c}
{0 $\thicksim 2^{64}$ - 1 \\ or \\-$2^{63} \thicksim 2^{63}$ - 1}} &
\multicolumn{1}{c|}{No limited} &
\multicolumn{1}{c|}{-} \\

\hline
\end{tabular}
\end{table}

Table \ref{table:algorithm:lishash_type_operation} shows what operation of each data type can do in Li's Hash.

\begin{table}[h]
\centering
\caption{Type of operation}
\label{table:algorithm:lishash_type_operation}
\begin{tabular}{|c|c|c|c|c|c|}

\hline
\multicolumn{1}{|c|}{Type name} &
\multicolumn{1}{c|}{STRING} &
\multicolumn{1}{c|}{BOOLEAN} &
\multicolumn{1}{c|}{INTEGER} &
\multicolumn{1}{c|}{REAL} &
\multicolumn{1}{c|}{BLOB} \\

\hline
\multicolumn{1}{|c|}{\tabincell{c}{Search\\(Exact matching)}} &
\multicolumn{1}{c|}{Y} &
\multicolumn{1}{c|}{Y} &
\multicolumn{1}{c|}{N} &
\multicolumn{1}{c|}{N} &
\multicolumn{1}{c|}{N} \\

\hline
\multicolumn{1}{|c|}{\tabincell{c}{Search\\(Prefix matching)}} &
\multicolumn{1}{c|}{Y} &
\multicolumn{1}{c|}{N} &
\multicolumn{1}{c|}{N} &
\multicolumn{1}{c|}{N} &
\multicolumn{1}{c|}{N} \\

\hline
\multicolumn{1}{|c|}{\tabincell{c}{Search\\(Suffix matching)}} &
\multicolumn{1}{c|}{Y} &
\multicolumn{1}{c|}{N} &
\multicolumn{1}{c|}{N} &
\multicolumn{1}{c|}{N} &
\multicolumn{1}{c|}{N} \\

\hline
\multicolumn{1}{|c|}{\tabincell{c}{Search\\(Partial matching)}} &
\multicolumn{1}{c|}{Y} &
\multicolumn{1}{c|}{N} &
\multicolumn{1}{c|}{N} &
\multicolumn{1}{c|}{N} &
\multicolumn{1}{c|}{N} \\

\hline
\multicolumn{1}{|c|}{Equal} &
\multicolumn{1}{c|}{Y} &
\multicolumn{1}{c|}{Y} &
\multicolumn{1}{c|}{Y} &
\multicolumn{1}{c|}{Y} &
\multicolumn{1}{c|}{N} \\

\hline
\multicolumn{1}{|c|}{\tabincell{c}{Equal\\(muti-value)}} &
\multicolumn{1}{c|}{Y} &
\multicolumn{1}{c|}{Y} &
\multicolumn{1}{c|}{Y} &
\multicolumn{1}{c|}{Y} &
\multicolumn{1}{c|}{N} \\

\hline
\multicolumn{1}{|c|}{Not equal} &
\multicolumn{1}{c|}{Y} &
\multicolumn{1}{c|}{Y} &
\multicolumn{1}{c|}{Y} &
\multicolumn{1}{c|}{Y} &
\multicolumn{1}{c|}{N} \\

\hline
\multicolumn{1}{|c|}{\tabincell{c}{Not equal\\(muti-value)}} &
\multicolumn{1}{c|}{Y} &
\multicolumn{1}{c|}{Y} &
\multicolumn{1}{c|}{Y} &
\multicolumn{1}{c|}{Y} &
\multicolumn{1}{c|}{N} \\

\hline
\multicolumn{1}{|c|}{Less than} &
\multicolumn{1}{c|}{N} &
\multicolumn{1}{c|}{N} &
\multicolumn{1}{c|}{Y} &
\multicolumn{1}{c|}{Y} &
\multicolumn{1}{c|}{N} \\

\hline
\multicolumn{1}{|c|}{Less than or equal} &
\multicolumn{1}{c|}{N} &
\multicolumn{1}{c|}{N} &
\multicolumn{1}{c|}{Y} &
\multicolumn{1}{c|}{Y} &
\multicolumn{1}{c|}{N} \\

\hline
\multicolumn{1}{|c|}{Greater than} &
\multicolumn{1}{c|}{N} &
\multicolumn{1}{c|}{N} &
\multicolumn{1}{c|}{Y} &
\multicolumn{1}{c|}{Y} &
\multicolumn{1}{c|}{N} \\

\hline
\multicolumn{1}{|c|}{Greater than or equal} &
\multicolumn{1}{c|}{N} &
\multicolumn{1}{c|}{N} &
\multicolumn{1}{c|}{Y} &
\multicolumn{1}{c|}{Y} &
\multicolumn{1}{c|}{N} \\

\hline
\multicolumn{1}{|c|}{Between} &
\multicolumn{1}{c|}{N} &
\multicolumn{1}{c|}{N} &
\multicolumn{1}{c|}{Y} &
\multicolumn{1}{c|}{Y} &
\multicolumn{1}{c|}{N} \\

\hline
\end{tabular}
\end{table}

And Table \ref{table:algorithm:data_type_represent_in_sql} is listed the data type will represent that data type in SQL database.

\begin{table}[h]
\centering
\caption{Data type represent in SQL database}
\label{table:algorithm:data_type_represent_in_sql}
\begin{tabular}{|c|c|c|c|c|c|}

\hline
\multicolumn{1}{|c|}{Type} &
\multicolumn{1}{c|}{STRING} &
\multicolumn{1}{c|}{BOOLEAN} &
\multicolumn{1}{c|}{INTEGER} &
\multicolumn{1}{c|}{REAL} &
\multicolumn{1}{c|}{BLOB} \\

\hline
\multicolumn{1}{|c|}{\tabincell{c}{
MySQL\\
\cite{web:mysql:data-types,web:mysql:data-types-store-requirements}
}} &
\multicolumn{1}{c|}{\tabincell{c}{
VARCHAR \\ VARBINARY \\ CHAR \\ BINARY
}} &
\multicolumn{1}{c|}{\tabincell{c}{
CHAR(1) \\ BINARY(1) \\ TINYINT(1)
}} &
\multicolumn{1}{c|}{\tabincell{c}{
TINYINT \\ SMALLINT \\ MEDIUMINT \\ INT \\ INTEGER \\ BIGINT
}} &
\multicolumn{1}{c|}{\tabincell{c}{
FLOAT \\ DOUBLE \\ DECIMAL \\ NUMERIC
}} &
\multicolumn{1}{c|}{\tabincell{c}{
TINYBLOB \\ TINYTEXT \\ BLOB \\ TEXT \\ MEDIUMBLOB \\ MEDIUMTEXT \\ LONGBLOB \\ LONGTEXT
}} \\

\hline
\multicolumn{1}{|c|}{\tabincell{c}{
SQLite\\
\cite{web:sqlite:data-types-3}
}} &
\multicolumn{1}{c|}{\tabincell{c}{
CHARACTER \\ VARCHAR \\ VARYING CHARACTER \\ NCHAR \\ NATIVE CHARACTER \\ NVARCHAR \\ TEXT \\ CLOB
}} &
\multicolumn{1}{c|}{\tabincell{c}{
BOOLEAN
}} &
\multicolumn{1}{c|}{\tabincell{c}{
INT \\ INTEGER \\ TINYINT \\ SMALLINT \\ MEDIUMINT \\ BIGINT \\ UNSIGNED BIG INT \\ INT2 \\ INT8 \\ NUMERIC
}} &
\multicolumn{1}{c|}{\tabincell{c}{
REAL \\ DOUBLE \\ FLOAT
}} &
\multicolumn{1}{c|}{\tabincell{c}{
BLOB
}} \\

\hline
\multicolumn{1}{|c|}{\tabincell{c}{
SQL server\\
\cite{web:transact-sql:data-types}
}} &
\multicolumn{1}{c|}{\tabincell{c}{
char \\ varchar \\ text \\ nchar \\ nvarchar \\ ntext
}} &
\multicolumn{1}{c|}{\tabincell{c}{
bit
}} &
\multicolumn{1}{c|}{\tabincell{c}{
bigint \\ numeric \\ smallint \\ decimal \\ smallmoney \\ int \\ tinyint \\ money
}} &
\multicolumn{1}{c|}{\tabincell{c}{
float \\ real
}} &
\multicolumn{1}{c|}{\tabincell{c}{
binary \\ varbinary \\ image
}} \\

\hline
\end{tabular}
\end{table}

\clearpage

% Index table section
\subsubsection{Index table}

To make the data queryable, Li's Hash design a custom indexing for each data type, so that each data type has their own index tables. The index table is the core concept of Li's Hash. The tables are JSON objects design, there are two kind of index table, \textit{"Index table"} and \textit{"Invert index table"}. Figure \ref{fig:algorithm:lishash_example} is shows the example that the tables looks like contain data \textit{"cow"} with the data type \textit{STRING} which will have more detail later.\\

The tables are using the multi-gram indexing but one table is use the invert way to do the indexing as its name. Because the table is a JSON object, so it is a name-value design, the \textit{"value"} is a collection which means it can store everything no matter an array or a data. In Li's Hash, the \textit{"value"} is pointing to a array which will contain element or data node, but for easy visualization that the array will using a list to show on figure, so call it as \textit{"Bucket"} is quite suitable. This means the index table in Li's Hash as a \textit{"name-bucket"} format.\\

Each element node contain few metadata: the value of this node storing (value), the count of this node is be using (count), and the count of the value repeating (repeat). The data nodes is pointing to the data \textit{id} of the data record.

\begin{figure}[h]
\centering
%\includegraphics[scale=0.4]{./algorithm/pic/index_table/table_format_v13.pdf}
\includegraphics[width=0.8\textwidth]{./algorithm/pic/index_table/table_format_v13.pdf}
\caption{A example of Li's Hash.}
\label{fig:algorithm:lishash_example}
\end{figure}

The time complexity of initial the tables is $O(1)$. About the meanings of all tables, which will more clearly in each operation of all data type.

\clearpage

% STRING section
\subsection{STRING type}

% Insertion section
\subsubsection{Insertion}

When insert a data into the empty table (Figure \ref{fig:algorithm:string:insertion:empty_table}), every data will assign a unique \textit{id} for this data, this \textit{id} will use for the index to look for.

\begin{figure}[h]
\centering
%\includegraphics[scale=0.4]{./algorithm/string/pic/insertion/empty_table_v2.pdf}
\includegraphics[width=0.5\textwidth]{./algorithm/string/pic/insertion/empty_table_v2.pdf}
\caption{A empty index table.}
\label{fig:algorithm:string:insertion:empty_table}
\end{figure}

So for example if inputting a data \textit{"book"}, first assign the \textit{id} as \textit{"\_d\_1\_"}, next step is separate the bytes using n-gram indexing with counting the repeat value to from the keys. So \textit{"book"} can count as \textit{b-\textgreater1}, \textit{o-\textgreater2} and \textit{k-\textgreater1} by counting the repeat value. And formed into \textit{"b"}, \textit{"bo"}, \textit{"boo"}, \textit{"book"}, \textit{"oo"}, \textit{"ook"} by n-gram but because using count the repeat byte that \textit{"bo"} is domained by \textit{"boo"}.\\

Every byte will have their own key such as \textit{'b'} and \textit{'o'} for record their repeat count, which is using in searching to know there have the keys like \textit{"'b':r1"} or \textit{"'o':r2"} which can use. The \textit{'r'} in key means the repeat times of that value. But the last byte wouldn't have that key in index table because of the key will exist in invert index table, also this happen in the opposite.\\

So the data will index as figure \ref{fig:algorithm:string:insertion:example_1}.

\begin{figure}[h]
\centering
%\includegraphics[scale=1.0]{./algorithm/string/pic/insertion/example_1_v5.pdf}
\includegraphics[width=0.7\textwidth]{./algorithm/string/pic/insertion/example_1_v5.pdf}
\caption{Insert data "book".}
\label{fig:algorithm:string:insertion:example_1}
\end{figure}

The time complexity of insertion should look like figure \ref{fig:algorithm:string:insertion:time_complexity}. The time complexity of insert a data should be $O(b)$, $b!$ is the number of for loop which is equal to the byte length of data, and $O(1)$ is means the key of the data like \textit{"'b':r1-'o':r2-'k':r1"} in the example.\\

\begin{figure}[h]
\centering
%\includegraphics[scale=1.0]{./algorithm/string/pic/insertion/time_complexity_v5.pdf}
\includegraphics[width=0.6\textwidth]{./algorithm/string/pic/insertion/time_complexity_v5.pdf}
\caption{Time complexity of insertion.}
\label{fig:algorithm:string:insertion:time_complexity}
\end{figure}

Next if insert the data \textit{"box"}, the tables will look like figure \ref{fig:algorithm:string:insertion:example_2}. As the figure shows that \textit{"box"} and \textit{"book"} have a common key, so some of the node will deduplicated.

\begin{figure}[h]
\centering
%\includegraphics[scale=0.6]{./algorithm/string/pic/insertion/example_2_v5.pdf}
\includegraphics[width=0.8\textwidth]{./algorithm/string/pic/insertion/example_2_v5.pdf}
\caption{Insert data \textit{"box"}.}
\label{fig:algorithm:string:insertion:example_2}
\end{figure}



% Deletion section
\subsubsection{Deletion}

In deletion, if a data need to be delete, and the bucket is only contain one node or data, then this mapping will be delete, otherwise only the count in the node will minus one. Next, delete the n-gram indexing in all index table.\\

Using the same example in figure \ref{fig:algorithm:string:insertion:example_2}. If removing the $"book"$ which will remove some nodes in both index table, and then the tables should look like figure \ref{fig:algorithm:string:deletion:example_1}. For better understanding, the nodes which need to remove will use the line over it.

\begin{figure}[h]
\centering
\includegraphics[scale=0.4]{./algorithm/string/pic/deletion/example_1_v3.pdf}
\caption{Delete $"book"$ from table.}
\label{fig:algorithm:string:deletion:example_1}
\end{figure}

The time needed is equals to $2 * O(b!)$, $b$ is the number of for loop which is equal to the byte length of string in one table (Normally it is equal, but in this case $(b = 3)$ which is because the $"o"$ have handled by the repeat counting). Because $2 * O(b!)$ is domain as $O(b)$, so the time complexity of delectation is $O(b)$.



% Modification section
\subsubsection{Modification}

Modify a data, actually is combining insert and delete, but without modified the data $id$. So the time complexity is $2 * O(b)$ operation but domain as $O(b)$.


% Selection section
\subsubsection{Selection}

The selection is the main core of Li's Hash, because the purposes of the index tables is to design for high-speed searching operation which means the Li's Hash can do the searching operation for key-value store.

Using the same example of figure \ref{fig:algorithm:string:insertion:example_2} from insertion section.

% Operation
\begin{enumerate}

% --------------------------------------------------------

\item \textbf{Exact matching}

As normal key-value store, if searching the data as \textit{"book"}, which convert into the key as \textit{"'b':r1-'o':r2-'k':r1"} and search in index table, and return the result of data node \textit{"\_d\_1\_"} in $O(1)$.

% --------------------------------------------------------

\item \textbf{Prefix matching}

If searching \textit{"bo"} as prefix which as same as \textit{"LIKE 'bo\%'"} in SQL. Using the key as \textit{"'b':r1"} from index table, will get the metadata that can know the next keys \textit{"'b':r1$-$'o':r1"} and \textit{"'b':r1$-$'o':r2"}. So fellow these two keys will get \textit{"'b':r1-'o':r1-'x':r1"} and \textit{"'b':r1-'o':r2-'k':r1"}, and using this two keys will get the data node \textit{"\_d\_1\_"} and \textit{"\_d\_2\_"}.

So the concept of search is fellow the metadata if get an element node, and add the data into the return list if pointing to a data node. Repeat search in table with this way until there is no element node can fellow.

And the time complexity should be $O(b)$.

% --------------------------------------------------------

\item \textbf{Suffix matching}

Similar as prefix matching, the different is using inverted index table rather than index table. So if try to search \textit{'x'} as suffix which as same as \textit{"LIKE '\%x'"} in SQL, using the key \textit{'x'} and get \textit{"'x':r1"} as return, and the remain is as same as the searching in prefix matching which will get the data node \textit{"\_d\_2\_"} in final. The time complexity is $O(b)$.

% --------------------------------------------------------

\item \textbf{Partial matching}

Partial matching is combining the prefix and suffix matching. For example if search \textit{'o'} for result which as same as using \textit{"LIKE '\%o\%'"} in SQL, then reassign as \textit{'o'} for prefix matching (\textit{"LIKE 'o\%'"} in SQL) and suffix matching (\textit{"LIKE '\%o'"} in SQL), and the last step is to do intersection to both result.

The reason of existing the special keys is to record the repeat time like\textit{'o'-\textgreater1} or \textit{2}, or some keys like \textit{"'o':r1}-\textit{'x':r1"} which are target on this operation.

If these keys are not created, the search like above wouldn't be searchable because the \textit{'o'} is middle of the keys, it can't be found unless there is some query function provided by non-relational database where are  range scan or full scan, but we have already mentioned before.

And the time complexity should be $O(b) + O(b)$ that domain as $O(b)$.

% --------------------------------------------------------

\item \textbf{Retrieve all}

Using the key of \textit{"root"} in index table will get all the prefix byte of all result, this can simple retrieve all data in the database which the time complexity be $O(b)$.

% --------------------------------------------------------

\end{enumerate}


% Summary section
\subsubsection{Summary}

Table \ref{table:algorithm:string:summary:time_complexity} is the summary the time complexity of each opration in \textit{STRING} type.

\begin{table}[h]
\centering
\caption{Time complexity for \textit{STRING} type.}
\label{table:algorithm:string:summary:time_complexity}
\begin{tabular}{|c|c|}

\hline
\multicolumn{1}{|c|}{Operation} &
\multicolumn{1}{c|}{\tabincell{c}{
Time complexity \\ ($b$: The byte length of data)
}} \\

\hline
\multicolumn{1}{|c|}{Insert} &
\multicolumn{1}{c|}{$O(b)$} \\

\hline
\multicolumn{1}{|c|}{Modify} &
\multicolumn{1}{c|}{$O(b)$} \\

\hline
\multicolumn{1}{|c|}{Delete} &
\multicolumn{1}{c|}{$O(b)$} \\

\hline
\multicolumn{1}{|c|}{\tabincell{c}{
Search \\ (Exact matching)
}} &
\multicolumn{1}{c|}{$O(1)$} \\

\hline
\multicolumn{1}{|c|}{\tabincell{c}{
Search \\ (Prefix matching)
}} &
\multicolumn{1}{c|}{$O(b)$} \\

\hline
\multicolumn{1}{|c|}{\tabincell{c}{
Search \\ (Suffix matching)
}} &
\multicolumn{1}{c|}{$O(b)$} \\

\hline
\multicolumn{1}{|c|}{\tabincell{c}{
Search \\ (Partial matching)
}} &
\multicolumn{1}{c|}{$O(b)$} \\

\hline
\multicolumn{1}{|c|}
{\tabincell{c}{
Search \\ (Retrieve all)
}} &
\multicolumn{1}{c|}
{\tabincell{c}{
$O(b)$ \\ ($b$ is the longest string of all data)
}} \\

\hline
\end{tabular}
\end{table}


\clearpage


% BOOLEAN section
\subsection{BOOLEAN type}

The indexing in \textit{BOOLEAN} type is the simplest than other type, the invert index table is not needed because it is useless by doing the invert indexing just one single byte. So the table can be simplify as figure \ref{fig:algorithm:boolean:example_1}.

\begin{figure}[ht]
\centering
%\includegraphics[scale=0.8]{./algorithm/boolean/pic/example_1_v1.pdf}
\includegraphics[width=0.8\textwidth]{./algorithm/boolean/pic/example_1_v1.pdf}
\caption{The indexing tables of in \textit{BOOLEAN} type.}
\label{fig:algorithm:boolean:example_1}
\end{figure}


% Insertion section
\subsubsection{Insertion}

When inserting a new data, it just need to add the data node into the bucket where the data belong with, so the time complexity should be $O(1)$.


% Deletion section
\subsubsection{Deletion}

Delete a data node is just get the bucket and remove the data node, so this is also a quick operation where time complexity be $O(1)$.


% Modification section
\subsubsection{Modification}

Modify the data is actually delete the node in a bucket and re-insert it again into another bucket. Time complexity should be $O(1)$.



% Selection section
\subsubsection{Selection}

Select operation is the simplest operation than the other, because the selection for \textit{BOOLEAN} type is only can select $"true"$ or $"false"$, so select the bucket then this will return the whole bucket, so this is also a quick operation. Also if user want to retrieve all, just retrieve both $"true"$ and $"false"$ bucket will get all data. That time complexity of both searching are $O(1)$.


% Summary section
\subsubsection{Summary}

Table \ref{table:algorithm:boolean:summary:time_complexity} is summarized the time complexity of each opration in \textit{BOOLEAN} type.

\begin{table}[h]
\centering
\caption{Time complexity for \textit{BOOLEAN} type.}
\label{table:algorithm:boolean:summary:time_complexity}
\begin{tabular}{|c|c|}

\hline
\multicolumn{1}{|c|}{Operation} &
\multicolumn{1}{c|}{Time complexity} \\

\hline
\multicolumn{1}{|c|}{Insert} &
\multicolumn{1}{c|}{$O(1)$} \\

\hline
\multicolumn{1}{|c|}{Modify} &
\multicolumn{1}{c|}{$O(1)$} \\

\hline
\multicolumn{1}{|c|}{Delete} &
\multicolumn{1}{c|}{$O(1)$} \\

\hline
\multicolumn{1}{|c|}{Selection} &
\multicolumn{1}{c|}{$O(1)$} \\

\hline
\end{tabular}
\end{table}


\clearpage


% INTEGER section
\subsection{INTEGER type}

The \textit{INTEGER} type is design as 8 bytes $({b} = 8)$, but for easy to explain the \textit{INTEGER} type design in Li's Hash, so the example below which will explain in 4 bytes $(b = 4)$.\\

As normal integer, the \textit{INTEGER} type can be also signed and unsigned, this information will record in the metadata, so this will not show in index table, the different between them is the operation have a little bit different, but they share the same index table. Same as the \textit{BOOLEAN} type, the invert index table is not needed. Because of the inverted index table cost spaces but don't provide a significant speed up for the operation.\\

We use figure \ref{fig:algorithm:integer:example_1} to explain these operations below:

\begin{figure}[h]
\centering
%\includegraphics[scale=0.45]{./algorithm/integer/pic/example_1_v3.pdf}
\includegraphics[width=0.8\textwidth]{./algorithm/integer/pic/example_1_v3.pdf}
\caption{The indexing tables of \textit{INTEGER} type.}
\label{fig:algorithm:integer:example_1}
\end{figure}

The index table is start with a root key \textit{'root'}. The root key is use to record the first byte of the data, so it will point to the range of 0 to 255.\\

In figure \ref{fig:algorithm:integer:example_1}, there are two data in the table. The four bytes of \textit{1} is \textit{0-0-0-1}, and \textit{167904506} is \textit{10-2-4-250} in byte. So in the table, the \textit{'root'} is pointing to \textit{'$\backslash0$'} and \textit{'$\backslash10$'}. After root key is finish its' indexing, the next step is just using n-gram indexing to index remain bytes to store data in the table.\\


% Insertion section
\subsubsection{Insertion}

Figure \ref{fig:algorithm:integer:example_1} already described some of the flow of insertion, so in here will show the table if insert a negative value into the table. Insert a -2147483647 (128-0-0-1) into table, which will become like figure \ref{fig:algorithm:integer:insertion:example_1}.

\begin{figure}[h]
\centering
%\includegraphics[scale=0.6]{./algorithm/integer/pic/insertion/example_1_v3.pdf}
\includegraphics[width=0.8\textwidth]{./algorithm/integer/pic/insertion/example_1_v3.pdf}
\caption{The table after insert a negative value.}
\label{fig:algorithm:integer:insertion:example_1}
\end{figure}

Figure \ref{fig:algorithm:integer:insertion:example_1} shows that even a negative value will store as the same way as the positive value. And time complexity is $O(b)$.



% Deletion section
\subsubsection{Deletion}

Deletion is just do the opposite insertion to remove the byte and decrease the count. Fellow the same rule, if the count become zero, then the node will be remove. The time complexity is also $O(b)$.


% Modification section
\subsubsection{Modification}

The modify flow are similar as \textit{STRING} type, remove the key which don't needed and add the count if the byte is the same. So follow the example in figure \ref{fig:algorithm:integer:insertion:example_1} and modify 167904506 (10-2-4-250) to 2 (0-0-0-2), the table will look like figure \ref{fig:algorithm:integer:modification:example_1} and time complexity is $O(b)$.

\begin{figure}[h]
\centering
%\includegraphics[scale=0.6]{./algorithm/integer/pic/modification/example_1_v3.pdf}
\includegraphics[width=0.8\textwidth]{./algorithm/integer/pic/modification/example_1_v3.pdf}
\caption{The table after modified the value.}
\label{fig:algorithm:integer:modification:example_1}
\end{figure}



% Selection section
\subsubsection{Selection}

Normally the database proves some function for compare the value of searching, so the Li's Hash musts also can do the same thing.

We use figure \ref{fig:algorithm:integer:modification:example_1} as example to demo the selection.

% Selection section enumerate
\begin{enumerate}

% --------------------------------------------------------

% Equal section
\item \textbf{Equal}

Compare the value is very simple. For example if we want to find data is equal to \textit{1 (0-0-0-1)}, we just need to use the key as $'\backslash0\backslash0\backslash0\backslash1'$ to search the index table. This should only take the time as $O(1)$.

% --------------------------------------------------------

% Not equal section
\item \textbf{Not equal}

Because it can't directly use the input value as key. So in this operation, it will start with the root key and parse all the result to find these element node until find the data node, but skip checking the key is as same as the input value. The time complexity be $O(b)$.

% --------------------------------------------------------

% Less than section
\item \textbf{Less than}

The \emph{"Less than"} comparison is similar as \emph{"Not equal"} comparison.

% Less than section description
\begin{description}

% Unsigned section
\item \textbf{Unsigned}

Start from root key, skip all the value which is \textit{"greater than"} the input value of the first byte, for example if the table contain \textit{10-X-X-X}, \textit{15-X-X-X} and \textit{20-X-X-X}, then if the input value is \textit{15-X-X-X}, the result will remain \textit{10-X-X-X} and \textit{15-X-X-X}. After that, search until to the last byte, then check the first byte if it is equal to the first byte of input value, and skip all the last byte which is \textit{"greater than or equal to"} the last byte of input value, otherwise keep all the result.

% Signed section
\item \textbf{Signed}

% Signed section enumerate
\begin{enumerate}

% Input value is a negative value
\item \textbf{Input value is a negative value}

If the input is a negative, then only start from the first byte is \textit{"greater than or equal to"} the inputs' first byte, and skip the key as same as the input.

% Input value is a positive value
\item \textbf{Input value is a positive value}

If the input is a positive, then start from the first byte is \textit{"less than or equal to"} the inputs' first byte and also skip the first byte is \textit{"greater than or equal to"} $\backslash128$ and the key as same as the input.

% Input value is equal to 0.0
\item \textbf{Input value is equal to 0.0}

If the input is zero, then then start from the first byte is \textit{"greater than or equal to"} $\backslash128$.

% End Signed section enumerate
\end{enumerate}

% End Less than section description
\end{description}

The \emph{"Less than or equal to"} comparison is just do the \emph{"Less than"} and \emph{"Equal"} operation and then combine both result for ouput. The time complexity is $O(b)$ for both operation.

% --------------------------------------------------------

% Greater than section
\item \textbf{Greater than}

This comparison flow is same as \emph{"Less than"}, and just need to convert all the \textit{"greater than"} to \textit{"less than"} and \textit{"less than"} to \textit{"greater than"}, the \emph{"Greater than or equal to"} will do the same thing. So time complexities are the same.

% --------------------------------------------------------

% Between section
\item \textbf{Between}

The \emph{"Between"} comparison is combining \emph{"Less than or equal to"} and \emph{"Greater than or equal to"} operation.

% Between section description
\begin{description}

% Unsigned section
\item \textbf{Unsigned}

Start with root key, but only keep the first byte which is only between and equal with the first byte of $minimum$ and $maximum$ input value.

For example if the table contain \textit{5-X-X-X}, \textit{10-X-X-X}, \textit{15-X-X-X}, \textit{20-X-X-X} and \textit{25-X-X-X}, and the input values are \textit{6-X-X-X} and \textit{20-X-X-X}, so \textit{10-X-X-X}, \textit{15-X-X-X} and \textit{20-X-X-X} are the result.

After that, search until to the last byte. Check the first byte if it is equal to the first byte of input value:

% Unsigned section enumerate
\begin{enumerate}[label=\bfseries \arabic*)]

\item When it is equal to the byte of $maximum$ input value, then skip all the last byte which is \textit{"greater than or equal to"} the last byte of $maximum$ input value.

\item When it is equal to the byte of $minimum$ input value, then skip all the last byte which is \textit{"less than or equal to"} the last byte of $minimum$ input value.

\item Otherwise keep all the result.

% End Unsigned section enumerate
\end{enumerate}


% Signed section
\item \textbf{Signed}

In signed integer, there are six cases for the \textbf{Between} operation:

% Signed section enumerate
\begin{enumerate}[label=\bfseries (\arabic*)]

% Case 1
\item \textbf{$minimum$ and $maximum$ are positive}

Use $minimum$ to do the \emph{"Greater than or equal to"}, and do \emph{"Less than or equal to"} by inputting the $maximum$. After that find the common result.

% Case 2
\item \textbf{$minimum$ and $maximum$ are negative}

As same as case \textbf{(1)}.

% Case 3
\item \textbf{$minimum$ is zero and $maximum$ is positive}

Use zero to do the \emph{"Greater than or equal to"}, and do \emph{"Less than or equal to"} by inputting the $maximum$. After that find the common result.

% Case 4
\item \textbf{$minimum$ is negative and $maximum$ is zero}

Use $minimum$ to do the \emph{"Greater than or equal to"}, and do \emph{"Less than or equal to"} by inputting the zero. After that find the common result.

% Case 5
\item \textbf{$minimum$ is negative and $maximum$ is positive}

Cut this into two part, the negative to zero part will do the \textbf{(4)}, another part will do \textbf{(3)}, after that find the common result.

% Case 6
\item \textbf{$minimum$ is positive value and $maximum$ are negative value}

This case should never happend because of the program should show warning message if the user really inputed like this.

% End Signed section enumerate
\end{enumerate}

The time complexity is $O(b)$.

% End Between section description
\end{description}
% --------------------------------------------------------

% End Selection section enumerate
\end{enumerate}



% Summary section
\subsubsection{Summary}

Table \ref{table:algorithm:integer:summary:time_complexity} is the summary the time complexity of each opration in \textit{INTEGER} type.

\begin{table}[h]
\centering
\caption{Time complexity for \textit{INTEGER} type.}
\label{table:algorithm:integer:summary:time_complexity}
\begin{tabular}{|c|c|}

\hline
\multicolumn{1}{|c|}{Operation} &
\multicolumn{1}{c|}{\tabincell{c}{
Time complexity \\ ($b$: The byte length of data, $b$ = 8)
}} \\

\hline
\multicolumn{1}{|c|}{Insert} &
\multicolumn{1}{c|}{$O(b)$)} \\

\hline
\multicolumn{1}{|c|}{Modify} &
\multicolumn{1}{c|}{$O(b)$)} \\

\hline
\multicolumn{1}{|c|}{Delete} &
\multicolumn{1}{c|}{$O(b)$)} \\

\hline
\multicolumn{1}{|c|}{Equal} &
\multicolumn{1}{c|}{$O(1)$} \\

\hline
\multicolumn{1}{|c|}{\tabincell{c}{Equal (muti-value)}} &
\multicolumn{1}{c|}{$O(1)$} \\

\hline
\multicolumn{1}{|c|}{Not equal} &
\multicolumn{1}{c|}{$O(b)$)} \\

\hline
\multicolumn{1}{|c|}{\tabincell{c}{Not equal (muti-value)}} &
\multicolumn{1}{c|}{$O(b)$)} \\

\hline
\multicolumn{1}{|c|}{Less than} &
\multicolumn{1}{c|}{$O(b)$)} \\

\hline
\multicolumn{1}{|c|}{Less than or equal} &
\multicolumn{1}{c|}{$O(b)$)} \\

\hline
\multicolumn{1}{|c|}{Greater than} &
\multicolumn{1}{c|}{$O(b)$)} \\

\hline
\multicolumn{1}{|c|}{Greater than or equal} &
\multicolumn{1}{c|}{$O(b)$)} \\

\hline
\multicolumn{1}{|c|}{Between} &
\multicolumn{1}{c|}{$O(b)$)} \\

\hline
\end{tabular}
\end{table}


\clearpage


% REAL section
\subsection{REAL type}

From 1960, there are many computer company has they own design of floating point, this cause a huge problem in data exchange andcommunication, this bring out the standard of IEEE 754 \cite{web:wiki:ieee-754}. Because of the hardware design, if want to have a range of the data, then it need to sacrifice the accuracy, this called the "Round-off error" \cite{web:wiki:round-off_error,web:c:handle-round-off_error}.\\

That's why the IEEE 754 specific the best between range and the accuracy, but still it can't provide 100\% accuracy, also the design of IEEE 754 is not suitable for do the operation like sorting and comparison. So if want to handle the data as a floating point, then this need to jump out from the IEEE 754, and building the own data structure. Then this can store the data no matter how big it is with 100\% accuracy, but this cost a little more space.\\

We study the design and description of $float$ and $double$ design from some of the existing relational database \cite{web:wiki:floating_point,web:wiki:double-precision_floating-point_format,web:wiki:real-number,web:vcpp:data-type-ranges,web:wiki:c-data-types,web:c:data-types,web:transact-sql:int-bigint-smallint-tinyint,web:transact-sql:effective_number_of_bits-decimal_places-length,web:transact-sql:float-real,web:csharp:decimal,web:sql-server:decimal-float-real,web:c-cpp:floating-point-precision,web:mysql:query-sorting-numbers,web:mysql:using-decimal-to-record-float-point,web:mysql:sql-manual-reference}. In these database, they usually design a data type as \textit{"Decimal"} to handle the problem of IEEE 754.\\

\textit{"Decimal"} is a unpack floating point which contain the sign, and the number are store as a string, this means each number will consume a byte to record it. When need to do sorting or comparison operation, it will need to read the data and convert it into string first, this means it need cost one more step before the process.\\

So if fellow the \textit{"Decimal"} design to handle the floating point, this will cost more spaces. Such as if store "100" as \textit{"Decimal"} type which will as cost 3 bytes ('1', '0', '0'), but if using the character type (char) to store it which just need 1 byte ('d' in ASCII). Also we want the Li's Hash can use the index table can do the sorting or comparison, so we create a data type as \textit{REAL} to handle the problem above.\\

\textit{REAL} (aka the \textit{"real number"} in mathematics \cite{web:wiki:real-number}) is combine the concept of \textit{"Decimal"} and design of $INTEGER$. First convert the floating point into string when inputting the value, then partition it into three part to store as figure \ref{fig:algorithm:real:data_format}: \textit{"Sign"}, \textit{"Integer"}, \textit{"Decimal"}. After that convert \textit{"Integer"} and \textit{"Decimal"} part back into bytes by using base256, so this can use less byte to store the value, also this can use the design in $INTEGER$ to do indexing, so that \textit{REAL} can also do sorting or comparison operation. Also becuase need keep the accuracy of the value, so the length of byte usage is dynamic.\\

\begin{figure}[h]
\centering
%\includegraphics[scale=1.0]{./algorithm/real/pic/design/data_format_v3.pdf}
\includegraphics[width=0.6\textwidth]{./algorithm/real/pic/design/data_format_v3.pdf}
\caption{Data format of \textit{REAL} type.}
\label{fig:algorithm:real:data_format}
\end{figure}

The data format (figure \ref{fig:algorithm:real:data_format}) of \textit{REAL} is little bit different than the normal data concept, the value is as same as normal, the number at the left hand side means larger.\\

But when in the storage view is different, the \textit{"Integer"} part is store in normal and inverted order which will explain in the example of each operation, but the \textit{"Decimal"} part is using inverted which means the value will inverted when it stored. We use figure \ref{fig:algorithm:real:data_store_inverted} to explain why we do this.

\begin{figure}[h]
\centering
    \begin{subfigure}[b]{0.4\textwidth}
        \includegraphics[width=\textwidth]{./algorithm/real/pic/design/data_store_inverted_1_v1.pdf}
        \caption{Normal order}
        \label{fig:algorithm:real:data_store_inverted_1}
    \end{subfigure}%
    ~ %add desired spacing between images, e. g. ~, \quad, \qquad etc.
          %(or a blank line to force the subfigure onto a new line)
    \begin{subfigure}[b]{0.4\textwidth}
        \includegraphics[width=\textwidth]{./algorithm/real/pic/design/data_store_inverted_2_v1.pdf}
        \caption{Inverted order}
        \label{fig:algorithm:real:data_store_inverted_2}
    \end{subfigure}

    \caption{Value storage}
    \label{fig:algorithm:real:data_store_inverted}
\end{figure}

If the data store as the same order as usual, the sample data will store like figure \ref{fig:algorithm:real:data_store_inverted_1}, the problem is if we treat \textit{'01'} as a value, it will convert into \textit{'1'} and lost its owns meaning, because \textit{'.1'} is not equal \textit{'.01'}. So if invert the \textit{"Decimal"} part, the value will look like figure \ref{fig:algorithm:real:data_store_inverted_2}. It shows the value can be store without missing value, the only is a additional convert operation is need to revert back into the real value when return data.\\

\begin{figure}[h]
\centering
%\includegraphics[scale=1.0]{./algorithm/real/pic/design/example_v4.pdf}
\includegraphics[width=0.8\textwidth]{./algorithm/real/pic/design/example_v4.pdf}
\caption{The index tables of \textit{REAL} type.}
\label{fig:algorithm:real:example}
\end{figure}

Figure \ref{fig:algorithm:real:example} is the example when storing data into index table. This table start with $root$ which is pointing to the \textit{"Sign"} part which is store with the last byte of \textit{"Integer"} part.

The \textit{"Integer"} is indexing using n-gram as normal, but it will index in two way:

\begin{enumerate}

\item When indexing the \textit{"Integer"} part only (like $'$+$\backslash44\backslash1'$ in figure), it is start from the last byte to the first, and then will pointing to the element node which contain a flag that means as $decimal$, which is represent to begin \textit{"Decimal"} part.

\item Like $'$+$\backslash1\backslash44\_\backslash2\backslash142'$ in figure, the order of the \textit{"Integer"} part is store as from the left to right when it is storing with the \textit{"Decimal"} part.

\end{enumerate}

And \textit{"Decimal"} is store from right to left, and because the inverted string value design which means if the value is small then it will become a greater value.

This order of \textit{"Integer"} and \textit{"Decimal"} part design is because this can do faster searching the key when doing sorting and comparison, this will explan detail in fellowing section.

% Insertion section
\subsubsection{Insertion}

In description and figure \ref{fig:algorithm:real:example} have already mentioned some of the flow of insertion, so in here will show the table if insert another value into the table. Insert $\pi (3.14159)$ into table, which will become like figure \ref{fig:algorithm:real:insertion:example}.

\begin{figure}[h]
\centering
%\includegraphics[scale=0.45]{./algorithm/real/pic/insertion/example_v4.png}
\includegraphics[width=0.8\textwidth]{./algorithm/real/pic/insertion/example_v4.pdf}
\caption{The table after inserted $\pi (3.14159)$.}
\label{fig:algorithm:real:insertion:example}
\end{figure}

Figure \ref{fig:algorithm:real:insertion:example} shows that the $\pi (+3.14159)$ is store the value by its \textit{"Sign"} $(+)$, \textit{"Integer"} $(3)$ and \textit{"Decimal"} $(95141)$.

The \textit{"Sign"} is pointing the last byte of the \textit{"Integer"}, the reason of point to the last byte is because of the dynamic length of \textit{REAL}, also assume the the byte of the data is longer than the input byte length:

\begin{enumerate}

\item  If the \textit{"Sign"} is pointing the first byte, then when we search the result for the input, this may need to compare more byte or we need to get the whole \textit{"Integer"} in the worst case to sure that this value is suitable or not to the input.

\item If \textit{"Sign"} is pointing the last byte, then we just need to compare few bytes or the same byte length of the input that we can immediately to know that is a result or not. So this indexing can speed up the searching.

\end{enumerate}

Time complexity should be $O(b!)$ which domain as $O(b)$, and $b$ is the length of the byte needed where $b = 4$ in this case.



% Deletion section
\subsubsection{Deletion}

Deletion is just do the opposite insertion to remove the byte and decrease the count. So time complexity be $O(b)$.



% Modification section
\subsubsection{Modification}

The modify flow are similar as \textit{INTEGER} type, remove the key which don't needed and add the count if the byte is the same. So follow the example in figure \ref{fig:algorithm:real:insertion:example} and then modify -$0.01$ to +$0.0$ (because zero don't contain positive or negative sign, so using positive sign should be fine), the table will look like figure \ref{fig:algorithm:real:modification:example} and the time complexity be $O(b)$.

\begin{figure}[h]
\centering
%\includegraphics[scale=0.5]{./algorithm/real/pic/modification/example_v4.pdf}
\includegraphics[width=0.8\textwidth]{./algorithm/real/pic/modification/example_v4.pdf}
\caption{The table after modified the value.}
\label{fig:algorithm:real:modification:example}
\end{figure}



% Selection section
\subsubsection{Selection}

The follow operations will use figure \ref{fig:algorithm:real:modification:example} as the example.

% Selection section enumerate
\begin{enumerate}

% --------------------------------------------------------

% Equal
\item \textbf{Equal}

If search \textit{0.0}, then convert it into three part: \textit{"Sign"}, \textit{"Integer"} and \textit{"Decimal"}, which means $'$+$\backslash0\backslash\_\backslash0'$ and use this as key to get the result, this should take $O(1)$.

% --------------------------------------------------------

% Not equal
\item \textbf{Not equal}

If searching the result which is not equal \textit{0.0}, strat of convert it into key, then start from root, then recursively to find the data nodes and only skip the key of input. This operation should take $O(b)$.

% --------------------------------------------------------

% Less than
\item \textbf{Less than}

When comparing the \emph{"Less than"} or \emph{"Greater}, the flow is way different as the other because of the dynamic byte design.

For example searching \textit{300.0} in table, then convert into string $'$+$\backslash44\backslash1\_\backslash0'$ which the length of \textit{"Integer"} is \textit{2} and \textit{1} for \textit{"Decimal"} that the two value is use to let the compare function know how deep of the byte need to search.

% Less than section enumerate
\begin{enumerate}

% Input value is a negative value
\item \textbf{Input value is a negative value}

\begin{enumerate}

\item Start from root and only get the nodes which the \textit{"Sign"} are \textit{'-'}.

\item When searching \textit{"Integer"} part, recursively to search the element node contain \textit{decimal} with the length is \textit{"longer than or equal to"} to the \textit{"Integer"} length of input (in this case is \textit{2}). Then only remain the nodes that the value of \textit{"Integer"} is \textit{"larger than or equal to"} the value in \textit{"Integer"} part of input.

\item Searching in \textit{"Decimal"} part:
\begin{enumerate}

\item If the length of \textit{"Integer"} part is \textit{"longer than"} the input, then get all the data nodes.

\item If the length of \textit{"Integer"} part is \textit{"equal to"} the input, then recursively and only get the element nodes that the length is \textit{"greater than or equal to"} the length of \textit{"Decimal"} of input. And check the value of \textit{"Decimal"} is \textit{"larger than"} the value in \textit{"Decimal"} part of input.
\end{enumerate}

\end{enumerate}

% Input value is a positive value
\item \textbf{Input value is a positive value}

\begin{enumerate}

\item Get all the data nodes start from root which the \textit{"Sign"} are \textit{'-'}.

\item Next start from root and get the nodes which the \textit{"Sign"} are \textit{'+'}.

\item When searching \textit{"Integer"} part, recursively to search the element node contain \textit{decimal} with the length is \textit{"shorter than or equal to"} to the \textit{"Integer"} length of input (in this case is \textit{2}). Then only remain the nodes that the value of \textit{"Integer"} is \textit{"smaller than or equal to"} than the value in \textit{"Integer"} part of input.

\item Searching in \textit{"Decimal"} part:
\begin{enumerate}

\item If the length of \textit{"Integer"} part is \textit{"shorter than"} the input, then get all the data nodes.

\item If the length of \textit{"Integer"} part is \textit{"equal to"} the input, then recursively and only get the element nodes that the length is \textit{"shorter than or equal to"} the length of \textit{"Decimal"} of input. And check the value of \textit{"Decimal"} is \textit{"smaller than"} the value in \textit{"Decimal"} part of input.
\end{enumerate}

\end{enumerate}

% Input value is equal to 0.0
\item \textbf{Input value is equal to \textit{0.0}}

Get all the data nodes start from root which the \textit{"Sign"} are \textit{'-'}.

% End Less than section enumerate
\end{enumerate}

The \emph{"Less than or equal to"} comparison is just do the \emph{"Less than"} and \emph{"Equal"} operation and then combine both result for ouput. The time complexity is $O(b)$ for both operation.

% --------------------------------------------------------

% Greater than
\item \textbf{Greater than}

This comparison flow is similar as \emph{"Less than"}.

% Greater than section enumerate
\begin{enumerate}

% Input value is a negative value
\item \textbf{Input value is a negative value}

\begin{enumerate}

\item Get all the data nodes start from root which the \textit{"Sign"} are \textit{'+'}.

\item Next start from root and get the nodes which the \textit{"Sign"} are \textit{'-'}.

\item When searching \textit{"Integer"} part, recursively to search the element node contain \textit{decimal} with the length is \textit{"shorter than or equal to"} to the \textit{"Integer"} length of input. Then only remain the nodes that the value of \textit{"Integer"} is \textit{"larger than or equal to"} than the value in \textit{"Integer"} part of input.

\item Searching in \textit{"Decimal"} part:
\begin{enumerate}

\item If the length of \textit{"Integer"} part is \textit{"shorter than"} the input, then get all the data nodes.

\item If the length of \textit{"Integer"} part is \textit{"equal to"} the input, then recursively and only get the element nodes that the length is \textit{"shorter than or equal to"} the length of \textit{"Decimal"} of input. And check the value of \textit{"Decimal"} is \textit{"larger than"} the value in \textit{"Decimal"} part of input.
\end{enumerate}

\end{enumerate}

% Input value is a positive value
\item \textbf{Input value is a positive value}

\begin{enumerate}

\item Start from root and only get the nodes which the \textit{"Sign"} are \textit{'+'}.

\item When searching \textit{"Integer"} part, recursively to search the element node contain \textit{decimal} with the length is \textit{"longer than or equal to"} to the \textit{"Integer"} length of input. Then only remain the nodes that the value of \textit{"Integer"} is \textit{"larger than or equal to"} the value in \textit{"Integer"} part of input.

\item Searching in \textit{"Decimal"} part:
\begin{enumerate}

\item If the length of \textit{"Integer"} part is \textit{"longer than"} the input, then get all the data nodes.

\item If the length of \textit{"Integer"} part is \textit{"equal to"} the input, then recursively and only get the element nodes that the length is \textit{"greater than or equal to"} the length of \textit{"Decimal"} of input. And check the value of \textit{"Decimal"} is \textit{"larger than"} the value in \textit{"Decimal"} part of input.
\end{enumerate}

\end{enumerate}

% Input value is equal to 0.0
\item \textbf{Input value is equal to \textit{0.0}}

Get all the data nodes start from root which the \textit{"Sign"} are \textit{'+'} but skip the key of \textit{+0.0}.

% End Greater than section enumerate
\end{enumerate}

The \emph{"Greater than or equal to"} comparison is just do the \emph{"Greater than"} and \emph{"Equal"} operation and then combine both result for ouput. The time complexity is $O(b)$ for both operation.

% --------------------------------------------------------

% Between
\item \textbf{Between}

The \textit{between} operation of \textit{REAL} is as same as the \textit{between} operation of the \textit{signed INTEGER}, so skip the description of this part. The time complexity is $O(b)$.

% --------------------------------------------------------

\end{enumerate}


% Summary section
\subsubsection{Summary}

Table \ref{table:algorithm:real:summary:time_complexity} is a summary the time complexity of each opration in \textit{REAL} type.

\begin{table}[h]
\centering
\caption{Time complexity for \textit{REAL} type.}
\label{table:algorithm:real:summary:time_complexity}
\begin{tabular}{|c|c|}

\hline
\multicolumn{1}{|c|}{Operation} &
\multicolumn{1}{c|}{\tabincell{c}{
Time complexity \\ ($b$: The byte length of data)
}} \\

\hline
\multicolumn{1}{|c|}{Insert} &
\multicolumn{1}{c|}{$O(b)$} \\

\hline
\multicolumn{1}{|c|}{Modify} &
\multicolumn{1}{c|}{$O(b)$} \\

\hline
\multicolumn{1}{|c|}{Delete} &
\multicolumn{1}{c|}{$O(b)$} \\

\hline
\multicolumn{1}{|c|}{Equal} &
\multicolumn{1}{c|}{$O(1)$} \\

\hline
\multicolumn{1}{|c|}{\tabincell{c}{Equal (muti-value)}} &
\multicolumn{1}{c|}{$O(1)$} \\

\hline
\multicolumn{1}{|c|}{Not equal} &
\multicolumn{1}{c|}{$O(b)$} \\

\hline
\multicolumn{1}{|c|}{\tabincell{c}{Not equal (muti-value)}} &
\multicolumn{1}{c|}{$O(b)$} \\

\hline
\multicolumn{1}{|c|}{Less than} &
\multicolumn{1}{c|}{$O(b)$} \\

\hline
\multicolumn{1}{|c|}{Less than or equal} &
\multicolumn{1}{c|}{$O(b)$} \\

\hline
\multicolumn{1}{|c|}{Greater than} &
\multicolumn{1}{c|}{$O(b)$} \\

\hline
\multicolumn{1}{|c|}{Greater than or equal} &
\multicolumn{1}{c|}{$O(b)$} \\

\hline
\multicolumn{1}{|c|}{Between} &
\multicolumn{1}{c|}{$O(b)$} \\

\hline
\end{tabular}
\end{table}

\textit{REAL} is target for the data type of \textit{"long double"}, the only disadvantage that it will need more byte to store the value compare with \textit{"long double"}. From table \ref{table:algorithm:real:design_data_type} shows that the \textit{"float"} can store the range beyond the \textit{"Bigint"}, so that this mean it will need many \textit{"Bigint"} to store the value in \textit{"long double"}.

\begin{table}[h]
\centering
\caption{Information about data type.}
\label{table:algorithm:real:design_data_type}
\begin{tabular}{|c|c|c|}

\hline
\multicolumn{1}{|c|}{Data type} &
\multicolumn{1}{c|}{Range} &
\multicolumn{1}{c|}{Bytes} \\

\hline
\multicolumn{1}{|c|}{float} &
\multicolumn{1}{c|}{$3.40282e^{+038}$ $\thicksim$ $1.17549e^{-038}$} &
\multicolumn{1}{c|}{4} \\

\hline
\multicolumn{1}{|c|}{double} &
\multicolumn{1}{c|}{$1.79769e^{+308}$ $\thicksim$ $2.22507e^{-308}$} &
\multicolumn{1}{c|}{8} \\

\hline
\multicolumn{1}{|c|}{long double} &
\multicolumn{1}{c|}{$1.18973e^{+4932}$ $\thicksim$ $3.3621e^{-4932}$} &
\multicolumn{1}{c|}{16} \\

\hline
\multicolumn{1}{|c|}{unsigned int} &
\multicolumn{1}{c|}{0 $\thicksim$ 4294697295} &
\multicolumn{1}{c|}{4} \\

\hline
\multicolumn{1}{|c|}{\tabincell{c}{
unsigned long long int \\ (Bigint)
}} &
\multicolumn{1}{c|}{0 $\thicksim$ 18446744073709551615} &
\multicolumn{1}{c|}{8} \\

\hline
\end{tabular}
\end{table}

But the advantage of \textit{REAL} that it can store the value with 100\% accuracy, also provide comparison and sorting, and it can store limitless data range. So no matter the basic use of the floating point such as Financial or Basic operations, these usage is hard to use more than five digital in \textit{"Decimal"} part. Also \textit{REAL} can store special data like science data such as the value in physics, this kind of usage may need to use up to thousand digital in \textit{"Decimal"} part, this is a normal range of \textit{"long double"}.\\



\clearpage



% BLOB section
\subsection{BLOB type}

Rather than the other data type, BLOB type is much more similar as the normal \textit{put()} because it don't do any indexing. So that it can't do any selection and comparison so that it need to work with other data type, there is a example show in figure \ref{fig:table_design:example} in section \ref{sec:table_design}.


\clearpage

% Summary and case study section
\subsection{Summary and case study}

Table \ref{table:algorithm:summary:time_complexity} is a summarization and the time complexity of each operation.\\

% Time complexity table
\begin{table}[h]
\centering
\caption{Time complexity of all data type ($b$: The byte length of data)}
\label{table:algorithm:summary:time_complexity}
\begin{tabular}{|c|c|c|c|c|c|}

%\hline
%\multicolumn{6}{|c|}{\tabincell{c}{
%Time complexity \\ ($b$: The byte length of data)
%}} \\

\hline
\multicolumn{1}{|c|}{Operation} &
\multicolumn{1}{c|}{BLOB} &
\multicolumn{1}{c|}{STRING} &
\multicolumn{1}{c|}{BOOLEAN} &
\multicolumn{1}{c|}{INTEGER} &
\multicolumn{1}{c|}{REAL} \\

\hline
\multicolumn{1}{|c|}{Insertion} &
\multicolumn{1}{c|}{X} &
\multicolumn{1}{c|}{$O(b)$} &
\multicolumn{1}{c|}{$O(1)$} &
\multicolumn{1}{c|}{$O(b)$} &
\multicolumn{1}{c|}{$O(b)$} \\

\hline
\multicolumn{1}{|c|}{Modification} &
\multicolumn{1}{c|}{X} &
\multicolumn{1}{c|}{$O(b)$} &
\multicolumn{1}{c|}{$O(1)$} &
\multicolumn{1}{c|}{$O(b)$} &
\multicolumn{1}{c|}{$O(b)$} \\

\hline
\multicolumn{1}{|c|}{Deletion} &
\multicolumn{1}{c|}{X} &
\multicolumn{1}{c|}{$O(b)$} &
\multicolumn{1}{c|}{$O(1)$} &
\multicolumn{1}{c|}{$O(b)$} &
\multicolumn{1}{c|}{$O(b)$} \\

\hline
\multicolumn{1}{|c|}{Equal} &
\multicolumn{1}{c|}{X} &
\multicolumn{1}{c|}{\tabincell{c}{$O(1)$\\(Exact matching)}} &
\multicolumn{1}{c|}{$O(1)$} &
\multicolumn{1}{c|}{$O(1)$} &
\multicolumn{1}{c|}{$O(1)$} \\

\hline
\multicolumn{1}{|c|}{\tabincell{c}{Equal (muti-value)}} &
\multicolumn{1}{c|}{X} &
\multicolumn{1}{c|}{\tabincell{c}{$O(1)$\\(Exact matching)}} &
\multicolumn{1}{c|}{X} &
\multicolumn{1}{c|}{$O(1)$} &
\multicolumn{1}{c|}{$O(1)$} \\

\hline
\multicolumn{1}{|c|}{Not equal} &
\multicolumn{1}{c|}{X} &
\multicolumn{1}{c|}{$O(b)$} &
\multicolumn{1}{c|}{$O(1)$} &
\multicolumn{1}{c|}{$O(b)$} &
\multicolumn{1}{c|}{$O(b)$} \\

\hline
\multicolumn{1}{|c|}{\tabincell{c}{Not equal (muti-value)}} &
\multicolumn{1}{c|}{X} &
\multicolumn{1}{c|}{$O(b)$} &
\multicolumn{1}{c|}{X} &
\multicolumn{1}{c|}{$O(b)$} &
\multicolumn{1}{c|}{$O(b)$} \\

\hline
\multicolumn{1}{|c|}{Less than} &
\multicolumn{1}{c|}{X} &
\multicolumn{1}{c|}{X} &
\multicolumn{1}{c|}{X} &
\multicolumn{1}{c|}{$O(b)$} &
\multicolumn{1}{c|}{$O(b)$} \\

\hline
\multicolumn{1}{|c|}{Less than or equal} &
\multicolumn{1}{c|}{X} &
\multicolumn{1}{c|}{X} &
\multicolumn{1}{c|}{X} &
\multicolumn{1}{c|}{$O(b)$} &
\multicolumn{1}{c|}{$O(b)$} \\

\hline
\multicolumn{1}{|c|}{Greater than} &
\multicolumn{1}{c|}{X} &
\multicolumn{1}{c|}{X} &
\multicolumn{1}{c|}{X} &
\multicolumn{1}{c|}{$O(b)$} &
\multicolumn{1}{c|}{$O(b)$} \\

\hline
\multicolumn{1}{|c|}{Greater than or equal} &
\multicolumn{1}{c|}{X} &
\multicolumn{1}{c|}{X} &
\multicolumn{1}{c|}{X} &
\multicolumn{1}{c|}{$O(b)$} &
\multicolumn{1}{c|}{$O(b)$} \\

\hline
\multicolumn{1}{|c|}{Between} &
\multicolumn{1}{c|}{X} &
\multicolumn{1}{c|}{X} &
\multicolumn{1}{c|}{X} &
\multicolumn{1}{c|}{$O(b)$} &
\multicolumn{1}{c|}{$O(b)$} \\

\hline
\multicolumn{1}{|c|}{\tabincell{c}{Search \\ (Exact matching)}} &
\multicolumn{1}{c|}{X} &
\multicolumn{1}{c|}{$O(1)$} &
\multicolumn{1}{c|}{X} &
\multicolumn{1}{c|}{X} &
\multicolumn{1}{c|}{X} \\

\hline
\multicolumn{1}{|c|}{\tabincell{c}{Search \\ (Prefix matching)}} &
\multicolumn{1}{c|}{X} &
\multicolumn{1}{c|}{$O(b)$} &
\multicolumn{1}{c|}{X} &
\multicolumn{1}{c|}{X} &
\multicolumn{1}{c|}{X} \\

\hline
\multicolumn{1}{|c|}{\tabincell{c}{Search \\ (Suffix matching)}} &
\multicolumn{1}{c|}{X} &
\multicolumn{1}{c|}{$O(b)$} &
\multicolumn{1}{c|}{X} &
\multicolumn{1}{c|}{X} &
\multicolumn{1}{c|}{X} \\

\hline
\multicolumn{1}{|c|}{\tabincell{c}{Search \\ (Partial matching)}} &
\multicolumn{1}{c|}{X} &
\multicolumn{1}{c|}{$O(b)$} &
\multicolumn{1}{c|}{X} &
\multicolumn{1}{c|}{X} &
\multicolumn{1}{c|}{X} \\

\hline
\end{tabular}
\end{table}

% Capacity table
\begin{table}[h]
\centering
\caption{Capacity needed of with all type (Unit: byte)}
\label{table:algorithm:summary:capacity}
\begin{tabular}{|c|c|c|c|c|c|}

\hline
\multicolumn{1}{|c|}{Type} &
\multicolumn{1}{c|}{BLOB} &
\multicolumn{1}{c|}{STRING} &
\multicolumn{1}{c|}{BOOLEAN} &
\multicolumn{1}{c|}{INTEGER} &
\multicolumn{1}{c|}{REAL} \\

\hline
\multicolumn{1}{|c|}{byte length} &
\multicolumn{1}{c|}{$n$} &
\multicolumn{1}{c|}{$n$} &
\multicolumn{1}{c|}{$n$} &
\multicolumn{1}{c|}{$8$} &
\multicolumn{1}{c|}
{\tabincell{c}{
$1 + n + m$ \\ ($n$ is byte length of \textit{"Integer"} part and \\
$m$ is byte length of \textit{"Decimal"} part)
}} \\

\hline
\multicolumn{1}{|c|}{index size} &
\multicolumn{1}{c|}{$0$} &
\multicolumn{1}{c|}{$$} &
\multicolumn{1}{c|}
{\tabincell{c}{
$2$ \\ (\textit{'+'} and \textit{'-'})
}} &
\multicolumn{1}{c|}{$$} &
\multicolumn{1}{c|}{$$} \\

\hline
\multicolumn{1}{|c|}{Total size} &
\multicolumn{1}{c|}{$n$} &
\multicolumn{1}{c|}{$$} &
\multicolumn{1}{c|}{$n + 2$} &
\multicolumn{1}{c|}{$$} &
\multicolumn{1}{c|}{$$} \\

\hline
\end{tabular}
\end{table}


\clearpage

% ------------------------------------------------
% End of page
% ------------------------------------------------
