
% ----------------------------------------------------------------------------
%                               Chinese abstract
%                                   中文摘要
% ----------------------------------------------------------------------------

% Set the line spacing
%\baselineskip = 26pt

% ------------------------------------------------

% Page start
\newpage
%\cleardoublepage
\phantomsection
\addcontentsline{toc}{chapter}{Abstract}

% ------------------------------------------------

\begin{center}
\large \textbf{摘要}
%\label{abstract}
\end{center}

% ------------------------------------------------

由於在嵌入式系統中,匯流排的數量急劇地上升,匯流排規劃的品質好壞成為決定嵌入式系統效能與系統功率消耗的重要指標。為了不讓匯流排的問題造成晶片設計流程後期的限制,在早期平面佈局時就將匯流排的因素加入考慮是比較理想的處理方式。近年來,匯流排導向平面佈局的問題吸引了許多人的注意,並且已有相當多的方法被提出於處理相關的問題。然而,目前的演算法都採取了比較簡單的問題定義,忽略了匯流排腳位的位置和方向等相關的資訊。忽略了上述的資訊可能會造成系統效能被高估的情況。因此,我們提出了一個匯流排導向的平面佈局演算法,我們所提出的演算法在規劃匯流排時也考慮到腳位的位置與方向的資訊,讓演算法所得到的結果能夠更符合實際情況。由於在規劃匯流排的同時也考慮匯流排腳位的資訊,匯流排轉彎的地方並不局限在匯流排所連接的方塊上。因此,在處理匯流排繞線問題時也比較有彈性。随著整個匯流排拓撲的形狀變得更有彈性,匯流排繞線的成功率也相對地提高,許多較佳的結果就能被發掘出來。將我們的演算法所得到的結果與目前最佳的匯流排導向平面佈局演算法所得到的結果比較,我們演算法的執行時間相對快了3.5倍,繞線成功率提高了1.2倍,匯流排長度少了1.8倍,並且將平面佈局中空白的區域降低了1.2倍。

% ------------------------------------------------

\begin{itemize}
\item {\bf 關鍵字:} 平面規劃, 匯流排規劃
\end{itemize}

% ------------------------------------------------

% End of page

% ------------------------------------------------
