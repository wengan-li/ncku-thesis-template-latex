% ------------------------------------------------
\StartChapter{Introduction}{chapter:how-to:write:introduction}
% ------------------------------------------------
\section{基本介紹}

  \begin{framed}
  \begin{verbatim}
    - 參數 -
    空掉: 是指可以寫成{} (即是不寫任何內容), 但是{}必須要存在.
    不填: 是指{}不必存在.

    但如果要使用最後的參數, 中間'可不填'則變成'可空掉',
    否則會錯誤或沒法使用.
  \end{verbatim}
  \end{framed}

這教學包含了原Latex和本模板特有的語法的使用方式和例子.

請注意原Latex語法會以英文小寫來顯示(\verb|\aabbcc|); 而本模板特有的語法會以英文大小寫混合(\verb|\AaBbCc|, 第一個字必定以大寫來顯示), 由於這些特有語法\textbf{不是}原Latex的語法, 所以不能直接應用在非本模板的Latex檔案上.

抄襲就是學習的第一步 (如同我們小時候去抄襲父母走路一樣), 如果你看完本教學都不知道怎樣寫的話, 証明是時候你親自下手實作, 我有留下了一些範本 (在'./context'下)以方便大家開始第一步, 之後就要靠大家自己的努力和實作, 再加上自己的探索能力了.

有問題的話, 可以有以下的地方找尋答案 (請使用這順序):
\begin{enumerate}
\item 直接研究在模板的Latex寫法 (在'./example'以下的所有檔案)
\item 查問懂得Latex的老師和同學 (這種人少得可憐, 可能去問Google老師會更快)
\item 去Wikibook\RefBib{web:latex:wikibooks} (這邊100\%會有你想要的答案, 問題是你會不會懂得使用)
\item 請求Google老師
\end{enumerate}

另外, 如果覺得本教學還缺少了什麼說明, 請告知.

% ------------------------------------------------
\EndChapter
% ------------------------------------------------
