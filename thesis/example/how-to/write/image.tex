% ------------------------------------------------
\StartChapter{Insert Image}{chapter:how-to:write:image}
% ------------------------------------------------
\section{介紹}

插入圖片其實有很多的玩法, 但是在畢業論文中, 它的放置位置則是非常固定的, 都是以中間為主, 之後就是插多張圖片. 因為它很固定, 所以我針對了插入單張或多張, 分別提供了以下的指令.

要注意的是, 圖片在畫面看到的大小, 跟真正寫到文件是不一樣的 (因為經過程式的自動縮放), 所以比例正常都要修改的.

\textbf{注意}: 由於Latex語法的設計上, 在2個'$\lbrace\rbrace$'之間不能斷行或出現空格, 否則會被判斷成其他的語法

% ------------------------------------------------
\section{單張}

  \begin{framed}
  \begin{verbatim}
   Scale:   比例 (1.0/空掉: 原大小, 0~1: 縮小, >1: 放大)
   Path:    位置
   Caption: 標題 (可空掉或不填)
   Label:   標簽 (可空掉或不填, 要使用標簽前必須同時使用標題)
   Angle:   角度 (可不填)

   插入圖片
   \InsertImage{Scale}{Path}{Caption}{Label}{Angle}

   插入圖片並置中
   \InsertCenterImage{Scale}{Path}{Caption}{Label}{Angle}
  \end{verbatim}
  \end{framed}

  \newpage

  {\bf 效果:}
  \begin{enumerate}
    \item
    {
      只填了比例和圖片位置\begin{verbatim}\InsertImage{1.0}{./image.png}\end{verbatim}
      \InsertImage{1.0}{./example/how-to/write/pic/Cc-by_new.svg.png}
    } % End of \item{}

    \item
    {
      使用置中的版本\begin{verbatim}\InsertCenterImage{1.0}{./image.png}\end{verbatim}
      \InsertCenterImage{1.0}{./example/how-to/write/pic/Cc-by_new.svg.png}
    } % End of \item{}

    \item
    {
      放大比例\begin{verbatim}\InsertCenterImage{1.5}{./image.png}\end{verbatim}
      \InsertCenterImage{1.5}{./example/how-to/write/pic/Cc-by_new.svg.png}
    } % End of \item{}

    \newpage

    \item
    {
      縮小比例\begin{verbatim}\InsertCenterImage{0.5}{./image.png}\end{verbatim}
      \InsertCenterImage{0.5}{./example/how-to/write/pic/Cc-by_new.svg.png}
    } % End of \item{}

    \item
    {
      增加標題並去掉比例的數字\begin{verbatim}\InsertCenterImage{}{./image.png}{Little man}\end{verbatim}
      \InsertCenterImage{}{./example/how-to/write/pic/Cc-by_new.svg.png}{Little man}
    } % End of \item{}

    \item
    {
      增加標簽\begin{verbatim}\InsertCenterImage{}{./image.png}{Little man No.1}{fig:little-man-no.1}\end{verbatim}
      之後可以使用ref去引用\begin{verbatim}\ref{fig:little-man-no.1}\end{verbatim}
      \InsertCenterImage{}{./example/how-to/write/pic/Cc-by_new.svg.png}{Little man No.1}{fig:little-man-no.1}

      e.g: 文中所指的人物一號 (Fig. \ref{fig:little-man-no.1}).
    } % End of \item{}

    \newpage

    \item
    {
      使用標簽, 但不需要標題\begin{verbatim}\InsertCenterImage{}{./image.png}{}{fig:little-man-no.2}\end{verbatim}
      使用ref去引用\begin{verbatim}\ref{fig:little-man-no.2}\end{verbatim}
      \InsertCenterImage{}{./example/how-to/write/pic/Cc-by_new.svg.png}{Little man No.2}{fig:little-man-no.2}

      e.g: 文中所指的人物二號 (在第\ref{fig:little-man-no.2}頁中曾提過). \\

      不使用標題但使用標簽有一個很大的分別是: 如果有標題的話, 那指向的是第幾個chapter的第幾張圖; 但如果沒有的話, 那指向的是出現那一張圖的頁碼.
    } % End of \item{}

    \item
    {
      使用角度去轉45度\begin{verbatim}\InsertCenterImage{}{./image.png}{Little man No.3}{fig:little-man-no.3}{45}\end{verbatim}
      使用ref去引用\begin{verbatim}\ref{fig:little-man-no.3}\end{verbatim}

      \InsertCenterImage{}{./example/how-to/write/pic/Cc-by_new.svg.png}{Little man No.3}{fig:little-man-no.3}{45}

      e.g: 文中所指的人物三號 (Fig. \ref{fig:little-man-no.3}).
    } % End of \item{}

  \end{enumerate}

% ------------------------------------------------
\newpage
\section{多張}

  如果要插入多張的話, 因為要能版面的範圍內, 同時又要能清楚顯示到你圖中的內容和文字, 理論上4張都已經算多的了. 所以多張的話, 分別放同不到頁面會比較好.

  設計上可插入最多8張的圖片, 同時會自動置中, 而且寫法會跟插入單張相近.

  \begin{framed}
  \begin{verbatim}
  Table:    設定放置多張圖片的表格
    ImagePerRow: 每一列多少張圖片
    Caption:     標題 (可空掉)
    Label:       標簽 (可空掉, 要使用標簽前必須同時使用標題)

  Image 1~8: 各張圖片的設定
    設定方式跟使用\InsertImage和\InsertCenterImage是
    幾乎一樣的, 但是沒了比例去設定, 會自動使用每一列算出的比例.

  插入多張圖片
    \InsertMultiImages{Table}{Image1}{Image2}{Image3}{Image4}{Image5}{Image6}{Image7}{Image8}

  同時可接受這種使用方式 ('%'是必須存在的,
  以防止被認為這是新段落, 而且不要有空格在當中)
    \InsertMultiImages%
    {%
      {ImagePerRow}{Caption}{Label}
    }%
    {%
      {Path}{Caption}{Label}{Angle}
    }%
    {%
      ...
    }%
    {%
      {Path}{Caption}{Label}{Angle}
    }%
  \end{verbatim}
  \end{framed}

  \newpage

  {\bf 效果:}
  \begin{enumerate}
    \item
    {
      插入2張圖片, 以1張圖為一列
      \begin{verbatim}
      \InsertMultiImages%
      {%
        {1}{2 images, 1 image per row}{}
      }%
      {%
        {./image.png}
      }%
      {%
        {./image.png}
      }%
      \end{verbatim}
      \InsertMultiImages%
      {%
        {1}{2 images, 1 image per row}{}
      }%
      {%
        {./example/how-to/write/pic/CC-BY-NC.png}
      }%
      {%
        {./example/how-to/write/pic/CC-BY-NC-ND.png}
      }%
    } % End of \item{}

    \newpage

    \item
    {
      插入5張圖片, 以2張圖為一列, 並有2張圖轉變角度, 同時有3張圖片做了標簽
      \begin{verbatim}
      \InsertMultiImages%
      {%
        {2}{5 images, 2 images per row}{fig:example:mi2:fig1}
      }%
      {%
        {./image.png}{Image 1}
      }%
      {%
        {./image.png}{Image 2}{fig:example:mi2:fig2}{-20}
      }%
      {%
        {./image.png}{Image 3}
      }%
      {%
        {./image.png}{Image 4}{fig:example:mi2:fig3}
      }%
      {%
        {./image.png}{Image 5}{}{45}
      }%
      \end{verbatim}
      \InsertMultiImages%
      {%
        {2}{5 images, 2 images per row}{fig:example:mi2:fig1}
      }%
      {%
        {./example/how-to/write/pic/CC-BY-NC.png}{Image 1}
      }%
      {%
        {./example/how-to/write/pic/CC-BY-NC-ND.png}{Image 2}{fig:example:mi2:fig2}{-20}
      }%
      {%
        {./example/how-to/write/pic/CC-BY-NC-SA.png}{Image 3}
      }%
      {%
        {./example/how-to/write/pic/CC-BY-ND.png}{Image 4}{fig:example:mi2:fig3}
      }%
      {%
        {./example/how-to/write/pic/CC-BY-SA.png}{Image 5}{}{45}
      }%
    } % End of \item{}

      e.g: 
      引用大圖 (Fig.\ref{fig:example:mi2:fig1}) ,
      引用小圖 (fig.\ref{fig:example:mi2:fig2}, fig.\ref{fig:example:mi2:fig3}).
  \end{enumerate}

% ------------------------------------------------
\EndChapter
% ------------------------------------------------
