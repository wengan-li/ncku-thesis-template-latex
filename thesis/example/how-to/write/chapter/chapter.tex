% ------------------------------------------------
\StartSection{章節 Chapter/Section}{chapter:how-to:write:chapter-section}

% ------------------------------------------------
\StartSubSection{介紹}

編寫任何的文章, 都會使用不同的章節來把內容進行分區. 例如學校的排版樣子大約:

\begin{minipage}{\textwidth}
  \begin{framed}
    \centerline{\LARGE Chapter X}
    \vspace{0.2cm}
    \centerline{\LARGE 這是標題}

    \vspace{0.5cm}
    \mbox{\Large X.1 子項目}
    \hspace{0.2cm}\mbox{項目內容 ...}

    \vspace{0.3cm}
    \hspace{0.2cm}\mbox{\large X.1.1 X.1的子項目}
    \hspace{0.4cm}\mbox{項目內容 ...}
  \end{framed}
\end{minipage}

所以針對這些功能, 本模版提供:
\begin{framed}
  \begin{verbatim}
  主要章節
  Title: 標題 (必填)
  Label: 標簽 (選填)
  \StartChapter{ Title }{ Label }
  \EndChapter

  次章節
  Title: 標題 (必填)
  Label: 標簽 (選填)
  \StartSection{ Title }{ Label }

  次章節的子章節
  Title: 標題 (必填)
  Label: 標簽 (選填)
  \StartSubSection{ Title }{ Label }

  子章節的子章節
  Title: 標題 (必填)
  Label: 標簽 (選填)
  \StartSubSubSection{ Title }{ Label }

  \end{verbatim}
\end{framed}

所以針對剛剛的例子, 它的Latex寫法為:

\begin{minipage}{\textwidth}
  \begin{framed}
  \begin{verbatim}
    \StartChapter{這是標題}

    \StartSection{子項目}
    項目內容 ...

    \StartSubSection{X.1的子項目}
    項目內容 ...

    \StartSubSubSection{X.1的子項目的子項目}
    項目內容 ...

    \EndChapter
  \end{verbatim}
  \end{framed}
\end{minipage}

