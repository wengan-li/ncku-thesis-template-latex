% ------------------------------------------------
\StartSection{圖片 Image}{chapter:how-to:write:image}
% ------------------------------------------------

\StartSubSection{介紹}

插入圖片其實有很多的玩法, 但是在畢業論文中, 它的放置位置則是非常固定的, 都是以中間為主, 之後就是插多張圖片. 因為它很固定, 所以本模板針對了插入單張或多張, 分別提供了以下的指令.

要注意的是, 圖片在畫面看到的大小, 跟真正寫到文件是不一樣的 (因為經過程式的自動縮放), 所以比例正常都要修改的.

留意的是, 圖片的路徑跟正常日常使用的路徑會不一樣, 是使用所謂的"相對路徑" (Relative path), 而起點是論文的主檔案(thesis.tex).

\noindent 例如 (以Windows的路徑為例子):\\
主檔案thesis.tex在: "\verb|C:\thesis\thesis.tex|"\\
圖片A: "\verb|C:\thesis\some_dir\A.png|"\\
圖片B: "\verb|C:\thesis\some_dir\B.png|"\\
使用時以"\verb|./some_dir/A.png|", "\verb|./some_dir/B.png|"的方式來使用, 注意是"$/$"而不是"$\backslash$".

% ------------------------------------------------
\newpage
\StartSubSection{單張}

  \begin{framed}
  \begin{verbatim}
  Path:   圖片位置 (必填)

  Options 設定 (使用','來分隔, 不分先後順序)
    scale:   比例 (選填, 預設: 1.0)
    (1.0: 原大小; 0.x ~ < 1.0: 縮小; > 1.0: 放大)
    caption: 標題 (選填)
    label:   標簽 (選填, 必須要配合caption使用, 否則無效)
    angle:   角度 (選填, 預設: 90)
    align:   center (置中)

  插入圖片
  \InsertImage[Options]{Path}

  E.g
    \InsertImage
    [caption={這 是 標 題}]
      {./image.png}

    \InsertImage
      [align = center, scale=0.5,
        angle=45,
        caption={這 是 標 題},
        label={this:is:label}]
      {./image.png}

    每一項資料可以使用斷行來分隔以保持可讀性.
    caption和label必須要使用'{}'才能有空格的句子.
  \end{verbatim}
  \end{framed}

  \newpage
  {\bf 效果:}
  \begin{enumerate}
  \item
  {
    只填了比例和圖片位置
    \begin{verbatim}
    \InsertImage
      {./image.png}
    \end{verbatim}
    \InsertImage
      {./example/how-to/write/image/pic/Cc-by_new.svg.png}
  } % End of \item{}

  \item
  {
    使用置中的版本
    \begin{verbatim}
    \InsertImage
      [align = center]
      {./image.png}
    \end{verbatim}
    \InsertImage
      [align = center]
    {./example/how-to/write/image/pic/Cc-by_new.svg.png}
  } % End of \item{}

  \item
  {
    放大比例
    \begin{verbatim}
    \InsertImage
      [align = center, scale=1.5]
      {./image.png}
    \end{verbatim}
    \InsertImage
      [align = center, scale=1.5]
      {./example/how-to/write/image/pic/Cc-by_new.svg.png}
  } % End of \item{}

  \newpage

  \item
  {
    縮小比例
    \begin{verbatim}
    \InsertImage
      [align = center, scale=0.5]
      {./image.png}
    \end{verbatim}
    \InsertImage
      [align = center, scale=0.5]
      {./example/how-to/write/image/pic/Cc-by_new.svg.png}
  } % End of \item{}

  \item
  {
    增加標題並去掉比例的數字
    \begin{verbatim}
    \InsertImage
      [align = center, caption={Little man}]
      {./image.png}
    \end{verbatim}
    \InsertImage
      [align = center, caption={Little man}]
      {./example/how-to/write/image/pic/Cc-by_new.svg.png}
  } % End of \item{}

  \newpage
  \item
  {
    增加標簽
    \begin{verbatim}
    \InsertImage
      [align = center, caption={Little man No.1},
        label={fig:little-man-no.1}]
      {./image.png}
    \end{verbatim}

    之後可以使用\verb| \RefTo |去引用 \verb| \RefTo{fig:little-man-no.1} |
    \InsertImage
      [align = center, caption={Little man No.1},
        label={fig:little-man-no.1}]
      {./example/how-to/write/image/pic/Cc-by_new.svg.png}

    e.g: 文中所指的人物一號 (Fig \RefTo{fig:little-man-no.1}).
  } % End of \item{}

  \item
  {
    使用角度去轉45度
    \begin{verbatim}
    \InsertImage
      [align = center, angle=45,
        caption={Little man No.2},
        label={fig:little-man-no.2}]
      {./image.png}
    \end{verbatim}

    使用\verb| \RefTo |去引用 \verb| \RefTo{fig:little-man-no.2} |
    \InsertImage
      [align = center, angle=45,
        caption={Little man No.2},
        label={fig:little-man-no.2}]
      {./example/how-to/write/image/pic/Cc-by_new.svg.png}

    e.g: 文中所指的人物二號 (Fig \RefTo{fig:little-man-no.2}).
  } % End of \item{}

  \newpage
  \item
  {
    把圖放在表格中, 這時候是使用\verb|\InsertImage{}|(如要置中則設定table就可), 而且不能使用caption和label (正常寫法應該用不到, 例如出現'Fig X.X'這種字在table中). (有關表格table的使用, 請參考Chap \RefTo{chapter:how-to:write:table})

    \begin{verbatim}
    \begin{table}[h]
    \centering
    \begin{tabular}{|c|c|}
      \hline
      \textbf{\underline{Website}} &
        \textbf{\underline{URL}} \\ \hline

      \begin{tabular}[c]{@{}c@{}}
      \includegraphics[scale=0.1]
        {./apple.png} \\ Apple
      \end{tabular} & \url{www.apple.com}  \\ \hline

      \begin{tabular}[c]{@{}c@{}}
      \includegraphics[scale=0.1]
        {./google.png} \\ Google
      \end{tabular} & \url{www.google.com} \\ \hline
    \end{tabular}
    \end{table}
    \end{verbatim}

    \begin{table}[h]
    \centering
    \label{table:how-to:write:image:insert-image-into-table}
    \begin{tabular}{|c|c|}
      \hline
      \textbf{\underline{Website}} &
        \textbf{\underline{URL}} \\ \hline

      \begin{tabular}[c]{@{}c@{}}
      \includegraphics[scale=0.1]
       {./example/how-to/write/image/pic/apple.jpg} \\ Apple
      \end{tabular} & \url{www.apple.com}  \\ \hline

      \begin{tabular}[c]{@{}c@{}}
      \includegraphics[scale=0.1]
        {./example/how-to/write/image/pic/google.png} \\ Google
      \end{tabular} & \url{www.google.com} \\ \hline
    \end{tabular}
    \end{table}

  } % End of \item{}
  \end{enumerate}

% ------------------------------------------------
\newpage
\StartSubSection{多張}

  如果要插入多張的話, 因為要能一頁版面的範圍內, 同時又要能清楚顯示到你圖中的內容和文字, 理論上4張都已經算多的了. 所以真的數量比較多的話, 分別放同不到頁面會比較好閱讀.

  設計上可插入1$\sim$8張的圖片, 而且寫法會跟插入單張相近.

  \begin{framed}
  \begin{verbatim}
  Options 主圖的設定 (使用','來分隔, 不分先後順序)
    perrow:  每一列多少張圖片 (選填, 預設: 1)
    caption: 標題 (選填)
    label:   標簽 (選填, 必須要配合caption使用, 否則無效)
    align:   center (置中)

  Image 1~8: 各張圖片的設定
    設定方式跟使用\InsertImage和\InsertCenterImage是一樣的
    [Image options] -> [Options]
    {Image path}  -> {Path}

  插入多張圖片
    \InsertMultiImages[Options] %
    {
      [Image options]{Image path}
    }%
    {
      ...
    }%
    {
      [Image options]{Image path}
    }
  ('%'是必須存在的, 以防止被Latex認為這是新段落)
  \end{verbatim}
  \end{framed}

  \newpage

  {\bf 效果:}
  \begin{enumerate}
  \item
  {
    插入2張圖片, 以1張圖為一列
    \begin{verbatim}
    \InsertMultiImages
    [align = center,
      caption = {2 images and 1 image per row}] %
    {
      {./image.png}
    }%
    {
      {./image.png}
    }
    \end{verbatim}
    \InsertMultiImages
    [align = center,
      caption = {2 images and 1 image per row}] %
    {
      {./example/how-to/write/image/pic/CC-BY-NC.png}
    }%
    {
      {./example/how-to/write/image/pic/CC-BY-NC-ND.png}
    }
  } % End of \item{}

  \item
  {
    插入2張圖片, 以2張圖為一列
    \begin{verbatim}
    \InsertMultiImages
    [perrow = 2, align = center,
      caption = {2 images and 2 images per row}] %
    {
      {./image.png}
    }%
    {
      {./image.png}
    }
    \end{verbatim}
    \InsertMultiImages
    [perrow = 2, align = center,
      caption = {2 images and 2 images per row}] %
    {
      {./example/how-to/write/image/pic/CC-BY-NC.png}
    }%
    {
      {./example/how-to/write/image/pic/CC-BY-NC-ND.png}
    }
  } % End of \item{}

  \newpage
  \item
  {
    插入3張圖片, 2張圖一列, 並有1張圖轉變角度, 同時主圖跟2張子圖片做了標簽
    \begin{verbatim}
    \InsertMultiImages
    [perrow = 2, align = center,
      caption = {5 images and 2 image per row},
      label = {fig:example:mi2:fig1}] %
    {
      [caption = {Image 1},
      label = {fig:example:mi2:fig1}]
      {./image.png}
    }%
    {
      [caption = {Image 2},
      label = {fig:example:mi2:fig2},
      angle = -20]
      {./image.png}
    }%
    {
      [caption = {Image 3}]
      {./image.png}
    }
    \end{verbatim}

%    \newpage
    效果會是這樣: \\
    \InsertMultiImages
    [perrow = 2, align = center,
      caption = {3 images and 2 images per row},
      label = {fig:example:mi2:mfig}] %
    {
      [caption = {Image 1},
      label = {fig:example:mi2:fig1}]
      {./example/how-to/write/image/pic/CC-BY-NC.png}
    }%
    {
      [caption = {Image 2},
      label = {fig:example:mi2:fig2},
      angle = -20]
      {./example/how-to/write/image/pic/CC-BY-NC-ND.png}
    }%
    {
      [caption = {Image 3}]
      {./example/how-to/write/image/pic/CC-BY-NC-SA.png}
    }

    e.g: 
    引用主圖 (Fig \RefTo{fig:example:mi2:mfig}) ,
    引用子圖片 (Fig \RefTo{fig:example:mi2:fig1}, Fig \RefTo{fig:example:mi2:fig2}).
  } % End of \item{}

  \newpage
  \item
  {
    插入4張圖片, 2張圖一列, 只有主圖做了標簽.\\
    如果需要不填內容, 但需要圖片的編號的話, 就在caption填寫'{ }'(有空格在中間), 而'{}'則會被認為沒有填寫.
    \begin{verbatim}
    \InsertMultiImages
    [perrow = 2, align = center,
      caption = {4 images and 2 image per row},
      label = {fig:example:mi3:mfig}] %
    {
      [caption = { }, label = {fig:example:mi3:fig1}]
      {./image.png}
    }%
    {
      [caption = {}, label = {fig:example:mi3:fig2}]
      {./image.png}
    }%
    {
      [caption = { }, label = {fig:example:mi3:fig3}]
      {./image.png}
    }%
    {
      [caption = { }, label = {fig:example:mi3:fig4}]
      {./image.png}
    }
    \end{verbatim}

    \InsertMultiImages
    [perrow = 2, align = center,
      caption = {4 images and 2 image per row},
      label = {fig:example:mi3:mfig}] %
    {
      [caption = { },
      label = {fig:example:mi3:fig1}]
      {./example/how-to/write/image/pic/CC-BY-NC.png}
    }%
    {
      [caption = {},
      label = {fig:example:mi3:fig2}]
      {./example/how-to/write/image/pic/CC-BY-NC-ND.png}
    }%
    {
      [caption = { },
      label = {fig:example:mi3:fig3}]
      {./example/how-to/write/image/pic/CC-BY-NC-SA.png}
    }%
    {
      [caption = { },
      label = {fig:example:mi3:fig4}]
      {./example/how-to/write/image/pic/CC-BY-ND.png}
    }

    可以看得出圖片的編號不一樣了\\
    引用主圖 (Fig \RefTo{fig:example:mi3:mfig})\\
    引用子圖片(a) (Fig \RefTo{fig:example:mi3:fig1})\\
    引用子圖片(b) (由於這張圖的caption是沒設定, 所以label無效)\\
    引用子圖片(c) (Fig \RefTo{fig:example:mi3:fig3})\\
    引用子圖片(d) (Fig \RefTo{fig:example:mi3:fig4})
  } % End of \item{}

  % 最後一張圖位置怪怪
  \newpage
  \item
  {
    插入8張圖片, 2張圖一列, 只有主圖填了標題.\\
    \begin{verbatim}
    \InsertMultiImages
    [perrow = 2, align = center,
      caption = {8 images and 2 image per row}] %
    {[caption = { }]{./image.png}}%
    {[caption = { }]{./image.png}}%
    {[caption = { }]{./image.png}}%
    {[caption = { }]{./image.png}}%
    {[caption = { }]{./image.png}}%
    {[caption = { }]{./image.png}}%
    {[caption = { }]{./image.png}}%
    {[caption = { }]{./image.png}}
    \end{verbatim}

    \InsertMultiImages
    [perrow = 2, align = center,
      caption = {8 images and 2 image per row}] %
    {[caption = { }]{./example/how-to/write/image/pic/CC-BY.png}}%
    {[caption = { }]{./example/how-to/write/image/pic/CC-BY-NC.png}}%
    {[caption = { }]{./example/how-to/write/image/pic/CC-BY-ND.png}}%
    {[caption = { }]{./example/how-to/write/image/pic/CC-BY-SA.png}}%
    {[caption = { }]{./example/how-to/write/image/pic/CC-BY.png}}%
    {[caption = { }]{./example/how-to/write/image/pic/CC-BY-NC.png}}%
    {[caption = { }]{./example/how-to/write/image/pic/CC-BY-ND.png}}%
    {[caption = { }]{./example/how-to/write/image/pic/CC-BY-SA.png}}
  } % End of \item{}
  \end{enumerate}

% ------------------------------------------------
\EndChapter
% ------------------------------------------------
