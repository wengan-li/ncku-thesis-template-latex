% ------------------------------------------------
\StartSection{檔案結構 Files structure}{chapter:how-to:use:files-struct}
% ------------------------------------------------

\StartSubSection{介紹}
這邊會對本模板的檔案位置進行簡單說明.

\begin{framed}
\begin{verbatim}
o
|-README.md     (本模板的一些基本說明)
|-LICENSE       (本模板的版權和使用條款)
|-CONTRIBUTE    (本模板的貢獻人員名單)
|_thesis        (本模板的主要內容)
  |
  |-spine.tex   (產生書脊用) [重要, 不能刪除]
  |-thesis.tex  (產生論文用) [重要, 不能刪除]
  | 
  |-ncku        (定義/設計模板用) [重要, 不能刪除]
  | |
  | ...
  |
  |-example     (本模板的說明文件內容) [可用來參考]
  | |
  | ...         在'conf/conf.tex'中, 如果你選擇的是\DemoMode,
  |             則會使用'./example/context.tex'中
  |             的模板說明文件內容.
  |             而如果使用\DemoMode, 但這資料夾已刪的,
  |             在產生論文時會回傳錯誤.
  |
  |-context     (你的論文內容) [重要, 不能刪除]
  | |
  | ...         在'conf/conf.tex'中,
  |             如果你選擇的是\ChiMode或\EngMode,
  |             就會使用'./context/context.tex'中的內容.
  |             所以請在這資料夾中編寫你的論文.
  |
  |_conf        [重要, 不能刪除]
    |
    conf.tex    (設定論文的一些基本資料用: 如題目, 人名等)
                [重要, 不能刪除]
\end{verbatim}
\end{framed}

