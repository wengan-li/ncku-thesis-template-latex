% ------------------------------------------------
%
%論文內容次序:
% 1.考試合格證明
% 2.中英文摘要(論文以中文撰寫者須附英文延伸摘要)
% 3.誌謝
% 4.目錄
% 5.表目錄
% 6.圖目錄
% 7.符號
% 8.主文
% 9.參考文獻
% 10.附錄
%
% 註: 參考文獻書寫注意事項:
% (1).
%    文學院之中文文獻依分類及年代順序排列。
%    其他學院所之文獻依英文姓氏第一個字母
%    (或中文姓氏第一個字筆劃)及年代順序排列。
%
% (2).
%    期刊文獻之書寫依序為:
%        姓名、文章名稱、期刊名、卷別、期別、頁別、年代。
%
% (3).
%    書寫之文獻依序為:
%        姓名、書名、出版商名、出版地、頁別、年代。
%
% ------------------------------------------------

% 封面內頁 Inner Cover
%
% 封面: 顯示所有封面內容, 但沒有學校Logo)
%     主要用在印刷版, 如精裝版 或 平裝版
%     (使用cover.tex來產生)
%
% 內頁: 顯示所有封面內容, 但有學校Logo
%     主要用在電子版 + 印刷版
%
% 只要是印刷版, 不論是精裝版或平裝版, 都是 封面 (殼/皮) + 內頁.
% 只有在電子版時, 第一頁就是封面內頁.

\DisplayInnerCover

% ------------------------------------------------

% 學位考試論文證明書
\DisplayOral

% ------------------------------------------------

% 摘要 Abstract
% 除了外籍生, 本地生和僑生都是要編寫中文和英文摘要
% 論文以中文撰寫須以英文補寫 800 至 1200 字數的英文延伸摘要 (Extended Abstract)
% 詳細可看附件的學校要求或看example中的英文延伸摘要

% ------------------------------------------------
\StartAbstractChi
% ------------------------------------------------

這是國立成功大學碩博士用畢業論文的LaTex模版. 這模版是使用學校最新的畢業論文要求來設計(參考: 附錄 - 撰寫論文須知 P.\RefPage{appendix:thesis-spec}).

這模版的目標是為了提供學生可以使用LaTex來寫畢業論文. 但是各系所有各自的格式, 故請在使用前先留意自己的系所有沒有格式要求 (參考: 附錄 - 可使用的系所 P.\RefPage{appendix:acceptable-dept}). 如果沒有, 則這模版應該用來使用; 否則要看系所上的格式, 是否跟這模版有相同的寫法.

這模版的內容是我參考了我所拿到的一些畢業論文的LaTex模版設計, 跟系上老師的一些對話, 和上課所聽得出的結論和想法而寫出的, 所以某些地方會帶有我們濃郁的資工系味道. 另外如果有任何的老師 (不論本系外系)可以提供一些意見或想法的話, 我會十分感謝的.

這模版盡量以全自動化方式去處理一些不用你去煩惱的部份, 如排版和設計. 只留下要你去填寫的部份, 所以只要選擇和填入你的內容, 就能得到一份符合學校要求的畢業論文.

最後, 希望你使用愉快.

% ------------------------------------------------
% 如果在conf.tex中的\SetAbstractChiKeywords有設定任何的關鍵字,
% 那中文版的關鍵字會在使用\EndAbstractChi後同時顯示出來.
\EndAbstractChi
% ------------------------------------------------
             % 中文版
% ------------------------------------------------
\StartAbstract
% ------------------------------------------------

Write your abstract here.

% ------------------------------------------------
\EndAbstract
% ------------------------------------------------
             % 英文版
% ------------------------------------------------
\StartExtendedAbstract
% ------------------------------------------------

\ExtAbstractSummary{%
The summary is a short, informative abstract of no more than 250 words. References should not be cited. The summary should (1) state the scope and objectives of the research, (2) describe the methods used, (3) summarize the results, and (4) state the principal conclusions. Text of the summary should be 12 pt Times New Roman font, single-spaced and justified. A single line space should be left below the title `SUMMARY'. Leave a single line space above the key words listed below.
} % End of \ExtAbstractSummary{}

% ------------------------------------------------

\ExtAbstractChapter{INTRODUCTION}
The purpose of the introduction is to tell readers why they should want to read your thesis/ dissertation. This section should provide sufficient background information to allow readers to understand and evaluate the paper's results.

The introduction should (1) present the nature and scope of the problem, (2) review related literature, (3) describe the materials used and method(s) of the study, and (4) describe the main results of the study.

All text in the main body of the extended abstract should be 12 pt Times New Roman font, single-spaced and justified. Main headings are placed in the centre of the column, in capital letters using 12 pt Times New Roman Bold font. Subheadings are placed on the left margin of the column and are typed in 12 pt Times New Roman Bold font.

% ------------------------------------------------

\ExtAbstractChapter{MATERIALS AND METHODS}
There is flexibility as to the naming of the section (or sections) that provide information on the method(s) or theories employed. The methodology employed inthe work must be described in sufficient detail or with sufficient references so that the results could be duplicated.

Your materials should be organised carefully. Include all the data necessary to support your conclusions, but exclude redundant or unnecessary data.

% ------------------------------------------------

\ExtAbstractChapter{RESULTS AND DISCUSSION}
The results and discussion sections present your research findings and your analysis of those findings. The results of experiments can be presented as tables or figures.

% ------------------------------------------------

\ExtAbstractSection{Figures and Tables}
Figures may be integrated within the results section of the extended abstract, or they can be appended to the end of the written text. Figures should be black \& white. They should be no wider than the width of the A4 page.

Tables can be created within Word. As noted for figures above, if a table is to be placed within the text, it can be no wider than the width of the A4 page. Larger tables will need to be placed at the end of the abstract.

Figures and tables should be numbered according to the order they are referenced in the paper. Figures and tables should be referred to by their number in the text. When referring to figures and tables in the text, spell out and capitalize the word Figure or Table. All figures and tables must have captions.

% ------------------------------------------------

\ExtAbstractSection{Captions}
Captions should clearly explain the significance of the figure or table without reference to the text. Details in captions should not be restated in the text. Parameters in figure captions should be included and presented in words rather than symbols.

Captions should be placed directly above the relevant table and beneath the relevant figure. The caption should be typed in 12 pt Times New Roman Bold font. Spell out the word `Table' or `Figure' in full. An example table and a figure follow.

% ------------------------------------------------

\InsertTable
  [caption={Specifications of the engine}]
  {
    \begin{tabular}{llll}
    \hline
    Engine &  &  & OPEL Astra C16SE \\ \hline
    Displacement (cc) &  &  & 1598 \\
    Bore x stroke(mm x mm) &  &  & 79 x 81.5 \\
    Value mechanism &  &  & SOHC \\
    Number of valves &  &  & Intake 4, exhaust 4 \\
    Compression ratio &  &  & 9.8:1 \\
    Torque &  &  & 135/3400 Nm/rpm \\
    Power &  &  & 74/5800 kW/rpm \\
    Ignition sequence &  &  & 1-3-4-2 \\
    Spark plug &  &  & BPR6ES \\
    Fuel &  &  & 95 unleaded gasoline \\
    Cylinder arrangment &  &  & In-line 4 cylinders \\ \hline
    \end{tabular}
  } % End of  \InsertTable{}

\InsertFigure
  [scale=0.5,
    caption={HC emission as a function of equivalence ratio}]
  {./example/abstract/pic/extended-abstract-2.jpg}

% ------------------------------------------------

\ExtAbstractChapter{CONCLUSION}
This section should include (1) the main points of your paper and why they are significant, (2) any exceptions to, problems with, or limitations to your argument, (3) agreements or disagreements with previously published work, (4) theoretical and practical implications of the work, and (5) conclusions drawn.

% ------------------------------------------------
\EndExtendedAbstract
% ------------------------------------------------
        % 英文延伸摘要

% ------------------------------------------------

% 誌謝 Acknowledgments
% 誌謝正常應該只要寫一種版本就可,
% 提供2種以自行選擇所顯示的語言.
% 2種同時編寫都是可以的.

% ------------------------------------------------
\StartAbstractChi
% ------------------------------------------------

這是國立成功大學碩博士用畢業論文的LaTex模版. 這模版是使用學校最新的畢業論文要求來設計(參考: 附錄 - 撰寫論文須知 P.\RefPage{appendix:thesis-spec}).

這模版的目標是為了提供學生可以使用LaTex來寫畢業論文. 但是各系所有各自的格式, 故請在使用前先留意自己的系所有沒有格式要求 (參考: 附錄 - 可使用的系所 P.\RefPage{appendix:acceptable-dept}). 如果沒有, 則這模版應該用來使用; 否則要看系所上的格式, 是否跟這模版有相同的寫法.

這模版的內容是我參考了我所拿到的一些畢業論文的LaTex模版設計, 跟系上老師的一些對話, 和上課所聽得出的結論和想法而寫出的, 所以某些地方會帶有我們濃郁的資工系味道. 另外如果有任何的老師 (不論本系外系)可以提供一些意見或想法的話, 我會十分感謝的.

這模版盡量以全自動化方式去處理一些不用你去煩惱的部份, 如排版和設計. 只留下要你去填寫的部份, 所以只要選擇和填入你的內容, 就能得到一份符合學校要求的畢業論文.

最後, 希望你使用愉快.

% ------------------------------------------------
% 如果在conf.tex中的\SetAbstractChiKeywords有設定任何的關鍵字,
% 那中文版的關鍵字會在使用\EndAbstractChi後同時顯示出來.
\EndAbstractChi
% ------------------------------------------------
             % 中文版
% ------------------------------------------------
\StartAbstract
% ------------------------------------------------

Write your abstract here.

% ------------------------------------------------
\EndAbstract
% ------------------------------------------------
             % 英文版

% ------------------------------------------------

% 目錄 (內容, 圖表和圖片) Index of contents, tables and figures.
% 內容會自動產生 The indices will generate in automate.
\DisplayIndex         % 顯示索引
\DisplayTablesIndex   % 顯示表格索引
\DisplayFiguresIndex  % 顯示圖片索引

% ------------------------------------------------

% Nomenclature
% ------------------------------------------------
\StartSection{術語/符號 Nomenclature}{chapter:how-to:write:nomenclature}
% ------------------------------------------------

Nomenclature在定義一些在整份論文中所會用到的變數是很常用到的. 它的位置會出現在文章當中或是在Chapter 1之前. 它的設計沒有一個標準答案, 在不同的情況下可能有不同顯示方式, 但它基本上跟一張Table是沒差的. 而它在Latex中是使用一個package名為`nomencl'.

但經過研究了一下package `nomencl'或tabbing這些用來建Nomenclature的方式後, 發現`nomencl'在設計上反而會增加在產生論文時的步驟; 而tabbing要自行定義一個闊度才能弄得比較好看, 但同時內容卻出現沒法置中和設計上等一些問題. 故最後決定直接套用Table來讓同學更能自由的設計不同的Nomenclature table.

設計Nomenclature table需要2個知識或工具:\\
1) 設計一張Table, 這邊請參考P. \RefPage{chapter:how-to:write:table}.\\
2) 有關所需要用到的符號, 請參考Equation (P. \RefPage{chapter:how-to:write:equation})中所使用到的工具, Texmarker左邊的工具列, 或看這幾個網頁\RefBib{web:symbols:site1}\RefBib{web:symbols:site2}\RefBib{web:symbols:site3}, 應該已經足夠同學們寫出合適的符號.

% ------------------------------------------------
%\newpage
\StartSubSection{使用方式}

如果是指是在Chapter 1之前的一大張的Nomenclature table, 為Nomenclature Chapter.
  \begin{verbatim}
  \StartNomChapter{ NAME }{ LABEL }
  \EndNomChapter
  \end{verbatim}
Nomenclature Chapter跟一般Chapter的使用方式是一樣的, 但差別在於不會出現`Chapter'這字眼. 而由於大家的Nomenclature Chapter name可能不一樣, 故跟Chapter一樣可設定自行的name.

而如果是在文章當中的Nomenclature table. 基本上就是使用同一個的`\verb|\InsertTable|', 但還可以使用`nomtitle'來設定標題. `nomtitle'跟`caption'的差別是, 使用`nomtitle'所顯示出來的標題是沒有`Table XX:'為開頭, 同樣都是使用`pos'來控制題目的位置.

  \begin{DescriptionFrame}
  \begin{verbatim}
  Options 設定
    nomtitle:   Nomenclature 標題 (選填)
    ...

  E.g
    \InsertTable
    [nomtitle={這是Nomenclature Table的標題}]
      {
        ...
      }
  \end{verbatim}
  \end{DescriptionFrame}

有關這個的用法可參考`example/nomenclature/nomenclature.tex'中的Nomenclature Chapter所demo的例子, 那2個例子只是最簡單的Nomenclature table設計, 應該足夠同學們去弄出合適自己的Nomenclature table的設計.


% ------------------------------------------------

% Introduction section
% ------------------------------------------------
\StartChapter{Introduction}{chapter:introduction}
% ------------------------------------------------

這是國立成功大學碩博士用畢業論文的LaTex模版. 本模版是使用學校最新的畢業論文要求來設計(參考: 附錄 - 撰寫論文須知 P.\RefPage{appendix:thesis-spec}).

雖然本模版的目標是為了提供學生可以使用LaTex來寫畢業論文. 但是各系所有各自的格式, 所以做了一個表列出已知的系所情況(參考: 附錄 - 可使用的系所 P.\RefPage{appendix:acceptable-dept}), 故請在使用前先留意自己的系所有沒有格式要求. 如果沒有, 則本模版應該是可以用來使用; 否則要看系所上的格式, 是否跟本模版有相同的寫法.

本模版分以下幾個主要部份來進行教學:

\begin{enumerate}
  \item 本模版的架構設計
  \item 設定本模版的一些資料以轉成你的論文
  \item 介紹LaTex和本模版所提供的語法
  \item 最後有一個chapter為``老師們的話''(Chap.\RefTo{chapter:words-from-professor})寫了一些老師對論文的想法和意見, 以供同學們留意
\end{enumerate}

同學們只要閱讀完後, 把部份的檔案直接copy和修改內容, 應該很快就能上手本模版去寫自己的論文.

另外在附錄(Appendix)附上了一些重要的學校的文件, 由於本模版很接近完善, 故直接使用本模版後可不需再閱過學校相關規定之文件, 所以該類文件置於此僅為備考用.

% ------------------------------------------------
\newpage

\begin{description}
  \item[版權 License]\hfill\\
  詳細請看`LICENSE'這檔案中的條款說明.\\

  \InsertFigure
    [scale=0.8,
      caption={CC Attribution-NonCommercial-ShareAlike License},
      label={fig:appendix:by-nc}]
    {./example/introduction/pic/by-nc-sa.png}

    本著作(ncku-thesis-template-latex\RefBib{web:this-project:github})採用創用 CC 姓名標示-非商業性-相同方式分享 4.0 授權條款.

    This work(ncku-thesis-template-latex\RefBib{web:this-project:github}) is licensed under Creative Commons Attribution-NonCommercial-ShareAlike 4.0 International License.\\

  而本模版所使用到的國立成功大學浮水印則由國立成功大學擁有\textbf{所有}相關的權利. 故如使用這浮水印到論文以外的應用, 請跟`成大圖書館 系統管理組-數位論文小組'聯絡.\\

  \item[版本修改 ChangeLog]\hfill\\
  詳細請看`ChangeLog.md'這檔案中的說明.
\end{description}

% ------------------------------------------------
\EndChapter
% ------------------------------------------------


% Objective section
% ------------------------------------------------
\StartChapter{Objective}{chapter:objective}
% ------------------------------------------------

\section{起因}

做這個模板的原因其實很簡單:

\begin{enumerate}
  \item
  {
    去投國外paper時, 對方可能會要求使用LaTeX, 所以未來要懂LaTeX是不意外的.
  } % End of \item{}

  \item
  {
    想拿LaTeX來寫畢業論文, 卻發現學校只提供Mircosoft Word模板, 但卻沒有提供LaTeX的, 所以證明本模板對學校是有存在價值的.
  } % End of \item{}

  \item
  {
    因為看到發現台灣科技大學\cite{web:latex:template:ntust}, 台灣大學\cite{web:latex:template:ntu}, 元智大學\cite{web:latex:yzu}都能找到LaTeX的模板, 連大陸那邊都有一些學校有在提供.

    那些學校的畢業論文模板不只提供是Mircosoft Word版本(.doc), 是會連Latex(.tex)版本都有, 而我們學校卻沒有. 唯一我們學校在Google上找到的有提到的卻是數學系系網頁上的功能\cite{web:latex:ncku_math_introduction}和建在數學系上的一個討論區\cite{web:latex:ncku_math_forum}.
  } % End of \item{}

  \item
  {
    因為學校對Phd跟Master的畢業論文要求是同一個格式, 所以如果完成後對學校任何學生應該都有其好處.

    對大家都有多一個選擇來寫畢業論文, 而不是被限在使用Mircosoft Word來寫.
  } % End of \item{}

  \item
  {
    經過詢問我們資訊工程系(CSIE)的系上一些老師後, 意外發現原來某些實驗室其實已經有各自的版本存在(參考: Acknowledgment P.\pageref{chapter:acknowledgments-chi}), 但每個版本都有各自的優缺點.

    E.g :
    \begin{enumerate}

      \item
      {
        新的使用者或接手的人不容易修改或使用.
      } % End of \item{}

      \item
      {
        或是需要安裝的步驟十分麻煩 (e.g cwTeX\cite{web:latex:cwtex}).
      } % End of \item{}

      \item
      {
        另外有一些因為是只針對英文版本, 沒有考量在編寫或初稿時會有中英混雜的時候(同時因學校奇怪的要求, 例如英文內容的論文卻要寫中文論文名字等), 所以需要把整個論文分開成不同格式的檔案.
      } % End of \item{}

      \item
      {
        etc.
      } % End of \item{}
    \end{enumerate}
  } % End of \item{}
\end{enumerate}

% ------------------------------------------------

\section{目標}
所以為了解決以上的問題, 這個模板針對了好幾點來處理:

\begin{enumerate}

  \item
  {
    把本模板做到連笨蛋都可以很快懂得使用(所謂的Books for Dummies), 所以只留下使用者要填寫的部份外, 其他都交由模板去負責.
  } % End of \item{}

  \item
  {
    希望做到使用者只讀這份模板, 就會懂得去修改和寫自己所需的內容(所謂的Self-contained. 但其實是不太可能的, 因為Latex的使用手冊就算寫成一本幾百頁的書, 都可以缺少很多東西), 所以會同時提供很基本使用Latex的方式, 和填寫本模板步驟.
  } % End of \item{}

  \item
  {
    希望一份模板, 能同時應用在中文或是英文版本, 只要修改內容和一些的設定.
  } % End of \item{}

  \item
  {
    把本模板open source, 讓以後任何的同學們都可以使用和修改, 以合適當時的需求.
  } % End of \item{}

\end{enumerate}

而選擇使用XeLaTex的原因, 是我分析了cwTeX, CJK和XeLaTex.
cwTeX的寫法太糟, 要背多新一種語法, 而且安裝複雜\cite{web:latex:cwtex}; 而CJK有一定程度的設定才能在整個論文中自由使用, 感覺設定麻煩而不太能笨蛋化來用, 所以放棄選用; 故最後選用最簡單加一些包裝, 就可以簡單使用中英混合的XeLaTex.

% ------------------------------------------------

\section{缺點}
但是同樣任何東西都會有缺點, 故本模板都不意外:

\begin{enumerate}

  \item
  {
    本模板是以台灣國立成功大學所最新訂下的畢業論文要求(參考: 附錄 - 撰寫論文須知 P.\pageref{appendix:thesis-spec})來設計, 所以不一定能對非本校的人有用.
  } % End of \item{}

  \item
  {
    對沒有程式基礎, 只會用Mircosoft Word的人來講, 可能會在修改或使用上會十分吃力.
  } % End of \item{}

  \item
  {
    因為我針對某些使用者不用去接觸的部份, 進行了大量的包裝(Wrapper), 所以如果懂得Latex的人可能會覺得我破壞了Latex的語法. 但是本模板是針對笨蛋化和全自動, 我相信對不會的人來講, 才不管這問題 (如同一般理論派和應用派的差別).
  } % End of \item{}

  \item
  {
    某些包裝出來的語法, 可能會在一些情況下會產生衝突而令Latex不接受, 這時候有2種做法:
    \begin{enumerate}
      \item
      {
        不使用某些寫法, 例如已知的\begin{verbatim}\InsertImage和\InsertCenterImage\end{verbatim}沒法被包在Table, minipage或framebox中.
      } % End of \item{}

      \item
      {
        如真的要使用那些情況, 那只好自己真的不使用我的語法, 而直接去寫Latex原版的語法.
      } % End of \item{}
    \end{enumerate}
  } % End of \item{}
\end{enumerate}

% ------------------------------------------------

\section{總結}

以上是個人對這份模板的一些個人想法和起源.

如果以上的話都阻擋不了你想使用的話, 那歡迎翻到下一頁開始學習如何使用Latex或本模板, 同時都希望本模板能對你提供到幫助.

% ------------------------------------------------
\EndChapter
% ------------------------------------------------


% How-to-use section
% ------------------------------------------------
\StartChapter{Template使用教學}
% ------------------------------------------------

% ------------------------------------------------
\StartSection{論文基本資料設定 Tesis base configure}{chapter:how-to:use:conf}
% ------------------------------------------------

\StartSubSection{介紹}

'conf/conf.tex'是用來設定一些論文需要的資料: 如題目, 人名等. 故以下的章節會一個一個資料說明要怎麼填寫或修改:

\begin{enumerate}
  \item
  {
    \textbf{論文的編寫語言 / 用途}

    有3個選擇, 但只能用其中一個.

    \verb|\ChiMode|:  編寫的為中英混合版, 有提供基本的所需的檔案.\\
    \verb|\EngMode|:  編寫的為全英文版, 有提供基本的所需的檔案.\\
    \verb|\DemoMode|: 完整教學 或 樣板測試用.

    如果你選擇的是\verb|\DemoMode|, 則會使用'./example/context.tex'中的模板說明文件內容. 而如果使用\verb|\DemoMode|, 但這資料夾已刪的, 在產生論文時會回傳錯誤.

    如果你選擇的是\verb|\ChiMode|或\verb|\EngMode|, 就會使用'./context/context.tex'中的內容. 所以這時候請在這資料夾中編寫你的論文.
  } % End of \item{}

  \item
  {
    \textbf{封面名字顯示方式}

    如果你是使用\verb|\ChiMode|, 則可無視這個設定.

    而如果你選擇是\verb|\EngMode|, 預設在封面上只會顯示英文名字而已, 把\verb|\CDBothName|中的\verb|'%'|拿掉以設定同時顯示中英文名字

  } % End of \item{}

  \item
  {
    \textbf{Title 論文題目}

    要填寫你的中文和(或)英文論文題目.

    如果題目內有必須以數學模式表示的符號,請用\verb|\mbox{}|包住數學模式. 如:\\
    \verb|\SetTitle{題目題目}{New equation \mbox{$E = mc^4$} here}|\\

    而如果覺得自動產生出來的題目斷行位置不適合, 可以手動加'\verb|\\|'來強制斷行. 如:\\
    \verb|\SetTitle{題目題目}{Title Tooooooooooo \\ Longgggggggggggg}|\\

    有3種可使用, 可獨立使用, 但只有最後設定的一方有效\\
    \verb|\SetTitle{你的題目}{Your Title}|: 同時設定中英文題目\\
    \verb|\SetChiTitle{你的題目}|: 只設定中文題目\\
    \verb|\SetEngTitle{Your Title}|: 只設定英文題目\\

    如:\\
    \verb|\SetTitle %|\\
    \verb|{中文題目中文題目} %|\\
    \verb|{Your Title Your Title}|\\
    \verb|'%'|是必須的, 是用來跟Latex說這3行是同一句話.

    或\\
    \verb|\SetChiTitle{中文題目中文題目}|\\
    \verb|\SetEngTitle{Your Title \\ Your Title}|

    圖書館說不管你是編寫中英混合或全英文版, 都\textbf{必須}同時存在中英題目.
  } % End of \item{}

  \item
  {
    \textbf{Degree name 學位}

    設定這論文是碩士或是博士學位論文.\\
    有2種可選擇, 但只有最後設定的一方有效.\\
    \verb|\PhdDegree|: 博士學位\\
    \verb|\MasterDegree|: 碩士學位
  } % End of \item{}

  \item
  {
    \textbf{Your name 你的名字}

    填寫你的中文和(或)英文.\\
    有3種可使用, 可獨立使用, 但只有最後設定的一方有效.\\
    \verb|\SetMyName{你的名字}{Your name}|: 同時設定你的中英文名字\\
    \verb|\SetMyChiName{你的名字}|: 只設定你的中文名字\\
    \verb|\SetMyEngName{Your name}|: 只設定你的英文名字
  } % End of \item{}

  \item
  {
    \textbf{論文封面上的日期}

    設定西元的年月, 會自動計算出民國的年份, 和英文的月份轉換.\\
    次序為: \verb|\SetCoverDate{年份}{月份}|\\
    如: \verb|\SetCoverDate{2014}{12}|

    \textbf{注意}: 依本校研究生學位考試細則第十條規定:
      \begin{description}
        \item[碩士班]: \hfill
          論文日期:上學期為〇〇〇年1月;下學期為〇〇〇年6月,
          以該學期結束日期(一月或六月)為準。
          (如:在上學期101年9月~102年1月期間口試,
              不論是在此期間何月份口試,其日期均固定為102年1月).
          另碩士生如101上學期完成口試,101下學期申請出國,102上學期辦理離校,
          則論文封面為103年1月

        \item[博士班]: \hfill
        以當學期通過學位口試,則論文日期為口試日期(如〇〇〇年〇〇月〇〇日),
        若論文有修改致延至次學期,則以論文上傳日期為主。
      \end{description}
    故本模版會根據你的學位, 來選擇顯示在封面的日期格式.
  } % End of \item{}

  \item
  {
    \textbf{口試的日期}

    設定西元的年月日, 會自動計算出民國的年份, 和英文的月份轉換.\\
    次序為: \verb|\SetOralDate{年份}{月份}{日}|\\
    如: \verb|\SetOralDate{2014}{12}{31}|
  } % End of \item{}

  \item
  {
    \textbf{指導老師 Advisor(s)}

    在封面上預算了最多3位的空間, 中文名字固定以'教授'為結尾, 英文名字固定以'Prof.'為開頭.

    有3種可使用, 用來設定3位老師的名字\\
    \verb|\SetAdvisorNameX{老師的名字}{Professor's name}|: 同時設定中英文名字\\
    \verb|\SetAdvisorChiNameX{老師的名字}|: 只設定中文名字\\
    \verb|\SetAdvisorEngNameX{Professor's name}|: 只設定英文名字\\
    (NameX為NameA, NameB, NameC)

    使用\verb|\SetAdvisorNameA|是必須的, 而如果你的指導教授有2或3位, 那只要增加\verb|\SetAdvisorNameB|和\verb|\SetAdvisorNameC|則可.

    如: \verb|\SetAdvisorNameA{老師的中文名字}{老師的英文名字}|
  } % End of \item{}

  \item
  {
    \textbf{口試証明文件 Oral presentation document}

    口試証明文件是使用'範例'或是'自己的檔案', 只能選擇其中一方.

    如果要用的是範例:\\
    \verb|\DisplayOralTemplate|: 顯示 / 使用 口試範例版本.\\
    \verb|\SetCommitteeSize{8}|: 口試委員數量, 要配合\verb|\DisplayOralTemplate|來使用, 至少4位, 最多8位, 預設為8位.\\

    而如果要用的是自己的檔案:\\
    把你的圖片放在'context/oral'下, 之後設定中英文版所對應是哪一個檔案.\\
    例子用的'oral-chi.pdf'和'oral-eng.pdf'已放在'context/oral'中.
    \verb|\DisplayOralImage|: 設定要顯示圖片
    \verb|\SetOralImageChi{oral-chi.pdf}|: 設定中文口試檔名
    \verb|\SetOralImageEng{oral-eng.pdf}|: 設定英文口試檔名

    雖然沒有限定圖片的格式, 但是推薦使用PDF, 而且是沒法使用SVG.
  } % End of \item{}

  \item
  {
    \textbf{關鍵字 Keyword}

    可設定最多5個關鍵字.使用方式:\\
    \verb|\SetKeywords{Keyword A}{Keyword B}{Keyword C}{Keyword D}{Keyword E}|
  } % End of \item{}

  \item
  {
    \textbf{書脊 Spine}

    用來控制當使用spine.tex來產生書脊時內容控制. 預設書脊上會使用英文題目, 使用\verb|\SpineTitleChi|(把\verb|'%'|拿掉)以設定改使用中文題目.
  } % End of \item{}

  \item
  {
    \textbf{系所 Department or Institute}

    設定你的系所名字, 如:\\
    \verb|\SetDeptMath|: 數學系\\
    \verb|\SetDeptCSIE|: 資訊工程學系

    只要設定系所名字, 會自動進行適當的斷行和填入學院名稱等處理.\\

    這部份的資料是使用學校的教學單位資料中英文版(某些系所的中英的URL會不一樣或錯誤的)\RefBib{web:school:academics}.\\
    縮寫是靠學校給的Domain name所得出的, 故可能會有錯誤的時候.\\
    所以如果錯了的話, 就請告知真正的寫法(或縮寫)是什麼.\\

    設定系所名字則請參考下面的名單.

    \newpage
    \begin{table}[h]
      \caption{系所名字 Part 1}
      \begin{tabular}{|l|l|}
        \hline
        寫法 & 系所名字 \\ \hline

        \verb|\SetDeptChinese| &
        \begin{tabular}[c]{@{}l@{}}
          中國文學系\\
          Department of Chinese Literature
        \end{tabular} \\ \hline

        \verb|\SetDeptArt| &
        \begin{tabular}[c]{@{}l@{}}
          藝術研究所\\
          Institute of Art
        \end{tabular} \\ \hline

        \verb|\SetDeptMinNan| &
        \begin{tabular}[c]{@{}l@{}}
          閩南文化研究中心\\
          Min-Nan Culture Studies Center
        \end{tabular} \\ \hline

        \verb|\SetDeptFLLD| &
        \begin{tabular}[c]{@{}l@{}}
          外國語文學系\\
          Department of Foreign Languages and Literature
        \end{tabular} \\ \hline

        \verb|\SetDeptTWL| &
        \begin{tabular}[c]{@{}l@{}}
          臺灣文學系\\
          Department of Taiwanese Literature
        \end{tabular} \\ \hline

        \verb|\SetDeptKCLC| &
        \begin{tabular}[c]{@{}l@{}}
          華語中心\\
          Chinese Language Center
        \end{tabular} \\ \hline

        \verb|\SetDeptLang| &
        \begin{tabular}[c]{@{}l@{}}
          外語中心\\
          Foreign Language Center
        \end{tabular} \\ \hline

        \verb|\SetDeptHis| &
        \begin{tabular}[c]{@{}l@{}}
          歷史學系\\
          Department of History
        \end{tabular} \\ \hline

        \verb|\SetDeptMath| &
        \begin{tabular}[c]{@{}l@{}}
          數學系\\
          Department of Mathematics
        \end{tabular} \\ \hline

        \verb|\SetDeptDPS| &
        \begin{tabular}[c]{@{}l@{}}
          光電科學與工程學系\\
          Departmment of Photonics
        \end{tabular} \\ \hline

        \verb|\SetDeptPhys| &
        \begin{tabular}[c]{@{}l@{}}
          物理學系\\
          Department of Physics
        \end{tabular} \\ \hline

        \verb|\SetDeptCh| &
        \begin{tabular}[c]{@{}l@{}}
          化學系\\
          Department of Chemistry
        \end{tabular} \\ \hline

        \verb|\SetDeptEarth| &
        \begin{tabular}[c]{@{}l@{}}
          地球科學系\\
          Department of Earth Sciences
        \end{tabular} \\ \hline

        \verb|\SetDeptPSSC| &
        \begin{tabular}[c]{@{}l@{}}
          太空與電漿科學研究所\\
          Institute of Space and Plasma Sciences
        \end{tabular} \\ \hline

        \verb|\SetDeptNCTS| &
        \begin{tabular}[c]{@{}l@{}}
          國家理論科學研究中心\\
          National Center for Theoretical Sciences (South)
        \end{tabular} \\ \hline

        \verb|\SetDeptME| &
        \begin{tabular}[c]{@{}l@{}}
          機械工程學系\\
          Department of Mechanical Engineering
        \end{tabular} \\ \hline

        \verb|\SetDeptChe| &
        \begin{tabular}[c]{@{}l@{}}
          化學工程學系\\
          Department of Chemical Engineering
        \end{tabular} \\ \hline

        \verb|\SetDeptCivil| &
        \begin{tabular}[c]{@{}l@{}}
          土木工程學系\\
          Department of Civil Engineering
        \end{tabular} \\ \hline

        \verb|\SetDeptMSE| &
        \begin{tabular}[c]{@{}l@{}}
          材料科學及工程學系\\
          Department of Materials Science and Engineering
        \end{tabular} \\ \hline

      \end{tabular}
    \end{table}

    \newpage
    \begin{table}[h]
      \caption{系所名字 Part 2}
      \begin{tabular}{|l|l|}
        \hline
        寫法 & 系所名字 \\ \hline

        \verb|\SetDeptHyd| &
        \begin{tabular}[c]{@{}l@{}}
          水利及海洋工程學系\\
          Department of Hydraulic and Ocean Engineering
        \end{tabular} \\ \hline

        \verb|\SetDeptES| &
        \begin{tabular}[c]{@{}l@{}}
          工程科學系\\
          Department of Engineering Science
        \end{tabular} \\ \hline

        \verb|\SetDeptSNAME| &
        \begin{tabular}[c]{@{}l@{}}
          系統及船舶機電工程學系\\
          Department of System and Naval Mechatronic Engineering
        \end{tabular} \\ \hline

        \verb|\SetDeptIAA| &
        \begin{tabular}[c]{@{}l@{}}
          航空太空工程學系\\
          Department of Aeronautics and Astronautics
        \end{tabular} \\ \hline

        \verb|\SetDeptMP| &
        \begin{tabular}[c]{@{}l@{}}
          資源工程學系\\
          Department of Resources Engineering
        \end{tabular} \\ \hline

        \verb|\SetDeptEV| &
        \begin{tabular}[c]{@{}l@{}}
          環境工程學系\\
          Department of Environmental Engineering
        \end{tabular} \\ \hline

        \verb|\SetDeptBME| &
        \begin{tabular}[c]{@{}l@{}}
          生物醫學工程學系\\
          Department of BioMedical Engineering
        \end{tabular} \\ \hline

        \verb|\SetDeptGeomatics| &
        \begin{tabular}[c]{@{}l@{}}
          測量及空間資訊學系\\
          Department of Geomatics
        \end{tabular} \\ \hline

        \verb|\SetDeptIOTMA| &
        \begin{tabular}[c]{@{}l@{}}
          海洋科技與事務研究所\\
          Institute of Ocean Technology and Marine Affairs
        \end{tabular} \\ \hline

        \verb|\SetDeptICA| &
        \begin{tabular}[c]{@{}l@{}}
          民航研究所\\
          Institute of Civil Aviation
        \end{tabular} \\ \hline

        \verb|\SetDeptIBDPE| &
        \begin{tabular}[c]{@{}l@{}}
          能源國際學士學位學程\\
          International Bachelor Degree Program on Energy
        \end{tabular} \\ \hline

        \verb|\SetDeptICAMP| &
        \begin{tabular}[c]{@{}l@{}}
          尖端材料國際碩士學位學程\\
          International Curriculum for Advanced Materials Program
        \end{tabular} \\ \hline

        \verb|\SetDeptINHMM| &
        \begin{tabular}[c]{@{}l@{}}
          自然災害減災及管理國際碩士學位學程\\
          International Master Program on \\
          Natural Hazards Mitigation and Management
        \end{tabular} \\ \hline

        \verb|\SetDeptICEM| &
        \begin{tabular}[c]{@{}l@{}}
          工程管理碩士在職專班\\
          International Graduate Program of \\
          Civil Engineering and Management
        \end{tabular} \\ \hline

        \verb|\SetDeptEE| &
        \begin{tabular}[c]{@{}l@{}}
          電機工程學系\\
          Department of Electrical Engineering
        \end{tabular} \\ \hline

        \verb|\SetDeptCSIE| &
        \begin{tabular}[c]{@{}l@{}}
          資訊工程學系\\
          Insitute of Computer Science and Information Engineering
        \end{tabular} \\ \hline

        \verb|\SetDeptIME| &
        \begin{tabular}[c]{@{}l@{}}
          微電子工程研究所\\
          Institute of Microelectronics
        \end{tabular} \\ \hline

        \verb|\SetDeptCCE| &
        \begin{tabular}[c]{@{}l@{}}
          電腦與通信工程研究所\\
          Institute of Computer \& Communication Engineering
        \end{tabular} \\ \hline

        \verb|\SetDeptIMIS| &
        \begin{tabular}[c]{@{}l@{}}
          製造資訊與系統研究所\\
          Institute of Manufacturing Information and Systems
        \end{tabular} \\ \hline

      \end{tabular}
    \end{table}

    \newpage
    \begin{table}[h]
      \caption{系所名字 Part 3}
      \begin{tabular}{|l|l|}
        \hline
        寫法 & 系所名字 \\ \hline

        \verb|\SetDeptIMI| &
        \begin{tabular}[c]{@{}l@{}}
          醫學資訊研究所\\
          Institute of Medical Informatics
        \end{tabular} \\ \hline

        \verb|\SetDeptSTAT| &
        \begin{tabular}[c]{@{}l@{}}
          統計學系\\
          Department of Statistics
        \end{tabular} \\ \hline

        \verb|\SetDeptACC| &
        \begin{tabular}[c]{@{}l@{}}
          會計學系\\
          Department of Accountancy
        \end{tabular} \\ \hline

        \verb|\SetDeptTCM| &
        \begin{tabular}[c]{@{}l@{}}
          交通管理科學系\\
          Department of Transportation and \\
          Communication Management Science
        \end{tabular} \\ \hline

        \verb|\SetDeptBA| &
        \begin{tabular}[c]{@{}l@{}}
          企業管理學系暨國際企業研究所\\
          Department of Business Administration and\\
          Graduate Institute of International Business
        \end{tabular} \\ \hline

        \verb|\SetDeptTM| &
        \begin{tabular}[c]{@{}l@{}}
          電信管理研究所\\
          Institute of Telecommunications Management
        \end{tabular} \\ \hline

        \verb|\SetDeptIIM| &
        \begin{tabular}[c]{@{}l@{}}
          工業與資訊管理學系暨資訊管理研究所\\
          Institute of Information Management
        \end{tabular} \\ \hline

        \verb|\SetDeptFin| &
        \begin{tabular}[c]{@{}l@{}}
          財務金融研究所\\
          Institute of Finance \& Banking
        \end{tabular} \\ \hline

        \verb|\SetDeptPHEI| &
        \begin{tabular}[c]{@{}l@{}}
          體育健康與休閒研究所\\
          Institute of Physical Education, Health \& Leisure Studies
        \end{tabular} \\ \hline

        \verb|\SetDeptEMBA| &
        \begin{tabular}[c]{@{}l@{}}
          高階管理碩士在職專班\\
          Executive Master of Business Administration (EMBA)
        \end{tabular} \\ \hline

        \verb|\SetDeptIMBA| &
        \begin{tabular}[c]{@{}l@{}}
          國際經營管理研究所\\
          Institute of International Management (IMBA)
        \end{tabular} \\ \hline

        \verb|\SetDeptAMBA| &
        \begin{tabular}[c]{@{}l@{}}
          經營管理碩士班\\
          Advanced Master of Business Administration (AMBA)
        \end{tabular} \\ \hline

        \verb|\SetDeptPolSci| &
        \begin{tabular}[c]{@{}l@{}}
          政治學系\\
          Department of Political Science
        \end{tabular} \\ \hline

        \verb|\SetDeptEconomic| &
        \begin{tabular}[c]{@{}l@{}}
          經濟學系\\
          Department of Economics
        \end{tabular} \\ \hline

        \verb|\SetDeptPsychology| &
        \begin{tabular}[c]{@{}l@{}}
          心理學系\\
          Department of Psychology
        \end{tabular} \\ \hline

        \verb|\SetDeptLaw| &
        \begin{tabular}[c]{@{}l@{}}
          法律學系\\
          Department of Law and \\
          Institute of Law in Science and Technology
        \end{tabular} \\ \hline

        \verb|\SetDeptED| &
        \begin{tabular}[c]{@{}l@{}}
          教育研究所\\
          Institute of Education
        \end{tabular} \\ \hline

        \verb|\SetDeptIOCS| &
        \begin{tabular}[c]{@{}l@{}}
          認知科學研究所\\
          Institute of Cognitive Science
        \end{tabular} \\ \hline

        \verb|\SetDeptGIPE| &
        \begin{tabular}[c]{@{}l@{}}
          政治經濟學研究所\\
          Institute of Political Economy
        \end{tabular} \\ \hline

      \end{tabular}
    \end{table}

    \newpage
    \begin{table}[h]
      \caption{系所名字 Part 4}
      \begin{tabular}{|l|l|}
        \hline
        寫法 & 系所名字 \\ \hline

        \verb|\SetDeptFMRI| &
        \begin{tabular}[c]{@{}l@{}}
          心智影像研究中心\\
          Mind Research and Image Center
        \end{tabular} \\ \hline

        \verb|\SetDeptArch| &
        \begin{tabular}[c]{@{}l@{}}
          建築學系\\
          Department of Architecture
        \end{tabular} \\ \hline

        \verb|\SetDeptUP| &
        \begin{tabular}[c]{@{}l@{}}
          都市計劃學系\\
          Department of Urban Planning
        \end{tabular} \\ \hline

        \verb|\SetDeptID| &
        \begin{tabular}[c]{@{}l@{}}
          工業設計學系\\
          Department of Industrial Design
        \end{tabular} \\ \hline

        \verb|\SetDeptICID| &
        \begin{tabular}[c]{@{}l@{}}
          創意產業設計研究所\\
          Institute of Creative Industry Design
        \end{tabular} \\ \hline

        \verb|\SetDeptBio| &
        \begin{tabular}[c]{@{}l@{}}
          生命科學系\\
          Department of Life Sciences
        \end{tabular} \\ \hline

        \verb|\SetDeptBioTech| &
        \begin{tabular}[c]{@{}l@{}}
          生物科技研究所\\
          Institute of Biotechnology
        \end{tabular} \\ \hline

        \verb|\SetDeptIBBT| &
        \begin{tabular}[c]{@{}l@{}}
          生物資訊與訊息傳遞研究所\\
          Institute of Bioinformatics and Biosignal Transduction
        \end{tabular} \\ \hline

        \verb|\SetDeptITPS| &
        \begin{tabular}[c]{@{}l@{}}
          熱帶植物科學研究所\\
          Institute of Tropical Plant Sciences
        \end{tabular} \\ \hline

        \verb|\SetDeptEDUC| &
        \begin{tabular}[c]{@{}l@{}}
          醫學系\\
          School of Medicine
        \end{tabular} \\ \hline

        \verb|\SetDeptBiohem| &
        \begin{tabular}[c]{@{}l@{}}
          生物化學暨分子生物學研究所\\
          Department of Biochemistry and Molecular Biology
        \end{tabular} \\ \hline

        \verb|\SetDeptPath| &
        \begin{tabular}[c]{@{}l@{}}
          病理學科\\
          Department of Pathology
        \end{tabular} \\ \hline

        \verb|\SetDeptIntMed| &
        \begin{tabular}[c]{@{}l@{}}
          內科學科\\
          Department of Internal Medicine
        \end{tabular} \\ \hline

        \verb|\SetDeptPhysMed| &
        \begin{tabular}[c]{@{}l@{}}
          生理學研究所\\
          Department of Physiology
        \end{tabular} \\ \hline

        \verb|\SetDeptSurgery| &
        \begin{tabular}[c]{@{}l@{}}
          外科學科\\
          Department of Surgery
        \end{tabular} \\ \hline

        \verb|\SetDeptPed| &
        \begin{tabular}[c]{@{}l@{}}
          小兒學科\\
          Department of Pediatrics
        \end{tabular} \\ \hline

        \verb|\SetDeptAnatomy| &
        \begin{tabular}[c]{@{}l@{}}
          解剖學科暨細胞生物與解剖學研究所\\
          Department of Cell Biology and Anatomy
        \end{tabular} \\ \hline

        \verb|\SetDeptObsGyn| &
        \begin{tabular}[c]{@{}l@{}}
          婦產學科\\
          Department of Obstetrics and Gynecology
        \end{tabular} \\ \hline

        \verb|\SetDeptBone| &
        \begin{tabular}[c]{@{}l@{}}
          骨科學科\\
          Department of Orthopaedics
        \end{tabular} \\ \hline

        \verb|\SetDeptPhMed| &
        \begin{tabular}[c]{@{}l@{}}
          公共衛生學科暨公共衛生研究所\\
          Department of Public Health
        \end{tabular} \\ \hline

      \end{tabular}
    \end{table}

    \newpage
    \begin{table}[h]
      \caption{系所名字 Part 5}
      \begin{tabular}{|l|l|}
        \hline
        寫法 & 系所名字 \\ \hline

        \verb|\SetDeptNeuro| &
        \begin{tabular}[c]{@{}l@{}}
          神經學科\\
          Department of Neurology
        \end{tabular} \\ \hline

        \verb|\SetDeptPsy| &
        \begin{tabular}[c]{@{}l@{}}
          精神學科\\
          Department of Psychiatry
        \end{tabular} \\ \hline

        \verb|\SetDeptParasite| &
        \begin{tabular}[c]{@{}l@{}}
          寄生蟲學科\\
          Department of Parasitology
        \end{tabular} \\ \hline

        \verb|\SetDeptOphth| &
        \begin{tabular}[c]{@{}l@{}}
          眼科學科\\
          Department of Ophthalmology
        \end{tabular} \\ \hline

        \verb|\SetDeptOtolaryngo| &
        \begin{tabular}[c]{@{}l@{}}
          耳鼻喉學科\\
          Department of Otolaryngology
        \end{tabular} \\ \hline

        \verb|\SetDeptDEOH| &
        \begin{tabular}[c]{@{}l@{}}
          工業衛生學科暨環境醫學研究所\\
          Department of Environmental and Occupational Health
        \end{tabular} \\ \hline

        \verb|\SetDeptDerm| &
        \begin{tabular}[c]{@{}l@{}}
          皮膚學科\\
          Department of Dermatology
        \end{tabular} \\ \hline

        \verb|\SetDeptUro| &
        \begin{tabular}[c]{@{}l@{}}
          泌尿學科\\
          Department of Urology
        \end{tabular} \\ \hline

        \verb|\SetDeptPharmaco| &
        \begin{tabular}[c]{@{}l@{}}
          藥理學科暨藥理學研究所\\
          Department of Pharmacology
        \end{tabular} \\ \hline

        \verb|\SetDeptAnesth| &
        \begin{tabular}[c]{@{}l@{}}
          麻醉學科\\
          Department of Anesthesiology
        \end{tabular} \\ \hline

        \verb|\SetDeptRehab| &
        \begin{tabular}[c]{@{}l@{}}
          復健學科\\
          Department of Physical Medicine and Rehabilitation
        \end{tabular} \\ \hline

        \verb|\SetDeptMicrobio| &
        \begin{tabular}[c]{@{}l@{}}
          微生物學及免疫研究所\\
          Department of Microbiology and Immunology
        \end{tabular} \\ \hline

        \verb|\SetDeptRad| &
        \begin{tabular}[c]{@{}l@{}}
          放射線學科\\
          Department of Diagnostic Radiology
        \end{tabular} \\ \hline

        \verb|\SetDeptNM| &
        \begin{tabular}[c]{@{}l@{}}
          核子醫學科\\
          Department of Nuclear Medicine
        \end{tabular} \\ \hline

        \verb|\SetDeptFamily| &
        \begin{tabular}[c]{@{}l@{}}
          家庭醫學科\\
          Department of Family Medicine
        \end{tabular} \\ \hline

        \verb|\SetDeptEmergency| &
        \begin{tabular}[c]{@{}l@{}}
          急診學科\\
          Department of Emergency Medicine
        \end{tabular} \\ \hline

        \verb|\SetDeptDentistry| &
        \begin{tabular}[c]{@{}l@{}}
          牙科學科\\
          Department of Dentistry
        \end{tabular} \\ \hline

        \verb|\SetDeptOEM| &
        \begin{tabular}[c]{@{}l@{}}
          職業及環境醫學科\\
          Department of Occupational and Environmental Medicine
        \end{tabular} \\ \hline

        \verb|\SetDeptForensic| &
        \begin{tabular}[c]{@{}l@{}}
          法醫學科\\
          Department of Forensic Medicine
        \end{tabular} \\ \hline

        \verb|\SetDeptNursing| &
        \begin{tabular}[c]{@{}l@{}}
          護理學系\\
          Department of Nursing
        \end{tabular} \\ \hline

      \end{tabular}
    \end{table}

    \newpage
    \begin{table}[h]
      \caption{系所名字 Part 6}
      \begin{tabular}{|l|l|}
        \hline
        寫法 & 系所名字 \\ \hline

        \verb|\SetDeptMT| &
        \begin{tabular}[c]{@{}l@{}}
          醫學檢驗生物技術學系\\
          Department of Medical Laboratory Science and Biotechnology
        \end{tabular} \\ \hline

        \verb|\SetDeptPT| &
        \begin{tabular}[c]{@{}l@{}}
          物理治療學系\\
          Department of Physical Therapy
        \end{tabular} \\ \hline

        \verb|\SetDeptOT| &
        \begin{tabular}[c]{@{}l@{}}
          職能治療學系\\
          Department of Occupational Therapy
        \end{tabular} \\ \hline

        \verb|\SetDeptPharmacy| &
        \begin{tabular}[c]{@{}l@{}}
          藥學系\\
          School of Pharmacy
        \end{tabular} \\ \hline

        \verb|\SetDeptBasicMed| &
        \begin{tabular}[c]{@{}l@{}}
          基礎醫學研究所\\
          Institute of Basic Medical Sciences
        \end{tabular} \\ \hline

        \verb|\SetDeptBehMed| &
        \begin{tabular}[c]{@{}l@{}}
          行為醫學研究所\\
          Institute of Behavioral Medicine
        \end{tabular} \\ \hline

        \verb|\SetDeptCLPARM| &
        \begin{tabular}[c]{@{}l@{}}
          臨床藥學與藥物科技研究所\\
          Institute of Clinical Pharmacy and Pharmaceutical Sciences
        \end{tabular} \\ \hline

        \verb|\SetDeptIMM| &
        \begin{tabular}[c]{@{}l@{}}
          分子醫學研究所\\
          Institute of Molecular Medicine
        \end{tabular} \\ \hline

        \verb|\SetDeptIOM| &
        \begin{tabular}[c]{@{}l@{}}
          口腔醫學研究所\\
          Institute of Oral Medicine
        \end{tabular} \\ \hline

        \verb|\SetDeptICMMed| &
        \begin{tabular}[c]{@{}l@{}}
          臨床醫學研究所\\
          Institute of Clinical Medicine
        \end{tabular} \\ \hline

        \verb|\SetDeptAlliedHealth| &
        \begin{tabular}[c]{@{}l@{}}
          健康照護科學研究所\\
          Institute of Allied Health Sciences
        \end{tabular} \\ \hline

        \verb|\SetDeptIOG| &
        \begin{tabular}[c]{@{}l@{}}
          老年學研究所\\
          Institute of Gerontology
        \end{tabular} \\ \hline

      \end{tabular}
    \end{table}
  } % End of \item{}
\end{enumerate}


% ------------------------------------------------
\StartSection{產生論文 Generate Thesis}{chapter:how-to:use:generate}
% ------------------------------------------------

\StartSubSection{介紹}
這邊會簡單講解如何安裝基本的程式來產生你的論文.

\StartSubSection{MiKTeX安裝}
我們需要MiKTeX來幫我們來轉LaTex成PDF.

  \InsertFigure
    [scale=0.5,
      caption={MiKTeX Logo}]
    {./example/how-to/use/build/pic/miktex/logo.png}

首先去MiKTeX的網頁\RefBib{web:miktex:website}來下載它回來, 它預設在'Recommended Download'是32-bit的, 所以如果你要下載64-bit的話, 就要按'Other Downloads'中的第一個.

  \InsertFigure
    [scale=0.2,
      caption={Download MiKTeX}]
    {./example/how-to/use/build/pic/miktex/mdownload.png}

  \InsertFigure
    [scale=0.35,
      caption={安裝MiKTeX}]
    {./example/how-to/use/build/pic/miktex/minstall.png}

安裝MiKTeX其實沒有什麼需要太在意的東西, 但由獨有一個東西需要設定, 在不停按下一步時, 會出現fig \RefTo{fig:how-to:use:build:package-download}這個畫面, 在這邊推薦選擇'\textbf{Yes}', 因為這邊是用來設定自動幫你下載一些你需要使用的工具來產生論文.

  \InsertFigure
    [scale=0.35,
      caption={Download Package},
      label={fig:how-to:use:build:package-download}]
    {./example/how-to/use/build/pic/miktex/auto-download.png}

最後就要等待安裝, 由於內容滿多, 所以在這邊可能需要等待幾分鐘.

  \InsertFigure
    [scale=0.35,
      caption={等待安裝完成}]
    {./example/how-to/use/build/pic/miktex/installing.png}

% ------------------------------------------------
\newpage
\StartSubSection{Texmaker安裝}

我們需要Texmaker來幫我們處理產生流程和看PDF用.

  \InsertFigure
    [scale=0.5,
      caption={Texmaker Logo}]
    {./example/how-to/use/build/pic/texmaker/logo.png}

首先去Texmaker的網頁\RefBib{web:texmaker:website}來下載它回來, 推薦使用'Executable file for windows', 同時使用'Alternative download link' (因為這個line是使用Google Drive, 所以速度能有保證).

  \InsertFigure
    [scale=0.2,
      caption={Download Texmaker}]
    {./example/how-to/use/build/pic/texmaker/download.png}

安裝Texmaker其實沒有什麼要設定的東西, 不停按下一步就行了.

  \InsertFigure
    [scale=0.35,
      caption={安裝Texmaker}]
    {./example/how-to/use/build/pic/texmaker/install.png}

  \InsertFigure
    [scale=0.35,
      caption={安裝Texmaker}]
    {./example/how-to/use/build/pic/texmaker/install-2.png}

% ------------------------------------------------
\newpage
\StartSubSection{產生論文和封面}

當安裝完Texmaker和MiKTeX後, 直接點開'thesis.tex', 可以看到這個畫面(Fig\RefTo{fig:how-to:use:build:texmaker:thesis.tex}).

  \InsertFigure
    [scale=0.25,
      caption={Texmaker打開thesis.tex畫面},
      label={fig:how-to:use:build:texmaker:thesis.tex}]
    {./example/how-to/use/build/pic/texmaker/1.png}

產生論文的方式為在上方(Fig\RefTo{fig:how-to:use:build:texmaker:to_xelatex})由'快速編譯'改成'XeLaTeX', 之後按左方的箭頭就可以進行產生的處理(註: 如果是第一次使用, 那這時候背後MiKTeX會自動下載一些工具回來, 所以會等待比較久).

  \InsertFigure
    [scale=0.5,
      caption={改使用XeLaTeX},
      label={fig:how-to:use:build:texmaker:to_xelatex}]
    {./example/how-to/use/build/pic/texmaker/2.png}

\newpage
之後只要等待下方出現一些結果(Fig\RefTo{fig:how-to:use:build:texmaker:gen_message})就是說明已產生完成.

  \InsertFigure
    [scale=1.0,
      caption={處理的結果},
      label={fig:how-to:use:build:texmaker:gen_message}]
    {./example/how-to/use/build/pic/texmaker/2-5.png}

如果PDF產生成功, 那接旁邊'瀏覽PDF'的箭頭, 會出現一個視窗(Fig\RefTo{fig:how-to:use:build:texmaker:new_pdf})來顯示那個PDF檔.

  \InsertFigure
    [scale=0.25,
      caption={瀏覽PDF},
      label={fig:how-to:use:build:texmaker:new_pdf}]
    {./example/how-to/use/build/pic/texmaker/3.png}

\newpage
把以上的步驟用在'cover.tex'上(Fig\RefTo{fig:how-to:use:build:texmaker:cover.tex})就能去產生你的封面.

  \InsertFigure
    [scale=0.3,
      caption={Texmaker打開cover.tex畫面},
      label={fig:how-to:use:build:texmaker:cover.tex}]
    {./example/how-to/use/build/pic/texmaker/4.png}

當如果你已經把'thesis.tex'和'cover.tex'都產生了PDF, 你的資料夾應該會有這些檔案和資料夾(Fig\RefTo{fig:how-to:use:build:texmaker:dir}), 那2個PDF正是你需要的東西.

  \InsertFigure
    [scale=0.5,
      caption={資料夾內容},
      label={fig:how-to:use:build:texmaker:dir}]
    {./example/how-to/use/build/pic/texmaker/5.png}

% ------------------------------------------------

\newpage
\StartSubSection{產生PDF的流程}

在編譯LaTex時成PDF時, 必須注意內部的引用(\verb|\RefBib{}|)情況.

如果只是編寫內容, 引用的內容和號碼不是重要的話, 那直接使用:\\
XeLaTeX -> 瀏覽PDF\\
即可.

但如果你的PDF是最終版本, 那你的流程則需要使用:\\
XeLaTeX -> BibTex -> XeLaTeX -> 瀏覽PDF\\
才對.

因為第一次的XeLaTeX是用來產生'thesis.aux', 而有這個檔案才能對你的內容中Reference的引用來進行連接, 而BibTex正是做這個的處理以產出'thesis.bbl', 而第二次的XeLaTeX會使用'thesis.bbl'來把你的Reference中的號碼在內容中設定.

以上步驟只需用在'thesis.tex'的部份, 而'cover.tex'則不用做這個行為.

% ------------------------------------------------


% ------------------------------------------------
\EndChapter
% ------------------------------------------------


% How-to-write section
% ------------------------------------------------
\StartChapter{LaTex編寫教學}
% ------------------------------------------------

% ------------------------------------------------
\StartSection{基本介紹 Introduction}{chapter:how-to:write:intro}
% ------------------------------------------------

這教學包含了原LaTex和本模版特有的語法的使用方式和例子. (真正完完整整的LaTex教學手冊可不只單單幾百頁的厚度, 所以減少大家的時間, 所以本模版教學只講一些幾乎大家100\%會需要使用的語法).

請注意原LaTex語法會以英文小寫來顯示(\verb|\aabbcc|); 而本模版特有的語法會以英文大小寫混合(\verb|\AaBbCc|, 第一個字必定以大寫來顯示), 由於這些特有語法\textbf{不是}原LaTex的語法, 所以不能直接應用在非本模版的LaTex檔案上.

抄襲就是學習的第一步 (如同我們小時候去抄襲父母走路一樣), 所以本模版有留下了一些範本 (在`./context'下)以方便大家開始第一步, 之後就要靠大家自己的努力和實作, 再加上自己的探索能力了.

%\newpage
有問題的話, 可以有以下的地方找尋答案 (請使用這順序):
\begin{enumerate}
  \item 請一步一步增加內容, 如發生錯誤, 就把剛剛新增的內容拿掉, 以找出錯誤的地方
  \item 直接研究在模版的LaTex寫法 (在 './example' 以下的所有檔案)
  \item 查問懂得LaTex的老師和同學
  \item 去LaTex的Wikibook \RefBib{web:latex:wikibooks}\\
        這邊有大量的例子, 但是這些例子都是獨立的, 所以潛在語法混合後的會發生沖突的可能性; 另外都十分推薦去讀 '大家來學LaTeX' \RefBib{web:latex:latex123}
  \item 請求Google老師
\end{enumerate}

另外, 如果覺得本教學還缺少了什麼說明, 請告知.

% ------------------------------------------------

% Section
\newpage% ------------------------------------------------
\StartChapter{Latex教學 - 基本語法 Basic}{chapter:how-to:write:basic}

% http://www.hitripod.com/blog/2012/05/latex-thesis-template-quick-reference/

% ------------------------------------------------
\section{介紹}

這邊會講解一些最基本的功能.

% ------------------------------------------------
\section{字體變化}

\begin{itemize}
  \item
  {
    正常

    這是文字 This is text
  } % End of \item{}

  \item
  {
    粗體

    寫法: \verb|\textbf{這是文字 This is text}|

    效果: \textbf{這是文字 This is text}
  } % End of \item{}

  \item
  {
    斜体

    寫法: \verb|\textit{這是文字 This is text}|

    效果: \textit{這是文字 This is text}\\
    (中文的斜体並不太明顯)
  } % End of \item{}
\end{itemize}

% ------------------------------------------------
\newpage
\section{清單 List Structures}

  日常的清單主要有3種:

\begin{itemize}
  \item
  {
    數字

    可以有2種寫法, 使用\verb|\item xxxx|來只寫一行, 或是用\verb|{...}|可把內容包起來.
    \begin{framed}
    \begin{verbatim}
      \begin{enumerate}
        \item Item1

        \item Item2

        \item
        {
          Item3

          Item3's context
        }

        \item
        {
          Item4

          Item4's context
        }
      \end{enumerate}
    \end{verbatim}
    \end{framed}

    效果:
    \begin{enumerate}
      \item Item1

      \item Item2

      \item
      {
        Item3

        Item3's context
      }

      \item
      {
        Item4

        Item4's context
      }
    \end{enumerate}
  } % End of \item{}

  \newpage
  \item
  {
    符號

    \begin{framed}
    \begin{verbatim}
      \begin{itemize}
        \item Item1

        \item Item2

        \item
        {
          Item3

          Item3's context
        }

        \item
        {
          Item4

          Item4's context
        }
      \end{itemize}
    \end{verbatim}
    \end{framed}

    效果:
    \begin{itemize}
      \item Item1

      \item Item2

      \item
      {
        Item3

        Item3's context
      }

      \item
      {
        Item4

        Item4's context
      }
    \end{itemize}
  } % End of \item{}

  \newpage
  \item
  {
    文字

    可以有2種寫法, 使用\verb|\item[xxxx] xxxx|來只寫一行, \\
    或是用\verb|\hfill \\|把內容放到第2行才開始.

    \begin{framed}
    \begin{verbatim}
      \begin{description}
        \item[Item1] Item1's context
        \item[Item2] Item2's context
        \item[Item3] \hfill \\
          Item3's context
      \end{description}
    \end{verbatim}
    \end{framed}

    效果:
    \begin{description}
      \item[Item1] Item1's context
      \item[Item2] Item2's context
      \item[Item3] \hfill \\
        Item3's context
    \end{description}
  } % End of \item{}

  \newpage
  \item
  {
    巢狀表單

    表單應該最多只會用到第4層, 但是其實當你需要用到第3層時, 這時候你應該考慮的不是怎使用表單, 而是要怎換另外一種寫法了.

    \begin{framed}
    \begin{verbatim}
      \begin{enumerate}
        \item
        {
          Level-1 Item 1
          \begin{enumerate}
          \item Nested Item 1

          \item
          {
            Level-2 Item 2

            \begin{enumerate}
            \item
            {
              Level-3 Item 1

              \begin{enumerate}
              \item Level-4 Item 1
              \item Level-4 Item 2
              \end{enumerate}
            }

            \item Level-3 Item 2
            \end{enumerate}
          }
          \end{enumerate}
        }
      \end{enumerate}

      \begin{itemize}
        \item
        {
          Level-1 Item 1

          \begin{itemize}
          \item
          {
            Level-2 Item 2

            \begin{itemize}
            \item Level-3 Item 1
            \item Level-3 Item 2
            \end{itemize}
          }

          \item Level-2 Item 2
          \end{itemize}
        }
      \end{itemize}
    \end{verbatim}
    \end{framed}

    效果:
    \begin{enumerate}
      \item
      {
        Level-1 Item 1
        \begin{enumerate}
        \item Nested Item 1

        \item
        {
          Level-2 Item 2

          \begin{enumerate}
          \item
          {
            Level-3 Item 1

            \begin{enumerate}
            \item Level-4 Item 1
            \item Level-4 Item 2
            \end{enumerate}
          }

          \item Level-3 Item 2
          \end{enumerate}
        }
        \end{enumerate}
      }
    \end{enumerate}

    \begin{itemize}
      \item
      {
        Level-1 Item 1

        \begin{itemize}
        \item
        {
          Level-2 Item 2

          \begin{itemize}
          \item Level-3 Item 1
          \item Level-3 Item 2
          \end{itemize}
        }

        \item Level-2 Item 2
        \end{itemize}
      }
    \end{itemize}

  } % End of \item{}
\end{itemize}

% ------------------------------------------------
\newpage
\section{標記 Label}

  跟原本的Latex語法不一樣, 本模板提供幾個語法來引用Label (以重寫名字來直接理解要引用什麼).

  但是如果你是懂得原Latex的寫法(\verb|\ref{}, \cite{}, etc.|), 都可以直接使用原本的寫法, 其實是同一個東西.

    \begin{framed}
    \begin{verbatim}
      引用 公式(Equation)
      \RefEquation{...}
      \RefEquationB{...}

      引用 參考資料(References)
      \RefBib{...}

      引用 頁碼
      \RefPage{...}

      引用 其他任何的東西: 如圖片, 表格,
            chapter, section, subsection, etc.
      \RefTo{...}
    \end{verbatim}
    \end{framed}

    e.g:

% ------------------------------------------------
\EndChapter
% ------------------------------------------------

\newpage% ------------------------------------------------
\StartSection{章節 Chapter/Section}{chapter:how-to:write:chapter-section}
% ------------------------------------------------

\StartSubSection{介紹}

編寫任何的文章, 都會使用不同的章節來把內容進行分區. 例如學校的排版樣子大約:

  \EmptyLine
\begin{fmpage}{\textwidth}
  \centerline{\LARGE Chapter X}
  \vspace{0.2cm}
  \centerline{\LARGE 這是標題}

  \vspace{0.5cm}
  \mbox{\Large X.1 子項目}\\
  \mbox{\hspace{1.2cm}項目內容 ...}

  \vspace{0.3cm}
  \mbox{\large X.1.1 子項目}\\
  \mbox{\hspace{1.2cm}項目內容 ...}
\end{fmpage}
  \EmptyLine

所以針對這些功能, 本模版提供:

  \EmptyLine
\begin{fmpage}{\textwidth}
  \begin{verbatim}
    主要章節
    Title: 標題 (必填)
    Label: 標簽 (選填)
    \StartChapter{ Title }{ Label }
    \EndChapter % 用來保證你的內容在這Chapter內

    次章節
    Title: 標題 (必填)
    Label: 標簽 (選填)
    \StartSection{ Title }{ Label }

    次章節的子章節
    Title: 標題 (必填)
    Label: 標簽 (選填)
    \StartSubSection{ Title }{ Label }
  \end{verbatim}
\end{fmpage}
  \EmptyLine

所以針對剛剛的例子, 它的Latex寫法為:

  \EmptyLine
\begin{fmpage}{\textwidth}
  \begin{verbatim}
    \StartChapter{這是標題}

    \StartSection{子項目}
    項目內容 ...

    \StartSubSection{X.1的子項目}
    項目內容 ...

    \EndChapter
  \end{verbatim}
\end{fmpage}


\newpage% ------------------------------------------------
\StartSection{圖片 Figure}{chapter:how-to:write:figure}
% ------------------------------------------------

\StartSubSection{圖片小知識}

圖片為兩類別: 點陣圖或向量圖 (參考 Fig \RefTo{fig:how-to:write:figure:format}) .\\

  \InsertFigure
    [scale=0.3,
      caption={圖片類別},
      label={fig:how-to:write:figure:format}]
    {./example/how-to/write/figure/pic/graph-format.png}

點陣圖(如.jpg, .gif,.png, .tiff)在相機和網頁中十分常見, 優點是幾乎能應用在不同地方/工具, 缺點在放大縮小時會出現失真的情況, 所以為了更清晰, 則要更高的解析度的圖片, 這時候就會大大增加圖片大小, 而這大小會直接影響所產生出來論文PDF的大小.

向量圖(如.pdf, .eps, .svg)在學術界內用來放在論文中是非常常見. 優點是不會因放大縮小而造成內容變型, 所以是十分有用的. 缺點是必須使用一些特定的工具才能顯示或產生出來. 而LaTex主要使用這一類圖檔. 但LaTex對SVG的支援十分不好, 故這模版沒無提供插入SVG檔, 故強烈推薦使用PDF為主要的格式.

PDF可同時為點陣圖或向量圖, 主要是看你提供什麼圖片格式來轉成PDF. 另一種Windows增加型中繼檔(.emf)都同樣是點陣圖或向量圖, 而由於Word是沒法直接插入PDF/EPS/SVG的圖檔, 所以需先轉成這格式, 以保留向量圖的品質.

% ------------------------------------------------
\newpage
\StartSubSection{轉換格式}

以下為一些已知的方式可把圖像轉成向量圖格式, 按鈕的位置有可能不一樣, 但應該都會在那些地方中. (以下使用工具版本為: Adobe Acrobat XI Pro, Adobe Illustor CS5, Adobe Photoshop CS5, Microsoft Visio Professional 2013, Microsoft Office Professional 2013 [Excel/PowerPoint])

\begin{description}
  \item[SVG - > PDF] \hfill\\
    如果你有安裝學校的Adobe Acrobat Pro (沒有的話都推薦你安裝. 因為交給圖書館的電子檔時, 你起碼要對PDF檔上鎖), 直接對那個SVG檔右鍵, 就會有`轉換成Adobe PDF' 這個選項.

  \item[Adobe Illustor - > EMF] \hfill\\
    `檔案' -> `轉存' -> 格式拉到最下面就有了.

  \item[Adobe Illustor - > PDF] \hfill\\
    `檔案' -> `另存新檔' -> 在格式中間位置 -> 如果你的是成品的話, 則可考慮把 `保留Illustrator編輯能力', `內嵌頁面縮圖'都拿掉以減少PDF檔的大小.

  \item[Adobe Illustor - > SVG] \hfill\\
    `檔案' -> `另存新檔' -> 在格式最底的位置 -> 如果你想不到有什麼設定, 直接按`確定'就好了. 同時都推薦在`影像'的選項設定為`嵌入'以去掉任何影像有位置不對而造成SVG檔有什麼問題.

  \item[Adobe Photoshop - > PDF] \hfill\\
    `檔案' -> `另存新檔' -> 在格式中下的位置 -> 如果你的是成品的話, 則可考慮把 `保留Photoshop編輯能力'拿掉以減少PDF檔的大小.

  \item[Visio - > PDF] \hfill\\
    `檔案' -> `匯出' -> `建立PDF/XPS'.

  \item[Visio - > SVG] \hfill\\
    `檔案' -> `匯出' -> `變更檔案類型' -> `SVG可縮放向量圖形' -> 下面的`另存新檔'.

  \item[Visio - > EMF] \hfill\\
    `檔案' -> `匯出' -> `變更檔案類型' -> `EMF 增加型中繼檔' -> 下面的`另存新檔'.

  \newpage
  \item[PowerPoint - > EMF] \hfill
    \begin{enumerate}
      \item `檔案' -> `匯出' -> `變更檔案類型' -> `儲存成其他檔案類型'
      \item 下面的`另存新檔' -> 在格式最底部位置選擇`Windows增加型中繼檔' -> `僅此投影片'即可.
      \item 當然都可選擇`所有投影片', 只是`僅此投影片'即可儲存跟PowerPoint同檔名的EMF檔. 而`所有投影片'會把多張的EMF檔放在一個資料夾中.
    \end{enumerate}

  \item[Excel - > PDF] Excel可以把所做出來的圖表轉成PDF內容.
    \begin{enumerate}
      \item 在Excel中對某個圖表按一下左鍵, 之後 `檔案' -> `匯出' -> `建立PDF/XPS'.
      \item 所做出來的PDF應該是一張A4, 所以要做裁切. 打開那個PDF, 右上方有一個`工具' -> 右邊多了一個工具列.
      \item 按`裁切', 之後在圖中的任何地方按2下左鍵, 把`移除白色邊距'打勾 (如右圖中沒任何反應, 重新打勾一下看看), 之後按`確定'即可.
      \item 按`Ctrl + S'來儲存即可.
    \end{enumerate}

  \item[Excel - > EMF] Excel沒有任何直接方式把圖表轉成EMF, 但有方法來間接轉換.
    \begin{enumerate}
      \item 先做一次\textbf{Excel - > PDF}的方法.
      \item `檔案' -> `另存新檔' -> 在格式中間位置選擇`PowerPoint簡報(*.pptx)'.
      \item 之後做一次\textbf{PowerPoint - > EMF}的方法即可.
    \end{enumerate}
\end{description}

\noindent \textbf{注意:} 一些系所可能會使用自己的程式或一些工具以製造出SVG檔, 但以我經驗發現有時候有些工具所產出的SVG檔並不能在日常的工具或瀏覽器正常顯示 (應該是SVG中的XML有某程度的內容跟公認的內容不太一樣所造成的). 所以我會推薦先檢查這SVG檔是否在手上的工具或程式都顯示正常, 之後再把這SVG轉成PDF, 以方便匯入到論文之中, 同時能保證這SVG中的內容沒有出現任何變型或錯誤.

% ------------------------------------------------
\newpage
\StartSubSection{使用介紹}

插入圖片其實有很多的玩法, 但是在畢業論文中, 它的放置位置則是非常固定的, 都是以中間為主, 並插入單/多張圖片. 因為圖片位置都是固定的, 所以本模版針對了插入單張或多張來設計, 之後的章節會對這2個方式的使用作詳細說明.

要注意的是, 圖片在畫面看到的大小, 跟真正寫到文件是不一樣的 (因為經過程式的自動縮放), 所以比例正常都要修改的.

留意的是, 圖片的路徑跟正常日常使用的路徑會不一樣, 是使用所謂的"相對路徑" (Relative path), 而起點是論文的主檔案(thesis.tex).

\noindent 例如 (以Windows的路徑為例子):\\
主檔案thesis.tex在: "\verb|C:\thesis\thesis.tex|"\\
圖片A: "\verb|C:\thesis\some_dir\A.png|"\\
圖片B: "\verb|C:\thesis\some_dir\B.png|"\\
使用時以"\verb|./some_dir/A.png|", "\verb|./some_dir/B.png|"的方式來使用, 注意是"$/$"而不是"$\backslash$".

還有檔名的文字中間不要在非格式的位置(如最後的.png, .pdf等)出現`.'這個符號 (如AB.CD-EF.png), 否則有可能會出現潛在的錯誤.

% ------------------------------------------------
\newpage
\StartSubSection{單張}

  \begin{DescriptionFrame}
  \begin{verbatim}
  Path:   圖片位置 (必填)

  Options 設定 (使用','來分隔, 不分先後順序)
    scale:   比例 (選填, 預設: 1.0)
    (1.0: 原大小; 0.x ~ < 1.0: 縮小; > 1.0: 放大)
    (設計上你是可以無限放大, 但還是推薦你使用大圖, 之後縮小)
    caption: 標題 (選填)
    label:   標簽 (選填, 必須要配合caption使用, 否則無效)
    angle:   角度 (選填, 預設: 0度)
    opacity: 背景顏色透明度, 預設使用白色為背景 (選填, 預設: 0.75)
    (0.x ~ < 1.0: 透明; => 1.0: 不透明)

  插入圖片
  \InsertFigure[Options]{Path}

  E.g
    \InsertFigure
    [caption={這 是 標 題}]
      {./figure.png}

    \InsertFigure
      [scale=0.5,
        angle=45,
        caption={這 是 標 題},
        label={this:is:label}]
      {./figure.png}

    每一項資料可以使用斷行來分隔以保持可讀性.
    caption和label必須要使用'{}'才能有空格的句子.

    補充:
        LaTex對SVG檔的支援並不理想, 故推薦先對SVG進行加工,
        如轉成.eps或.pdf (推薦).
  \end{verbatim}
  \end{DescriptionFrame}

  \newpage
  {\bf 效果:}
  \begin{enumerate}
  \item
  {
    只填了圖片位置
    \begin{verbatim}
    \InsertFigure
      {./figure.png}
    \end{verbatim}
    \InsertFigure
      {./example/how-to/write/figure/pic/Cc-by_new.png}
  } % End of \item{}

  \item
  {
    放大比例
    \begin{verbatim}
    \InsertFigure
      [scale=1.5]
      {./figure.png}
    \end{verbatim}
    \InsertFigure
      [scale=1.5]
      {./example/how-to/write/figure/pic/Cc-by_new.png}
  } % End of \item{}

  \item
  {
    縮小比例
    \begin{verbatim}
    \InsertFigure
      [scale=0.5]
      {./figure.png}
    \end{verbatim}
    \InsertFigure
      [scale=0.5]
      {./example/how-to/write/figure/pic/Cc-by_new.png}
  } % End of \item{}

  \newpage
  \item
  {
    增加標題並去掉比例的數字
    \begin{verbatim}
    \InsertFigure
      [caption={Little man}]
      {./figure.png}
    \end{verbatim}
    \InsertFigure
      [caption={Little man}]
      {./example/how-to/write/figure/pic/Cc-by_new.png}
  } % End of \item{}

  \item
  {
    增加標簽
    \begin{verbatim}
    \InsertFigure
      [caption={Little man No.1},
        label={fig:little-man-no.1}]
      {./figure.png}
    \end{verbatim}

    之後可以使用\verb| \RefTo |去引用 \verb| \RefTo{fig:little-man-no.1} |
    \InsertFigure
      [caption={Little man No.1},
        label={fig:little-man-no.1}]
      {./example/how-to/write/figure/pic/Cc-by_new.png}

    e.g: 文中所指的人物一號 (Fig \RefTo{fig:little-man-no.1}).
  } % End of \item{}

  \newpage
  \item
  {
    使用角度去轉45度
    \begin{verbatim}
    \InsertFigure
      [angle=45,
        caption={Little man No.2},
        label={fig:little-man-no.2}]
      {./figure.png}
    \end{verbatim}

    使用\verb| \RefTo |去引用 \verb| \RefTo{fig:little-man-no.2} |
    \InsertFigure
      [angle=45,
        caption={Little man No.2},
        label={fig:little-man-no.2}]
      {./example/how-to/write/figure/pic/Cc-by_new.png}

    e.g: 文中所指的人物二號 (Fig \RefTo{fig:little-man-no.2}).
  } % End of \item{}

  \newpage
  \item
  {
    使用透明度以能看到頁面中的學校浮水印, 不過除非你是使用很小的圖, 否則還是會被你的圖蓋到而會感覺不出來的.\\

    \vspace{2.0cm}

    \begin{verbatim}
    \InsertFigure
      [caption={opacity使用預設}]
      {./figure.png}
    \end{verbatim}

    \InsertFigure
      [scale=0.5,
        caption={opacity使用預設}]
      {./example/abstract/pic/extended-abstract-2.jpg}

    \newpage
    \EmptyLine
    \vspace{2.5cm}

    \begin{verbatim}
    \InsertFigure
      [opacity=0.4,
      caption={opacity使用0.4}]
      {./figure.png}
    \end{verbatim}

    \InsertFigure
      [scale=0.5,
        caption={opacity使用0.4},
        opacity=0.4]
      {./example/abstract/pic/extended-abstract-2.jpg}

    \newpage
    \EmptyLine
    \vspace{7.0cm}

    \InsertFigures
    [caption={opacity使用預設}] %
    {
      {./example/how-to/write/figure/pic/CC-BY-NC.png}
    }%
    {
      {./example/how-to/write/figure/pic/CC-BY-NC-ND.png}
    }

    \vspace{1.0cm}

    \InsertFigures
    [caption={opacity使用0.4},
    opacity=0.4]
    {
      {./example/how-to/write/figure/pic/CC-BY-NC.png}
    }%
    {
      {./example/how-to/write/figure/pic/CC-BY-NC-ND.png}
    }

  } % End of \item{}

  \newpage
  \item
  {
    把圖放在表格中, 這時候是使用\verb|\InsertFigure{}|, 是不能使用caption和label (正常應該用不到這種+寫法, 例如出現`Fig X.X'這種字在table中). (有關表格table的使用, 請參考Chap \RefTo{chapter:how-to:write:table}). 如真的想使用, 則考慮這邊的寫法.

    \begin{verbatim}
    \begin{table}[H]
    \centering
    \begin{tabular}{|c|c|}
      \hline
      \textbf{\underline{Website}} &
        \textbf{\underline{URL}} \\ \hline

      \begin{tabular}[c]{@{}c@{}}
      \includegraphics[scale=0.1]
        {./apple.png} \\ Apple
      \end{tabular} & \url{www.apple.com}  \\ \hline

      \begin{tabular}[c]{@{}c@{}}
      \includegraphics[scale=0.1]
        {./google.png} \\ Google
      \end{tabular} & \url{www.google.com} \\ \hline
    \end{tabular}
    \end{table}
    \end{verbatim}

    \begin{table}[H]
    \centering
    \label{table:how-to:write:figure:insert-figure-into-table}
    \begin{tabular}{|c|c|}
      \hline
      \textbf{\underline{Website}} &
        \textbf{\underline{URL}} \\ \hline

      \begin{tabular}[c]{@{}c@{}}
      \includegraphics[scale=0.1]
       {./example/how-to/write/figure/pic/apple.jpg} \\ Apple
      \end{tabular} & \url{www.apple.com}  \\ \hline

      \begin{tabular}[c]{@{}c@{}}
      \includegraphics[scale=0.1]
        {./example/how-to/write/figure/pic/google.png} \\ Google
      \end{tabular} & \url{www.google.com} \\ \hline
    \end{tabular}
    \end{table}

  } % End of \item{}
  \end{enumerate}

% ------------------------------------------------
\newpage
\StartSubSection{多張}

  如果要同時顯示多張的話, 因為要能一頁版面的範圍內, 同時又要能清楚顯示到你圖中的內容和文字, 大約4張都已經算多的了. 所以真的數量比較多的話, 推薦分別放同不到頁面會比較好閱讀.

  多張是使用\verb|\InsertFigures| ({\bf 注意:} Figure是複數, 有一個`s'), 設計上可插入1$\sim$8張的圖片, 而且寫法會跟插入單張相近.\\

  \begin{DescriptionFrame}
  \begin{verbatim}
  Options 主圖的設定 (使用','來分隔, 不分先後順序)
    perrow:  每一列多少張圖片 (選填, 預設: 1. 最小: 1, 最大: 4)
    caption: 標題 (選填)
    label:   標簽 (選填, 必須要配合caption使用, 否則無效)
    opacity: 背景顏色透明度, 預設使用白色為背景 (選填, 預設: 0.75)
    (0.x ~ < 1.0: 透明; => 1.0: 不透明)

  Figure 1~8: 各張圖片的設定
    設定方式跟使用\InsertFigure是一樣的
    [Figure options] -> [Options]
    {Figure path}  -> {Path}

  插入多張圖片
    \InsertFigures[Options] %
    {
      [Figure options]{Figure path}
    }%
    {
      ...
    }%
    {
      [Figure options]{Figure path}
    }
  ('%'是必須存在的, 以防止被LaTex認為這是新段落)
  \end{verbatim}
  \end{DescriptionFrame}

  \newpage

  {\bf 效果:}
  \begin{enumerate}
  \item
  {
    插入2張圖片, 以1張圖為一列
    \begin{verbatim}
    \InsertFigures
    [caption = {2 figures and 1 figure per row}] %
    {
      {./figure.png}
    }%
    {
      {./figure.png}
    }
    \end{verbatim}
    \InsertFigures
    [caption = {2 figures and 1 figure per row}] %
    {
      {./example/how-to/write/figure/pic/CC-BY-NC.png}
    }%
    {
      {./example/how-to/write/figure/pic/CC-BY-NC-ND.png}
    }
  } % End of \item{}
  \item
  {
    插入2張圖片, 以2張圖為一列
    \begin{verbatim}
    \InsertFigures
    [perrow = 2,
      caption = {2 figures and 2 figures per row}] %
    {
      {./figure.png}
    }%
    {
      {./figure.png}
    }
    \end{verbatim}
    \InsertFigures
    [perrow = 2,
      caption = {2 figures and 2 figures per row}] %
    {
    [caption = {2 figures and 2 figures per row}]%
      {./example/how-to/write/figure/pic/CC-BY-NC.png}
    }%
    {
      {./example/how-to/write/figure/pic/CC-BY-NC-ND.png}
    }
  } % End of \item{}

  \newpage
  \item
  {
    插入3張圖片, 2張圖一列, 並有1張圖轉變角度, 同時主圖跟2張子圖片做了標簽
    \begin{verbatim}
    \InsertFigures
    [perrow = 2,
      caption = {3 figures and 2 figures per row},
      label = {fig:example:mi2:fig1}] %
    {
      [caption = {Figure 1},
      label = {fig:example:mi2:fig1}]
      {./figure.png}
    }%
    {
      [caption = {Figure 2},
      label = {fig:example:mi2:fig2}, angle = -20]
      {./figure.png}
    }%
    {
      [caption = {Figure 3}]
      {./figure.png}
    }
    \end{verbatim}

%    \newpage
    效果會是這樣: \\
    \InsertFigures
    [perrow = 2,%
      caption = {3 figures and 2 figures per row},
      label = {fig:example:mi2:mfig}] %
    {
      [caption = {Figure 1},
      label = {fig:example:mi2:fig1}]
      {./example/how-to/write/figure/pic/CC-BY-NC.png}
    }%
    {
      [caption = {Figure 2},
      label = {fig:example:mi2:fig2},
      angle = -20]
      {./example/how-to/write/figure/pic/CC-BY-NC-ND.png}
    }%
    {
      [caption = {Figure 3}]
      {./example/how-to/write/figure/pic/CC-BY-NC-SA.png}
    }%

    e.g:
    引用主圖 (Fig \RefTo{fig:example:mi2:mfig}) ,
    引用子圖片 (Fig \RefTo{fig:example:mi2:fig1}, Fig \RefTo{fig:example:mi2:fig2}).
  } % End of \item{}

  \newpage
  \item
  {
    插入4張圖片, 2張圖一列, 只有主圖做了標簽.\\
    如果需要不填內容, 但需要圖片的編號的話, 就在caption填寫`{ }'(有空格在中間), 而`{}'則會被認為沒有填寫.
    \begin{verbatim}
    \InsertFigures
    [perrow = 2,
      caption = {4 figures and 2 figures per row},
      label = {fig:example:mi3:mfig}] %
    {
      [caption = { }, label = {fig:example:mi3:fig1}]
      {./figure.png}
    }%
    {
      [caption = {}, label = {fig:example:mi3:fig2}]
      {./figure.png}
    }%
    {
      [caption = { }, label = {fig:example:mi3:fig3}]
      {./figure.png}
    }%
    {
      [caption = { }, label = {fig:example:mi3:fig4}]
      {./figure.png}
    }
    \end{verbatim}

    \InsertFigures
    [perrow = 2,
      caption = {4 figures and 2 figures per row},
      label = {fig:example:mi3:mfig}] %
    {
      [caption = { },
      label = {fig:example:mi3:fig1}]
      {./example/how-to/write/figure/pic/CC-BY-NC.png}
    }%
    {
      [caption = {},
      label = {fig:example:mi3:fig2}]
      {./example/how-to/write/figure/pic/CC-BY-NC-ND.png}
    }%
    {
      [caption = { },
      label = {fig:example:mi3:fig3}]
      {./example/how-to/write/figure/pic/CC-BY-NC-SA.png}
    }%
    {
      [caption = { },
      label = {fig:example:mi3:fig4}]
      {./example/how-to/write/figure/pic/CC-BY-ND.png}
    }

    可以看得出圖片的編號不一樣了\\
    引用主圖 (Fig \RefTo{fig:example:mi3:mfig})\\
    引用子圖片(a) (Fig \RefTo{fig:example:mi3:fig1})\\
    引用子圖片(b) (由於這張圖的caption是沒設定, 所以label無效)\\
    引用子圖片(c) (Fig \RefTo{fig:example:mi3:fig3})\\
    引用子圖片(d) (Fig \RefTo{fig:example:mi3:fig4})
  } % End of \item{}

  %
  \newpage
  \item
  {
    插入8張圖片, 2張圖一列, 只有主圖填了標題.\\
    \begin{verbatim}
    \InsertFigures
    [perrow = 2,
      caption = {8 figures and 2 figures per row}] %
    {[caption = { }]{./figure.png}}%
    {[caption = { }]{./figure.png}}%
    {[caption = { }]{./figure.png}}%
    {[caption = { }]{./figure.png}}%
    {[caption = { }]{./figure.png}}%
    {[caption = { }]{./figure.png}}%
    {[caption = { }]{./figure.png}}%
    {[caption = { }]{./figure.png}}
    \end{verbatim}

    \InsertFigures
    [perrow = 2,
      caption = {8 figures and 2 figures per row}] %
    {[caption = { }]{./example/how-to/write/figure/pic/CC-BY.png}}%
    {[caption = { }]{./example/how-to/write/figure/pic/CC-BY-NC.png}}%
    {[caption = { }]{./example/how-to/write/figure/pic/CC-BY-ND.png}}%
    {[caption = { }]{./example/how-to/write/figure/pic/CC-BY-SA.png}}%
    {[caption = { }]{./example/how-to/write/figure/pic/CC-BY.png}}%
    {[caption = { }]{./example/how-to/write/figure/pic/CC-BY-NC.png}}%
    {[caption = { }]{./example/how-to/write/figure/pic/CC-BY-ND.png}}%
    {[caption = { }]{./example/how-to/write/figure/pic/CC-BY-SA.png}}
  } % End of \item{}

  %
  \newpage
  \item
  {
    插入8張圖片, 3張圖一列, 只有主圖填了標題.\\
    \begin{verbatim}
    \InsertFigures
    [perrow = 3,
      caption = {8 figures and 3 figures per row}] %
    {[caption = { }]{./figure.png}}%
    {[caption = { }]{./figure.png}}%
    {[caption = { }]{./figure.png}}%
    {[caption = { }]{./figure.png}}%
    {[caption = { }]{./figure.png}}%
    {[caption = { }]{./figure.png}}%
    {[caption = { }]{./figure.png}}%
    {[caption = { }]{./figure.png}}
    \end{verbatim}

    \InsertFigures
    [perrow = 3,
      caption = {8 figures and 3 figures per row}] %
    {[caption = { }]{./example/how-to/write/figure/pic/CC-BY.png}}%
    {[caption = { }]{./example/how-to/write/figure/pic/CC-BY.png}}%
    {[caption = { }]{./example/how-to/write/figure/pic/CC-BY.png}}%
    {[caption = { }]{./example/how-to/write/figure/pic/CC-BY.png}}%
    {[caption = { }]{./example/how-to/write/figure/pic/CC-BY.png}}%
    {[caption = { }]{./example/how-to/write/figure/pic/CC-BY.png}}%
    {[caption = { }]{./example/how-to/write/figure/pic/CC-BY.png}}%
    {[caption = { }]{./example/how-to/write/figure/pic/CC-BY.png}}
  } % End of \item{}

  %
  \newpage
  \item
  {
    插入8張圖片, 4張圖一列, 只有主圖填了標題.\\
    \begin{verbatim}
    \InsertFigures
    [perrow = 4,
      caption = {8 figures and 4 figures per row}] %
    {[caption = { }]{./figure.png}}%
    {[caption = { }]{./figure.png}}%
    {[caption = { }]{./figure.png}}%
    {[caption = { }]{./figure.png}}%
    {[caption = { }]{./figure.png}}%
    {[caption = { }]{./figure.png}}%
    {[caption = { }]{./figure.png}}%
    {[caption = { }]{./figure.png}}
    \end{verbatim}

    \InsertFigures
    [perrow = 4,
      caption = {8 figures and 4 figures per row}] %
    {[caption = { }]{./example/how-to/write/figure/pic/CC-BY.png}}%
    {[caption = { }]{./example/how-to/write/figure/pic/CC-BY-NC.png}}%
    {[caption = { }]{./example/how-to/write/figure/pic/CC-BY-ND.png}}%
    {[caption = { }]{./example/how-to/write/figure/pic/CC-BY-SA.png}}%
    {[caption = { }]{./example/how-to/write/figure/pic/CC-BY.png}}%
    {[caption = { }]{./example/how-to/write/figure/pic/CC-BY-NC.png}}%
    {[caption = { }]{./example/how-to/write/figure/pic/CC-BY-ND.png}}%
    {[caption = { }]{./example/how-to/write/figure/pic/CC-BY-SA.png}}
  } % End of \item{}
  \end{enumerate}
% ------------------------------------------------
\EndChapter
% ------------------------------------------------

\newpage% ------------------------------------------------
\StartChapter{Insert Table}{chapter:how-to:write:image}
% ------------------------------------------------
\section{介紹}

% http://www.tablesgenerator.com/

% ------------------------------------------------
\EndChapter
% ------------------------------------------------

\newpage% ------------------------------------------------

\newpage
\StartSection{Equation}

\EquationBegin
  x = &a + b + c + \\
  &d + e + f + g + \\
  &h + i + j + k
\EquationEnd

\EquationBegin{testpage:equation:eq1}E = mc^2\EquationEnd

% ------------------------------------------------

\newpage% ------------------------------------------------
\StartSection{術語/符號 Nomenclature}{chapter:how-to:write:nomenclature}
% ------------------------------------------------

Nomenclature在定義一些在整份論文中所會用到的變數是很常用到的. 它的位置會出現在文章當中或是在Chapter 1之前. 它的設計沒有一個標準答案, 在不同的情況下可能有不同顯示方式, 但它基本上跟一張Table是沒差的. 而它在Latex中是使用一個package名為`nomencl'.

但經過研究了一下package `nomencl'或tabbing這些用來建Nomenclature的方式後, 發現`nomencl'在設計上反而會增加在產生論文時的步驟; 而tabbing要自行定義一個闊度才能弄得比較好看, 但同時內容卻出現沒法置中和設計上等一些問題. 故最後決定直接套用Table來讓同學更能自由的設計不同的Nomenclature table.

設計Nomenclature table需要2個知識或工具:\\
1) 設計一張Table, 這邊請參考P. \RefPage{chapter:how-to:write:table}.\\
2) 有關所需要用到的符號, 請參考Equation (P. \RefPage{chapter:how-to:write:equation})中所使用到的工具, Texmarker左邊的工具列, 或看這幾個網頁\RefBib{web:symbols:site1}\RefBib{web:symbols:site2}\RefBib{web:symbols:site3}, 應該已經足夠同學們寫出合適的符號.

% ------------------------------------------------
%\newpage
\StartSubSection{使用方式}

如果是指是在Chapter 1之前的一大張的Nomenclature table, 為Nomenclature Chapter.
  \begin{verbatim}
  \StartNomChapter{ NAME }{ LABEL }
  \EndNomChapter
  \end{verbatim}
Nomenclature Chapter跟一般Chapter的使用方式是一樣的, 但差別在於不會出現`Chapter'這字眼. 而由於大家的Nomenclature Chapter name可能不一樣, 故跟Chapter一樣可設定自行的name.

而如果是在文章當中的Nomenclature table. 基本上就是使用同一個的`\verb|\InsertTable|', 但還可以使用`nomtitle'來設定標題. `nomtitle'跟`caption'的差別是, 使用`nomtitle'所顯示出來的標題是沒有`Table XX:'為開頭, 同樣都是使用`pos'來控制題目的位置.

  \begin{DescriptionFrame}
  \begin{verbatim}
  Options 設定
    nomtitle:   Nomenclature 標題 (選填)
    ...

  E.g
    \InsertTable
    [nomtitle={這是Nomenclature Table的標題}]
      {
        ...
      }
  \end{verbatim}
  \end{DescriptionFrame}

有關這個的用法可參考`example/nomenclature/nomenclature.tex'中的Nomenclature Chapter所demo的例子, 那2個例子只是最簡單的Nomenclature table設計, 應該足夠同學們去弄出合適自己的Nomenclature table的設計.

\newpage% ------------------------------------------------
\StartSection{文獻引用 Bibliography/Reference}{chapter:how-to:write:bib}
% ------------------------------------------------

\StartSubSection{介紹}

Reference對論文來講十分重要的東西, 所以如果你引用的paper數量不少, 那在整理上會有點麻煩, 所以世界上有不少東西來管理這部份的資料, 如用的Word的話會配合Endnote.

而本模版是使用LaTex中的BibTex來管理, 你可以在`./content/references'找到3個`.bib'檔, 那正是你可以把你所引用的內容放在裡面.

Bib的分類滿多 (參考\RefBib{web:latex:bib_manage}), 但論文主要都是引用`book' (課本, 書籍等), `misc' (網頁, 任何其他東西), `inproceedings' (論文類)中的內容, 所以本模版提供的樣板檔案為`book.bib', `misc.bib' 跟 `paper.bib'.

\StartSubSection{使用方式}

任何放置論文的出版社(如ACM, IEEE, DBLP等), 都會為了方便別人去引用, 都會提供一些資料以給放在論文中引用. Fig \RefTo{fig:write:bib:1} 是以ACM Digital Library例子, 簡單說明如何使用BibTex來管理.

\InsertFigure
  [caption={ACM Digital Library例子},
    label={fig:write:bib:1}, scale=0.5]
  {./example/how-to/write/bib/pic/1.png}

\InsertFigure
  [caption={BibTex的位置},
    label={fig:write:bib:2}, scale=0.4]
  {./example/how-to/write/bib/pic/2.png}

在畫面右方會看到`Export Formats'的位置, 會看到如fig \RefTo{fig:write:bib:2}中一個的BibTex的按鈕.

\InsertFigure
  [caption={BibTex資料},
    label={fig:write:bib:3}, scale=0.5]
  {./example/how-to/write/bib/pic/3.png}

按它後就會出現如fig \RefTo{fig:write:bib:3}這個畫面, 這個就是要填進Bib的資料, 所以把這個東西複製到Bib檔內.

\InsertFigure
  [caption={整理/使用BibTex},
    label={fig:write:bib:4}, scale=0.5]
  {./example/how-to/write/bib/pic/4.png}

但複製完後要改一個東西, 第一行是所謂的label部份(參考Chap \RefTo{chapter:how-to:write:label}), 所以要改成一個自己能記得的label以方便在內容中來引用.

%有什麼問題可以去問Google\cite{website:google}老師. (如果有設定references用的檔案, 即使用了ReferencesFiles, 那必須至少要存在一個cite才不會顯示錯誤.)

% ------------------------------------------------
\EndChapter
% ------------------------------------------------

\newpage% ------------------------------------------------
\StartSection{虛擬程式碼(Pseudocode)}{chapter:how-to:write:pseudocode}
% ------------------------------------------------

\StartSubSection{介紹}

Pseudocode在資訊類的paper是很常見, 雖然這東西冷門, 但是有它的存在意義.

而由於真的要寫Pseudocode的人, 理論上都100\%會寫程式, 所以有關這邊會直接使用例子(基本的function, if-elseif-else, while, return, switch-case)來說明, 靠例子應該就能寫出你所要的Pseudocode.

唯一注意的是需要使用:\\
'\verb|\Statex|'來斷一行空行\\
'\verb|\State|'來斷一行以寫新code在後面

% ------------------------------------------------

\newpage
\begin{algorithm}
  \caption{My algorithm (function A)}
  \label{algo:functionA}

  \begin{algorithmic}[1]
    \Function{function\_name\_a}{arg1, arg2}
      \If{conditionA}
        \State ...
      \ElsIf{conditionB}
        \State ...
      \Else
        \State ...
      \EndIf
      \Statex
      \If{condition1}
        \State ...
      \Else
        \If{condition2}
          \State ...
        \Else
          \State ...
        \EndIf
      \EndIf
      \Statex
      \For{condition}
        \State ...
      \EndFor
    \EndFunction
  \end{algorithmic}
\end{algorithm}

\newpage
針對function A (Algorithm \RefTo{algo:functionA}), 它的Latex寫法為:
\begin{framed}
  \begin{verbatim}
\begin{algorithm}
  \caption{My algorithm (function A)}
  \label{algo:functionA}

  \begin{algorithmic}[1]
    \Function{function\_name\_a}{arg1, arg2}
      \If{conditionA}
        \State ...
      \ElsIf{conditionB}
        \State ...
      \Else
        \State ...
      \EndIf
      \Statex
      \If{condition1}
        \State ...
      \Else
        \If{condition2}
          \State ...
        \Else
          \State ...
        \EndIf
      \EndIf
      \Statex
      \For{condition}
        \State ...
      \EndFor
    \EndFunction
  \end{algorithmic}
\end{algorithm}
  \end{verbatim}
\end{framed}

% ------------------------------------------------

\newpage
\begin{algorithm}
  \caption{My algorithm (function B)}
  \label{algo:functionB}

  \begin{algorithmic}[1]
    \Function{functionNameB}{}
      \State ...
      \State Some code here
      \State ...
      \Statex
      \While{condition3}
        \State ...
      \EndWhile
      \Statex
      \Repeat
        \State ...
      \Until{condition3}
      \Statex
      \Switch{condition4}
        \Case{condition5} ... \Break \EndCase
        \Statex
        \Case{condition6}
          \State ...
          \State \Break
        \EndCase
        \Statex
        \Default
          \State ...
        \EndDefault
      \EndSwitch

      \Statex\State \Return retValue
    \EndFunction
  \end{algorithmic}
\end{algorithm}

\newpage
針對function B (Algorithm \RefTo{algo:functionB}), 它的Latex寫法為:
\begin{framed}
  \begin{verbatim}
\begin{algorithm}
  \caption{My algorithm (function B)}
  \label{algo:functionB}

  \begin{algorithmic}[1]
    \Function{functionNameB}{}
      \State ...
      \State Some code here
      \State ...
      \Statex
      \While{condition3}
        \State ...
      \EndWhile
      \Statex
      \Repeat
        \State ...
      \Until{condition3}
      \Statex
      \Switch{condition4}
        \Case{condition5} ... \Break \EndCase
        \Statex
        \Case{condition6}
          \State ...
          \State \Break
        \EndCase
        \Statex
        \Default
          \State ...
        \EndDefault
      \EndSwitch

      \Statex\State \Return retValue
    \EndFunction
  \end{algorithmic}
\end{algorithm}
  \end{verbatim}
\end{framed}


% ------------------------------------------------
\EndChapter
% ------------------------------------------------


% Words from professors section
% ------------------------------------------------
\StartChapter{老師們的話 Words from Professors}{chapter:words-from-professor}
% ------------------------------------------------

這部份的內容節錄於我跟系上老師的一些對話, 和上課所聽得出的結論和想法而整理出來的, 所以某些地方會帶有我們濃郁的資工系味道. 另外如果有任何的老師 (不論本系外系)可以提供一些意見或想法的話, 我會十分感謝的.

% ------------------------------------------------
\StartSection{想法}

\begin{enumerate}
  \item
  {
    有用才算創新, 要站在使用者的角度去想
  } % End of \item{}

  \item
  {
    技術$\neq$研究, 是研究才有系統跟技術
  } % End of \item{}

  \item
  {
    研究
    \begin{itemize}
      \item
      {
        就是去想問題, 以不同的角度去想東西跟解決的方法
      } % End of \item{}

      \item
      {
        十分重要的是, 為什麼要這樣做, 這跟別人有什麼差別, 而且這樣做好處是什麼
      } % End of \item{}

      \item
      {
        `工程科系'是以多答案去解決一個問題, 而`理科'是提出一個標準的答案
      } % End of \item{}

      \item
      {
        不要相信直覺, 要所有東西都要證據
      } % End of \item{}

      \item
      {
        找研究題目的方法
      } % End of \item{}

      \begin{enumerate}
        \item
        {
          針對傳統的問題, 用方法不一樣去處理它
        } % End of \item{}

        \item
        {
          把一個問題的原本假設, 環境和條件之類的進行變動, 以得到不同的結果
        } % End of \item{}
      \end{enumerate}
    \end{itemize}
  } % End of \item{}

  \item
  {
    在coding中, Bug就是你的盲點或你所不懂的.

    所以如果你在debug時是位置``經驗法則''來預估bug的位置, 這即是進行while loop, 永遠都找不到.

    你首先要得到bug所經過的code, 一個一個地插入debug message來分析變數和跑到哪去, 慢慢地縮少範圍, 這樣有數據式debug, 會比``經驗法則''來得快.
  } % End of \item{}
\end{enumerate}

% ------------------------------------------------
\StartSection{投影片/presentation}

\begin{enumerate}
  \item
  {
    口試用PPT
    \begin{itemize}
      \item
      {
        要有outline, 而且要講大約要用多少時間來講
      } % End of \item{}

      \item
      {
        每個chapter都有一個頁面用來做分頁, 以讓口試委員知道聽到哪一個部份了.
      } % End of \item{}

      \item
      {
        需要一些backup slide; 例如只講5張而已, 但backup用50張; 內容主要是一些data的來源, 和名詞解釋等
      } % End of \item{}

      \item
      {
        最核心要留時間用心讀給口試委員知道, 就算慢慢講用了2-3分鐘都是非常值得的.
      } % End of \item{}

      \item
      {
        ppt每一個section都要有一頁做summary/換頁作為結尾, 以讓聽的人回憶, 記憶, 剛說了什麼.
      } % End of \item{}

      \item
      {
        如果內容是多個block的流程, 要想辦法顯示出自己在講的位置, 否則別人會幾頁後就忘了前面在說什麼.
      } % End of \item{}

      \item
      {
        如果下頁是一個demo, 畫面之類等圖案, 圖片. 在上一頁的結尾是說``下一頁會展示這東西xxxx的畫面''.
      } % End of \item{}

      \item
      {
        在ppt中, 在說明自己的方法, 如``Result - Method A'', Method A 應用斜體字.
      } % End of \item{}

    \end{itemize}
  } % End of \item{}

  \item
  {
      1分鐘的報告
    \begin{itemize}
      \item
      {
        用one slide
      } % End of \item{}

      \item
      {
        主要使用graph
      } % End of \item{}

      \item
      {
        1,2 句的text
      } % End of \item{}

      \item
      {
        Some data
      } % End of \item{}

      \item
      {
        要做得能吸引眼睛
      } % End of \item{}
    \end{itemize}
  } % End of \item{}

  \item
  {
    一般報告paper
    \begin{itemize}
      \item
      {
        報1張ppt的時間應該是只有1分鐘左右 (除非詳細的系統架構圖), 因為讀1個中文字大約0.3秒
      } % End of \item{}

      \item
      {
        總原則
        \begin{enumerate}
          \item
          {
            解決了\textbf{`什麼'}的問題, 一定要非常清楚, 簡潔有力的說明
          } % End of \item{}

          \item
          {
            多用圖, 文字要讀完才能理解, 但圖可以有一看就懂的效果
          } % End of \item{}
        \end{enumerate}
      } % End of \item{}

      \item
      {
        Introduction
        \begin{enumerate}
          \item
          {
            什麼環境
          } % End of \item{}

          \item
          {
            什麼應用而造成這個問題
          } % End of \item{}

          \item
          {
            Given什麼條件
          } % End of \item{}

          \item
          {
            Find什麼條件
          } % End of \item{}

          \item
          {
            在什麼狀態下
          } % End of \item{}

          \item
          {
            Idea of the solution\\
            把最基本的精神講出來就可以, 不需要講detail
          } % End of \item{}
        \end{enumerate}
      } % End of \item{}

      \item
      {
        Related Work
        \begin{enumerate}
          \item
          {
            講解相關的研究
          } % End of \item{}

          \item
          {
            在1x分鐘中的報告是不用講, 除非如果不講相關的研究, 接下去觀眾就會完全不懂, 這才需要去提到 (因為是非常相關)
          } % End of \item{}
        \end{enumerate}
      } % End of \item{}

      \item
      {
        演算法
        \begin{itemize}
          \item
          {
            No
            \begin{enumerate}
              \item
              {
                不要講變數
              } % End of \item{}

              \item
              {
                不要把整個演算法顯示出來一步步講
              } % End of \item{}

              \item
              {
                不要用pseudocode
              } % End of \item{}
            \end{enumerate}
          } % End of \item{}

          \item
          {
            Yes
            \begin{enumerate}
              \item
              {
                盡量使用圖片來講解演算法
              } % End of \item{}
            \end{enumerate}
          } % End of \item{}
        \end{itemize}
      } % End of \item{}

      \item
      {
        公式
        \begin{enumerate}
          \item
          {
            不用講detail\\
            $ P( Q_{ni} ) = \frac{ 2^{k} - 1}{ 2^{n} - 1} $\\
            右邊部份不用說明\\
            只要講一整個公式的用途是在算什麼就行了
          } % End of \item{}

          \item
          {
            $ REL = A + B + C $\\
            只要講$REL$在算什麼就行了\\
            (除非別人不懂在講要算什麼, 才要把A, B, C都講出來大約算什麼則行了)
          } % End of \item{}
        \end{enumerate}
      } % End of \item{}

      \item
      {
        Theorem定理
        \begin{enumerate}
          \item
          {
            Definition\\
            在以後會常常說明的觀念, 為了以後方便講解和使用, 則使用Definition.\\
            在1x分鐘的報告中, 如果不常用,則不用講Definition, 如需要或常常會使用才需要.
          } % End of \item{}

          \item
          {
            Lemma\\
            是Theorem分開用來簡單說明的一個東西
          } % End of \item{}

          \item
          {
            Theorem\\
            是Lemma集合出的一個理論
          } % End of \item{}

          \item
          {
            Corollary\\
            在Theorem的結果用另一種條件或什麼得出的另一結果
          } % End of \item{}

          \item
          {
            Proposition\\
            以上的看情況來決定要不要講, 如果是跟algorithm無關的, 則不用講, 否則要講一點點.\\

            如果不講定理, 都能講懂algorithm, 那則不用講.\\
            而如果algorithm會使用到一個小小的定理, 即只要講定理的結果.
          } % End of \item{}

          \item
          {
            Proof\\
            在報告時是絕對不用講的
          } % End of \item{}
        \end{enumerate}
      } % End of \item{}

      \item
      {
        Performance\\
        除非作者沒有提供任何做實驗的數據, 否則正常情況下都要講解這部份的內容.
        (有一些研究方向或實驗室, 沒有要求對這部份作要求的話, 那是可以不用說明的)\\
        必須說明作者使用的dataset是什麼, 環境是什麼等一些基本資料. 之後作者做了什麼實驗, 效果如何, 發現了什麼.

        但是注意的是要對內容進行選擇, 不必要100\%的實驗資料都要拿出來講解, 只要講解這演算法最核心的一些實驗(如系統架構)就可以了.
      } % End of \item{}

      \item
      {
        優缺點, 建議 (十分重要)\\
        優點其實作者就會大力的說明, 所以不難找到.\\
        但是更重要的是, 作者一般都不會點出這演算法的缺點, 所以必須要看懂缺點在哪, 有什麼建議, 有什麼可以改進的方法, 或是有什麼方法可以用來延伸.
      } % End of \item{}

      \item
      {
        總結
        \begin{enumerate}
          \item
          {
            愈快讓人明白整個paper的要點.
          } % End of \item{}

          \item
          {
            最好能用圖片來說明.
          } % End of \item{}

          \item
          {
            Top-Down manner\\
            先講整體的觀念, 後才一部份一部份的講內容
          } % End of \item{}
        \end{enumerate}
      } % End of \item{}
    \end{itemize}

    科技論文, 是一開始就把結果說出來; 而其他的作文, 則是使用`起、承、轉、合'的手法. 但這是對論文是不對的.
  } % End of \item{}
\end{enumerate}

% ------------------------------------------------
\StartSection{投論文的目標}

\begin{enumerate}
  \item
  {
    學位論文不影響以後把內容用來投去什麼的地方, 例如可以把學位論文100\%把內容移到journal中. 所以最重要的要做是優先把學位論文寫的, 才考慮投去哪
  } % End of \item{}

  \item
  {
    找paper用來投的地方, 可以到``系網->學生事務->碩博士->期刊,會議點數''
  } % End of \item{}

  \item
  {
    寫完才考慮投去哪裡, 才把資料修成那邊要的樣子
  } % End of \item{}
\end{enumerate}

% ------------------------------------------------
\StartSection{實驗的比較對象}

\begin{enumerate}
  \item
  {
    千萬不能對不同架構, 規模不一樣的對象來進行比較
  } % End of \item{}

  \item
  {
    用電腦系統來講
    \begin{itemize}
      \item
      {
        Single server只能跟single server比較
      } % End of \item{}

      \item
      {
        Distributed system只能跟distributed system來比
      } % End of \item{}
    \end{itemize}
  } % End of \item{}
\end{enumerate}

% ------------------------------------------------
\StartSection{Related work}

只要有提到的對象, 就要去跟它比較; 不能比的就要去講差別; 有paper的就要去實作別人的部份功能

% ------------------------------------------------
\StartSection{References}

\begin{enumerate}
  \item
  {
    要拿去哪投哪裡, 就起碼最少要引用一篇那邊的paper, 否則對方一般都不太想去看 (利益問題)
  } % End of \item{}

  \item
  {
    References所選的對象, 要根據這排名去選, 越高越有說服力:
    \begin{itemize}
      \item
      {
        Paper / book\\
        Paper所提出的東西一定會做過實驗或計算, 所以有一定的正確性. 但更新速度快, 所以會有很大量的.\\
        Book是經過好多年才會把一些正確的知識整合起來, 所以速度較慢, 但是以當代來講是最正確.
      } % End of \item{}

      \item
      {
        Tech report / Datasheet\\
        Tech report是一些人對某種東西去做研究或實驗, 所以他們會先把那個東西進行分析和理解, 故所寫出來的東西都經過他們的分析和研究, 雖然沒有paper那種程度的說服力, 但還是可以被人用來學習和引用的.\\
        Datasheet是一個系統或library的開發者所寫下來的, 因為他們是最懂得那東西, 所以使用他們的資料是可以被接受的.
      } % End of \item{}

      \item
      {
        Article\\
        Article是某些或某人去對一個主題去做, 所以所寫所說都是他們的立場或想法, 不一定100\%是正確; 但這些Article都是在反映人們對某主題的分析或理解, 所以可以代表以當代來講, 人們在意的部份是什麼.
      } % End of \item{}

      \item
      {
        URL / Website\\
        URL是最不應該當成References來使用, URL出現只能當符合以下情況:
        \begin{itemize}
          \item
          {
            URL所指向的是系統, tools, library的官方網站
          } % End of \item{}

          \item
          {
            URL所指向的是有關所使用的系統, tools, library的Datasheet
          } % End of \item{}

          \item
          {
            URL所指向的是Related work中要比較的對象它相關的資料會使用在這篇論文中, 如source code.
          } % End of \item{}
        \end{itemize}

        否則的話, 千萬不要放, 因為越多的URL, 說服力會越低.\\

        另外都千萬不要使用Wikipedia當成References, 雖然Wikipedia是知識解說的地方, 但Wikipedia正因為太普遍, 所以完全沒有任何特殊的說服力; 如同介紹人們去做search網頁時, 可以使用google, yahoo是同一個道理.
      } % End of \item{}
    \end{itemize}
  } % End of \item{}
\end{enumerate}

% ------------------------------------------------
\StartSection{圖上的文字 / 表格}

除非特殊要求, 否則不能比正常的文字小 (必須 $ >= $ 10 pt), 要令人感覺每一個文字都是一樣大的, 要讓讀者可以一口氣看, 而不用做放大放小的行為.

% ------------------------------------------------
\StartSection{寫作技術}

每一個新section的開頭段落, 不能以`所以', `so'等文字, 而是必須要再用一些文字當起點, 如`前一章提到xxxxxx'.

% ------------------------------------------------
\StartSection{內容}

\begin{enumerate}
  \item
  {
    Paper必須要做到self-contained, 要把用到的其他知識時, 必須要有example以解釋這thesis在說什麼
  } % End of \item{}

  \item
  {
    不能使用了別的東西, 而完全沒解釋是什麼意思, 要讀者去查References的資料去理解這thesis在做什麼
  } % End of \item{}
\end{enumerate}

% ------------------------------------------------
\StartSection{公式}

\begin{enumerate}
  \item
  {
    避免重複使用\\
    $
      \begin{array}{ll}
            P(X) = \ldots & (A)\\
            P(X) = \ldots & (B)
      \end{array}
    $\\
    但2個$ P(X) $都代表不同的意思
  } % End of \item{}

  \item
  {
    大小寫不能一起用\\
    如 $ rel (a) $, $ REL(a) $, 但是不同意思
  } % End of \item{}

  \item
  {
    Subscript/superscript\\
    上下標是用來區分用的\\
    如$ w_{1} $, $ w_{2} $, $ w_{3} $ $ \ldots $.\\

    但不需要的話, 就不要加這個東西, 如:\\
    $
      \left\{
        \begin{array}{ll}
          r_{1}(a) & = \ldots\\
          r_{2}(a) & = \ldots
      	\end{array}
      \right.
    $\\
    但2個$ r(a) $代表同一個意思
  } % End of \item{}

  \item
  {
    名字不要太長\\
    如$ sim(a,b) $\\
    因為很像similarity (相似), 所以可以使用, 但沒有近像的字, 就不要用寫這樣
  } % End of \item{}

  \item
  {
    變數沒用就不要寫\\
    如$ pv(u,a) = 1 / distance $\\
    $U$和$a$都是沒意義, 所以可以去掉
  } % End of \item{}
\end{enumerate}

% ------------------------------------------------
\EndChapter
% ------------------------------------------------


% ------------------------------------------------

% 參考文獻 References
% References and bibliography

% Import the files that contain your references.
% If you set some references file,
% you need to use at least one cite to make Latex work.

\ReferencesFiles{./context/references/paper}{./context/references/misc}{./context/references/book}


% ------------------------------------------------

% 附錄 Appendix
% ------------------------------------------------
\StartAppendix
% ------------------------------------------------

% ------------------------------------------------
% Page start
\newpage
\phantomsection
% ------------------------------------------------

\chapter{可使用這模版的系所}
\label{appendix:acceptable-dept}

這邊列出一些\textbf{應該可使用}或\textbf{不可使用}這模版的系所名字, 這表可能會有不正確, 所以還是先問系辦確定會比較好.\\

因為這名單都是靠網路上能找多少而得出的結果, 而如果沒有分類的話, 很高機會是使用圖書館的要求. \\

而如果這表名單中沒有顯示你的系所, 但你已經\textbf{知道}是否能使用, 請告知以供更新.

\clearpage

\section{應該可使用}

  可使用的原因幾乎都是系所自己沒有特殊要求, 所以直接使用圖書館的要求, 而本模版就是跟隨圖書館所定下的要求來設計.

  \begin{table*}[pht]
  \centering
  \caption{應該可使用的系所}
  \label{table:acceptable-dept:acceptable}
  \begin{tabular}{|c|c|}

  \hline
  \multicolumn{1}{|c|}{資訊工程學系} &
  \multicolumn{1}{c|}{Department of Computer Science and Information Engineering} \\

  \hline
  \end{tabular}
  \end{table*}

\section{應該不可使用}

  不可使用的原因是系所自己已經有提供一份樣版出來, 而那份樣版的要求有沒有跟本模版一樣設計, 這個就不作詳細分析. 故如果已經有樣版, 那我就會把它們分類成\textit{無法使用這本模版}比較好, 但如果分類錯誤, 請告知.

  \begin{table*}[pht]
  \centering
  \caption{應該不可使用的系所}
  \label{table:acceptable-dept:unacceptable}
    \begin{tabular}{|c|c|c|}

    \hline
    \multicolumn{1}{|c|}{生物科技研究所} &
    \multicolumn{1}{c|}{Institute of Biotechnology} &
    \multicolumn{1}{c|}{\href{www.biotech.ncku.edu.tw/files/archive/331_4b79187a.doc}{Link}} \\

    \hline
    \multicolumn{1}{|c|}{體育健康與休閒研究所} &
    \multicolumn{1}{c|}{
      \tabincell{l}{
      Institute of Physical Education,\\
      Health and Leisure Studies
      }
    } &
    \multicolumn{1}{c|}{\href{www.ncku.edu.tw/~deprb/docs/Thesis\%20Regulation\%20.doc}{Link}} \\

    \hline
    \end{tabular}
  \end{table*}

% ------------------------------------------------
% End of page
% ------------------------------------------------

% ------------------------------------------------
\StartChapter{各系所博碩士撰寫論文須知}{appendix:thesis-spec}
% ------------------------------------------------

這部份資料來源是使用'電子學位論文服務'提供'國立成功大學博碩士學位論文格式規範'\RefBib{web:ncku:thesis-need-to-know}.\\

\setboolean{@twoside}{false}
\includepdf[pages=-]{./example/appendix/pdf/thesis-spec-a.pdf}

% ------------------------------------------------
\EndChapter
% ------------------------------------------------

% ------------------------------------------------
\StartChapter{電子論文上傳前檢查事項}{appendix:e-paper_upload}
% ------------------------------------------------

\section{介紹}
這部份資料來源是使用'成功大學電子學位論文服務'中的'電子論文上傳前檢查事項'的'2012090001.pdf'.\\

同樣原檔案沒法顯示, 故需要進行另儲存.\\

檔案位置:\\
新: 'example/appendix/pdf/2012090001-a.pdf'\\
原: 'example/appendix/pdf/2012090001.pdf'\\

\setboolean{@twoside}{false}
\includepdf[pages=-]{./example/appendix/pdf/2012090001-a.pdf}

% ------------------------------------------------
\EndChapter
% ------------------------------------------------

% ------------------------------------------------
\StartChapter{學位論文上傳說明}{appendix:e-paper_upload_ppt}
% ------------------------------------------------

\StartSection{介紹}
這部份資料來源是使用圖書館提供的'2014 學位論文上傳說明會'修改而成的, 只抽出使用本模板後, 還要做什麼的行為.\\

\setboolean{@twoside}{false}
\includepdf[pages=-]{./example/appendix/pdf/2012050003-short-a}

% ------------------------------------------------
\EndChapter
% ------------------------------------------------



% ------------------------------------------------
\EndAppendix
% ------------------------------------------------


% ------------------------------------------------

% 用來對內容進行不同的設定測試用的測試頁
% ------------------------------------------------
\StartChapter{測試頁 Testing Pages}
% ------------------------------------------------

這邊是用來放置不同的內容來進行測試或設計樣版用的.

% ------------------------------------------------

% 半形字元與全形字元
%% ------------------------------------------------
%
% Reference from:
%
% 半形字元與全形字元
% <https://zh.wikipedia.org/wiki/%E5%85%A8%E5%BD%A2%E5%92%8C%E5%8D%8A%E5%BD%A2>
%
% ------------------------------------------------

\StartSection{半形字元與全形字元的比較 (ASCII字元)}
用來檢查半形/全形字元顯示正常\\

\begin{verbatim}
   全形    半形        |        全形    半形        |        全形    半形
   " "     " "                 !       !                  "       "
    #       #                  $       $                  %       %
    &       &                  '       '                  (       (
    )       )                  *       *                  +       +
    ,       ,                  -       -                  .       .
    /       /

    0       0                 1       1                  2       2
    3       3                 4       4                  5       5
    6       6                 7       7                  8       8
    9       9

    :       :                 ;       ;                  <       <
    =       =                 >       >                  ?       ?
    @       @

    A       A                 B       B                  C       C
    D       D                 E       E                  F       F
    G       G                 H       H                  I       I
    J       J                 K       K                  L       L
    M       M                 N       N                  O       O
    P       P                 Q       Q                  R       R
    S       S                 T       T                  U       U
    V       V                 W       W                  X       X
    Y       Y                 Z       Z

    [       [                 \       \                  ]       ]
    ^       ^                 _       _                  `       `

    a       a                 b       b                  c       c
    d       d                 e       e                  f       f
    g       g                 h       h                  i       i
    j       j                 k       k                  l       l
    m       m                 n       n                  o       o
    p       p                 q       q                  r       r
    s       s                 t       t                  u       u
    v       v                 w       w                  x       x
    y       y                 z       z

    {       {                 |       |                  }       }
    ~       ~
\end{verbatim}

% ------------------------------------------------


% ------------------------------------------------
\UseDefaultLinesSpacing
% ------------------------------------------------

% 英文內容
%% ------------------------------------------------

\newpage
\StartSection{英文內容 (只用段落來分段)}
用來看內容, 符號, 段距, 字元之間的距離等東西\\

NCKU offers an open learning environment characterized by classical Western and modern eastern landscape. NCKU has attracted numerous distinguished visitors around the world for its rich and welcoming campuses.

In addition to two off-university campuses, An-Nan and Gueiren, the main campus of NCKU consists of 7 satellite campuses adjacent to one another. NCKU occupies a total of more than 180 hectares of land, which tops many other universities in Taiwan. NCKU started out as having only one campus, Cheng Kung, but continued to expand to its current scale. Each campus in the major part of NCKU is closely interlinked and tightly developed as a city within the university.

NCKU's spirit of ``pristine practicality'' and its motto of ``intellectual development through persistent pursuit of knowledge'' have been intrinsic to the thick cultural heritage of the city of Tainan. While members of NCKU are well-integrated with one another, they also work independently. Patrons as NCKU enjoy abundant learning resources.

Through continuous evolution and progression, NCKU has become an active participant in global academia and has earned an excellent reputation in teaching and research. For example, NCKU has ranked the 256th in 2010 Academic Ranking of World Universities announced by Shanghai Jiaotong University and the 80th in 2011 Webometrics Ranking of World Universities published by Centre for Scientific Information and Documentation, Spain, only second to that of NTU, Todai and Kyodai in the Asian region.

% ------------------------------------------------

\newpage
\StartSection{英文內容 (只用強制斷行)}
用來看內容, 符號, 段距, 字元之間的距離等東西\\

NCKU offers an open learning environment characterized by classical Western and modern eastern landscape. NCKU has attracted numerous distinguished visitors around the world for its rich and welcoming campuses.\\
In addition to two off-university campuses, An-Nan and Gueiren, the main campus of NCKU consists of 7 satellite campuses adjacent to one another. NCKU occupies a total of more than 180 hectares of land, which tops many other universities in Taiwan. NCKU started out as having only one campus, Cheng Kung, but continued to expand to its current scale. Each campus in the major part of NCKU is closely interlinked and tightly developed as a city within the university.\\
NCKU's spirit of ``pristine practicality'' and its motto of ``intellectual development through persistent pursuit of knowledge'' have been intrinsic to the thick cultural heritage of the city of Tainan. While members of NCKU are well-integrated with one another, they also work independently. Patrons as NCKU enjoy abundant learning resources.\\
Through continuous evolution and progression, NCKU has become an active participant in global academia and has earned an excellent reputation in teaching and research. For example, NCKU has ranked the 256th in 2010 Academic Ranking of World Universities announced by Shanghai Jiaotong University and the 80th in 2011 Webometrics Ranking of World Universities published by Centre for Scientific Information and Documentation, Spain, only second to that of NTU, Todai and Kyodai in the Asian region.

% ------------------------------------------------

\newpage
\StartSection{英文內容 (段落 + 強制斷行)}
用來看內容, 符號, 段距, 字元之間的距離等東西\\

NCKU offers an open learning environment characterized by classical Western and modern eastern landscape. NCKU has attracted numerous distinguished visitors around the world for its rich and welcoming campuses.\\

In addition to two off-university campuses, An-Nan and Gueiren, the main campus of NCKU consists of 7 satellite campuses adjacent to one another. NCKU occupies a total of more than 180 hectares of land, which tops many other universities in Taiwan. NCKU started out as having only one campus, Cheng Kung, but continued to expand to its current scale. Each campus in the major part of NCKU is closely interlinked and tightly developed as a city within the university.\\

NCKU's spirit of ``pristine practicality'' and its motto of ``intellectual development through persistent pursuit of knowledge'' have been intrinsic to the thick cultural heritage of the city of Tainan. While members of NCKU are well-integrated with one another, they also work independently. Patrons as NCKU enjoy abundant learning resources.\\

Through continuous evolution and progression, NCKU has become an active participant in global academia and has earned an excellent reputation in teaching and research. For example, NCKU has ranked the 256th in 2010 Academic Ranking of World Universities announced by Shanghai Jiaotong University and the 80th in 2011 Webometrics Ranking of World Universities published by Centre for Scientific Information and Documentation, Spain, only second to that of NTU, Todai and Kyodai in the Asian region.

% ------------------------------------------------


% ------------------------------------------------
\UseDefaultLinesSpacing
% ------------------------------------------------

% 中文內容
%% ------------------------------------------------

\newpage
\StartSection{中文內容 (只用段落來分段)}
用來看內容, 符號, 段距, 字元之間的距離等東西\\

本校創校於西元1931年(昭和6年,民國20年)1月15日,原名為「臺南高等工業學校」;1944年(昭和19年,民國33年)改稱為「臺南工業專門學校」。民國34年臺灣光復。本校於民國35年2月改制為「臺灣省立臺南工業專科學校」,由王石安博士擔任校長;35年10月改制為「臺灣省立工學院」,仍由王石安博士擔任校長。彼時僅有成功校區,39年增購勝利校區。41年2月,由秦大鈞博士接任校長。

45年8月,本校改制為「臺灣省立成功大學」,仍由秦大鈞博士擔任校長;同時增設文理學院及商學院。46年8月,由閻振興博士接任校長。54年1月,由羅雲平博士接任校長。55年增購光復校區。58年10月,將文理學院分為文學院及理學院。60年8月,改制為「國立成功大學」,並由倪超博士接任校長;同年增購建國校區。67年8月,由王唯農博士接任校長。

69年8月,夏漢民博士接任校長;同年將商學院更名為管理學院。72年8月,增設醫學院,並增購自強校區及敬業校區。74年增購力行校區部分校地。76年增購歸仁校區,設置航空太空實驗場。77年6月本校醫學院附設醫院正式營運。77年8月,馬哲儒博士接任校長。80年增購自強校區北半部。82年陸續增購台南市「文大五」用地,闢為本校安南校區。

83年8月,吳京院士接任校長。吳校長於85年6月入閣擔任教育部長,由副校長黃定加博士代理校長。86年2月,翁政義博士接任校長。86年8月增設社會科學院。88年完成收購安南校區全部校地;同年6月增購原陸軍八○四醫院用地(91年6月完成撥用手續)。翁校長於89年5月出任國科會主委,由副校長翁鴻山博士代理校長。90年2月,高強博士接任校長。92年8月,增設電機資訊學院、規劃與設計學院。94年7月,本校配合國軍斗六醫院精實案,接管該院營運權,設置為雲林縣斗六校區,並改制為本校醫學院附設醫院斗六分院。94年8月,增設生物科學與科技學院。94年10月,本校得到教育部的肯定,獲選為「發展國際一流大學及頂尖研究中心計畫」的兩所重點大學之一。

96年2月,賴明詔院士接任校長,97年2月教育部公佈「發展國際一流大學及頂尖研究中心計畫」第二梯次的審議結果,本校繼續獲得教育部的肯定與補助,積極朝國際一流大學的目標邁進。97年10月增購歸仁校區北側台糖土地(正式登記為本校管有)。100年2月,黃煌煇博士接任校長。100年4月本校獲得教育部第二期頂尖大學計畫補助,持續朝國際一流大學的目標邁進。104年2月,蘇慧貞博士接任校長。

% ------------------------------------------------

\newpage
\StartSection{中文內容 (只用強制斷行)}
用來看內容, 符號, 段距, 字元之間的距離等東西\\

本校創校於西元1931年(昭和6年,民國20年)1月15日,原名為「臺南高等工業學校」;1944年(昭和19年,民國33年)改稱為「臺南工業專門學校」。民國34年臺灣光復。本校於民國35年2月改制為「臺灣省立臺南工業專科學校」,由王石安博士擔任校長;35年10月改制為「臺灣省立工學院」,仍由王石安博士擔任校長。彼時僅有成功校區,39年增購勝利校區。41年2月,由秦大鈞博士接任校長。\\
45年8月,本校改制為「臺灣省立成功大學」,仍由秦大鈞博士擔任校長;同時增設文理學院及商學院。46年8月,由閻振興博士接任校長。54年1月,由羅雲平博士接任校長。55年增購光復校區。58年10月,將文理學院分為文學院及理學院。60年8月,改制為「國立成功大學」,並由倪超博士接任校長;同年增購建國校區。67年8月,由王唯農博士接任校長。\\
69年8月,夏漢民博士接任校長;同年將商學院更名為管理學院。72年8月,增設醫學院,並增購自強校區及敬業校區。74年增購力行校區部分校地。76年增購歸仁校區,設置航空太空實驗場。77年6月本校醫學院附設醫院正式營運。77年8月,馬哲儒博士接任校長。80年增購自強校區北半部。82年陸續增購台南市「文大五」用地,闢為本校安南校區。\\
83年8月,吳京院士接任校長。吳校長於85年6月入閣擔任教育部長,由副校長黃定加博士代理校長。86年2月,翁政義博士接任校長。86年8月增設社會科學院。88年完成收購安南校區全部校地;同年6月增購原陸軍八○四醫院用地(91年6月完成撥用手續)。翁校長於89年5月出任國科會主委,由副校長翁鴻山博士代理校長。90年2月,高強博士接任校長。92年8月,增設電機資訊學院、規劃與設計學院。94年7月,本校配合國軍斗六醫院精實案,接管該院營運權,設置為雲林縣斗六校區,並改制為本校醫學院附設醫院斗六分院。94年8月,增設生物科學與科技學院。94年10月,本校得到教育部的肯定,獲選為「發展國際一流大學及頂尖研究中心計畫」的兩所重點大學之一。\\
96年2月,賴明詔院士接任校長,97年2月教育部公佈「發展國際一流大學及頂尖研究中心計畫」第二梯次的審議結果,本校繼續獲得教育部的肯定與補助,積極朝國際一流大學的目標邁進。97年10月增購歸仁校區北側台糖土地(正式登記為本校管有)。100年2月,黃煌煇博士接任校長。100年4月本校獲得教育部第二期頂尖大學計畫補助,持續朝國際一流大學的目標邁進。104年2月,蘇慧貞博士接任校長。

% ------------------------------------------------

\newpage
\StartSection{中文內容 (段落 + 強制斷行)}
用來看內容, 符號, 段距, 字元之間的距離等東西\\

本校創校於西元1931年(昭和6年,民國20年)1月15日,原名為「臺南高等工業學校」;1944年(昭和19年,民國33年)改稱為「臺南工業專門學校」。民國34年臺灣光復。本校於民國35年2月改制為「臺灣省立臺南工業專科學校」,由王石安博士擔任校長;35年10月改制為「臺灣省立工學院」,仍由王石安博士擔任校長。彼時僅有成功校區,39年增購勝利校區。41年2月,由秦大鈞博士接任校長。\\

45年8月,本校改制為「臺灣省立成功大學」,仍由秦大鈞博士擔任校長;同時增設文理學院及商學院。46年8月,由閻振興博士接任校長。54年1月,由羅雲平博士接任校長。55年增購光復校區。58年10月,將文理學院分為文學院及理學院。60年8月,改制為「國立成功大學」,並由倪超博士接任校長;同年增購建國校區。67年8月,由王唯農博士接任校長。\\

69年8月,夏漢民博士接任校長;同年將商學院更名為管理學院。72年8月,增設醫學院,並增購自強校區及敬業校區。74年增購力行校區部分校地。76年增購歸仁校區,設置航空太空實驗場。77年6月本校醫學院附設醫院正式營運。77年8月,馬哲儒博士接任校長。80年增購自強校區北半部。82年陸續增購台南市「文大五」用地,闢為本校安南校區。\\

83年8月,吳京院士接任校長。吳校長於85年6月入閣擔任教育部長,由副校長黃定加博士代理校長。86年2月,翁政義博士接任校長。86年8月增設社會科學院。88年完成收購安南校區全部校地;同年6月增購原陸軍八○四醫院用地(91年6月完成撥用手續)。翁校長於89年5月出任國科會主委,由副校長翁鴻山博士代理校長。90年2月,高強博士接任校長。92年8月,增設電機資訊學院、規劃與設計學院。94年7月,本校配合國軍斗六醫院精實案,接管該院營運權,設置為雲林縣斗六校區,並改制為本校醫學院附設醫院斗六分院。94年8月,增設生物科學與科技學院。94年10月,本校得到教育部的肯定,獲選為「發展國際一流大學及頂尖研究中心計畫」的兩所重點大學之一。\\

96年2月,賴明詔院士接任校長,97年2月教育部公佈「發展國際一流大學及頂尖研究中心計畫」第二梯次的審議結果,本校繼續獲得教育部的肯定與補助,積極朝國際一流大學的目標邁進。97年10月增購歸仁校區北側台糖土地(正式登記為本校管有)。100年2月,黃煌煇博士接任校長。100年4月本校獲得教育部第二期頂尖大學計畫補助,持續朝國際一流大學的目標邁進。104年2月,蘇慧貞博士接任校長。

% ------------------------------------------------


% ------------------------------------------------
\UseDefaultLinesSpacing
% ------------------------------------------------

% 混合的中英文內容
%% ------------------------------------------------

\newpage
\StartSection{混合的中英文內容 (只用段落來分段)}
用來看內容, 符號, 段距, 字元之間的距離等東西\\

Google公司(英語:Google Inc.; 中文:穀歌[3]、穀歌[4]、科高[5]), 是一家美國的跨國科技企業, 業務範圍涵蓋互聯網搜索、雲計算、廣告技術等領域, 開發並提供大量基於互聯網的產品與服務[6], 其主要利潤來自於AdWords等廣告服務[7][8].

\begin{description}
\item [Google] Google由在斯坦福大學攻讀理工博士的拉裡•佩奇和謝爾蓋•布林共同創建, 因此兩人也被稱為``Google Guys''[9][10][11]. 1998年9月4日, Google以私營公司的形式創立, 目的是設計並管理互聯網搜尋引擎``Google搜索''. 2004年8月19日, Google公司在納斯達克上市, 後來被稱為``三駕馬車''的公司兩位共同創始人與出任首席執行官的埃裡克•施密特在此時承諾:共同在Google工作至少二十年, 即至2024年止[12].

\item [Google]\hfill\\ Google由在斯坦福大學攻讀理工博士的拉裡•佩奇和謝爾蓋•布林共同創建, 因此兩人也被稱為``Google Guys''[9][10][11]. 1998年9月4日, Google以私營公司的形式創立, 目的是設計並管理互聯網搜尋引擎``Google搜索''. 2004年8月19日, Google公司在納斯達克上市, 後來被稱為``三駕馬車''的公司兩位共同創始人與出任首席執行官的埃裡克•施密特在此時承諾:共同在Google工作至少二十年, 即至2024年止[12].

\item Google的宗旨是``整合全球範圍的資訊, 使人人皆可訪問並從中受益''(To organize the world's information and make it universally accessible and useful)[13]; 而非正式的口號則為``不作惡''(Don't be evil), 由工程師阿米特•派特爾(Amit Patel)所創[14], 並得到了保羅•布赫海特的支持[15][16]. Google公司的總部稱為``Googleplex'', 位於美國加州聖克拉拉縣的山景城. 2011年4月, 佩奇接替施密特擔任首席執行官[17].

\item 在2015年8月, Google進行宣佈資產重組. 重組後, Google劃歸新成立的Alphabet底下. 同時, 此舉把Google旗下的核心搜索和廣告業務與Google無人車等新興業務分離開來[18].
\end{description}

據估計, Google在全世界的資料中心內運營著上百萬台的伺服器, [19]每天處理數以億計的搜索請求[20]和約二十四PB使用者生成的資料. [21][22][23][24] Google自創立起開始的快速成長同時也帶動了一系列的產品研發、並購事項與合作關係, 而不僅僅是公司核心的網路搜索業務. Google公司提供豐富的線上軟體服務, 如雲端硬碟、Gmail電子郵件, 包括Orkut、Google Buzz以及Google+在內的社交網路服務. Google的產品同時也以應用軟體的形式進入使用者桌面, 例如Google Chrome網頁流覽器、Picasa圖片整理與編輯軟體、Google Talk即時通訊工具等. 另外, Google還進行了移動設備的Android作業系統以及Google Chrome OS作業系統的開發. [25]

資訊分析網站Alexa資料顯示, Google的主功能變數名稱google.com是全世界訪問量最高的網站, Google搜索在其他國家或地區域名下的多個網站(google.co.in、google.de、google.com.hk等等), 及旗下的YouTube、Blogger、Orkut等的訪問量都在前一百名之內. [26]其中, 社交網路服務Orkut於2014年9月關閉. [27]

Facebook(原本稱作thefacebook)是一家位於美國加州聖馬刁郡門洛派克市的線上社交網路服務網站. 其名稱的靈感來自美國高中提供給學生包含照片和聯絡資料的通訊錄(或稱花名冊)暱稱「face book」[6][7].

除了文字訊息之外, 使用者可傳送圖片、影片和聲音媒體訊息(現在也可以傳送其他檔案類型如.doc,.docx,.xls,.xlsx等, 但是.exe可能會被禁止傳送)給其他使用者, 以及透過整合的地圖功能分享使用者的所在位置. Facebook是在2004年2月4日由馬克•紮克伯格與他的哈佛大學室友們所創立[8]. Facebook的會員最初只限於哈佛學生加入, 但後來逐漸擴展到其他在波士頓區域的同學也能使用, 包括一些常春藤名校、MIT、紐約大學、史丹福大學等. 接著逐漸支援讓其他大學和高中學生加入, 並在最後開放給任何13歲或以上的人使用.  現在Facebook允許任何聲明自己年滿13歲的使用者註冊[9].

使用者必須註冊才能使用Facebook, 註冊後他們可以創建個人檔案、將其他使用者加為好友、傳遞訊息, 並在其他使用者更新個人檔案時獲得自動通知. 此外使用者也可以加入有相同興趣的群組, 這些群組依據工作地點、學校或其他特性分類. 使用者亦可將朋友分別加入不同的列表中管理, 例如「同事」或「摯友」等. 截至2012年9月, Facebook內已有超過十幾億個活躍使用者[10], 其中約有9\%的不實使用者[11]. 截至2012年, Facebook每年共產生180拍位元組(PB)的資料, 並以每24小時0.5拍位元元組的速度增加[12]. 統計顯示, Facebook上每天上傳3億5千萬張圖片. [13]

Facebook創始人馬克•紮克伯格是世界上最著名的CEO之一. 而馬克•紮克伯格曾經的朋友與商業合作夥伴愛德華多•薩維林在新加坡亦十分知名[14].

Google公司(英語:Google Inc.; 中文:穀歌[3]、穀歌[4]、科高[5]), 是一家美國的跨國科技企業, 業務範圍涵蓋互聯網搜索、雲計算、廣告技術等領域, 開發並提供大量基於互聯網的產品與服務[6], 其主要利潤來自於AdWords等廣告服務[7][8].

\begin{itemize}
\item Google由在斯坦福大學攻讀理工博士的拉裡•佩奇和謝爾蓋•布林共同創建, 因此兩人也被稱為``Google Guys''[9][10][11]. 1998年9月4日, Google以私營公司的形式創立, 目的是設計並管理互聯網搜尋引擎``Google搜索''. 2004年8月19日, Google公司在納斯達克上市, 後來被稱為``三駕馬車''的公司兩位共同創始人與出任首席執行官的埃裡克•施密特在此時承諾:共同在Google工作至少二十年, 即至2024年止[12].

\item Google的宗旨是``整合全球範圍的資訊, 使人人皆可訪問並從中受益''(To organize the world's information and make it universally accessible and useful)[13];
\begin{itemize}
\item Google的宗旨是``整合全球範圍的資訊, 使人人皆可訪問並從中受益''(To organize the world's information and make it universally accessible and useful)[13]; 而非正式的口號則為``不作惡''(Don't be evil), 由工程師阿米特•派特爾(Amit Patel)所創[14], 並得到了保羅•布赫海特的支持[15][16]. Google公司的總部稱為``Googleplex'', 位於美國加州聖克拉拉縣的山景城. 2011年4月, 佩奇接替施密特擔任首席執行官[17].
\item Google的宗旨是``整合全球範圍的資訊, 使人人皆可訪問並從中受益''(To organize the world's information and make it universally accessible and useful)[13].
\item Google公司的總部稱為``Googleplex'', 位於美國加州聖克拉拉縣的山景城. 2011年4月, 佩奇接替施密特擔任首席執行官[17].
\end{itemize}

\item 在2015年8月, Google進行宣佈資產重組. 重組後, Google劃歸新成立的Alphabet底下. 同時, 此舉把Google旗下的核心搜索和廣告業務與Google無人車等新興業務分離開來[18].
\end{itemize}

據估計, Google在全世界的資料中心內運營著上百萬台的伺服器, [19]每天處理數以億計的搜索請求[20]和約二十四PB使用者生成的資料. [21][22][23][24] Google自創立起開始的快速成長同時也帶動了一系列的產品研發、並購事項與合作關係, 而不僅僅是公司核心的網路搜索業務. Google公司提供豐富的線上軟體服務, 如雲端硬碟、Gmail電子郵件, 包括Orkut、Google Buzz以及Google+在內的社交網路服務. Google的產品同時也以應用軟體的形式進入使用者桌面, 例如Google Chrome網頁流覽器、Picasa圖片整理與編輯軟體、Google Talk即時通訊工具等. 另外, Google還進行了移動設備的Android作業系統以及Google Chrome OS作業系統的開發. [25]

資訊分析網站Alexa資料顯示, Google的主功能變數名稱google.com是全世界訪問量最高的網站, Google搜索在其他國家或地區域名下的多個網站(google.co.in、google.de、google.com.hk等等), 及旗下的YouTube、Blogger、Orkut等的訪問量都在前一百名之內. [26]其中, 社交網路服務Orkut於2014年9月關閉. [27]

Facebook(原本稱作thefacebook)是一家位於美國加州聖馬刁郡門洛派克市的線上社交網路服務網站. 其名稱的靈感來自美國高中提供給學生包含照片和聯絡資料的通訊錄(或稱花名冊)暱稱「face book」[6][7].
\begin{enumerate}
\item 除了文字訊息之外, 使用者可傳送圖片、影片和聲音媒體訊息(現在也可以傳送其他檔案類型如.doc,.docx,.xls,.xlsx等, 但是.exe可能會被禁止傳送)給其他使用者, 以及透過整合的地圖功能分享使用者的所在位置.

\item Facebook是在2004年2月4日由馬克•紮克伯格與他的哈佛大學室友們所創立[8]. Facebook的會員最初只限於哈佛學生加入, 但後來逐漸擴展到其他在波士頓區域的同學也能使用, 包括一些常春藤名校、MIT、紐約大學、史丹福大學等.

\item 接著逐漸支援讓其他大學和高中學生加入, 並在最後開放給任何13歲或以上的人使用.  現在Facebook允許任何聲明自己年滿13歲的使用者註冊[9].
\end{enumerate}
使用者必須註冊才能使用Facebook, 註冊後他們可以創建個人檔案、將其他使用者加為好友、傳遞訊息, 並在其他使用者更新個人檔案時獲得自動通知. 此外使用者也可以加入有相同興趣的群組, 這些群組依據工作地點、學校或其他特性分類. 使用者亦可將朋友分別加入不同的列表中管理, 例如「同事」或「摯友」等. 截至2012年9月, Facebook內已有超過十幾億個活躍使用者[10], 其中約有9\%的不實使用者[11]. 截至2012年, Facebook每年共產生180拍位元組(PB)的資料, 並以每24小時0.5拍位元元組的速度增加[12]. 統計顯示, Facebook上每天上傳3億5千萬張圖片. [13]

Facebook創始人馬克•紮克伯格是世界上最著名的CEO之一. 而馬克•紮克伯格曾經的朋友與商業合作夥伴愛德華多•薩維林在新加坡亦十分知名[14].

% ------------------------------------------------


% ------------------------------------------------
\UseDefaultLinesSpacing
% ------------------------------------------------

% 表格
% ------------------------------------------------
\StartChapter{Insert Table}{chapter:how-to:write:image}
% ------------------------------------------------
\section{介紹}

% http://www.tablesgenerator.com/

% ------------------------------------------------
\EndChapter
% ------------------------------------------------


% ------------------------------------------------
\EndChapter
% ------------------------------------------------


% ------------------------------------------------
