%
% This file is part of ncku-thesis-templete.
%
% ncku-thesis-templete is distributed in the hope that it will be useful,
% you can redistribute it and/or modify
% it under the terms of the Attribution-NonCommercial-ShareAlike
% 4.0 International.
%
% You should have received a copy of the
% Attribution-NonCommercial-ShareAlike 4.0 International
% along with ncku-thesis-templete.
%
% If not, see <http://creativecommons.org/licenses/by-nc-sa/4.0/legalcode.txt>.
%

% Some common helper function

% ----------------------------------------------------------------------------

% Some helper functions

\newcommand{\GetMonthInEng}[1]
{
  \ifthenelse{\equal{#1}{1}}{January}{}
  \ifthenelse{\equal{#1}{2}}{February}{}
  \ifthenelse{\equal{#1}{3}}{March}{}
  \ifthenelse{\equal{#1}{4}}{April}{}
  \ifthenelse{\equal{#1}{5}}{May}{}
  \ifthenelse{\equal{#1}{6}}{June}{}
  \ifthenelse{\equal{#1}{7}}{July}{}
  \ifthenelse{\equal{#1}{8}}{August}{}
  \ifthenelse{\equal{#1}{9}}{September}{}
  \ifthenelse{\equal{#1}{10}}{October}{}
  \ifthenelse{\equal{#1}{11}}{November}{}
  \ifthenelse{\equal{#1}{12}}{December}{}
} % End of \newcommand{}

% 計算出台灣民國幾年
% Get the year using Taiwans' year
\newcommand{\SetOralTaiwanYear}[1]
{
  \FPeval{\OralTaiwanYearResult}{clip(#1 - 1911)}
} % End of \newcommand{}

% 計算出台灣民國幾年
% Get the year using Taiwans' year
\newcommand{\SetThesisTaiwanYear}[1]
{
  \FPeval{\ThesisTaiwanYearResult}{clip(#1 - 1911)}
} % End of \newcommand{}

% ----------------------------------------------------------------------------

% --- 關鍵字 Keyword ---
\newcommand{\KeywordsA}{} % Default
\newcommand{\KeywordsB}{} % Default
\newcommand{\KeywordsC}{} % Default
\newcommand{\KeywordsD}{} % Default
\newcommand{\KeywordsE}{} % Default
\newcommand{\SetKeywordsA}[1]{\renewcommand{\KeywordsA}{#1}}
\newcommand{\SetKeywordsB}[1]{\renewcommand{\KeywordsB}{#1}}
\newcommand{\SetKeywordsC}[1]{\renewcommand{\KeywordsC}{#1}}
\newcommand{\SetKeywordsD}[1]{\renewcommand{\KeywordsD}{#1}}
\newcommand{\SetKeywordsE}[1]{\renewcommand{\KeywordsE}{#1}}
\newcommand{\GetKeywordsA}[0]{\KeywordsA}
\newcommand{\GetKeywordsB}[0]{\KeywordsB}
\newcommand{\GetKeywordsC}[0]{\KeywordsC}
\newcommand{\GetKeywordsD}[0]{\KeywordsD}
\newcommand{\GetKeywordsE}[0]{\KeywordsE}
\DeclareDocumentCommand{\SetKeywords}{
  m G{\empty} G{\empty} G{\empty} G{\empty}}
{
  \SetKeywordsA{#1}
  \ifthenelse{\equal{#2}{\empty}}{}{\SetKeywordsB{#2}}
  \ifthenelse{\equal{#3}{\empty}}{}{\SetKeywordsC{#3}}
  \ifthenelse{\equal{#4}{\empty}}{}{\SetKeywordsD{#4}}
  \ifthenelse{\equal{#5}{\empty}}{}{\SetKeywordsE{#5}}
} % End of \newcommand{}

% ----------------------------------------------------------------------------

% In the minimal example below the macro \modulo{<a>}{<b>} stores the result of <a> mod <b> in the macro \result
\newcommand{\modulo}[2]{%
  \FPeval{\result}{trunc(#1-(#2*trunc(#1/#2,0)),0)}%
}

% ----------------------------------------------------------------------------

