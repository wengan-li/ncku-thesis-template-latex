
% Some helper function about insert image

% 用\begin{figure} .. \end{figure}
% 可能會出現問題
% http://www.tex.ac.uk/cgi-bin/texfaq2html?label=ouparmd

\DeclareDocumentCommand{\SetImageCaption}{G{\empty}}
{
  \ifthenelse{\equal{#1}{\empty}}
  {}
  {\IfNoValueF{#1}{\caption{#1}}}
} % End of \DeclareDocumentCommand{}

\DeclareDocumentCommand{\SetImageLabel}{G{\empty}}
{
  \ifthenelse{\equal{#1}{\empty}}
  {}
  {\IfNoValueF{#1}{\label{#1}}}
} % End of \DeclareDocumentCommand{}

% -----------------------------------------------------------------

\pgfkeys
{
  /InsertImage/.is family, /InsertImage,
  default/.style = 
  {
    scale = 1.0,
    angle = 0,
    caption = \empty,
    label = \empty,
    align = \empty,
  },
  scale/.estore in = \InsertImageValueScale,
  angle/.estore in = \InsertImageValueAngle,
  caption/.estore in = \InsertImageValueCaption,
  label/.estore in = \InsertImageValueLabel,
  align/.estore in = \InsertImageValueAlign,
} % End of \pgfkeys{}

% Insert a single image
\newcommand{\InsertImage}[2][\empty]
{
  % Parse the input
  \pgfkeys{/InsertImage, default, #1}
  %
  % Debug if needed
%  Scale: \InsertImageValueScale\\
%  Angle: \InsertImageValueAngle\\
%  Caption: \InsertImageValueCaption\\
%  Label: \InsertImageValueLabel\\
%  Align: \InsertImageValueAlign\\
%  \ifthenelse{\equal{#2}{\empty}}
%    {Path: -EMPTY-\\}{Path: #2\\}
%  \IfNoValueF{#2}{Path: #2\\}
  %
  %
  % Check any caption first
  \ifthenelse{\equal{\InsertImageValueCaption}{\empty}}
  { %
    %
    %No Caption
    %
    % Align
    \ifthenelse{\equal{\InsertImageValueAlign}{center}}
      {\begin{center}}{}
    % Insert image
    \includegraphics
      [scale=\InsertImageValueScale,
        angle=\InsertImageValueAngle]{#2}
    % Align
    \ifthenelse{\equal{\InsertImageValueAlign}{center}}
      {\end{center}}{}
  } % End of if{}
  { %
    %
    %Have Caption
    %
    \begin{figure*}[h]
      % Do center if needed
      \ifthenelse{\equal{\InsertImageValueAlign}{center}}
        {\center}{}
      % Insert image
      \includegraphics
        [scale=\InsertImageValueScale,
          angle=\InsertImageValueAngle]{#2}
      % Set Caption
      \SetImageCaption{\InsertImageValueCaption}
      % Set Label
      \ifthenelse{\equal{\InsertImageValueCaption}{\empty}}
        {}{\SetImageLabel{\InsertImageValueLabel}}
    \end{figure*}
  } % End of else{}
} % End of \newcommand{}

% Insert a single image, but place it in center
\newcommand{\InsertCenterImage}[2][\empty]
{
  \InsertImage[#1, align=center]{#2}
} % End of \newcommand{}

% -----------------------------------------------------------------

% For multi images
\newcommand\MultiImagesPerRow{1}
\newcommand{\SetMultiImagesPerRow}[1]
  {\renewcommand{\MultiImagesPerRow}{#1}}
\newcommand{\GetMultiImagesPerRow}[0]{\MultiImagesPerRow}

%add desired spacing between images, e. g. ~, \quad, \qquad etc.
%(or a blank line to force the subfigure onto a new line)
\newcommand{\SubfigureBreakSpaceLine}[1]
{
  \ifthenelse{\equal{\intcalcMod{
    \GetMultiImageId - 1}{\GetMultiImagesPerRow}}{0}}%
  {%
    % Echo blank line
    %

  } % End of if{}
  {%
    ~
  } % End of else{}
} % End of \newcommand{}

% Confingure the main image

% Low-level insert image

\newcommand{\InsertSubfigureImage}[5]
{%
  \begin{subfigure}[b]{\WidthOfImagePerRow\textwidth}%
    \center
    \includegraphics[scale=#1, angle=#5]{#2}
    \SetImageCaption{#3}
    \SetImageLabel{#4}
  \end{subfigure}%
} % End of \newcommand{}

\newcommand{\SetWidthOfImagePerRow}
{
  %
%  GetMultiImagesPerRow: \GetMultiImagesPerRow\\
%  GetMultiImageTotalValue: \GetMultiImageTotalValue\\
  %
  \ifthenelse{\equal{\GetMultiImagesPerRow}{1}}
  {
    %perrow = 1
    \FPeval{\WidthOfImagePerRow}{1.0}
  } % End of if{}
  {
    %perrow > 1
    %
    % Flag
    \provideboolean{WidthOfImagePerRowTmpIsDone}
    \setboolean{WidthOfImagePerRowTmpIsDone}{false}
    %
    % Calc remain
    \FPeval{\TmpRemain}{
      clip((\GetMultiImageTotalValue) - (\GetMultiImageId))}
    %TmpRemain: \TmpRemain\\
    %
    % Get mod
    \modulo{\GetMultiImageTotalValue}{\GetMultiImagesPerRow}
    %mod: \result\\
    %
    \FPiflt{\TmpRemain}{\GetMultiImagesPerRow}
    \else
      % Gt
      \setboolean{WidthOfImagePerRowTmpIsDone}{true}
      %
      % Default
      \FPeval{
        \WidthOfImagePerRow}{
          clip(clip(1 / (\GetMultiImagesPerRow)) - 0.1)}
    \fi % End of \FPiflt
    %
    \ifthenelse{\boolean{WidthOfImagePerRowTmpIsDone}}
    {}
    {
      % Lt
      \ifthenelse{\result = 0}
      {
        % Last row
        %
        \FPeval{
          \WidthOfImagePerRow}{
            clip(clip(1 / (\GetMultiImagesPerRow)) - 0.1)}
      } % End of if{}
      {
        % Diff row
        %
        \FPifeq{\TmpRemain}{0}
          \FPeval{\WidthOfImagePerRow}{1.0}
        \else
          \FPeval{
            \WidthOfImagePerRow}{
              clip(clip(1 / (\GetMultiImagesPerRow)) - 0.1)}
        \fi % End of \FPiflt
      } % End of else{}
    } % End of else{}
  } % End of else{}
  %
%  \WidthOfImagePerRow
  %
} % End of \newcommand{}

\newcommand\MultiImageTotalValue{0}
\newcommand{\SetMultiImageTotalValue}[1]
  {\renewcommand{\MultiImageTotalValue}{#1}}
\newcommand{\GetMultiImageTotalValue}[0]{\MultiImageTotalValue}

\newcommand\MultiImageId{0}
\newcommand{\SetMultiImageId}[1]
  {\renewcommand{\MultiImageId}{#1}}
\newcommand{\GetMultiImageId}[0]{\MultiImageId}

% Insert multi-image
% Arg: 1st: Table configure
%      2~9th: Images (Max 8 Images)
%
% ---------------------------------------
% 使用 \InsertMultiImageLL#2
% 是跟 \InsertMultiImageLL{#2} 不一樣的
% 直接連接#2是指把#2整個當成function的所有參數
% 而{#2}是指把#2當成單一個參數傳給function
% ---------------------------------------
%
\newcommand\WidthOfImagePerRow{0}

\pgfkeys
{
  /InsertMultiImagesTest/.is family, /InsertMultiImagesTest,
  default/.style = 
  {
    perrow = 1,
    caption = \empty,
    label = \empty,
  },
  perrow/.estore in = \InsertMultiImagesTestValueImagePerRow,
  caption/.estore in = \InsertMultiImagesTestValueCaption,
  label/.estore in = \InsertMultiImagesTestValueLabel,
} % End of \pgfkeys{}

%\newcommand{\InsertMultiImagesTest}[2][\empty]
%\newcommand{\InsertMultiImagesTest}[2][\empty]
\DeclareDocumentCommand{\InsertMultiImagesTest}{
  +O{\empty} +m +G{\empty} +G{\empty} +G{\empty}
  +G{\empty} +G{\empty} +G{\empty} +G{\empty}}
{
  % Parse the input
  \pgfkeys{/InsertMultiImagesTest, default, #1}
%  perrow: \InsertMultiImagesTestValueImagePerRow\\
%  caption: \InsertMultiImagesTestValueCaption\\
%  label: \InsertMultiImagesTestValueLabel\\
  %
  %Set Multi Image Total
  \ifthenelse{\equal{#2}{\empty}}{}{\SetMultiImageTotalValue{1}}
  \ifthenelse{\equal{#3}{\empty}}{}{\SetMultiImageTotalValue{2}}
  \ifthenelse{\equal{#4}{\empty}}{}{\SetMultiImageTotalValue{3}}
  \ifthenelse{\equal{#5}{\empty}}{}{\SetMultiImageTotalValue{4}}
  \ifthenelse{\equal{#6}{\empty}}{}{\SetMultiImageTotalValue{5}}
  \ifthenelse{\equal{#7}{\empty}}{}{\SetMultiImageTotalValue{6}}
  \ifthenelse{\equal{#8}{\empty}}{}{\SetMultiImageTotalValue{7}}
  \ifthenelse{\equal{#9}{\empty}}{}{\SetMultiImageTotalValue{8}}
  %
  % Set per row
  \FPeval{\MultiImagesPerRow}{clip(
    \InsertMultiImagesTestValueImagePerRow)} %
  %
  \begin{figure*}[h]
    %
    % Image 1
    %\#2: P #2 P\\
    \ifthenelse{\equal{#2}{\empty}}
      {}
      {
        \SetMultiImageId{1}
        \SubfigureBreakSpaceLine{1}
        \InsertMultiImagesTestSubfigure#2
      } % End of else{}
    %
    % Image 2
    %\#3: P #3 P\\
    \ifthenelse{\equal{#3}{\empty}}
      {}
      {
        \SetMultiImageId{2}
        \SubfigureBreakSpaceLine{2}
        \InsertMultiImagesTestSubfigure#3
      } % End of else{}
    %
    % Image 3
    %\#4: P #4 P\\
    \ifthenelse{\equal{#4}{\empty}}
      {}
      {
        \SetMultiImageId{3}
        \SubfigureBreakSpaceLine{3}
        \InsertMultiImagesTestSubfigure#4
      } % End of else{}
    %
    % Image 4
    %\#5: P #5 P\\
    \ifthenelse{\equal{#5}{\empty}}
      {}
      {
        \SetMultiImageId{4}
        \SubfigureBreakSpaceLine{4}
        \InsertMultiImagesTestSubfigure#5
      } % End of else{}
    %
    % Image 5
    %\#6: P #6 P\\
    \ifthenelse{\equal{#6}{\empty}}
      {}
      {
        \SetMultiImageId{5}
        \SubfigureBreakSpaceLine{5}
        \InsertMultiImagesTestSubfigure#6
      } % End of else{}
    %
    % Image 6
    %\#7: P #7 P\\
    \ifthenelse{\equal{#7}{\empty}}
      {}
      {
        \SetMultiImageId{6}
        \SubfigureBreakSpaceLine{6}
        \InsertMultiImagesTestSubfigure#7
      } % End of else{}
    %
    % Image 7
    %\#8: P #8 P\\
    \ifthenelse{\equal{#8}{\empty}}
      {}
      {
        \SetMultiImageId{7}
        \SubfigureBreakSpaceLine{7}
        \InsertMultiImagesTestSubfigure#8
      } % End of else{}
    %
    % Image 8
    %\#9: P #9 P\\
    \ifthenelse{\equal{#9}{\empty}}
      {}
      {
        \SetMultiImageId{8}
        \SubfigureBreakSpaceLine{8}
        \InsertMultiImagesTestSubfigure#9
      } % End of else{}
    %
    % Set Caption
    \SetImageCaption{\InsertMultiImagesTestValueCaption}
    % Set Label
    \ifthenelse{\equal{\InsertMultiImagesTestValueCaption}{\empty}}
      {}{\SetImageLabel{\InsertMultiImagesTestValueLabel}}
    %
  \end{figure*}
} % End of \newcommand{}

\pgfkeys
{
  /InsertMultiImagesTestSubfigure/.is family, /InsertMultiImagesTestSubfigure,
  default/.style = 
  {
    scale = 1.0,
    angle = 0,
    caption = \empty,
    label = \empty,
    align = \empty,
  },
  scale/.estore in = \InsertMultiImagesTestSubfigureValueScale,
  angle/.estore in = \InsertMultiImagesTestSubfigureValueAngle,
  caption/.estore in = \InsertMultiImagesTestSubfigureValueCaption,
  label/.estore in = \InsertMultiImagesTestSubfigureValueLabel,
  align/.estore in = \InsertMultiImagesTestSubfigureValueAlign,
} % End of \pgfkeys{}

\newcommand{\InsertMultiImagesTestSubfigure}[2][\empty]
{
  % Set width
  \SetWidthOfImagePerRow
  %
  % Parse the input
  \pgfkeys{/InsertMultiImagesTestSubfigure, default, #1}
  %
  \WidthOfImagePerRow
  \begin{subfigure}[b]{\WidthOfImagePerRow\textwidth}
    %
    \center
    %
    \includegraphics[
      scale=\InsertMultiImagesTestSubfigureValueScale,
      angle=\InsertMultiImagesTestSubfigureValueAngle]
      {#2}
    %
    % Set Caption
    \SetImageCaption{\InsertMultiImagesTestSubfigureValueCaption}
    % Set Label
    \ifthenelse{\equal{\InsertMultiImagesTestSubfigureValueCaption}{\empty}}
      {}{\SetImageLabel{\InsertMultiImagesTestSubfigureValueLabel}}
  \end{subfigure}%
} % End of \newcommand{}

% ----------------------------------------------------------------------------

% http://tex.stackexchange.com/questions/34312/how-to-create-a-command-with-key-values
