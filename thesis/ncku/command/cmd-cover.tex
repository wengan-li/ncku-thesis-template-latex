%
% This file is part of ncku-thesis-template.
%
% ncku-thesis-template is distributed in the hope of usefuling to someone,
% you can redistribute it and/or modify
% it under the terms of the Attribution-NonCommercial-ShareAlike
% 4.0 International.
%
% You should have received a copy of the
% Attribution-NonCommercial-ShareAlike 4.0 International
% along with ncku-thesis-template.
%
% If not, see <http://creativecommons.org/licenses/by-nc-sa/4.0/legalcode.txt>.
%

% Some helper function use in cover

% ----------------------------------------------------------------------------
\newcommand{\StartCover}
{
  \StartNewPage
  %
  % 設定使用 無頁碼
  \thispagestyle{empty}
  %
  \EnableCoverPageStyle
  %
  % Set the line spacing to single for the titles (to compress the lines)
  \renewcommand{\baselinestretch}{1}   %行距 1 倍
  %
  % Aligned to the center of the page
  \begin{center}
} % End of \newcommand{}

\newcommand{\EndCover}
{
  % End of alignment
  \end{center}
  \DisableCoverPageStyle
  \EndOfPage
} % End of \newcommand{}
% ----------------------------------------------------------------------------

% --- University name 學校名字 ---
% 基本上是寫死, 但是如果是別校的人, 可直接使用\SetSchoolName來修改
\newcommand\univCname{國立成功大學}           % Default
\newcommand\univEname{National Cheng Kung University} % Default
\newcommand{\SetSchoolChiName}[1]{\renewcommand{\univCname}{#1}}
\newcommand{\SetSchoolEngName}[1]{\renewcommand{\univEname}{#1}}
\newcommand{\SetSchoolName}[2]
{
  \SetSchoolChiName{#1}
  \SetSchoolEngName{#2}
} % End of \newcommand{}

\newcommand{\GetSchoolChiName}{\univCname}
\newcommand{\GetSchoolEngName}{\univEname}
% ----------------------------------------------------------------------------

% --- Chinese / English title 中英文論文題目 ---
\newcommand{\cTitle}{Chinese Title Here} % Default
\newcommand{\eTitle}{English Title Here} % Default
\newcommand{\SetChiTitle}[1]{\renewcommand{\cTitle}{#1}}
\newcommand{\SetEngTitle}[1]{\renewcommand{\eTitle}{#1}}
\newcommand{\SetTitle}[2]
{
  \SetChiTitle{#1}
  \SetEngTitle{#2}
} % End of \newcommand{}

\newcommand{\GetChiTitle}{\cTitle}
\newcommand{\GetEngTitle}{\eTitle}
% ----------------------------------------------------------------------------

% --- User's name 使用者名字 ---
\newcommand{\myCname}{你的名字}     % Default
\newcommand{\myEname}{Your name}   % Default
\newcommand{\SetMyChiName}[1]{\renewcommand{\myCname}{#1}}
\newcommand{\SetMyEngName}[1]{\renewcommand{\myEname}{#1}}
\newcommand{\SetMyName}[2]
{
  \SetMyChiName{#1}
  \SetMyEngName{#2}
} % End of \newcommand{}

\newcommand{\GetAuthorChiName}{\myCname}
\newcommand{\GetAuthorEngName}{\myEname}

% ----------------------------------------------------------------------------

% --- Degree name 學位 ---
% thesis 是指論文的通稱
% dissertation 指的是博士的論文

% 碩士論文  Master's thesis
% 博士論文  Doctoral dissertation

\newcommand{\ValueDegreeMaster}{0}
\newcommand{\ValueDegreePhd}{1}
\newcommand{\FlagDegreeType}{\ValueDegreePhd} % Default
\newcommand{\GetFlagDegreeType}{\FlagDegreeType}
\newcommand{\SetFlagDegreeType}[1]{\renewcommand{\FlagDegreeType}{#1}}

\newcommand{\degreeCname}{碩士/博士} % Default
\newcommand{\degreeEname}{Master / Doctor} % Default
\newcommand{\degreeThesisEname}{Master's Thesis / Doctoral Dissertation} % Default

\newcommand{\GetChiDegree}{\degreeCname}
\newcommand{\GetEngDegree}{\degreeEname}
\newcommand{\GetEngDegreeThesis}{\degreeThesisEname}
\newcommand{\SetChiDegree}[1]{\renewcommand{\degreeCname}{#1}}
\newcommand{\SetEngDegree}[1]{\renewcommand{\degreeEname}{#1}}
\newcommand{\SetEngDegreeThesis}[1]{\renewcommand{\degreeThesisEname}{#1}}

\newcommand{\PhdDegree}
{
  \SetFlagDegreeType{\ValueDegreePhd}
  \SetChiDegree{博士}
  \SetEngDegree{Doctor}
  \SetEngDegreeThesis{Doctoral Dissertation}
} % End of \newcommand{}

\newcommand{\MasterDegree}
{
  \SetFlagDegreeType{\ValueDegreeMaster}
  \SetChiDegree{碩士}
  \SetEngDegree{Master}
  \SetEngDegreeThesis{Master's Thesis}
} % End of \newcommand{}

% ----------------------------------------------------------------------------

% --- Date 日期 ---

% --- 論文的日期 ---
\newcommand{\ThesisYear}{2014}  % Default
\newcommand{\ThesisMonth}{1}    % Default

\newcommand{\SetThesisDate}[2]{\SetThesisDate{#1}{#2}} % For backporting
\newcommand{\SetCoverDate}[2]
{
  \SetThesisTaiwanYear{#1}
  \renewcommand{\ThesisYear}{#1}
  \renewcommand{\ThesisMonth}{#2}
} % End of \newcommand{}

\newcommand{\GetThesisYear}{\ThesisYear}
\newcommand{\GetThesisYearInTaiwanYear}{\ThesisTaiwanYearResult}
\newcommand{\GetThesisMonth}{\ThesisMonth}
\newcommand{\GetThesisMonthInEng}{\GetMonthInEng{\ThesisMonth}}

% ---  口試的日期 ---
\newcommand{\OralChiYear}{101}      % Default
\newcommand{\OralChiMonth}{1}       % Default
\newcommand{\OralChiDay}{1}         % Default
\newcommand{\OralEngYear}{2014}     % Default
\newcommand{\OralEngMonth}{January} % Default
\newcommand{\OralEngDay}{1}         % Default

\newcommand{\GetOralChiYear}{\OralChiYear}
\newcommand{\GetOralYearInTaiwanYear}
{\SetThesisTaiwanYear{\OralEngYear}\ThesisTaiwanYearResult}
\newcommand{\GetOralChiMonth}{\OralChiMonth}
\newcommand{\GetOralChiDay}{\OralChiDay}
\newcommand{\GetOralEngYear}{\OralEngYear}
\newcommand{\GetOralEngMonth}{\OralEngMonth}
\newcommand{\GetOralEngDay}{\OralEngDay}

\newcommand{\SetOralChiDate}[3]
{
  \SetOralTaiwanYear{#1}
  \renewcommand{\OralChiYear}{\OralTaiwanYearResult}
  \renewcommand{\OralChiMonth}{#2}
  \renewcommand{\OralChiDay}{#3}
} % End of \newcommand{}

\newcommand{\SetOralEngDate}[3]
{
  \renewcommand{\OralEngYear}{#1}
  \renewcommand{\OralEngMonth}{\GetMonthInEng{#2}}
  \renewcommand{\OralEngDay}{#3}
} % End of \newcommand{}

\newcommand{\SetOralDate}[3]
{
  \SetOralChiDate{#1}{#2}{#3}
  \SetOralEngDate{#1}{#2}{#3}
} % End of \newcommand{}

% ----------------------------------------------------------------------------

% --- 學院 College, 系所 Department and Institute ---

% --------------------------- College ---------------------------
\newcommand{\collCname}{學院 C}
\newcommand{\collEname}{College of C}
\newcommand{\SetCollChiName}[1]{\renewcommand{\collCname}{#1}}
\newcommand{\SetCollEngName}[1]{\renewcommand{\collEname}{#1}}
\newcommand{\SetCollName}[2]
{
  \SetCollChiName{#1}
  \SetCollEngName{#2}
} % End of \newcommand{}

\newcommand{\GetCollChiName}{\collCname}
\newcommand{\GetCollEngName}{\collEname}

% --------------------------- Department ---------------------------
\newcommand{\deptCname}{A 系 / 所}
%\newcommand{\deptEname}{DeptA} % Short form of department
\newcommand{\fulldeptEname}{Department / Insitute A} % Full name of department
\newcommand{\SetDeptChiName}[1]{\renewcommand{\deptCname}{#1}}
%\newcommand{\SetDeptEngShortName}[1]{\renewcommand{\deptEname}{#1}}
\newcommand{\SetDeptEngName}[1]{\renewcommand{\fulldeptEname}{#1}}
\newcommand{\SetDeptName}[3]
{
  \SetDeptChiName{#1}
%  \SetDeptEngShortName{#2}
  \SetDeptEngName{#3}
} % End of \newcommand{}

\newcommand{\GetDeptChiName}{\deptCname}
\newcommand{\GetDeptEngName}{\fulldeptEname}

% ----------------------------------------------------------------------------

% --- 指導老師 Advisor(s) ---
% 在封面上預算了最多3位的空間
% 中文名字固定以 博士 結尾
% 英文名字固定以 Dr. 開頭

\newcommand{\advisorCnameA}{X}
\newcommand{\advisorEnameA}{X}
\newcommand{\advisorCnameB}{}
\newcommand{\advisorEnameB}{}
\newcommand{\advisorCnameC}{}
\newcommand{\advisorEnameC}{}

\newcommand{\GetAdvisorChiNameA}{\advisorCnameA}
\newcommand{\GetAdvisorEngNameA}{\advisorEnameA}
\newcommand{\GetAdvisorChiNameB}{\advisorCnameB}
\newcommand{\GetAdvisorEngNameB}{\advisorEnameB}
\newcommand{\GetAdvisorChiNameC}{\advisorCnameC}
\newcommand{\GetAdvisorEngNameC}{\advisorEnameC}

\newcommand{\SetAdvisorChiNameA}[1]{\renewcommand{\advisorCnameA}{#1}}
\newcommand{\SetAdvisorEngNameA}[1]{\renewcommand{\advisorEnameA}{#1}}
\newcommand{\SetAdvisorChiNameB}[1]{\renewcommand{\advisorCnameB}{#1}}
\newcommand{\SetAdvisorEngNameB}[1]{\renewcommand{\advisorEnameB}{#1}}
\newcommand{\SetAdvisorChiNameC}[1]{\renewcommand{\advisorCnameC}{#1}}
\newcommand{\SetAdvisorEngNameC}[1]{\renewcommand{\advisorEnameC}{#1}}

\newcommand{\SetAdvisorNameA}[2]
{
  \SetAdvisorChiNameA{#1}
  \SetAdvisorEngNameA{#2}
} % End of \newcommand{}

\newcommand{\SetAdvisorNameB}[2]
{
  \SetAdvisorChiNameB{#1}
  \SetAdvisorEngNameB{#2}
} % End of \newcommand{}

\newcommand{\SetAdvisorNameC}[2]
{
  \SetAdvisorChiNameC{#1}
  \SetAdvisorEngNameC{#2}
} % End of \newcommand{}

% ----------------------------------------------------------------------------

% Use to create cover
\newcommand{\CreateCover}%
{
  \begin{document}
  % ------------------------------------------------

% 請根據你的需求去選用中文或英文封面
%%%%%%%%%%%%%%%%%%%%%%%%%%%%%%%
%       NCKU cover 封面
%%%%%%%%%%%%%%%%%%%%%%%%%%%%%%%
%
\begin{titlepage}

% 判斷是否要浮水印?
\ifx\mywatermark\undefined 
  \thispagestyle{empty}  % 無頁碼、無 header (無浮水印)
\else
  \thispagestyle{EmptyWaterMarkPage} % 無頁碼、有浮水印
\fi

% aligned to the center of the page
\begin{center}
% font size (relative to 12 pt):
% \large (14pt) < \Large (18pt) < \LARGE (20pt) < \huge (24pt)< \Huge (24 pt)
%
\makebox[10cm][s]{\Huge\univCname}\\  %顯示中文校名
\vspace{1cm}
\makebox[8cm][s]{\Huge\deptCname}\\ %顯示中文系所名
\vspace{1cm}
\makebox[5cm][s]{\Huge\degreeCname 論文}\\ %顯示論文種類 (中文)
\vspace{1.5cm}
\vspace{2cm}
\vspace{2.5cm}

%
% Set the line spacing to single for the titles (to compress the lines)
\renewcommand{\baselinestretch}{1}   %行距 1 倍
%\large % it needs a font size changing command to be effective
\Large\cTitle\\  % 中文題目
%
\vspace{0.5cm}
%
\Large\eTitle\\ %英文題目

\vspace{3.5cm}

% \makebox is a text box with specified width;
% option s: stretch; option l: left aligned
% use \makebox to make sure
% 「研究生」 與「指導教授」occupy the same width
% Names are filled in a box with pre-defined width
% the left and right sides of 「:」occupy the same width (use \hspace{} to fill the short)
% to guarantee 「:」is at the center
% assume the width of a Chinese character is 1.2em
% 4.8em is determined by the length of the longest string "指導教授"
% 7.2em is determined by the length of the possibly longest name + title "歐陽明志博士"
\hspace{2.4em}%
\makebox[4.8em][s]{\Large 研究生}%
\makebox[1em][c]{\Large :}%
\makebox[7.2em][l]{\Large\myCname}\\  % 顯示作者中文名

\vspace{0.5cm}

\hspace{2.4em}%
\makebox[4.8em][s]{\Large 指導教授}%
\makebox[1em][c]{\Large :}%
\makebox[7.2em][l]{\Large\advisorCnameA}\\  %顯示指導教授A中文名
%
% 判斷是否有共同指導的教授 B
\ifx \advisorCnameB  \itsempty
\relax % 沒有 B 教授,所以不佔版面,不印任何空白
\else
\vspace{0.1cm}
% 共同指導的教授 B
\hspace{2.4em}%
\makebox[4.8em][s]{}%
\makebox[1em][c]{}%
\makebox[7.2em][l]{\Large\advisorCnameB}\\%顯示指導教授B中文名
\fi
%
% 判斷是否有共同指導的教授 C
\ifx \advisorCnameC  \itsempty
\relax % 沒有 C 教授,所以不佔版面,不印任何空白
\else
\vspace{0.1cm}
% 共同指導的教授 C
\hspace{2.4em}%
\makebox[4.8em][s]{}%
\makebox[1em][c]{}%
\makebox[7.2em][l]{\Large\advisorCnameC}\\%顯示指導教授B中文名
\fi
%
\vfill

\vspace{1cm}
\makebox[8cm][s]{\Large 中華民國\cYear 年\cMonth 月}%
%
\end{center}

\end{titlepage}
%%%%%%%%%%%%%%

\input{./cover/cover-chi-2.tex}

% ----------------------------------------------------------------------------
%                                English cover
%                                   英文封面
% ----------------------------------------------------------------------------

% Set the line spacing to single for the titles (to compress the lines)
\renewcommand{\baselinestretch}{1}   %行距 1 倍

% ------------------------------------------------

% Page start
\newpage
\phantomsection

% Add to "Table of Contents"
\addcontentsline{toc}{chapter}{Cover}

% 設定使用 無頁碼, 有浮水印
\thispagestyle{empty}

% Aligned to the center of the page
\begin{center}

% ------------------------------------------------

% 顯示 校名, 系所名, 論文種類
\makebox[10cm][c]{\Huge\univEname}\\
\vspace{0.5cm}
\makebox[8cm][c]{\LARGE\fulldeptEname}\\
\vspace{0.5cm}
\makebox[5cm][c]{\LARGE Master's Thesis}\\

% ------------------------------------------------

% Space
\vspace{7cm}

% ------------------------------------------------

% English title 英文題目
\LARGE\eTitle\\
\vspace{4.5cm}

% ------------------------------------------------

% 顯示學生和老師的名字

% --------------------------

% 顯示 學生 的名字
\makebox[4.8em][r]{\Large Student}
\makebox[1em][c]{\Large:}
\makebox[7.2em][l]{\Large\myEname}\\

\vspace{0.5cm}

% --------------------------

% 顯示 老師 (指導老師 A) 的名字
\makebox[4.8em][r]{\Large Advisor}
\makebox[1em][c]{\Large:}
\makebox[7.2em][l]{\Large\advisorEnameA}\\

% --------------------------

% 指導老師 B
\ifx \advisorEnameB  \itsempty
    % 沒有 老師B 的話, 不佔版面和不印任何空白
    \relax
\else
    % 顯示 老師B 的名字
    \vspace{0.1cm}
    \hspace{6.3em}
    \makebox[7.2em][l]{\Large\advisorEnameB}\\
\fi

% --------------------------

% 指導老師 C
\ifx \advisorEnameC  \itsempty
    % 沒有 老師C 的話, 不佔版面和不印任何空白
    \relax
\else
    % 顯示 老師C 的名字
    \vspace{0.1cm}
    \hspace{6.3em}
    \makebox[7.2em][l]{\Large\advisorEnameC}\\
\fi

% ------------------------------------------------

% Date 日期
\vspace{1.1cm}
\makebox[8cm][c]{\Large \eMonth\ \ \eYear}

% ------------------------------------------------

% End of alignment
\end{center}

% End of page

% ------------------------------------------------


% ------------------------------------------------

  \end{document}
} % End of \newcommand{}

% Use to include and display inner cover
\newcommand{\DisplayInnerCover}{%
% This file is part of the project of
% National Cheng Kung University (NCKU) Thesis/Dissertation Template in LaTex.
% This project is hold at
%     <https://github.com/wengan-li/ncku-thesis-template-latex>
% by Wen-Gan Li.
%
% This project is distributed in the hope of usefuling to someone,
% you can redistribute it and/or modify it under the terms of the
% Attribution-NonCommercial-ShareAlike 4.0 International.
%
% You should have received a copy of the
% Attribution-NonCommercial-ShareAlike 4.0 International
% along with this project.
% If not, see <http://creativecommons.org/licenses/by-nc-sa/4.0/legalcode.txt>.
%
% Please feel free to fork it, modify it, and try it.
% Have fun !!!
%

% ------------------------------------------------

% 根據user的需求去選用中文或英文封面內頁
%
% 封面: 顯示所有封面內容, 但沒有學校Logo
% 內頁: 顯示所有封面內容, 但有學校Logo
% 不論是精裝版或平裝版都是 封面 (殼/皮) + 內頁
%
\if \GetDisplayCoverLang \ValueDisplayCoverLangEng
  
% ----------------------------------------------------------------------------
%                                English cover
%                                   英文封面
% ----------------------------------------------------------------------------

% Set the line spacing to single for the titles (to compress the lines)
\renewcommand{\baselinestretch}{1}   %行距 1 倍

% ------------------------------------------------

% Page start
\newpage
\phantomsection

% Add to "Table of Contents"
\addcontentsline{toc}{chapter}{Cover}

% 設定使用 無頁碼, 有浮水印
\thispagestyle{empty}

% Aligned to the center of the page
\begin{center}

% ------------------------------------------------

% 顯示 校名, 系所名, 論文種類
\makebox[10cm][c]{\Huge\univEname}\\
\vspace{0.5cm}
\makebox[8cm][c]{\LARGE\fulldeptEname}\\
\vspace{0.5cm}
\makebox[5cm][c]{\LARGE Master's Thesis}\\

% ------------------------------------------------

% Space
\vspace{7cm}

% ------------------------------------------------

% English title 英文題目
\LARGE\eTitle\\
\vspace{4.5cm}

% ------------------------------------------------

% 顯示學生和老師的名字

% --------------------------

% 顯示 學生 的名字
\makebox[4.8em][r]{\Large Student}
\makebox[1em][c]{\Large:}
\makebox[7.2em][l]{\Large\myEname}\\

\vspace{0.5cm}

% --------------------------

% 顯示 老師 (指導老師 A) 的名字
\makebox[4.8em][r]{\Large Advisor}
\makebox[1em][c]{\Large:}
\makebox[7.2em][l]{\Large\advisorEnameA}\\

% --------------------------

% 指導老師 B
\ifx \advisorEnameB  \itsempty
    % 沒有 老師B 的話, 不佔版面和不印任何空白
    \relax
\else
    % 顯示 老師B 的名字
    \vspace{0.1cm}
    \hspace{6.3em}
    \makebox[7.2em][l]{\Large\advisorEnameB}\\
\fi

% --------------------------

% 指導老師 C
\ifx \advisorEnameC  \itsempty
    % 沒有 老師C 的話, 不佔版面和不印任何空白
    \relax
\else
    % 顯示 老師C 的名字
    \vspace{0.1cm}
    \hspace{6.3em}
    \makebox[7.2em][l]{\Large\advisorEnameC}\\
\fi

% ------------------------------------------------

% Date 日期
\vspace{1.1cm}
\makebox[8cm][c]{\Large \eMonth\ \ \eYear}

% ------------------------------------------------

% End of alignment
\end{center}

% End of page

% ------------------------------------------------

\else
  %%%%%%%%%%%%%%%%%%%%%%%%%%%%%%%
%       NCKU cover 封面
%%%%%%%%%%%%%%%%%%%%%%%%%%%%%%%
%
\begin{titlepage}

% 判斷是否要浮水印?
\ifx\mywatermark\undefined 
  \thispagestyle{empty}  % 無頁碼、無 header (無浮水印)
\else
  \thispagestyle{EmptyWaterMarkPage} % 無頁碼、有浮水印
\fi

% aligned to the center of the page
\begin{center}
% font size (relative to 12 pt):
% \large (14pt) < \Large (18pt) < \LARGE (20pt) < \huge (24pt)< \Huge (24 pt)
%
\makebox[10cm][s]{\Huge\univCname}\\  %顯示中文校名
\vspace{1cm}
\makebox[8cm][s]{\Huge\deptCname}\\ %顯示中文系所名
\vspace{1cm}
\makebox[5cm][s]{\Huge\degreeCname 論文}\\ %顯示論文種類 (中文)
\vspace{1.5cm}
\vspace{2cm}
\vspace{2.5cm}

%
% Set the line spacing to single for the titles (to compress the lines)
\renewcommand{\baselinestretch}{1}   %行距 1 倍
%\large % it needs a font size changing command to be effective
\Large\cTitle\\  % 中文題目
%
\vspace{0.5cm}
%
\Large\eTitle\\ %英文題目

\vspace{3.5cm}

% \makebox is a text box with specified width;
% option s: stretch; option l: left aligned
% use \makebox to make sure
% 「研究生」 與「指導教授」occupy the same width
% Names are filled in a box with pre-defined width
% the left and right sides of 「:」occupy the same width (use \hspace{} to fill the short)
% to guarantee 「:」is at the center
% assume the width of a Chinese character is 1.2em
% 4.8em is determined by the length of the longest string "指導教授"
% 7.2em is determined by the length of the possibly longest name + title "歐陽明志博士"
\hspace{2.4em}%
\makebox[4.8em][s]{\Large 研究生}%
\makebox[1em][c]{\Large :}%
\makebox[7.2em][l]{\Large\myCname}\\  % 顯示作者中文名

\vspace{0.5cm}

\hspace{2.4em}%
\makebox[4.8em][s]{\Large 指導教授}%
\makebox[1em][c]{\Large :}%
\makebox[7.2em][l]{\Large\advisorCnameA}\\  %顯示指導教授A中文名
%
% 判斷是否有共同指導的教授 B
\ifx \advisorCnameB  \itsempty
\relax % 沒有 B 教授,所以不佔版面,不印任何空白
\else
\vspace{0.1cm}
% 共同指導的教授 B
\hspace{2.4em}%
\makebox[4.8em][s]{}%
\makebox[1em][c]{}%
\makebox[7.2em][l]{\Large\advisorCnameB}\\%顯示指導教授B中文名
\fi
%
% 判斷是否有共同指導的教授 C
\ifx \advisorCnameC  \itsempty
\relax % 沒有 C 教授,所以不佔版面,不印任何空白
\else
\vspace{0.1cm}
% 共同指導的教授 C
\hspace{2.4em}%
\makebox[4.8em][s]{}%
\makebox[1em][c]{}%
\makebox[7.2em][l]{\Large\advisorCnameC}\\%顯示指導教授B中文名
\fi
%
\vfill

\vspace{1cm}
\makebox[8cm][s]{\Large 中華民國\cYear 年\cMonth 月}%
%
\end{center}

\end{titlepage}
%%%%%%%%%%%%%%

\fi

% ------------------------------------------------
}

% ----------------------------------------------------------------------------

\newcommand{\ValueDisplayCoverLangEng}{0}
\newcommand{\ValueDisplayCoverLangChi}{1}
\newcommand{\VarDisplayCoverLang}{\ValueDisplayCoverLangEng}
\newcommand{\GetDisplayCoverLang}{\VarDisplayCoverLang}
\newcommand{\DisplayCoverInChi}{\renewcommand{\VarDisplayCoverLang}{\ValueDisplayCoverLangChi}}
\newcommand{\DisplayCoverInEng}{\renewcommand{\VarDisplayCoverLang}{\ValueDisplayCoverLangEng}}

% ----------------------------------------------------------------------------

% Display Chinese and English name in english cover
\newcommand{\CoverDisplayNameChiEng}{0} % Default not display both
\newcommand{\SetCDBothName}{\renewcommand{\CoverDisplayNameChiEng}{1}}
\newcommand{\GetCDBothName}{\CoverDisplayNameChiEng}
\newcommand{\CDBothName}{\SetCDBothName}

% ----------------------------------------------------------------------------

% 顯示 '(初稿)' (中文版) 和 '(Draft)' (英文版) 在封面
\newcommand{\GetTextDraftChi}{(初稿)}
\newcommand{\GetTextDraftEng}{(Draft)}
\newcommand{\VarCoverDisplayDraft}{0} % Don't display in default
\newcommand{\EnableFlagDisplayDraft}{\renewcommand{\VarCoverDisplayDraft}{1}}
\newcommand{\DisplayDraft}{\EnableFlagDisplayDraft}
\newcommand{\GetFlagDisplayDraft}{\VarCoverDisplayDraft}
% ----------------------------------------------------------------------------
