
% Some helper function use for oral document

% ----------------------------------------------------------------------------

% 口試委員 Committee member(s)
\newcommand{\CommitteeSize}{8} % Default
\newcommand{\GetCommitteeSize}[0]{\CommitteeSize}
\newcommand{\SetCommitteeSize}[1]
{
  \ifthenelse{#1 < 4}
  {
    \renewcommand{\CommitteeSize}{4}
  } % End of if{}
  {
    \ifthenelse{#1 > 8}
    {\renewcommand{\CommitteeSize}{8}}
    {\renewcommand{\CommitteeSize}{#1}}
  } % End of else{}
} % End of \newcommand{}

% ----------------------------------------------------------------------------

% Signature line
\newcommand{\namesigdate}[0]{\rule{5.5cm}{1pt}}

% ----------------------------------------------------------------------------

% 口試証明文件 Document of oral presentation

% 要顯示圖片還是範例
\newcommand{\OralDocumentImage}{0}
\newcommand{\OralDocumentTemplete}{1}
\newcommand{\OralDocumentEngOnly}{0}
\newcommand{\OralDocument}{\OralDocumentTemplete} % Default

\newcommand{\DisplayOralTemplete}[0]
  {\renewcommand{\OralDocument}{\OralDocumentTemplete}}
\newcommand{\DisplayOralImage}[0]
  {\renewcommand{\OralDocument}{\OralDocumentImage}}
\newcommand{\DisplayOralEngOnly}[0]
  {\renewcommand{\OralDocumentEngOnly}{1}}

% The path of the image that the oral document
\newcommand\OralDocumentImageChiPath{./oral/example/oral-chi.pdf} % Default
\newcommand\OralDocumentImageEngPath{./oral/example/oral-eng.pdf} % Default
\newcommand{\SetOralImageChi}[1]{\renewcommand{\OralDocumentImageChiPath}{#1}}
\newcommand{\SetOralImageEng}[1]{\renewcommand{\OralDocumentImageEngPath}{#1}}

\newcommand{\SetOralImage}[2]
{
  \SetOralImageChi{#1}
  \SetOralImageEng{#2}
} % End of \newcommand{}

\newcommand{\GetOralImageChiPath}[0]{\OralDocumentImageChiPath}
\newcommand{\GetOralImageEngPath}[0]{\OralDocumentImageEngPath}

% ----------------------------------------------------------------------------

