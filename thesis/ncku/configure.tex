%
% This file is part of ncku-thesis-template.
%
% ncku-thesis-template is distributed in the hope of usefuling to someone,
% you can redistribute it and/or modify
% it under the terms of the Attribution-NonCommercial-ShareAlike
% 4.0 International.
%
% You should have received a copy of the
% Attribution-NonCommercial-ShareAlike 4.0 International
% along with ncku-thesis-template.
%
% If not, see <http://creativecommons.org/licenses/by-nc-sa/4.0/legalcode.txt>.
%

% ------------------------------------------------

\documentclass[12pt, a4paper, onecolumn, english]{report}

% ------------------------------------------------

% XeLaTex檢查點, 以要求必須使用XeLaTex來處理模版
\usepackage{ifxetex}
\ifxetex\else\errmessage{模版: 請使用XeLaTex來產生論文.}\stop\fi

% ------------------------------------------------

% 引用字體的基本設定
%
% No longer need \usepackage[T1]{fontenc} and
% \usepackage[utf8]{inputenc} when using XeLaTeX and LuaLaTeX as the engine.
%

% 引用fontspec以提供控制英文字型
\usepackage{fontspec}
\defaultfontfeatures{Ligatures=TeX} % To support LaTeX quoting style

% 引用xeCJK以提供控制中文字型
\usepackage{xeCJK}

% ------------------------------------------------

% 引用需要的LaTex packages

% Some base packages
\usepackage{fp}
\usepackage{ifthen}
\usepackage{pgfkeys}
\usepackage{xparse}
\usepackage{amsmath}
\usepackage{framed}
\usepackage{url}

% For floats
% flafter package will make sure that the floats are
% not placed before their definition
\usepackage{flafter}

% For list
% Ref: <http://ftp.yzu.edu.tw/CTAN/macros/latex/contrib/enumitem/enumitem.pdf>
\usepackage{enumitem}
%\setlist{noitemsep, nosep}
%nolistsep

% For paragraphs
\usepackage{parskip}

% For line spacing
% Ref: <https://en.wikibooks.org/wiki/LaTeX/Paragraph_Formatting>
\usepackage{setspace}

% For PDF
\usepackage{hyperref}
\usepackage{pdfpages}

% For figure
\usepackage{graphicx}
\usepackage{caption}
\usepackage{subcaption}

% For table
\usepackage{array}
\usepackage{multirow}
\usepackage{booktabs}
\usepackage{diagbox}

% For comment
\usepackage{comment}

% For 目錄
\usepackage[tocgraduated]{tocstyle}
\usetocstyle{standard}

% For bib
% everal features of babel that do not make sense in the XeLaTex world
% (like font encodings, shorthands, etc.) are not supported.
\usepackage{polyglossia}
\setdefaultlanguage{english}

% For chinese number in title
% http://ftp.yzu.edu.tw/CTAN/macros/latex/contrib/zhnumber/zhnumber.pdf
\usepackage{zhnumber}

% For pseudocode
\usepackage{algorithm}
\usepackage[noend]{algpseudocode}
\algnewcommand\algorithmicswitch{\textbf{switch}}
\algnewcommand\algorithmiccase{\textbf{case}}
\algnewcommand\algorithmicdefault{\textbf{default}}
\algnewcommand\algorithmicbreak{\textbf{break}}
\algdef{SE}[SWITCH]{Switch}{EndSwitch}[1]{\algorithmicswitch\ #1\ }{\algorithmicend\ \algorithmicswitch}%
\algdef{SE}[CASE]{Case}{EndCase}[1]{\algorithmiccase\ #1:}{\algorithmicend\ \algorithmiccase}%
\algdef{SE}[CASE]{Default}{EndDefault}[0]{\algorithmicdefault:}{\algorithmicend\ \algorithmiccase}%
\algtext*{EndSwitch}%
\algtext*{EndCase}%
\algtext*{EndDefault}%
\def\Break{\algorithmicbreak}

% ------------------------------------------------

% 有關學校對論文要求的設定

% --------------------------

% 一些用來設定function和variable的command

% ----------------------------------------------------------------------------
% 一些用來設定function和variable的command
% Some function and variable that let user use and configure
%
% 此處只是一些預設值和function
% 修改內容是在'conf/conf.tex'
% ----------------------------------------------------------------------------

% Static variable and some provided API

% Some common helper function

% ----------------------------------------------------------------------------

% Some helper functions

\newcommand{\GetMonthInEng}[1]
{
  \ifthenelse{\equal{#1}{1}}{January}{}
  \ifthenelse{\equal{#1}{2}}{February}{}
  \ifthenelse{\equal{#1}{3}}{March}{}
  \ifthenelse{\equal{#1}{4}}{April}{}
  \ifthenelse{\equal{#1}{5}}{May}{}
  \ifthenelse{\equal{#1}{6}}{June}{}
  \ifthenelse{\equal{#1}{7}}{July}{}
  \ifthenelse{\equal{#1}{8}}{August}{}
  \ifthenelse{\equal{#1}{9}}{September}{}
  \ifthenelse{\equal{#1}{10}}{October}{}
  \ifthenelse{\equal{#1}{11}}{November}{}
  \ifthenelse{\equal{#1}{12}}{December}{}
} % End of \newcommand{}

% 計算出台灣民國幾年
% Get the year using Taiwans' year
\newcommand{\SetOralTaiwanYear}[1]
{
  \FPeval{\OralTaiwanYearResult}{clip(#1 - 1911)}
} % End of \newcommand{}

% 計算出台灣民國幾年
% Get the year using Taiwans' year
\newcommand{\SetThesisTaiwanYear}[1]
{
  \FPeval{\ThesisTaiwanYearResult}{clip(#1 - 1911)}
} % End of \newcommand{}

% ----------------------------------------------------------------------------


\newcommand{\BeginThesis}[0]
{
  % Start of paper
  \begin{document}

  % Set the line spacing for whole thesis
  \baselineskip = 26pt
} % End of \newcommand{}

\newcommand{\EndThesis}[0]
{
  % End of paper
  \end{document}
} % End of \newcommand{}

% ----------------------------------------------------------------------------

\newcommand{\StartNewPage}
{
  % Page start
  \newpage
  \phantomsection
} % End of \newcommand{}

\newcommand{\EndOfPage}
{
  % End of page
  \clearpage
} % End of \newcommand{}

% ----------------------------------------------------------------------------

% Abstract
\newcommand{\StartAbstract}
{
  \StartNewPage

  % Add to "Table of Contents"
  \addcontentsline{toc}{chapter}{Abstract}

  % Title
  \centerline{\Large \textbf{Abstract}}
} % End of \newcommand{}

\def \EndChiAbstract
{
  % Keyword
  \AbstractKeyword

  \EndOfPage
} % End of \def{}

\def \EndAbstract
{
  % Keyword
  \ExtendedAbstractKeyword

  \EndOfPage
} % End of \def{}

% Extended Abstract
\newcommand{\StartExtendedAbstract}
{
  \StartNewPage

  % Set page
  \baselineskip=20pt
  \setlength{\parindent}{0.0pt}

  % Set style
  \pagestyle{empty}

  % 設定段落之間的距離
  \setlength{\parskip}{0.5cm}

  % Add to "Table of Contents"
  \addcontentsline{toc}{chapter}{Extended Abstract}

  % Add title
  \centerline{\large \textbf{\eTitle}}
  \vspace{0.5cm}

  % Add name
  \centerline{\large \GetAuthorEngName}

  % Add names
  \centerline{\large Prof. \thinspace \GetAdvisorEngNameA}
  \ifthenelse{\equal{\GetAdvisorEngNameB}{\empty}}
    {}
    {\centerline{\large Prof. \thinspace \GetAdvisorEngNameB}}

  \ifthenelse{\equal{\GetAdvisorEngNameC}{\empty}}
    {}
    {\centerline{\large Prof. \thinspace \GetAdvisorEngNameC}}

  % Add department
  \centerline{\large \GetDeptEngName}
} % End of \newcommand{}

\def \EndExtendedAbstract
{
  \EndOfPage

  % 設定段落之間的距離
  \setlength{\parskip}{0.3cm}

  % Reset page
  \setlength{\parindent}{1.5em}
  \baselineskip=26pt

  % Re-set style
  \pagestyle{plain}
} % End of \def{}

% Summary in Extended Abstract
\newcommand{\ExtAbstractSummary}[1]{
  \begin{framed}
    \ExtAbstractChapter{SUMMARY}{#1}

    % Keyword
    \ExtendedAbstractKeyword
  \end{framed}
} % End of \newcommand{}

% Chapter in Extended Abstract
\newcommand{\ExtAbstractChapter}[2]
{
  \vspace{0.4cm}
  \centerline{\textbf{#1}}

  #2
} % End of \newcommand{}

% Sub-chapter in Extended Abstract
\newcommand{\ExtAbstractSubChapter}[2]
{
  \textbf{#1}

  #2
} % End of \newcommand{}

\newcommand{\AbstractKeyword}[0]
{
  % Keyword
  \par
  \ifthenelse{\equal{\GetKeywordsE}{\empty}}
  {
    \ifthenelse{\equal{\GetKeywordsD}{\empty}}
    {
      \ifthenelse{\equal{\GetKeywordsC}{\empty}}
      {
        \ifthenelse{\equal{\GetKeywordsB}{\empty}}
        {
          \ifthenelse{\equal{\GetKeywordsA}{\empty}}
          {}
          {
            {\noindent \bf 關鍵字:} \GetKeywordsA
          } % End of else{}
        } % End of if{}
        {
          {\noindent \bf 關鍵字:} \GetKeywordsA, \GetKeywordsB
        } % End of else{}
      } % End of if{}
      {
        {\noindent \bf 關鍵字:} \GetKeywordsA, \GetKeywordsB, \GetKeywordsC
      } % End of else{}
    } % End of if{}
    {
      {\noindent \bf 關鍵字:} \GetKeywordsA, \GetKeywordsB, \GetKeywordsC, \GetKeywordsD
    } % End of else{}
  } % End of if{}
  {
    {\noindent\bf 關鍵字:} \GetKeywordsA, \GetKeywordsB, \GetKeywordsC, \GetKeywordsD, \GetKeywordsE
  } % End of else{}
} % End of \newcommand{}

\newcommand{\ExtendedAbstractKeyword}[0]
{
  % Keyword
  \par
  \ifthenelse{\equal{\GetKeywordsE}{\empty}}
  {
    \ifthenelse{\equal{\GetKeywordsD}{\empty}}
    {
      \ifthenelse{\equal{\GetKeywordsC}{\empty}}
      {
        \ifthenelse{\equal{\GetKeywordsB}{\empty}}
        {
          \ifthenelse{\equal{\GetKeywordsA}{\empty}}
          {}
          {
            {\noindent \bf Key words:} \GetKeywordsA
          } % End of else{}
        } % End of if{}
        {
          {\noindent \bf Key words:} \GetKeywordsA, \GetKeywordsB
        } % End of else{}
      } % End of if{}
      {
        {\noindent \bf Key words:} \GetKeywordsA, \GetKeywordsB, \GetKeywordsC
      } % End of else{}
    } % End of if{}
    {
      {\noindent \bf Key words:} \GetKeywordsA, \GetKeywordsB, \GetKeywordsC, \GetKeywordsD
    } % End of else{}
  } % End of if{}
  {
    {\noindent\bf Key words:} \GetKeywordsA, \GetKeywordsB, \GetKeywordsC, \GetKeywordsD, \GetKeywordsE
  } % End of else{}
} % End of \newcommand{}

% ----------------------------------------------------------------------------

% Chapters
\DeclareDocumentCommand{\StartChapter}{m g}
{
  \chapter{#1}
  \IfNoValueF{#2}{\label{#2}}
%  \baselineskip=26pt
%  \thispagestyle{empty}
} % End of \DeclareDocumentCommand{}

\def \EndChapter {\EndOfPage}

% Acknowledgments
\newcommand{\StartAcknowledgments}[1]
{
  \StartNewPage

  % Add to "Table of Contents"
  \addcontentsline{toc}{chapter}{Acknowledgments}

  % Title
  \centerline{\Large \textbf{Acknowledgments}}

  \IfNoValueF{#1}{\label{#1}}
} % End of \newcommand{}

\def \EndAcknowledgments {\EndOfPage}

\def \StartAppendix
{
  \StartNewPage
  % Appendix begins here (For one appendix chapter only)
  %\cleardoublepage
  \appendix
  %\clearpage
  %\setcounter{page}{\thepage}
  % If more than one appendix chapters
  %\appendices
}

\def \EndAppendix {\EndOfPage}

% ----------------------------------------------------------------------------

% --- Chinese / English title 中英文論文題目 ---
\newcommand{\KeywordsA}{} % Default
\newcommand{\KeywordsB}{} % Default
\newcommand{\KeywordsC}{} % Default
\newcommand{\KeywordsD}{} % Default
\newcommand{\KeywordsE}{} % Default
\newcommand{\SetKeywordsA}[1]{\renewcommand{\KeywordsA}{#1}}
\newcommand{\SetKeywordsB}[1]{\renewcommand{\KeywordsB}{#1}}
\newcommand{\SetKeywordsC}[1]{\renewcommand{\KeywordsC}{#1}}
\newcommand{\SetKeywordsD}[1]{\renewcommand{\KeywordsD}{#1}}
\newcommand{\SetKeywordsE}[1]{\renewcommand{\KeywordsE}{#1}}
\newcommand{\GetKeywordsA}[0]{\KeywordsA}
\newcommand{\GetKeywordsB}[0]{\KeywordsB}
\newcommand{\GetKeywordsC}[0]{\KeywordsC}
\newcommand{\GetKeywordsD}[0]{\KeywordsD}
\newcommand{\GetKeywordsE}[0]{\KeywordsE}
\DeclareDocumentCommand{\SetKeywords}{
  m G{\empty} G{\empty} G{\empty} G{\empty}}
{
  \SetKeywordsA{#1}
  \ifthenelse{\equal{#2}{\empty}}{}{\SetKeywordsB{#2}}
  \ifthenelse{\equal{#3}{\empty}}{}{\SetKeywordsC{#3}}
  \ifthenelse{\equal{#4}{\empty}}{}{\SetKeywordsD{#4}}
  \ifthenelse{\equal{#5}{\empty}}{}{\SetKeywordsE{#5}}
} % End of \newcommand{}

% ----------------------------------------------------------------------------



% Some helper function about insert image

% 用\begin{figure} .. \end{figure}
% 可能會出現問題
% http://www.tex.ac.uk/cgi-bin/texfaq2html?label=ouparmd

\DeclareDocumentCommand{\SetImageCaption}{G{\empty}}
{
  \ifthenelse{\equal{#1}{\empty}}
  {}
  {\IfNoValueF{#1}{\caption{#1}}}
} % End of \DeclareDocumentCommand{}

\DeclareDocumentCommand{\SetImageLabel}{G{\empty}}
{
  \ifthenelse{\equal{#1}{\empty}}
  {}
  {\IfNoValueF{#1}{\label{#1}}}
} % End of \DeclareDocumentCommand{}

\pgfkeys
{
  /InsertImage/.is family, /InsertImage,
  default/.style = 
  {
    scale = 1.0,
    angle = 0,
    caption = \empty,
    label = \empty,
    align = \empty,
  },
  scale/.estore in = \InsertImageValueScale,
  angle/.estore in = \InsertImageValueAngle,
  caption/.estore in = \InsertImageValueCaption,
  label/.estore in = \InsertImageValueLabel,
  align/.estore in = \InsertImageValueAlign,
} % End of \pgfkeys{}

% Insert a single image
\newcommand{\InsertImage}[2][\empty]
{
  % Parse the input
  \pgfkeys{/InsertImage, default, #1}
  %
  % Debug if needed
%  Scale: \InsertImageValueScale\\
%  Angle: \InsertImageValueAngle\\
%  Caption: \InsertImageValueCaption\\
%  Label: \InsertImageValueLabel\\
%  Align: \InsertImageValueAlign\\
%  \ifthenelse{\equal{#2}{\empty}}
%    {Path: -EMPTY-\\}{Path: #2\\}
%  \IfNoValueF{#2}{Path: #2\\}
  %
  %
  % Check any caption first
  \ifthenelse{\equal{\InsertImageValueCaption}{\empty}}
  { %
    %
    %No Caption
    %
    % Align
    \ifthenelse{\equal{\InsertImageValueAlign}{center}}
      {\begin{center}}{}
    % Insert image
    \includegraphics
      [scale=\InsertImageValueScale,
        angle=\InsertImageValueAngle]{#2}
    % Align
    \ifthenelse{\equal{\InsertImageValueAlign}{center}}
      {\end{center}}{}
  } % End of if{}
  { %
    %
    %Have Caption
    %
    \begin{figure}[h]
      % Do center if needed
      \ifthenelse{\equal{\InsertImageValueAlign}{center}}
        {\center}{}
      % Insert image
      \includegraphics
        [scale=\InsertImageValueScale,
          angle=\InsertImageValueAngle]{#2}
      % Set Caption
      \SetImageCaption{\InsertImageValueCaption}
      % Set Label
      \ifthenelse{\equal{\InsertImageValueCaption}{\empty}}
        {}{\SetImageLabel{\InsertImageValueLabel}}
    \end{figure}
  } % End of else{}
} % End of \newcommand{}

% Insert a single image, but place it in center
\newcommand{\InsertCenterImage}[2][\empty]
{
  \InsertImage[#1, align=center]{#2}
} % End of \newcommand{}

% For multi images
\newcommand\MultiImagesPerRow{1}
\newcommand{\SetMultiImagesPerRow}[1]
  {\renewcommand{\MultiImagesPerRow}{#1}}
\newcommand{\GetMultiImagesPerRow}[0]{\MultiImagesPerRow}

%add desired spacing between images, e. g. ~, \quad, \qquad etc.
%(or a blank line to force the subfigure onto a new line)
\newcommand{\SubfigureBreakSpaceLine}[1]
{
  \ifthenelse{\equal{\intcalcMod{
    \GetMultiImageId - 1}{\GetMultiImagesPerRow}}{0}}%
  {%
    % Echo blank line
    %

  } % End of if{}
  {%
    ~
  } % End of else{}
} % End of \newcommand{}

% Confingure the main image
\DeclareDocumentCommand{\SetMultiImageRoot}{m g g}
{
  % Set per row
  \FPeval{\MultiImagesPerRow}{clip(#1)} %
} % End of \newcommand{}

\DeclareDocumentCommand{\SetMultiImageRootCaptionAndLabel}{m g g}
{
  \SetImageCaption{#2} %
  \SetImageLabel{#3} %
} % End of \newcommand{}

% Low-level insert image
\newcommand\ImageRotateAngle{0}
\newcommand{\SetImageRotateAngle}[1]
  {\renewcommand{\ImageRotateAngle}{#1}}
\newcommand{\GetImageRotateAngle}[0]{\ImageRotateAngle}

\newcommand{\InsertSubfigureImage}[5]
{%
  \begin{subfigure}[b]{\WidthOfImagePerRow\textwidth}%
    \center
    \includegraphics[scale=#1, angle=#5]{#2}
    \SetImageCaption{#3}
    \SetImageLabel{#4}
  \end{subfigure}%
} % End of \newcommand{}

\newcommand{\InsertMultiImageInterface}[2]
{
  \ifthenelse{\equal{#2}{\empty}}
    {}
    {
      \SetMultiImageId{#1}
      \SubfigureBreakSpaceLine{#1}
      \ifthenelse{\equal{\GetMultiImagesPerRow}{1}}%
      {%
        \FPeval{\WidthOfImagePerRow}{1.0} %
      } % End of if{}
      {%
        \ifthenelse{\equal{\GetMultiImageId}{\GetMultiImageTotalValue}}%
        {%
          \FPeval{\WidthOfImagePerRow}{1.0} %
        } % End of if{}
        {%
          \FPeval{\WidthOfImagePerRow}{clip(clip(1 / \GetMultiImagesPerRow) - 0.1)} %
        } % End of else{}
      } % End of else{}
      \InsertMultiImageLL#2
    } % End of else{}
} % End of \newcommand{}

\DeclareDocumentCommand{\InsertMultiImageLL}{m m g g g}
{
  \SetImageRotateAngle{0}
  \IfNoValueF{#5}{\SetImageRotateAngle{#5}}
  \InsertSubfigureImage{#1}{#2}{#3}{#4}{\ImageRotateAngle}%
} % End of \DeclareDocumentCommand{}

\newcommand\MultiImageTotalValue{0}
\newcommand{\SetMultiImageTotalValue}[1]
  {\renewcommand{\MultiImageTotalValue}{#1}}
\newcommand{\GetMultiImageTotalValue}[0]{\MultiImageTotalValue}

\newcommand\MultiImageId{0}
\newcommand{\SetMultiImageId}[1]
  {\renewcommand{\MultiImageId}{#1}}
\newcommand{\GetMultiImageId}[0]{\MultiImageId}

% Insert multi-image
% Arg: 1st: Table configure
%      2~9th: Images (Max 8 Images)
%
% ---------------------------------------
% 使用 \InsertMultiImageLL#2
% 是跟 \InsertMultiImageLL{#2} 不一樣的
% 直接連接#2是指把#2整個當成function的所有參數
% 而{#2}是指把#2當成單一個參數傳給function
% ---------------------------------------
%
\DeclareDocumentCommand{\InsertMultiImages}{%
  G{\empty} G{\empty} G{\empty} %
  G{\empty} G{\empty} G{\empty} %
  G{\empty} G{\empty} G{\empty}}
{
  %Set Multi Image Total
  \ifthenelse{\equal{#2}{\empty}}{}{\SetMultiImageTotalValue{1}}
  \ifthenelse{\equal{#3}{\empty}}{}{\SetMultiImageTotalValue{2}}
  \ifthenelse{\equal{#4}{\empty}}{}{\SetMultiImageTotalValue{3}}
  \ifthenelse{\equal{#5}{\empty}}{}{\SetMultiImageTotalValue{4}}
  \ifthenelse{\equal{#6}{\empty}}{}{\SetMultiImageTotalValue{5}}
  \ifthenelse{\equal{#7}{\empty}}{}{\SetMultiImageTotalValue{6}}
  \ifthenelse{\equal{#8}{\empty}}{}{\SetMultiImageTotalValue{7}}
  \ifthenelse{\equal{#9}{\empty}}{}{\SetMultiImageTotalValue{8}}

  \begin{figure}[hbtp]
    \centering
    \SetMultiImageRoot#1
    % Image 1
    \InsertMultiImageInterface{1}{#2}
    % Image 2
    \InsertMultiImageInterface{2}{#3}
    % Image 3
    \InsertMultiImageInterface{3}{#4}
    % Image 4
    \InsertMultiImageInterface{4}{#5}
    % Image 5
    \InsertMultiImageInterface{5}{#6}
    % Image 6
    \InsertMultiImageInterface{6}{#7}
    % Image 7
    \InsertMultiImageInterface{7}{#8}
    % Image 8
    \InsertMultiImageInterface{8}{#9}
    \SetMultiImageRootCaptionAndLabel#1
  \end{figure}
} % End of \DeclareDocumentCommand{}

% ----------------------------------------------------------------------------

\DeclareDocumentCommand{\TestImages}{%
  O{1.0} O{0} m}
{
        #1 \\
        #2 \\
        #3 \\
} % End of \DeclareDocumentCommand{}

% http://tex.stackexchange.com/questions/34312/how-to-create-a-command-with-key-values

\pgfkeys
{
  /InsertImageB/.is family, /InsertImageB,
  default/.style = 
  {
    scale = 1.0,
    angle = 0,
    caption = \empty,
    label = \empty
  },
  scale/.estore in = \InsertImageBScale,
  angle/.estore in = \InsertImageBAngle,
  caption/.estore in = \InsertImageBCaption,
  label/.estore in = \InsertImageBLabel,
} % End of \pgfkeys{}

\pgfkeys
{
  /InsertImageCD/.is family, /InsertImageCD,
  default/.style = 
  {
    scale = 1.0,
    angle = 0,
    caption = \empty,
    label = \empty
  },
  scale/.estore in = \InsertImageCDScale,
  angle/.estore in = \InsertImageCDAngle,
  caption/.estore in = \InsertImageCDCaption,
  label/.estore in = \InsertImageCDLabel,
} % End of \pgfkeys{}

\newcommand\InsertImageB[2][\empty]
{
  \pgfkeys{/InsertImageB, default, #1}%
  Scale: \InsertImageBScale\\
  Angle: \InsertImageBAngle\\
  caption: \InsertImageBCaption\\
  label: \InsertImageBLabel\\
  Path: #2\\
}

\newcommand\InsertImageCD[2][\empty]
{
  \pgfkeys{/InsertImageCD, default, #1}%
  Scale: \InsertImageCDScale\\
  Angle: \InsertImageCDAngle\\
  caption: \InsertImageCDCaption\\
  label: \InsertImageCDLabel\\
  Path: #2\\
}

%
% This file is part of the project of
% National Cheng Kung University (NCKU) Thesis/Dissertation Template in LaTex.
% This project is hold at
%     <https://github.com/wengan-li/ncku-thesis-template-latex>
% by Wen-Gan Li.
%
% This project is distributed in the hope of usefuling to someone,
% you can redistribute it and/or modify it under the terms of the
% Attribution-NonCommercial-ShareAlike 4.0 International.
%
% You should have received a copy of the
% Attribution-NonCommercial-ShareAlike 4.0 International
% along with this project.
% If not, see <http://creativecommons.org/licenses/by-nc-sa/4.0/legalcode.txt>.
%
% Please feel free to fork it, modify it, and try it.
% Have fun !!!
%

% Some helper function use for oral document

% ----------------------------------------------------------------------------
\def \OralIndexHeader {Oral presentation document}
% ----------------------------------------------------------------------------

\newcommand{\StartOralTemplateDocChi}
{
  % 由於中文版的watermark是用文字, 所以先把Logo版關掉
  \ClearWatermarkStyle
  %
  % 使用文字版watermark
  \UseWatermarkTextStyle
  %
  \singlespacing%
  %
  \StartNewPage
  %
  % 設定使用 無頁碼
  \thispagestyle{empty}
  %
  % Aligned to the center of the page
  \begin{center}
} % End of \newcommand{}

\newcommand{\StartOralTemplateDocEng}
{
  %
  \singlespacing%
  %
  \StartNewPage
  %
  % 設定使用 無頁碼
  \thispagestyle{empty}
  %
  % Aligned to the center of the page
  \begin{center}
} % End of \newcommand{}

\newcommand{\EndOralTemplateDoc}
{
  % End of alignment
  \end{center}
  %
  % End of page
  \EndOfPage
  \UseDefaultLineStretch
  %
  % 重新使用學校浮水印 Watermark
  \ClearWatermarkStyle
  \UseWatermarkFigureStyle
} % End of \newcommand{}

% ----------------------------------------------------------------------------

% 口試委員 Committee member(s)
\newcommand{\CommitteeSize}{9} % Default
\newcommand{\GetCommitteeSize}{\CommitteeSize}
\newcommand{\SetCommitteeSize}[1]
{
  \ifthenelse{#1 < 2}
  {
    \renewcommand{\CommitteeSize}{2}
  } % End of if{}
  {
    \ifthenelse{#1 > 9}
    {\renewcommand{\CommitteeSize}{9}}
    {\renewcommand{\CommitteeSize}{#1}}
  } % End of else{}
} % End of \newcommand{}

% 口試委員簽名區
\newcommand{\DisplayCommitteeSignatureArea}
{
  \par
  % 口試委員 至少2位
  \ifthenelse{\CommitteeSize = 2}
  {
    \begin{minipage}[c][8.0cm][c]{\textwidth}
      \makebox[0.5\textwidth][c]{\namesigdate}
      \makebox[0.5\textwidth][c]{\namesigdate}
    \end{minipage}
  } % End of if{}
  {} % End of else{}
  %
  \ifthenelse{\CommitteeSize = 3}
  {
    \begin{minipage}[c][4.0cm][c]{\textwidth}
      \makebox[0.5\textwidth][c]{\namesigdate}
      \makebox[0.5\textwidth][c]{\namesigdate}
    \end{minipage}
    %
    \begin{minipage}[c][4.0cm][c]{\textwidth}
      \makebox[\textwidth][c]{\namesigdate}
    \end{minipage}
  } % End of if{}
  {} % End of else{}
  %
  \ifthenelse{\CommitteeSize = 4}
  {
    \begin{minipage}[c][4.0cm][c]{\textwidth}
      \makebox[0.5\textwidth][c]{\namesigdate}
      \makebox[0.5\textwidth][c]{\namesigdate}
    \end{minipage}
    %
    \begin{minipage}[c][4.0cm][c]{\textwidth}
      \makebox[0.5\textwidth][c]{\namesigdate}
      \makebox[0.5\textwidth][c]{\namesigdate}
    \end{minipage}
  } % End of if{}
  {} % End of else{}
  %
  \ifthenelse{\CommitteeSize = 5}
  {
    \begin{minipage}[c][2.65cm][c]{\textwidth}
      \makebox[0.5\textwidth][c]{\namesigdate}
      \makebox[0.5\textwidth][c]{\namesigdate}
    \end{minipage}
    %
    \begin{minipage}[c][2.65cm][c]{\textwidth}
      \makebox[0.5\textwidth][c]{\namesigdate}
      \makebox[0.5\textwidth][c]{\namesigdate}
    \end{minipage}
    %
    \begin{minipage}[c][2.65cm][c]{\textwidth}
      \makebox[\textwidth][c]{\namesigdate}
    \end{minipage}
  } % End of if{}
  {} % End of else{}
  %
  \ifthenelse{\CommitteeSize = 6}
  {
    \begin{minipage}[c][2.65cm][c]{\textwidth}
      \makebox[0.5\textwidth][c]{\namesigdate}
      \makebox[0.5\textwidth][c]{\namesigdate}
    \end{minipage}
    %
    \begin{minipage}[c][2.65cm][c]{\textwidth}
      \makebox[0.5\textwidth][c]{\namesigdate}
      \makebox[0.5\textwidth][c]{\namesigdate}
    \end{minipage}
    %
    \begin{minipage}[c][2.65cm][c]{\textwidth}
      \makebox[0.5\textwidth][c]{\namesigdate}
      \makebox[0.5\textwidth][c]{\namesigdate}
    \end{minipage}
  } % End of if{}
  {} % End of else{}
  %
  \ifthenelse{\CommitteeSize = 7}
  {
    \begin{minipage}[c][2.0cm][c]{\textwidth}
      \makebox[0.5\textwidth][c]{\namesigdate}
      \makebox[0.5\textwidth][c]{\namesigdate}
    \end{minipage}
    %
    \begin{minipage}[c][2.0cm][c]{\textwidth}
      \makebox[0.5\textwidth][c]{\namesigdate}
      \makebox[0.5\textwidth][c]{\namesigdate}
    \end{minipage}
    %
    \begin{minipage}[c][2.0cm][c]{\textwidth}
      \makebox[0.5\textwidth][c]{\namesigdate}
      \makebox[0.5\textwidth][c]{\namesigdate}
    \end{minipage}
    %
    \begin{minipage}[c][2.0cm][c]{\textwidth}
      \makebox[\textwidth][c]{\namesigdate}
    \end{minipage}
  } % End of if{}
  {} % End of else{}
  %
  \ifthenelse{\CommitteeSize = 8}
  {
    \begin{minipage}[c][2.0cm][c]{\textwidth}
      \makebox[0.5\textwidth][c]{\namesigdate}
      \makebox[0.5\textwidth][c]{\namesigdate}
    \end{minipage}
    %
    \begin{minipage}[c][2.0cm][c]{\textwidth}
      \makebox[0.5\textwidth][c]{\namesigdate}
      \makebox[0.5\textwidth][c]{\namesigdate}
    \end{minipage}
    %
    \begin{minipage}[c][2.0cm][c]{\textwidth}
      \makebox[0.5\textwidth][c]{\namesigdate}
      \makebox[0.5\textwidth][c]{\namesigdate}
    \end{minipage}
    %
    \begin{minipage}[c][2.0cm][c]{\textwidth}
      \makebox[0.5\textwidth][c]{\namesigdate}
      \makebox[0.5\textwidth][c]{\namesigdate}
    \end{minipage}
  } % End of if{}
  {} % End of else{}
  %
  \ifthenelse{\CommitteeSize = 9}
  {
    \begin{minipage}[c][1.6cm][c]{\textwidth}
      \makebox[0.5\textwidth][c]{\namesigdate}
      \makebox[0.5\textwidth][c]{\namesigdate}
    \end{minipage}
    %
    \begin{minipage}[c][1.6cm][c]{\textwidth}
      \makebox[0.5\textwidth][c]{\namesigdate}
      \makebox[0.5\textwidth][c]{\namesigdate}
    \end{minipage}
    %
    \begin{minipage}[c][1.6cm][c]{\textwidth}
      \makebox[0.5\textwidth][c]{\namesigdate}
      \makebox[0.5\textwidth][c]{\namesigdate}
    \end{minipage}
    %
    \begin{minipage}[c][1.6cm][c]{\textwidth}
      \makebox[0.5\textwidth][c]{\namesigdate}
      \makebox[0.5\textwidth][c]{\namesigdate}
    \end{minipage}
    %
    \begin{minipage}[c][1.6cm][c]{\textwidth}
      \makebox[\textwidth][c]{\namesigdate}
    \end{minipage}
  } % End of if{}
  {} % End of else{}
} % End of \newcommand{}

% ----------------------------------------------------------------------------

% Signature line
\newcommand{\namesigdate}{\rule{5.5cm}{1pt}}

% ----------------------------------------------------------------------------

% 學位考試論文證明書 Defense Certificate

% 要顯示圖片還是範例
\newcommand{\ValueOralDocumentTypeImage}{0}
\newcommand{\ValueOralDocumentTypeTemplate}{1}
\newcommand{\FlagOralDocumentType}{\OralDocumentTemplate} % Default
\newcommand{\GetOralDocumentType}{\FlagOralDocumentType} % Default

\newcommand{\DisplayOralTemplate}
  {\renewcommand{\FlagOralDocumentType}{\ValueOralDocumentTypeTemplate}}
\newcommand{\DisplayOralImage}
  {\renewcommand{\FlagOralDocumentType}{\ValueOralDocumentTypeImage}}

% The path of the image that the oral document
\newcommand\OralDocumentImageChiPath{\empty} % Default
\newcommand\OralDocumentImageEngPath{\empty} % Default
\newcommand{\SetOralImageChi}[1]
{\renewcommand{\OralDocumentImageChiPath}{./context/oral/#1}}
\newcommand{\SetOralImageEng}[1]
{\renewcommand{\OralDocumentImageEngPath}{./context/oral/#1}}

\newcommand{\GetOralImageChiPath}{\OralDocumentImageChiPath}
\newcommand{\GetOralImageEngPath}{\OralDocumentImageEngPath}

\newcommand{\ValueDisplayOralTemplateOn}{1}
\newcommand{\ValueDisplayOralTemplateOff}{0}
\newcommand{\VarDisplayOralTemplateChi}{\ValueDisplayOralTemplateOff}
\newcommand{\VarDisplayOralTemplateEng}{\ValueDisplayOralTemplateOff}

\newcommand{\DisplayOralChiTemplate}
  {\renewcommand{\VarDisplayOralTemplateChi}{\ValueDisplayOralTemplateOn}}
\newcommand{\GetDisplayOralChiTemplate}{\VarDisplayOralTemplateChi}

\newcommand{\DisplayOralEngTemplate}
  {\renewcommand{\VarDisplayOralTemplateEng}{\ValueDisplayOralTemplateOn}}
\newcommand{\GetDisplayOralTemplateEng}{\VarDisplayOralTemplateEng}

% ----------------------------------------------------------------------------

% Use to include the oral files
\newcommand{\DisplayOral}{
% ----------------------------------------------------------------------------
%                           Document of oral presentation
%                                   口試証明文件
% ----------------------------------------------------------------------------

% Page start
\newpage
\phantomsection

% 設定顯示在目錄
\addcontentsline{toc}{chapter}{Document of oral presentation}

% ----------------------------------------------------------------------------

% Chinese version 中文版本
%\phantomsection
%\addcontentsline{toc}{section}{Chinese version}
%\thispagestyle{empty}

% Image of document
%\includepdf{./oral/oral-chi.pdf}
\input{./oral/templete-chi.tex}
% ----------------------------------------------------------------------------

% English version 英文版本
\phantomsection
\addcontentsline{toc}{section}{English version}
\thispagestyle{empty}

% Image of document
\includepdf{./oral/oral-eng.pdf}
%\put{./oral/oral-eng.pdf}

% ----------------------------------------------------------------------------

% End of page
}

% ----------------------------------------------------------------------------

%
% This file is part of the project of
% National Cheng Kung University (NCKU) Thesis/Dissertation Template in LaTex.
% This project is hold at
%     <https://github.com/wengan-li/ncku-thesis-template-latex>
% by Wen-Gan Li.
%
% This project is distributed in the hope of usefuling to someone,
% you can redistribute it and/or modify it under the terms of the
% Attribution-NonCommercial-ShareAlike 4.0 International.
%
% You should have received a copy of the
% Attribution-NonCommercial-ShareAlike 4.0 International
% along with this project.
% If not, see <http://creativecommons.org/licenses/by-nc-sa/4.0/legalcode.txt>.
%
% Please feel free to fork it, modify it, and try it.
% Have fun !!!
%

% Some common helper function

% ----------------------------------------------------------------------------

% 使用 hyperref 在 pdf 簡介欄裡填入相關資料
\newcommand{\FillInPDFData}
{
  \ifx \hypersetup \undefined
    % do nothing
    \relax
  \else
    \ifx \GetChiTitle \undefined
      \hypersetup
      {
        pdftitle  = {\GetEngTitle},
        pdfauthor = {\GetAuthorEngName},
      }
    \else
      \hypersetup
      {
        pdftitle  = {\GetEngTitle\ (\GetChiTitle)},
        pdfauthor = {\GetAuthorEngName\ (\GetAuthorChiName)},
      }
    \fi

    \hypersetup
    {
      unicode     = true,
      pdfcreator  = {\GetUniversityEngName},
%      pdfproducer = {\GetUniversityEngName},
      pdfsubject  = {},
    }

    \ifthenelse{\equal{\GetPDFKeywords}{\empty}}{}{%
      \hypersetup{pdfkeywords = {\GetPDFKeywords}}}
  \fi
} % End of \newcommand{}

% ----------------------------------------------------------------------------


% Helper function for different page or chapter
%
% This file is part of the project of
% National Cheng Kung University (NCKU) Thesis/Dissertation Template in LaTex.
% This project is hold at
%     <https://github.com/wengan-li/ncku-thesis-template-latex>
% by Wen-Gan Li.
%
% This project is distributed in the hope of usefuling to someone,
% you can redistribute it and/or modify it under the terms of the
% Attribution-NonCommercial-ShareAlike 4.0 International.
%
% You should have received a copy of the
% Attribution-NonCommercial-ShareAlike 4.0 International
% along with this project.
% If not, see <http://creativecommons.org/licenses/by-nc-sa/4.0/legalcode.txt>.
%
% Please feel free to fork it, modify it, and try it.
% Have fun !!!
%

% Some helper function about thesis

% ----------------------------------------------------------------------------

% Variable
\newcommand{\VarDemoModeOn}{1}
\newcommand{\VarDemoModeOff}{0}
\newcommand{\ValueDemoMode}{\VarDemoModeOff}
\newcommand{\GetDemoMode}{\ValueDemoMode}

% Mode
\newcommand{\DemoMode}{\renewcommand{\ValueDemoMode}{\VarDemoModeOn}}

% ----------------------------------------------------------------------------

\newcommand{\BeginThesis}
{
  % Start of paper
  \begin{document}
} % End of \newcommand{}

\newcommand{\EndThesis}
{
  % End of paper
  \end{document}
} % End of \newcommand{}

\newcommand{\CreateThesis}
{
  % Start of thesis
  \BeginThesis
  %
  % 內容 context
  \ifthenelse{\GetDemoMode = \VarDemoModeOn}%
  {% ------------------------------------------------

% 封面內頁 Inner Cover
%
% 封面: 顯示所有封面內容, 但沒有學校Logo)
%     主要用在印刷版, 如精裝版 或 平裝版
%     (使用cover.tex來產生)
%
% 內頁: 顯示所有封面內容, 但有學校Logo
%     主要用在電子版 + 印刷版
%
% 只要是印刷版, 不論是精裝版或平裝版, 都是 封面 (殼/皮) + 內頁.
% 只有在電子版時, 第一頁就是封面內頁.

\DisplayInnerCover

% ------------------------------------------------

% 口試証明文件
\DisplayOral

% ------------------------------------------------

% 摘要 Abstract
% 除了外籍生, 本地生和僑生都是要編寫中文和英文摘要
% 論文以中文撰寫須以英文補寫 800 至 1200 字數的英文延伸摘要 (Extended Abstract)
% 詳細可看附件的學校要求或看example中的英文延伸摘要

\input{./example/abstract/chi}             % 中文版
\input{./example/abstract/eng}             % 英文版
%\input{./example/abstract/extended}        % 英文延伸摘要

% ------------------------------------------------

% 誌謝 Acknowledgments
% 誌謝正常應該只要寫一種版本就可,
% 提供2種以自行選擇所顯示的語言.
% 2種同時編寫都是可以的.

%\input{./example/acknowledgments/chi}             % 中文版
\input{./example/acknowledgments/eng}             % 英文版

% ------------------------------------------------

% 目錄 (內容, 圖表和圖片) Index of contents, tables and figures.
% 內容會自動產生 The indices will generate in automate.
\DisplayIndex                 % 顯示索引
\DisplayTablesIndex   % 顯示表格索引
\DisplayFiguresIndex  % 顯示圖片索引

% ------------------------------------------------

% Introduction section
\input{./example/introduction/introduction}

% Objective section
\input{./example/objective/objective}

% How-to-use section
\input{./example/how-to/use/use}

% How-to-write section
\input{./example/how-to/write/write}

% Words from teacher section
\input{./example/words-from-teacher/words-from-teacher}

% ------------------------------------------------

% 參考文獻 References
\input{./example/references/references}

% ------------------------------------------------

% 附錄 Appendix
\input{./example/appendix/appendix}

% ------------------------------------------------
}%
  {% ------------------------------------------------

% 封面內頁 Inner Cover
%
% 封面: 顯示所有封面內容, 但沒有學校Logo)
%     主要用在印刷版, 如精裝版 或 平裝版
%     (使用cover.tex來產生)
%
% 內頁: 顯示所有封面內容, 但有學校Logo
%     主要用在電子版 + 印刷版
%
% 只要是印刷版, 不論是精裝版或平裝版, 都是 封面 (殼/皮) + 內頁.
% 只有在電子版時, 第一頁就是封面內頁.

\DisplayInnerCover

% ------------------------------------------------

% 口試証明文件
\DisplayOral

% ------------------------------------------------

% 摘要 Abstract
% 除了外籍生, 本地生和僑生都是要編寫中文和英文摘要
% 論文以中文撰寫須以英文補寫 800 至 1200 字數的英文延伸摘要 (Extended Abstract)
% 詳細可看附件的學校要求或看example中的英文延伸摘要

\input{./example/abstract/chi}             % 中文版
\input{./example/abstract/eng}             % 英文版
%\input{./example/abstract/extended}        % 英文延伸摘要

% ------------------------------------------------

% 誌謝 Acknowledgments
% 誌謝正常應該只要寫一種版本就可,
% 提供2種以自行選擇所顯示的語言.
% 2種同時編寫都是可以的.

%\input{./example/acknowledgments/chi}             % 中文版
\input{./example/acknowledgments/eng}             % 英文版

% ------------------------------------------------

% 目錄 (內容, 圖表和圖片) Index of contents, tables and figures.
% 內容會自動產生 The indices will generate in automate.
\DisplayIndex                 % 顯示索引
\DisplayTablesIndex   % 顯示表格索引
\DisplayFiguresIndex  % 顯示圖片索引

% ------------------------------------------------

% Introduction section
\input{./example/introduction/introduction}

% Objective section
\input{./example/objective/objective}

% How-to-use section
\input{./example/how-to/use/use}

% How-to-write section
\input{./example/how-to/write/write}

% Words from teacher section
\input{./example/words-from-teacher/words-from-teacher}

% ------------------------------------------------

% 參考文獻 References
\input{./example/references/references}

% ------------------------------------------------

% 附錄 Appendix
\input{./example/appendix/appendix}

% ------------------------------------------------
}
  %
  % End of thesis
  \EndThesis
} % End of \newcommand{}

% ----------------------------------------------------------------------------

%
% This file is part of ncku-thesis-template.
%
% ncku-thesis-template is distributed in the hope of usefuling to someone,
% you can redistribute it and/or modify
% it under the terms of the Attribution-NonCommercial-ShareAlike
% 4.0 International.
%
% You should have received a copy of the
% Attribution-NonCommercial-ShareAlike 4.0 International
% along with ncku-thesis-template.
%
% If not, see <http://creativecommons.org/licenses/by-nc-sa/4.0/legalcode.txt>.
%

% Some helper function about page

% ----------------------------------------------------------------------------

\newcommand{\StartNewPage}
{
  % Page start
  \newpage
  \phantomsection
} % End of \newcommand{}

\newcommand{\EndOfPage}
{
  % End of page
  \clearpage
} % End of \newcommand{}

% ----------------------------------------------------------------------------


%
% This file is part of ncku-thesis-template.
%
% ncku-thesis-template is distributed in the hope of usefuling to someone,
% you can redistribute it and/or modify
% it under the terms of the Attribution-NonCommercial-ShareAlike
% 4.0 International.
%
% You should have received a copy of the
% Attribution-NonCommercial-ShareAlike 4.0 International
% along with ncku-thesis-template.
%
% If not, see <http://creativecommons.org/licenses/by-nc-sa/4.0/legalcode.txt>.
%

% Some helper function use in cover

% ----------------------------------------------------------------------------
\newcommand{\StartCover}
{
  \StartNewPage
  %
  % 設定使用 無頁碼
  \thispagestyle{empty}
  %
  \EnableCoverPageStyle
  %
  % Set the line spacing to single for the titles (to compress the lines)
  \renewcommand{\baselinestretch}{1}   %行距 1 倍
  %
  % Aligned to the center of the page
  \begin{center}
} % End of \newcommand{}

\newcommand{\EndCover}
{
  % End of alignment
  \end{center}
  \DisableCoverPageStyle
  \EndOfPage
} % End of \newcommand{}
% ----------------------------------------------------------------------------

% --- University name 學校名字 ---
% 基本上是寫死, 但是如果是別校的人, 可直接使用\SetSchoolName來修改
\newcommand\univCname{國立成功大學}           % Default
\newcommand\univEname{National Cheng Kung University} % Default
\newcommand{\SetSchoolChiName}[1]{\renewcommand{\univCname}{#1}}
\newcommand{\SetSchoolEngName}[1]{\renewcommand{\univEname}{#1}}
\newcommand{\SetSchoolName}[2]
{
  \SetSchoolChiName{#1}
  \SetSchoolEngName{#2}
} % End of \newcommand{}

\newcommand{\GetSchoolChiName}{\univCname}
\newcommand{\GetSchoolEngName}{\univEname}
% ----------------------------------------------------------------------------

% --- Chinese / English title 中英文論文題目 ---
\newcommand{\cTitle}{Chinese Title Here} % Default
\newcommand{\eTitle}{English Title Here} % Default
\newcommand{\SetChiTitle}[1]{\renewcommand{\cTitle}{#1}}
\newcommand{\SetEngTitle}[1]{\renewcommand{\eTitle}{#1}}
\newcommand{\SetTitle}[2]
{
  \SetChiTitle{#1}
  \SetEngTitle{#2}
} % End of \newcommand{}

\newcommand{\GetChiTitle}{\cTitle}
\newcommand{\GetEngTitle}{\eTitle}
% ----------------------------------------------------------------------------

% --- User's name 使用者名字 ---
\newcommand{\myCname}{你的名字}     % Default
\newcommand{\myEname}{Your name}   % Default
\newcommand{\SetMyChiName}[1]{\renewcommand{\myCname}{#1}}
\newcommand{\SetMyEngName}[1]{\renewcommand{\myEname}{#1}}
\newcommand{\SetMyName}[2]
{
  \SetMyChiName{#1}
  \SetMyEngName{#2}
} % End of \newcommand{}

\newcommand{\GetAuthorChiName}{\myCname}
\newcommand{\GetAuthorEngName}{\myEname}

% ----------------------------------------------------------------------------

% --- Degree name 學位 ---
% thesis 是指論文的通稱
% dissertation 指的是博士的論文

% 碩士論文  Master's thesis
% 博士論文  Doctoral dissertation

\newcommand{\ValueDegreeMaster}{0}
\newcommand{\ValueDegreePhd}{1}
\newcommand{\FlagDegreeType}{\ValueDegreePhd} % Default
\newcommand{\GetFlagDegreeType}{\FlagDegreeType}
\newcommand{\SetFlagDegreeType}[1]{\renewcommand{\FlagDegreeType}{#1}}

\newcommand{\degreeCname}{碩士/博士} % Default
\newcommand{\degreeEname}{Master / Doctor} % Default
\newcommand{\degreeThesisEname}{Master's Thesis / Doctoral Dissertation} % Default

\newcommand{\GetChiDegree}{\degreeCname}
\newcommand{\GetEngDegree}{\degreeEname}
\newcommand{\GetEngDegreeThesis}{\degreeThesisEname}
\newcommand{\SetChiDegree}[1]{\renewcommand{\degreeCname}{#1}}
\newcommand{\SetEngDegree}[1]{\renewcommand{\degreeEname}{#1}}
\newcommand{\SetEngDegreeThesis}[1]{\renewcommand{\degreeThesisEname}{#1}}

\newcommand{\PhdDegree}
{
  \SetFlagDegreeType{\ValueDegreePhd}
  \SetChiDegree{博士}
  \SetEngDegree{Doctor}
  \SetEngDegreeThesis{Doctoral Dissertation}
} % End of \newcommand{}

\newcommand{\MasterDegree}
{
  \SetFlagDegreeType{\ValueDegreeMaster}
  \SetChiDegree{碩士}
  \SetEngDegree{Master}
  \SetEngDegreeThesis{Master's Thesis}
} % End of \newcommand{}

% ----------------------------------------------------------------------------

% --- Date 日期 ---

% --- 論文的日期 ---
\newcommand{\ThesisYear}{2014}  % Default
\newcommand{\ThesisMonth}{1}    % Default

\newcommand{\SetThesisDate}[2]{\SetThesisDate{#1}{#2}} % For backporting
\newcommand{\SetCoverDate}[2]
{
  \SetThesisTaiwanYear{#1}
  \renewcommand{\ThesisYear}{#1}
  \renewcommand{\ThesisMonth}{#2}
} % End of \newcommand{}

\newcommand{\GetThesisYear}{\ThesisYear}
\newcommand{\GetThesisYearInTaiwanYear}{\ThesisTaiwanYearResult}
\newcommand{\GetThesisMonth}{\ThesisMonth}
\newcommand{\GetThesisMonthInEng}{\GetMonthInEng{\ThesisMonth}}

% ---  口試的日期 ---
\newcommand{\OralChiYear}{101}      % Default
\newcommand{\OralChiMonth}{1}       % Default
\newcommand{\OralChiDay}{1}         % Default
\newcommand{\OralEngYear}{2014}     % Default
\newcommand{\OralEngMonth}{January} % Default
\newcommand{\OralEngDay}{1}         % Default

\newcommand{\GetOralChiYear}{\OralChiYear}
\newcommand{\GetOralYearInTaiwanYear}
{\SetThesisTaiwanYear{\OralEngYear}\ThesisTaiwanYearResult}
\newcommand{\GetOralChiMonth}{\OralChiMonth}
\newcommand{\GetOralChiDay}{\OralChiDay}
\newcommand{\GetOralEngYear}{\OralEngYear}
\newcommand{\GetOralEngMonth}{\OralEngMonth}
\newcommand{\GetOralEngDay}{\OralEngDay}

\newcommand{\SetOralChiDate}[3]
{
  \SetOralTaiwanYear{#1}
  \renewcommand{\OralChiYear}{\OralTaiwanYearResult}
  \renewcommand{\OralChiMonth}{#2}
  \renewcommand{\OralChiDay}{#3}
} % End of \newcommand{}

\newcommand{\SetOralEngDate}[3]
{
  \renewcommand{\OralEngYear}{#1}
  \renewcommand{\OralEngMonth}{\GetMonthInEng{#2}}
  \renewcommand{\OralEngDay}{#3}
} % End of \newcommand{}

\newcommand{\SetOralDate}[3]
{
  \SetOralChiDate{#1}{#2}{#3}
  \SetOralEngDate{#1}{#2}{#3}
} % End of \newcommand{}

% ----------------------------------------------------------------------------

% --- 學院 College, 系所 Department and Institute ---

% --------------------------- College ---------------------------
\newcommand{\collCname}{學院 C}
\newcommand{\collEname}{College of C}
\newcommand{\SetCollChiName}[1]{\renewcommand{\collCname}{#1}}
\newcommand{\SetCollEngName}[1]{\renewcommand{\collEname}{#1}}
\newcommand{\SetCollName}[2]
{
  \SetCollChiName{#1}
  \SetCollEngName{#2}
} % End of \newcommand{}

\newcommand{\GetCollChiName}{\collCname}
\newcommand{\GetCollEngName}{\collEname}

% --------------------------- Department ---------------------------
\newcommand{\deptCname}{A 系 / 所}
%\newcommand{\deptEname}{DeptA} % Short form of department
\newcommand{\fulldeptEname}{Department / Insitute A} % Full name of department
\newcommand{\SetDeptChiName}[1]{\renewcommand{\deptCname}{#1}}
%\newcommand{\SetDeptEngShortName}[1]{\renewcommand{\deptEname}{#1}}
\newcommand{\SetDeptEngName}[1]{\renewcommand{\fulldeptEname}{#1}}
\newcommand{\SetDeptName}[3]
{
  \SetDeptChiName{#1}
%  \SetDeptEngShortName{#2}
  \SetDeptEngName{#3}
} % End of \newcommand{}

\newcommand{\GetDeptChiName}{\deptCname}
\newcommand{\GetDeptEngName}{\fulldeptEname}

% ----------------------------------------------------------------------------

% --- 指導老師 Advisor(s) ---
% 在封面上預算了最多3位的空間
% 中文名字固定以 博士 結尾
% 英文名字固定以 Dr. 開頭

\newcommand{\advisorCnameA}{X}
\newcommand{\advisorEnameA}{X}
\newcommand{\advisorCnameB}{}
\newcommand{\advisorEnameB}{}
\newcommand{\advisorCnameC}{}
\newcommand{\advisorEnameC}{}

\newcommand{\GetAdvisorChiNameA}{\advisorCnameA}
\newcommand{\GetAdvisorEngNameA}{\advisorEnameA}
\newcommand{\GetAdvisorChiNameB}{\advisorCnameB}
\newcommand{\GetAdvisorEngNameB}{\advisorEnameB}
\newcommand{\GetAdvisorChiNameC}{\advisorCnameC}
\newcommand{\GetAdvisorEngNameC}{\advisorEnameC}

\newcommand{\SetAdvisorChiNameA}[1]{\renewcommand{\advisorCnameA}{#1}}
\newcommand{\SetAdvisorEngNameA}[1]{\renewcommand{\advisorEnameA}{#1}}
\newcommand{\SetAdvisorChiNameB}[1]{\renewcommand{\advisorCnameB}{#1}}
\newcommand{\SetAdvisorEngNameB}[1]{\renewcommand{\advisorEnameB}{#1}}
\newcommand{\SetAdvisorChiNameC}[1]{\renewcommand{\advisorCnameC}{#1}}
\newcommand{\SetAdvisorEngNameC}[1]{\renewcommand{\advisorEnameC}{#1}}

\newcommand{\SetAdvisorNameA}[2]
{
  \SetAdvisorChiNameA{#1}
  \SetAdvisorEngNameA{#2}
} % End of \newcommand{}

\newcommand{\SetAdvisorNameB}[2]
{
  \SetAdvisorChiNameB{#1}
  \SetAdvisorEngNameB{#2}
} % End of \newcommand{}

\newcommand{\SetAdvisorNameC}[2]
{
  \SetAdvisorChiNameC{#1}
  \SetAdvisorEngNameC{#2}
} % End of \newcommand{}

% ----------------------------------------------------------------------------

% Use to create cover
\newcommand{\CreateCover}%
{
  \begin{document}
  % ------------------------------------------------

% 請根據你的需求去選用中文或英文封面
\input{./cover/cover-chi.tex}
\input{./cover/cover-chi-2.tex}
\input{./cover/cover-eng.tex}

% ------------------------------------------------

  \end{document}
} % End of \newcommand{}

% Use to include and display inner cover
\newcommand{\DisplayInnerCover}{%
% This file is part of the project of
% National Cheng Kung University (NCKU) Thesis/Dissertation Template in LaTex.
% This project is hold at
%     <https://github.com/wengan-li/ncku-thesis-template-latex>
% by Wen-Gan Li.
%
% This project is distributed in the hope of usefuling to someone,
% you can redistribute it and/or modify it under the terms of the
% Attribution-NonCommercial-ShareAlike 4.0 International.
%
% You should have received a copy of the
% Attribution-NonCommercial-ShareAlike 4.0 International
% along with this project.
% If not, see <http://creativecommons.org/licenses/by-nc-sa/4.0/legalcode.txt>.
%
% Please feel free to fork it, modify it, and try it.
% Have fun !!!
%

% ------------------------------------------------

% 根據user的需求去選用中文或英文封面內頁
%
% 封面: 顯示所有封面內容, 但沒有學校Logo
% 內頁: 顯示所有封面內容, 但有學校Logo
% 不論是精裝版或平裝版都是 封面 (殼/皮) + 內頁
%
\if \GetDisplayCoverLang \ValueDisplayCoverLangEng
  \input{./template/cover/cover-eng}
\else
  \input{./template/cover/cover-chi}
\fi

% ------------------------------------------------
}

% ----------------------------------------------------------------------------

\newcommand{\ValueDisplayCoverLangEng}{0}
\newcommand{\ValueDisplayCoverLangChi}{1}
\newcommand{\VarDisplayCoverLang}{\ValueDisplayCoverLangEng}
\newcommand{\GetDisplayCoverLang}{\VarDisplayCoverLang}
\newcommand{\DisplayCoverInChi}{\renewcommand{\VarDisplayCoverLang}{\ValueDisplayCoverLangChi}}
\newcommand{\DisplayCoverInEng}{\renewcommand{\VarDisplayCoverLang}{\ValueDisplayCoverLangEng}}

% ----------------------------------------------------------------------------

% Display Chinese and English name in english cover
\newcommand{\CoverDisplayNameChiEng}{0} % Default not display both
\newcommand{\SetCDBothName}{\renewcommand{\CoverDisplayNameChiEng}{1}}
\newcommand{\GetCDBothName}{\CoverDisplayNameChiEng}
\newcommand{\CDBothName}{\SetCDBothName}

% ----------------------------------------------------------------------------

% 顯示 '(初稿)' (中文版) 和 '(Draft)' (英文版) 在封面
\newcommand{\GetTextDraftChi}{(初稿)}
\newcommand{\GetTextDraftEng}{(Draft)}
\newcommand{\VarCoverDisplayDraft}{0} % Don't display in default
\newcommand{\EnableFlagDisplayDraft}{\renewcommand{\VarCoverDisplayDraft}{1}}
\newcommand{\DisplayDraft}{\EnableFlagDisplayDraft}
\newcommand{\GetFlagDisplayDraft}{\VarCoverDisplayDraft}
% ----------------------------------------------------------------------------

%
% This file is part of ncku-thesis-template.
%
% ncku-thesis-template is distributed in the hope of usefuling to someone,
% you can redistribute it and/or modify
% it under the terms of the Attribution-NonCommercial-ShareAlike
% 4.0 International.
%
% You should have received a copy of the
% Attribution-NonCommercial-ShareAlike 4.0 International
% along with ncku-thesis-template.
%
% If not, see <http://creativecommons.org/licenses/by-nc-sa/4.0/legalcode.txt>.
%

% Some helper function about chapter and section

% ----------------------------------------------------------------------------

% Chapters
\DeclareDocumentCommand{\StartChapter}{+m +g}
{%
  \chapter{#1}%
  \IfNoValueF{#2}{\label{#2}}%
} % End of \DeclareDocumentCommand{}

\def \EndChapter {\EndOfPage}

% Section
\DeclareDocumentCommand{\StartSection}{+m +g}
{%
  \section{#1}%
  \IfNoValueF{#2}{\label{#2}}%
} % End of \DeclareDocumentCommand{}

% Sub-Section
\DeclareDocumentCommand{\StartSubSection}{+m +g}
{%
  \subsection{#1}%
  \IfNoValueF{#2}{\label{#2}}%
} % End of \DeclareDocumentCommand{}

% Sub-Sub-Section
\DeclareDocumentCommand{\StartSubSubSection}{+m +g}
{%
  \subsubsection{#1}%
  \IfNoValueF{#2}{\label{#2}}%
} % End of \DeclareDocumentCommand{}

% ----------------------------------------------------------------------------

% Chapter標題改使用 第1章 的方式來顯示
\newcommand\VarChapterTitleLangChi{1}
\newcommand\VarChapterTitleLangEng{0}
\newcommand\VarChapterTitleLang{\VarChapterTitleLangEng}
\newcommand\GetChapterTitleLang{\VarChapterTitleLang}
\newcommand{\ChapterTitleInChi}{\renewcommand{\VarChapterTitleLang}{\VarChapterTitleLangChi}}

% A wrapper to handle \ChapterSectionTitleInChi
\newcommand{\ChapterSectionTitleInChi}
{
  \ChapterTitleInChi
} % End of \newcommand{}

% ----------------------------------------------------------------------------

% 過去的API, 以 Error提醒不能再使用
\newcommand{\ChapterTitleNumInChi}{\errmessage{模版: 由v1.4.1開始, ChapterTitleNumInChi已不能再使用.}\stop}

% ----------------------------------------------------------------------------

%
% This file is part of ncku-thesis-template.
%
% ncku-thesis-template is distributed in the hope of usefuling to someone,
% you can redistribute it and/or modify
% it under the terms of the Attribution-NonCommercial-ShareAlike
% 4.0 International.
%
% You should have received a copy of the
% Attribution-NonCommercial-ShareAlike 4.0 International
% along with ncku-thesis-template.
%
% If not, see <http://creativecommons.org/licenses/by-nc-sa/4.0/legalcode.txt>.
%

% Some helper function use in abstract

% ----------------------------------------------------------------------------

% Abstract
\newcommand{\StartChiAbstract}{\StartAbstractChi}
\newcommand{\StartAbstractChi}
{
  \StartNewPage
  \chapter*{摘要}
  \addcontentsline{toc}{chapter}{摘要}
  \pagestyle{plain}
} % End of \newcommand{}

\newcommand{\StartAbstract}
{
  \StartNewPage
  \chapter*{Abstract}
  \addcontentsline{toc}{chapter}{Abstract}
  \pagestyle{plain}
} % End of \newcommand{}

\newcommand{\EndChiAbstract}{\EndAbstractChi}
\newcommand{\EndAbstractChi}
{
  % Keyword
  \ifthenelse{\equal{\GetAbstractChiKeywords}{\empty}}{}{%
      \par{\noindent \bf 關鍵字:} \GetAbstractChiKeywords}
  %
  \EndOfPage
} % End of \newcommand{}

\newcommand{\EndAbstract}
{
  % Keyword
  \ifthenelse{\equal{\GetAbstractEngKeywords}{\empty}}{}{%
      \par{\noindent \bf Keyword:} \GetAbstractEngKeywords}
  %
  \EndOfPage
} % End of \newcommand{}

% Extended Abstract
\newcommand{\StartExtendedAbstract}
{
  \StartNewPage

  % Set page
  \baselineskip=20pt
  \setlength{\parindent}{0.0pt}

  % Set style
  \pagestyle{empty}

  % 設定段落之間的距離
  \setlength{\parskip}{0.5cm}

  % Add to "Table of Contents"
  \addcontentsline{toc}{chapter}{Extended Abstract}

  % Add title
  \parbox{\textwidth}{\center \large \textbf{\eTitle}}
  \vspace{0.5cm}

  % Add name
  \centerline{\large \GetAuthorEngName}

  % Add names
  \centerline{Prof. \thinspace \GetAdvisorEngNameA}
  \ifthenelse{\equal{\GetAdvisorEngNameB}{\empty}}
    {}
    {\centerline{Prof. \thinspace \GetAdvisorEngNameB}}

  \ifthenelse{\equal{\GetAdvisorEngNameC}{\empty}}
    {}
    {\centerline{Prof. \thinspace \GetAdvisorEngNameC}}

  % Add department
  \centerline{\GetDeptEngName}
  \centerline{\GetCollEngName}
} % End of \newcommand{}

\newcommand{\EndExtendedAbstract}
{
  \EndOfPage

  % 設定段落之間的距離
  \setlength{\parskip}{0.3cm}

  % Reset page
  \setlength{\parindent}{1.5em}
  \baselineskip=26pt

  % Re-set style
  \pagestyle{plain}
} % End of \def{}

% Summary in Extended Abstract
\newcommand{\ExtAbstractSummary}[1]{
  \begin{framed}
    \ExtAbstractChapter{SUMMARY}{#1}

    % Keyword
    \par
    {\noindent \bf Keyword:} \GetAbstractEngKeywords
  \end{framed}
} % End of \newcommand{}

% Chapter in Extended Abstract
\newcommand{\ExtAbstractChapter}[2]
{
  \vspace{0.4cm}
  \centerline{\textbf{#1}}

  #2
} % End of \newcommand{}

% Sub-chapter in Extended Abstract
\newcommand{\ExtAbstractSection}[2]
{
  \textbf{#1}

  #2
} % End of \newcommand{}

% ----------------------------------------------------------------------------



% Some helper function about acknowledgments

% ----------------------------------------------------------------------------

% Acknowledgments
\newcommand{\StartAcknowledgments}[1]
{
  \StartNewPage

  % Add to "Table of Contents"
  \addcontentsline{toc}{chapter}{Acknowledgments}

  % Title
  \centerline{\Large \textbf{Acknowledgments}}

  \IfNoValueF{#1}{\label{#1}}
} % End of \newcommand{}

\newcommand{\StartAcknowledgmentsChi}[1]
{
  \StartNewPage

  % Add to "Table of Contents"
  \addcontentsline{toc}{chapter}{致謝}

  % Title
  \centerline{\Large \textbf{致謝}}

  \IfNoValueF{#1}{\label{#1}}
} % End of \newcommand{}

\def \EndAcknowledgments {\EndOfPage}

% ----------------------------------------------------------------------------



% Some helper function about appendix

% ----------------------------------------------------------------------------

\def \StartAppendix
{
  \StartNewPage
  % Appendix begins here (For one appendix chapter only)
  %\cleardoublepage
  \appendix
  %\clearpage
  %\setcounter{page}{\thepage}
  % If more than one appendix chapters
  %\appendices
}

\def \EndAppendix {\EndOfPage}

% ----------------------------------------------------------------------------



% Some function that use for school
% ----------------------------------------------------------------------
% -------------------------- 未完成 UNFINISH ----------------------------
% ----------------------------------------------------------------------

% ----------------------------------------------------------------------------
% The list of all department in NCKU

% Use the list from:
%   http://web.ncku.edu.tw/files/11-1000-182.php

% 所有學院跟系所的設定.
%
% 縮寫是靠學校給的Domain name所得出的, 故可能會有錯誤的時候.
% 所以如果錯了的話, 就請告知真正的寫法(或縮寫)是什麼.
%
% ----------------------------------------------------------------------------

% ----------------------------------------------------------------------------
%                       系所 Department and Institute 
% ----------------------------------------------------------------------------

% ----------------------------------------------------------------------------
% --------------------- 文學院 College of Liberal Arts ---------------------
% ----------------------------------------------------------------------------

% 中國文學系 Department of Chinese Literature
\newcommand{\SetDeptChinese}[0]
{
  \SetDeptName{中國文學系}{Chinese}{Department of Chinese Literature}
  \CollLiberalArts
} % End of \newcommand{}

% 藝術研究所 Institute of Art
\newcommand{\SetDeptArt}[0]
{
  \SetDeptName{藝術研究所}{Art}{Institute of Art}
  \CollLiberalArts
} % End of \newcommand{}

% 閩南文化研究中心 Min-Nan Culture Studies Center
\newcommand{\SetDeptMinNan}[0]
{
  \SetDeptName{閩南文化研究中心}{MinNan}{Min-Nan Culture Studies Center}
  \CollLiberalArts
} % End of \newcommand{}

% 外國語文學系 Department of Foreign Languages and Literature
\newcommand{\SetDeptFLLD}[0]
{
  \SetDeptName{外國語文學系}{FLLD}{Department of Foreign Languages and Literature}
  \CollLiberalArts
} % End of \newcommand{}

% 臺灣文學系 Department of Taiwanese Literature
\newcommand{\SetDeptTWL}[0]
{
  \SetDeptName{臺灣文學系}{TWL}{Department of Taiwanese Literature}
  \CollLiberalArts
} % End of \newcommand{}

% 華語中心 Chinese Language Center
\newcommand{\SetDeptKCLC}[0]
{
  \SetDeptName{華語中心}{KCLC}{Chinese Language Center}
  \CollLiberalArts
} % End of \newcommand{}

% 外語中心 Foreign Language Center
\newcommand{\SetDeptLang}[0]
{
  \SetDeptName{外語中心}{Lang}{Foreign Language Center}
  \CollLiberalArts
} % End of \newcommand{}

% 歷史學系 Department of History
\newcommand{\SetDeptHis}[0]
{
  \SetDeptName{歷史學系}{His}{Department of History}
  \CollLiberalArts
} % End of \newcommand{}

% ----------------------------------------------------------------------------
% --------------------- 理學院 College of Sciences ---------------------
% ----------------------------------------------------------------------------

% 數學系 Department of Mathematics
\newcommand{\SetDeptMath}[0]
{
  \SetDeptName{數學系}{Math}{Department of Mathematics}
  \CollSciences
} % End of \newcommand{}

% 光電科學與工程學系 Departmment of Photonics
\newcommand{\SetDeptDPS}[0]
{
  \SetDeptName{光電科學與工程學系}{DPS}{Departmment of Photonics}
  \CollSciences
} % End of \newcommand{}

% 物理學系 Department of Physics
\newcommand{\SetDeptPhys}[0]
{
  \SetDeptName{物理學系}{Phys}{Department of Physics}
  \CollSciences
} % End of \newcommand{}

% 化學系 Department of Chemistry
\newcommand{\SetDeptCh}[0]
{
  \SetDeptName{化學系}{Ch}{Department of Chemistry}
  \CollSciences
} % End of \newcommand{}

% 地球科學系 Department of Earth Sciences
\newcommand{\SetDeptEarth}[0]
{
  \SetDeptName{地球科學系}{Earth}{Department of Earth Sciences}
  \CollSciences
} % End of \newcommand{}

% 太空與電漿科學研究所 Institute of Space and Plasma Sciences
\newcommand{\SetDeptPSSC}[0]
{
  \SetDeptName{太空與電漿科學研究所}{PSSC}{Institute of Space and Plasma Sciences}
  \CollSciences
} % End of \newcommand{}

% 國家理論科學研究中心 National Center for Theoretical Sciences (South)
\newcommand{\SetDeptNCTS}[0]
{
  \SetDeptName{國家理論科學研究中心}{NCTS}{National Center for Theoretical Sciences (South)}
  \CollSciences
} % End of \newcommand{}

% ----------------------------------------------------------------------------
% --------------------- 工學院 College of Engineering ---------------------
% ----------------------------------------------------------------------------

% 機械工程學系 Engineering	Department of Mechanical Engineering
\newcommand{\SetDeptME}[0]
{
  \SetDeptName{機械工程學系}{ME}{Engineering	Department of Mechanical Engineering}
  \CollEngineering
} % End of \newcommand{}

% 化學工程學系 Department of Chemical Engineering
\newcommand{\SetDeptChe}[0]
{
  \SetDeptName{化學工程學系}{Che}{Department of Chemical Engineering}
  \CollEngineering
} % End of \newcommand{}

% 土木工程學系 Department of Civil Engineering
\newcommand{\SetDeptCivil}[0]
{
  \SetDeptName{土木工程學系}{Civil}{Department of Civil Engineering}
  \CollEngineering
} % End of \newcommand{}

% 材料科學及工程學系 Department of Materials Science and Engineering
\newcommand{\SetDeptMSE}[0]
{
  \SetDeptName{材料科學及工程學系}{MSE}{Department of Materials Science and Engineering}
  \CollEngineering
} % End of \newcommand{}

% 水利及海洋工程學系 Department of Hydraulic and Ocean Engineering
\newcommand{\SetDeptHyd}[0]
{
  \SetDeptName{水利及海洋工程學系}{Hyd}{Department of Hydraulic and Ocean Engineering}
  \CollEngineering
} % End of \newcommand{}

% 工程科學系 Department of Engineering Science
\newcommand{\SetDeptES}[0]
{
  \SetDeptName{工程科學系}{ES}{Department of Engineering Science}
  \CollEngineering
} % End of \newcommand{}

% 系統及船舶機電工程學系 Department of System and Naval Mechatronic Engineering
\newcommand{\SetDeptSNAME}[0]
{
  \SetDeptName{系統及船舶機電工程學系}{SNAME}{Department of System and Naval Mechatronic Engineering}
  \CollEngineering
} % End of \newcommand{}

% 航空太空工程學系 Department of Aeronautics and Astronautics
\newcommand{\SetDeptIAA}[0]
{
  \SetDeptName{航空太空工程學系}{IAA}{Department of Aeronautics and Astronautics}
  \CollEngineering
} % End of \newcommand{}

% 資源工程學系 Department of Resources Engineering
\newcommand{\SetDeptMP}[0]
{
  \SetDeptName{資源工程學系}{MP}{Department of Resources Engineering}
  \CollEngineering
} % End of \newcommand{}

% 環境工程學系 Department of Environmental Engineering
\newcommand{\SetDeptEV}[0]
{
  \SetDeptName{環境工程學系}{EV}{Department of Environmental Engineering}
  \CollEngineering
} % End of \newcommand{}

% 生物醫學工程學系 Department of BioMedical Engineering
\newcommand{\SetDeptBME}[0]
{
  \SetDeptName{生物醫學工程學系}{BME}{Department of BioMedical Engineering}
  \CollEngineering
} % End of \newcommand{}

% 測量及空間資訊學系 Department of Geomatics
\newcommand{\SetDeptGeomatics}[0]
{
  \SetDeptName{測量及空間資訊學系}{Geomatics}{Department of Geomatics}
  \CollEngineering
} % End of \newcommand{}

% 海洋科技與事務研究所 Institute of Ocean Technology and Marine Affairs
\newcommand{\SetDeptIOTMA}[0]
{
  \SetDeptName{海洋科技與事務研究所}{IOTMA}{Institute of Ocean Technology and Marine Affairs}
  \CollEngineering
} % End of \newcommand{}

% 民航研究所 Institute of Civil Aviation
\newcommand{\SetDeptICA}[0]
{
  \SetDeptName{民航研究所}{ICA}{Institute of Civil Aviation}
  \CollEngineering
} % End of \newcommand{}

% 能源國際學士學位學程 International Bachelor Degree Program on Energy
\newcommand{\SetDeptIBDPE}[0]
{
  \SetDeptName{能源國際學士學位學程}{IBDPE}{International Bachelor Degree Program on Energy}
  \CollEngineering
} % End of \newcommand{}

% 尖端材料國際碩士學位學程 International Curriculum for Advanced Materials Program
\newcommand{\SetDeptICAMP}[0]
{
  \SetDeptName{尖端材料國際碩士學位學程}{ICAMP}{International Curriculum for Advanced Materials Program}
  \CollEngineering
} % End of \newcommand{}

% 自然災害減災及管理國際碩士學位學程 International Graduate Program of Civil Engineering and Management
\newcommand{\SetDeptCivil}[0]
{
  \SetDeptName{自然災害減災及管理國際碩士學位學程}{Civil}{International Graduate Program of \\ Civil Engineering and Management}
  \CollEngineering
} % End of \newcommand{}

% 工程管理碩士在職專班 International Master Program on Natural Hazards Mitigation and Management
\newcommand{\SetDeptINHMM}[0]
{
  \SetDeptName{工程管理碩士在職專班}{INHMM}{International Master Program on \\ Natural Hazards Mitigation and Management}
  \CollEngineering
} % End of \newcommand{}

% ----------------------------------------------------------------------------
% ------ 電機資訊學院 College of Electrical Engineering & Computer Science ------
% ----------------------------------------------------------------------------

% 電機工程學系 Department of Electrical Engineering
\newcommand{\SetDeptEE}[0]
{
  \SetDeptName{電機工程學系}{EE}{Department of Electrical Engineering}
  \CollElectricalEngineeringAndComputerScience
} % End of \newcommand{}

% 資訊工程學系 Department of Computer Science and Information Engineering
\newcommand{\SetDeptCSIE}[0]
{
  \SetDeptName{資訊工程學系}{CSIE}{Department of Computer Science and Information Engineering}
  \CollElectricalEngineeringAndComputerScience
} % End of \newcommand{}

% 微電子工程研究所 Institute of Microelectronics
\newcommand{\SetDeptIME}[0]
{
  \SetDeptName{微電子工程研究所}{IME}{Institute of Microelectronics}
  \CollElectricalEngineeringAndComputerScience
} % End of \newcommand{}

% 電腦與通信工程研究所 Institute of Computer & Communication Engineering
\newcommand{\SetDeptCCE}[0]
{
  \SetDeptName{電腦與通信工程研究所}{CCE}{Institute of Computer & Communication Engineering}
  \CollElectricalEngineeringAndComputerScience
} % End of \newcommand{}

% 製造資訊與系統研究所 Institute of Manufacturing Information and Systems
\newcommand{\SetDeptIMIS}[0]
{
  \SetDeptName{製造資訊與系統研究所}{IMIS}{Institute of Manufacturing Information and Systems}
  \CollElectricalEngineeringAndComputerScience
} % End of \newcommand{}

% 醫學資訊研究所 Institute of Medical Informatics
\newcommand{\SetDeptIMI}[0]
{
  \SetDeptName{醫學資訊研究所}{IMI}{Institute of Medical Informatics}
  \CollElectricalEngineeringAndComputerScience
} % End of \newcommand{}

% ----------------------------------------------------------------------------
% --------------------- 管理學院 College of Management ---------------------
% ----------------------------------------------------------------------------

% 統計學系 Department of Statistics
\newcommand{\SetDeptSTAT}[0]
{
  \SetDeptName{統計學系}{STAT}{Department of Statistics}
  \CollManagement
} % End of \newcommand{}

% 會計學系 Department of Accountancy
\newcommand{\SetDeptACC}[0]
{
  \SetDeptName{會計學系}{ACC}{Department of Accountancy}
  \CollManagement
} % End of \newcommand{}

% 交通管理科學系 Department of Transportation and Communication Management Science
\newcommand{\SetDeptTCM}[0]
{
  \SetDeptName{交通管理科學系}{TCM}{Department of Transportation and Communication Management Science}
  \CollManagement
} % End of \newcommand{}

% 企業管理學系暨國際企業研究所 Department of Business Administration and Graduate Institute of International Business
\newcommand{\SetDeptBA}[0]
{
  \SetDeptName{企業管理學系暨國際企業研究所}{BA}{Department of Business Administration \\ and Graduate Institute of International Business}
  \CollManagement
} % End of \newcommand{}

% 電信管理研究所 Institute of Telecommunications Management
\newcommand{\SetDeptTM}[0]
{
  \SetDeptName{電信管理研究所}{TM}{Institute of Telecommunications Management}
  \CollManagement
} % End of \newcommand{}

% 工業與資訊管理學系暨資訊管理研究所 Institute of Information Management
\newcommand{\SetDeptIIM}[0]
{
  \SetDeptName{工業與資訊管理學系暨資訊管理研究所}{IIM}{Institute of Information Management}
  \CollManagement
} % End of \newcommand{}

% 財務金融研究所 Institute of Finance & Banking
\newcommand{\SetDeptFin}[0]
{
  \SetDeptName{財務金融研究所}{Fin}{Institute of Finance & Banking}
  \CollManagement
} % End of \newcommand{}

% 體育健康與休閒研究所 Institute of Physical Education, Health & Leisure Studies
\newcommand{\SetDeptPHEI}[0]
{
  \SetDeptName{體育健康與休閒研究所}{PHEI}{Institute of Physical Education, \\ Health & Leisure Studies}
  \CollManagement
} % End of \newcommand{}

% 高階管理碩士在職專班 (EMBA) Executive Master of Business Administration
\newcommand{\SetDeptEMBA}[0]
{
  \SetDeptName{高階管理碩士在職專班}{EMBA}{Executive Master of Business Administration}
  \CollManagement
} % End of \newcommand{}

% 國際經營管理研究所 (IMBA) Institute of International Management (IMBA)
\newcommand{\SetDeptIMBA}[0]
{
  \SetDeptName{國際經營管理研究所}{IMBA}{Institute of International Management}
  \CollManagement
} % End of \newcommand{}

% 經營管理碩士班 (AMBA) Advanced Master of Business Administration
\newcommand{\SetDeptAMBA}[0]
{
  \SetDeptName{經營管理碩士班}{AMBA}{Advanced Master of Business Administration}
  \CollManagement
} % End of \newcommand{}

% ----------------------------------------------------------------------------
% ------------------- 社會科學院 College of Social Science -------------------
% ----------------------------------------------------------------------------

% 政治學系	Department of Political Science
\newcommand{\SetDeptPolSci}[0]
{
  \SetDeptName{政治學系}{PolSci}{Department of Political Science}
  \CollSocialScience
} % End of \newcommand{}

% 經濟學系 Department of Economics
\newcommand{\SetDeptEconomic}[0]
{
  \SetDeptName{經濟學系}{Economic}{Department of Economics}
  \CollSocialScience
} % End of \newcommand{}

% 心理學系 Department of Psychology
\newcommand{\SetDeptPsychology}[0]
{
  \SetDeptName{心理學系}{Psychology}{Department of Psychology}
  \CollSocialScience
} % End of \newcommand{}

% 法律學系 Department of Law and Institute of Law in Science and Technology
\newcommand{\SetDeptLaw}[0]
{
  \SetDeptName{法律學系}{Law}{Department of Law and Institute of Law in Science and Technology}
  \CollSocialScience
} % End of \newcommand{}

% 教育研究所 Institute of Education
\newcommand{\SetDeptED}[0]
{
  \SetDeptName{教育研究所}{ED}{Institute of Education}
  \CollSocialScience
} % End of \newcommand{}

% 認知科學研究所 Institute of Cognitive Science
\newcommand{\SetDeptIOCS}[0]
{
  \SetDeptName{認知科學研究所}{IOCS}{Institute of Cognitive Science}
  \CollSocialScience
} % End of \newcommand{}

% 政治經濟學研究所 Institute of Political Economy
\newcommand{\SetDeptGIPE}[0]
{
  \SetDeptName{政治經濟學研究所}{GIPE}{Institute of Political Economy}
  \CollSocialScience
} % End of \newcommand{}

% 心智影像研究中心 Mind Research and Image Center
\newcommand{\SetDeptFMRI}[0]
{
  \SetDeptName{心智影像研究中心}{FMRI}{Mind Research and Image Center}
  \CollSocialScience
} % End of \newcommand{}

% ----------------------------------------------------------------------------
% ---------------- 規劃與設計學院 College of Planning & Design ----------------
% ----------------------------------------------------------------------------

% 建築學系 Department of Architecture
\newcommand{\SetDeptArch}[0]
{
  \SetDeptName{建築學系}{Arch}{Department of Architecture}
  \CollPlanningAndDesign
} % End of \newcommand{}

% 都市計劃學系 Department of Urban Planning
\newcommand{\SetDeptUP}[0]
{
  \SetDeptName{都市計劃學系}{UP}{Department of Urban Planning}
  \CollPlanningAndDesign
} % End of \newcommand{}

% 工業設計學系 Department of Industrial Design
\newcommand{\SetDeptID}[0]
{
  \SetDeptName{工業設計學系}{ID}{Department of Industrial Design}
  \CollPlanningAndDesign
} % End of \newcommand{}

% 創意產業設計研究所 Institute of Creative Industry Design
\newcommand{\SetDeptICID}[0]
{
  \SetDeptName{創意產業設計研究所}{ICID}{Institute of Creative Industry Design}
  \CollPlanningAndDesign
} % End of \newcommand{}

% ----------------------------------------------------------------------------
% ---------- 生物科學與科技學院 College of Bioscience & Biotechnology ----------
% ----------------------------------------------------------------------------

% 生命科學系 Bioscience & Biotechnology	Department of Life Sciences
\newcommand{\SetDeptBio}[0]
{
  \SetDeptName{生命科學系}{Bio}{Bioscience & Biotechnology	Department of Life Sciences}
  \CollBioscienceAndBiotechnology
} % End of \newcommand{}

% 生物科技研究所 Institute of Biotechnology
\newcommand{\SetDeptBioTech}[0]
{
  \SetDeptName{生物科技研究所}{BioTech}{Institute of Biotechnology}
  \CollBioscienceAndBiotechnology
} % End of \newcommand{}

% 生物資訊與訊息傳遞研究所 Institute of Bioinformatics and Biosignal Transduction
\newcommand{\SetDeptIBBT}[0]
{
  \SetDeptName{生物資訊與訊息傳遞研究所}{IBBT}{Institute of Bioinformatics and Biosignal Transduction}
  \CollBioscienceAndBiotechnology
} % End of \newcommand{}

% 熱帶植物科學研究所 Institute of Tropical Plant Sciences
\newcommand{\SetDeptITPS}[0]
{
  \SetDeptName{熱帶植物科學研究所}{ITPS}{Institute of Tropical Plant Sciences}
  \CollBioscienceAndBiotechnology
} % End of \newcommand{}

% ----------------------------------------------------------------------------
% --------------------- 醫學院 College of Medicine ---------------------
% ----------------------------------------------------------------------------

% 醫學系
% 護理學系
% 生物化學暨分子生物學研究所
% 病理學科
% 內科學科
% 醫學檢驗生物技術學系
% 生理學研究所
% 外科學科
% 小兒學科
% 物理治療學系
% 解剖學科暨細胞生物與解剖學研究所
% 婦產學科
% 骨科學科
% 職能治療學系
% 公共衛生學科暨公共衛生研究所
% 神經學科
% 精神學科
% 藥學系
% 寄生蟲學科
% 眼科學科
% 耳鼻喉學科
% 基礎醫學研究所
% 工業衛生學科暨環境醫學研究所
% 皮膚學科
% 泌尿學科
% 行為醫學研究所
% 藥理學科暨藥理學研究所
% 麻醉學科
% 復健學科
% 臨床藥學與藥物科技研究所
% 微生物學及免疫研究所
% 放射線學科
% 核子醫學科
% 分子醫學研究所
 
%  	家庭醫學科
% 急診學科
% 口腔醫學研究所
 
%  	牙科學科
% 職業及環境醫學科
% 臨床醫學研究所
 
%  	法醫學科
%  
% 健康照護科學研究所
 
%  
%  
%  	老年學研究所
 
% School of Medicine
% Department of Biochemistry and Molecular Biology
% Department of Forensic Medicine
% Department of Otolaryngology
% Department of Physiology
% Department of Pathology
% Department of Dermatology
% Department of Microbiology and Immunology
% Department of Internal Medicine
% Department of Urology
% Department of Pharmacology
% Department of Surgery
% Department of Anesthesiology
% Department of Environmental and Occupational Health
% Department of Pediatrics
% Department of Physical Medicine and Rehabilitation
% Department of Parasitology
% Department of Obstetrics Gynecology
% Department of Radiology
% Department of Public Health
% Department of Orthopaedics
% Department of Nuclear Medicine
% Department of Cell Biology and Anatomy
% Department of Neurology
% Department of Family Medicine
 
% Department of Psychiatry
% Department of Emergency Medicine
 
% Department of Ophthalmology
% Department of Dentistry
 
%  
% Department of Occupational and Environmental Medicine
 
%  
%  
% Department of Nursing
% Department of Medical Laboratory Science and Biotechnology
% Department of Physical Therapy
% Department of Occupational Therapy
% School of Pharmacy
% Institute of Clinical Pharmacy and Pharmaceutical Sciences
% Institute of Basic Medical Sciences
% Institute of Behavioral Medicine
% Institute of Clinical Medicine
% Institute of Molecular Medicine
% Institute of Oral Medicine
% Institute of Allied Health Sciences
% Institute of Gerontology
%  
%  
 

% ----------------------------------------------------------------------------

% ----------------------------------------------------------------------
% -------------------------- 未完成 UNFINISH ----------------------------
% ----------------------------------------------------------------------

%
% This file is part of ncku-thesis-templete.
%
% ncku-thesis-templete is distributed in the hope that it will be useful,
% you can redistribute it and/or modify
% it under the terms of the Attribution-NonCommercial-ShareAlike
% 4.0 International.
%
% You should have received a copy of the
% Attribution-NonCommercial-ShareAlike 4.0 International
% along with ncku-thesis-templete.
%
% If not, see <http://creativecommons.org/licenses/by-nc-sa/4.0/legalcode.txt>.
%

% Some helper function about watermark

% ----------------------------------------------------------------------------

% 學校Logo watermark
\newcommand{\UseSchoolWatermark}[0]
{
  \AddToShipoutPicture{
  \put(0,0){
  \parbox[b][\paperheight]{\paperwidth}{
  \vfill
  \centering
  \includegraphics[]{./ncku/watersymbol.jpg}
  \vfill
  }}}
} % End of \newcommand{}

% 學校文字的watermark
\newcommand{\UseSchoolTextWatermark}[0]
{
  \AddToShipoutPicture*{
  \put(0,0){
  \parbox[b][\paperheight]{\paperwidth}{
  \vfill
  \centering
   \makebox(0,0){\rotatebox{45}{\textcolor[gray]{0.75}%
          {\fontsize{2.0cm}{2.0cm}\selectfont{\GetSchoolChiName}}}}
  \vfill
  }}}
} % End of \newcommand{}

\newcommand{\ClearWatermark}[0]{\ClearShipoutPicture}

% ----------------------------------------------------------------------------



% ----------------------------------------------------------------------------



% --------------------------

% 論文有關資料
% ------------------------------------------------
\StartSection{論文基本資料設定 Tesis base configure}{chapter:how-to:use:conf}
% ------------------------------------------------

\StartSubSection{介紹}

'conf/conf.tex'是用來設定一些論文需要的資料: 如題目, 人名等. 故以下的章節會一個一個資料說明要怎麼填寫或修改:

\begin{enumerate}
  \item
  {
    \textbf{論文的編寫語言 / 用途}

    有3個選擇, 但只能用其中一個.

    \verb|\ChiMode|:  編寫的為中英混合版, 有提供基本的所需的檔案.\\
    \verb|\EngMode|:  編寫的為全英文版, 有提供基本的所需的檔案.\\
    \verb|\DemoMode|: 完整教學 或 樣板測試用.

    如果你選擇的是\verb|\DemoMode|, 則會使用'./example/context.tex'中的模板說明文件內容. 而如果使用\verb|\DemoMode|, 但這資料夾已刪的, 在產生論文時會回傳錯誤.

    如果你選擇的是\verb|\ChiMode|或\verb|\EngMode|, 就會使用'./context/context.tex'中的內容. 所以這時候請在這資料夾中編寫你的論文.
  } % End of \item{}

  \item
  {
    \textbf{封面名字顯示方式}

    如果你是使用\verb|\ChiMode|, 則可無視這個設定.

    而如果你選擇是\verb|\EngMode|, 預設在封面上只會顯示英文名字而已, 把\verb|\CDBothName|中的\verb|'%'|拿掉以設定同時顯示中英文名字

  } % End of \item{}

  \item
  {
    \textbf{Title 論文題目}

    要填寫你的中文和(或)英文論文題目.

    如果題目內有必須以數學模式表示的符號,請用\verb|\mbox{}|包住數學模式. 如:\\
    \verb|\SetTitle{題目題目}{New equation \mbox{$E = mc^4$} here}|\\

    而如果覺得自動產生出來的題目斷行位置不適合, 可以手動加'\verb|\\|'來強制斷行. 如:\\
    \verb|\SetTitle{題目題目}{Title Tooooooooooo \\ Longgggggggggggg}|\\

    有3種可使用, 可獨立使用, 但只有最後設定的一方有效\\
    \verb|\SetTitle{你的題目}{Your Title}|: 同時設定中英文題目\\
    \verb|\SetChiTitle{你的題目}|: 只設定中文題目\\
    \verb|\SetEngTitle{Your Title}|: 只設定英文題目\\

    如:\\
    \verb|\SetTitle %|\\
    \verb|{中文題目中文題目} %|\\
    \verb|{Your Title Your Title}|\\
    \verb|'%'|是必須的, 是用來跟Latex說這3行是同一句話.

    或\\
    \verb|\SetChiTitle{中文題目中文題目}|\\
    \verb|\SetEngTitle{Your Title \\ Your Title}|

    圖書館說不管你是編寫中英混合或全英文版, 都\textbf{必須}同時存在中英題目.
  } % End of \item{}

  \item
  {
    \textbf{Degree name 學位}

    設定這論文是碩士或是博士學位論文.\\
    有2種可選擇, 但只有最後設定的一方有效.\\
    \verb|\PhdDegree|: 博士學位\\
    \verb|\MasterDegree|: 碩士學位
  } % End of \item{}

  \item
  {
    \textbf{Your name 你的名字}

    填寫你的中文和(或)英文.\\
    有3種可使用, 可獨立使用, 但只有最後設定的一方有效.\\
    \verb|\SetMyName{你的名字}{Your name}|: 同時設定你的中英文名字\\
    \verb|\SetMyChiName{你的名字}|: 只設定你的中文名字\\
    \verb|\SetMyEngName{Your name}|: 只設定你的英文名字
  } % End of \item{}

  \item
  {
    \textbf{論文封面上的日期}

    設定西元的年月, 會自動計算出民國的年份, 和英文的月份轉換.\\
    次序為: \verb|\SetCoverDate{年份}{月份}|\\
    如: \verb|\SetCoverDate{2014}{12}|

    \textbf{注意}: 依本校研究生學位考試細則第十條規定:
      \begin{description}
        \item[碩士班]: \hfill
          論文日期:上學期為〇〇〇年1月;下學期為〇〇〇年6月,
          以該學期結束日期(一月或六月)為準。
          (如:在上學期101年9月~102年1月期間口試,
              不論是在此期間何月份口試,其日期均固定為102年1月).
          另碩士生如101上學期完成口試,101下學期申請出國,102上學期辦理離校,
          則論文封面為103年1月

        \item[博士班]: \hfill
        以當學期通過學位口試,則論文日期為口試日期(如〇〇〇年〇〇月〇〇日),
        若論文有修改致延至次學期,則以論文上傳日期為主。
      \end{description}
    故本模版會根據你的學位, 來選擇顯示在封面的日期格式.
  } % End of \item{}

  \item
  {
    \textbf{口試的日期}

    設定西元的年月日, 會自動計算出民國的年份, 和英文的月份轉換.\\
    次序為: \verb|\SetOralDate{年份}{月份}{日}|\\
    如: \verb|\SetOralDate{2014}{12}{31}|
  } % End of \item{}

  \item
  {
    \textbf{指導老師 Advisor(s)}

    在封面上預算了最多3位的空間, 中文名字固定以'教授'為結尾, 英文名字固定以'Prof.'為開頭.

    有3種可使用, 用來設定3位老師的名字\\
    \verb|\SetAdvisorNameX{老師的名字}{Professor's name}|: 同時設定中英文名字\\
    \verb|\SetAdvisorChiNameX{老師的名字}|: 只設定中文名字\\
    \verb|\SetAdvisorEngNameX{Professor's name}|: 只設定英文名字\\
    (NameX為NameA, NameB, NameC)

    使用\verb|\SetAdvisorNameA|是必須的, 而如果你的指導教授有2或3位, 那只要增加\verb|\SetAdvisorNameB|和\verb|\SetAdvisorNameC|則可.

    如: \verb|\SetAdvisorNameA{老師的中文名字}{老師的英文名字}|
  } % End of \item{}

  \item
  {
    \textbf{口試証明文件 Oral presentation document}

    口試証明文件是使用'範例'或是'自己的檔案', 只能選擇其中一方.

    如果要用的是範例:\\
    \verb|\DisplayOralTemplate|: 顯示 / 使用 口試範例版本.\\
    \verb|\SetCommitteeSize{8}|: 口試委員數量, 要配合\verb|\DisplayOralTemplate|來使用, 至少4位, 最多8位, 預設為8位.\\

    而如果要用的是自己的檔案:\\
    把你的圖片放在'context/oral'下, 之後設定中英文版所對應是哪一個檔案.\\
    例子用的'oral-chi.pdf'和'oral-eng.pdf'已放在'context/oral'中.
    \verb|\DisplayOralImage|: 設定要顯示圖片
    \verb|\SetOralImageChi{oral-chi.pdf}|: 設定中文口試檔名
    \verb|\SetOralImageEng{oral-eng.pdf}|: 設定英文口試檔名

    雖然沒有限定圖片的格式, 但是推薦使用PDF, 而且是沒法使用SVG.
  } % End of \item{}

  \item
  {
    \textbf{關鍵字 Keyword}

    可設定最多5個關鍵字.使用方式:\\
    \verb|\SetKeywords{Keyword A}{Keyword B}{Keyword C}{Keyword D}{Keyword E}|
  } % End of \item{}

  \item
  {
    \textbf{書脊 Spine}

    用來控制當使用spine.tex來產生書脊時內容控制. 預設書脊上會使用英文題目, 使用\verb|\SpineTitleChi|(把\verb|'%'|拿掉)以設定改使用中文題目.
  } % End of \item{}

  \item
  {
    \textbf{系所 Department or Institute}

    設定你的系所名字, 如:\\
    \verb|\SetDeptMath|: 數學系\\
    \verb|\SetDeptCSIE|: 資訊工程學系

    只要設定系所名字, 會自動進行適當的斷行和填入學院名稱等處理.\\

    這部份的資料是使用學校的教學單位資料中英文版(某些系所的中英的URL會不一樣或錯誤的)\RefBib{web:school:academics}.\\
    縮寫是靠學校給的Domain name所得出的, 故可能會有錯誤的時候.\\
    所以如果錯了的話, 就請告知真正的寫法(或縮寫)是什麼.\\

    設定系所名字則請參考下面的名單.

    \newpage
    \begin{table}[h]
      \caption{系所名字 Part 1}
      \begin{tabular}{|l|l|}
        \hline
        寫法 & 系所名字 \\ \hline

        \verb|\SetDeptChinese| &
        \begin{tabular}[c]{@{}l@{}}
          中國文學系\\
          Department of Chinese Literature
        \end{tabular} \\ \hline

        \verb|\SetDeptArt| &
        \begin{tabular}[c]{@{}l@{}}
          藝術研究所\\
          Institute of Art
        \end{tabular} \\ \hline

        \verb|\SetDeptMinNan| &
        \begin{tabular}[c]{@{}l@{}}
          閩南文化研究中心\\
          Min-Nan Culture Studies Center
        \end{tabular} \\ \hline

        \verb|\SetDeptFLLD| &
        \begin{tabular}[c]{@{}l@{}}
          外國語文學系\\
          Department of Foreign Languages and Literature
        \end{tabular} \\ \hline

        \verb|\SetDeptTWL| &
        \begin{tabular}[c]{@{}l@{}}
          臺灣文學系\\
          Department of Taiwanese Literature
        \end{tabular} \\ \hline

        \verb|\SetDeptKCLC| &
        \begin{tabular}[c]{@{}l@{}}
          華語中心\\
          Chinese Language Center
        \end{tabular} \\ \hline

        \verb|\SetDeptLang| &
        \begin{tabular}[c]{@{}l@{}}
          外語中心\\
          Foreign Language Center
        \end{tabular} \\ \hline

        \verb|\SetDeptHis| &
        \begin{tabular}[c]{@{}l@{}}
          歷史學系\\
          Department of History
        \end{tabular} \\ \hline

        \verb|\SetDeptMath| &
        \begin{tabular}[c]{@{}l@{}}
          數學系\\
          Department of Mathematics
        \end{tabular} \\ \hline

        \verb|\SetDeptDPS| &
        \begin{tabular}[c]{@{}l@{}}
          光電科學與工程學系\\
          Departmment of Photonics
        \end{tabular} \\ \hline

        \verb|\SetDeptPhys| &
        \begin{tabular}[c]{@{}l@{}}
          物理學系\\
          Department of Physics
        \end{tabular} \\ \hline

        \verb|\SetDeptCh| &
        \begin{tabular}[c]{@{}l@{}}
          化學系\\
          Department of Chemistry
        \end{tabular} \\ \hline

        \verb|\SetDeptEarth| &
        \begin{tabular}[c]{@{}l@{}}
          地球科學系\\
          Department of Earth Sciences
        \end{tabular} \\ \hline

        \verb|\SetDeptPSSC| &
        \begin{tabular}[c]{@{}l@{}}
          太空與電漿科學研究所\\
          Institute of Space and Plasma Sciences
        \end{tabular} \\ \hline

        \verb|\SetDeptNCTS| &
        \begin{tabular}[c]{@{}l@{}}
          國家理論科學研究中心\\
          National Center for Theoretical Sciences (South)
        \end{tabular} \\ \hline

        \verb|\SetDeptME| &
        \begin{tabular}[c]{@{}l@{}}
          機械工程學系\\
          Department of Mechanical Engineering
        \end{tabular} \\ \hline

        \verb|\SetDeptChe| &
        \begin{tabular}[c]{@{}l@{}}
          化學工程學系\\
          Department of Chemical Engineering
        \end{tabular} \\ \hline

        \verb|\SetDeptCivil| &
        \begin{tabular}[c]{@{}l@{}}
          土木工程學系\\
          Department of Civil Engineering
        \end{tabular} \\ \hline

        \verb|\SetDeptMSE| &
        \begin{tabular}[c]{@{}l@{}}
          材料科學及工程學系\\
          Department of Materials Science and Engineering
        \end{tabular} \\ \hline

      \end{tabular}
    \end{table}

    \newpage
    \begin{table}[h]
      \caption{系所名字 Part 2}
      \begin{tabular}{|l|l|}
        \hline
        寫法 & 系所名字 \\ \hline

        \verb|\SetDeptHyd| &
        \begin{tabular}[c]{@{}l@{}}
          水利及海洋工程學系\\
          Department of Hydraulic and Ocean Engineering
        \end{tabular} \\ \hline

        \verb|\SetDeptES| &
        \begin{tabular}[c]{@{}l@{}}
          工程科學系\\
          Department of Engineering Science
        \end{tabular} \\ \hline

        \verb|\SetDeptSNAME| &
        \begin{tabular}[c]{@{}l@{}}
          系統及船舶機電工程學系\\
          Department of System and Naval Mechatronic Engineering
        \end{tabular} \\ \hline

        \verb|\SetDeptIAA| &
        \begin{tabular}[c]{@{}l@{}}
          航空太空工程學系\\
          Department of Aeronautics and Astronautics
        \end{tabular} \\ \hline

        \verb|\SetDeptMP| &
        \begin{tabular}[c]{@{}l@{}}
          資源工程學系\\
          Department of Resources Engineering
        \end{tabular} \\ \hline

        \verb|\SetDeptEV| &
        \begin{tabular}[c]{@{}l@{}}
          環境工程學系\\
          Department of Environmental Engineering
        \end{tabular} \\ \hline

        \verb|\SetDeptBME| &
        \begin{tabular}[c]{@{}l@{}}
          生物醫學工程學系\\
          Department of BioMedical Engineering
        \end{tabular} \\ \hline

        \verb|\SetDeptGeomatics| &
        \begin{tabular}[c]{@{}l@{}}
          測量及空間資訊學系\\
          Department of Geomatics
        \end{tabular} \\ \hline

        \verb|\SetDeptIOTMA| &
        \begin{tabular}[c]{@{}l@{}}
          海洋科技與事務研究所\\
          Institute of Ocean Technology and Marine Affairs
        \end{tabular} \\ \hline

        \verb|\SetDeptICA| &
        \begin{tabular}[c]{@{}l@{}}
          民航研究所\\
          Institute of Civil Aviation
        \end{tabular} \\ \hline

        \verb|\SetDeptIBDPE| &
        \begin{tabular}[c]{@{}l@{}}
          能源國際學士學位學程\\
          International Bachelor Degree Program on Energy
        \end{tabular} \\ \hline

        \verb|\SetDeptICAMP| &
        \begin{tabular}[c]{@{}l@{}}
          尖端材料國際碩士學位學程\\
          International Curriculum for Advanced Materials Program
        \end{tabular} \\ \hline

        \verb|\SetDeptINHMM| &
        \begin{tabular}[c]{@{}l@{}}
          自然災害減災及管理國際碩士學位學程\\
          International Master Program on \\
          Natural Hazards Mitigation and Management
        \end{tabular} \\ \hline

        \verb|\SetDeptICEM| &
        \begin{tabular}[c]{@{}l@{}}
          工程管理碩士在職專班\\
          International Graduate Program of \\
          Civil Engineering and Management
        \end{tabular} \\ \hline

        \verb|\SetDeptEE| &
        \begin{tabular}[c]{@{}l@{}}
          電機工程學系\\
          Department of Electrical Engineering
        \end{tabular} \\ \hline

        \verb|\SetDeptCSIE| &
        \begin{tabular}[c]{@{}l@{}}
          資訊工程學系\\
          Insitute of Computer Science and Information Engineering
        \end{tabular} \\ \hline

        \verb|\SetDeptIME| &
        \begin{tabular}[c]{@{}l@{}}
          微電子工程研究所\\
          Institute of Microelectronics
        \end{tabular} \\ \hline

        \verb|\SetDeptCCE| &
        \begin{tabular}[c]{@{}l@{}}
          電腦與通信工程研究所\\
          Institute of Computer \& Communication Engineering
        \end{tabular} \\ \hline

        \verb|\SetDeptIMIS| &
        \begin{tabular}[c]{@{}l@{}}
          製造資訊與系統研究所\\
          Institute of Manufacturing Information and Systems
        \end{tabular} \\ \hline

      \end{tabular}
    \end{table}

    \newpage
    \begin{table}[h]
      \caption{系所名字 Part 3}
      \begin{tabular}{|l|l|}
        \hline
        寫法 & 系所名字 \\ \hline

        \verb|\SetDeptIMI| &
        \begin{tabular}[c]{@{}l@{}}
          醫學資訊研究所\\
          Institute of Medical Informatics
        \end{tabular} \\ \hline

        \verb|\SetDeptSTAT| &
        \begin{tabular}[c]{@{}l@{}}
          統計學系\\
          Department of Statistics
        \end{tabular} \\ \hline

        \verb|\SetDeptACC| &
        \begin{tabular}[c]{@{}l@{}}
          會計學系\\
          Department of Accountancy
        \end{tabular} \\ \hline

        \verb|\SetDeptTCM| &
        \begin{tabular}[c]{@{}l@{}}
          交通管理科學系\\
          Department of Transportation and \\
          Communication Management Science
        \end{tabular} \\ \hline

        \verb|\SetDeptBA| &
        \begin{tabular}[c]{@{}l@{}}
          企業管理學系暨國際企業研究所\\
          Department of Business Administration and\\
          Graduate Institute of International Business
        \end{tabular} \\ \hline

        \verb|\SetDeptTM| &
        \begin{tabular}[c]{@{}l@{}}
          電信管理研究所\\
          Institute of Telecommunications Management
        \end{tabular} \\ \hline

        \verb|\SetDeptIIM| &
        \begin{tabular}[c]{@{}l@{}}
          工業與資訊管理學系暨資訊管理研究所\\
          Institute of Information Management
        \end{tabular} \\ \hline

        \verb|\SetDeptFin| &
        \begin{tabular}[c]{@{}l@{}}
          財務金融研究所\\
          Institute of Finance \& Banking
        \end{tabular} \\ \hline

        \verb|\SetDeptPHEI| &
        \begin{tabular}[c]{@{}l@{}}
          體育健康與休閒研究所\\
          Institute of Physical Education, Health \& Leisure Studies
        \end{tabular} \\ \hline

        \verb|\SetDeptEMBA| &
        \begin{tabular}[c]{@{}l@{}}
          高階管理碩士在職專班\\
          Executive Master of Business Administration (EMBA)
        \end{tabular} \\ \hline

        \verb|\SetDeptIMBA| &
        \begin{tabular}[c]{@{}l@{}}
          國際經營管理研究所\\
          Institute of International Management (IMBA)
        \end{tabular} \\ \hline

        \verb|\SetDeptAMBA| &
        \begin{tabular}[c]{@{}l@{}}
          經營管理碩士班\\
          Advanced Master of Business Administration (AMBA)
        \end{tabular} \\ \hline

        \verb|\SetDeptPolSci| &
        \begin{tabular}[c]{@{}l@{}}
          政治學系\\
          Department of Political Science
        \end{tabular} \\ \hline

        \verb|\SetDeptEconomic| &
        \begin{tabular}[c]{@{}l@{}}
          經濟學系\\
          Department of Economics
        \end{tabular} \\ \hline

        \verb|\SetDeptPsychology| &
        \begin{tabular}[c]{@{}l@{}}
          心理學系\\
          Department of Psychology
        \end{tabular} \\ \hline

        \verb|\SetDeptLaw| &
        \begin{tabular}[c]{@{}l@{}}
          法律學系\\
          Department of Law and \\
          Institute of Law in Science and Technology
        \end{tabular} \\ \hline

        \verb|\SetDeptED| &
        \begin{tabular}[c]{@{}l@{}}
          教育研究所\\
          Institute of Education
        \end{tabular} \\ \hline

        \verb|\SetDeptIOCS| &
        \begin{tabular}[c]{@{}l@{}}
          認知科學研究所\\
          Institute of Cognitive Science
        \end{tabular} \\ \hline

        \verb|\SetDeptGIPE| &
        \begin{tabular}[c]{@{}l@{}}
          政治經濟學研究所\\
          Institute of Political Economy
        \end{tabular} \\ \hline

      \end{tabular}
    \end{table}

    \newpage
    \begin{table}[h]
      \caption{系所名字 Part 4}
      \begin{tabular}{|l|l|}
        \hline
        寫法 & 系所名字 \\ \hline

        \verb|\SetDeptFMRI| &
        \begin{tabular}[c]{@{}l@{}}
          心智影像研究中心\\
          Mind Research and Image Center
        \end{tabular} \\ \hline

        \verb|\SetDeptArch| &
        \begin{tabular}[c]{@{}l@{}}
          建築學系\\
          Department of Architecture
        \end{tabular} \\ \hline

        \verb|\SetDeptUP| &
        \begin{tabular}[c]{@{}l@{}}
          都市計劃學系\\
          Department of Urban Planning
        \end{tabular} \\ \hline

        \verb|\SetDeptID| &
        \begin{tabular}[c]{@{}l@{}}
          工業設計學系\\
          Department of Industrial Design
        \end{tabular} \\ \hline

        \verb|\SetDeptICID| &
        \begin{tabular}[c]{@{}l@{}}
          創意產業設計研究所\\
          Institute of Creative Industry Design
        \end{tabular} \\ \hline

        \verb|\SetDeptBio| &
        \begin{tabular}[c]{@{}l@{}}
          生命科學系\\
          Department of Life Sciences
        \end{tabular} \\ \hline

        \verb|\SetDeptBioTech| &
        \begin{tabular}[c]{@{}l@{}}
          生物科技研究所\\
          Institute of Biotechnology
        \end{tabular} \\ \hline

        \verb|\SetDeptIBBT| &
        \begin{tabular}[c]{@{}l@{}}
          生物資訊與訊息傳遞研究所\\
          Institute of Bioinformatics and Biosignal Transduction
        \end{tabular} \\ \hline

        \verb|\SetDeptITPS| &
        \begin{tabular}[c]{@{}l@{}}
          熱帶植物科學研究所\\
          Institute of Tropical Plant Sciences
        \end{tabular} \\ \hline

        \verb|\SetDeptEDUC| &
        \begin{tabular}[c]{@{}l@{}}
          醫學系\\
          School of Medicine
        \end{tabular} \\ \hline

        \verb|\SetDeptBiohem| &
        \begin{tabular}[c]{@{}l@{}}
          生物化學暨分子生物學研究所\\
          Department of Biochemistry and Molecular Biology
        \end{tabular} \\ \hline

        \verb|\SetDeptPath| &
        \begin{tabular}[c]{@{}l@{}}
          病理學科\\
          Department of Pathology
        \end{tabular} \\ \hline

        \verb|\SetDeptIntMed| &
        \begin{tabular}[c]{@{}l@{}}
          內科學科\\
          Department of Internal Medicine
        \end{tabular} \\ \hline

        \verb|\SetDeptPhysMed| &
        \begin{tabular}[c]{@{}l@{}}
          生理學研究所\\
          Department of Physiology
        \end{tabular} \\ \hline

        \verb|\SetDeptSurgery| &
        \begin{tabular}[c]{@{}l@{}}
          外科學科\\
          Department of Surgery
        \end{tabular} \\ \hline

        \verb|\SetDeptPed| &
        \begin{tabular}[c]{@{}l@{}}
          小兒學科\\
          Department of Pediatrics
        \end{tabular} \\ \hline

        \verb|\SetDeptAnatomy| &
        \begin{tabular}[c]{@{}l@{}}
          解剖學科暨細胞生物與解剖學研究所\\
          Department of Cell Biology and Anatomy
        \end{tabular} \\ \hline

        \verb|\SetDeptObsGyn| &
        \begin{tabular}[c]{@{}l@{}}
          婦產學科\\
          Department of Obstetrics and Gynecology
        \end{tabular} \\ \hline

        \verb|\SetDeptBone| &
        \begin{tabular}[c]{@{}l@{}}
          骨科學科\\
          Department of Orthopaedics
        \end{tabular} \\ \hline

        \verb|\SetDeptPhMed| &
        \begin{tabular}[c]{@{}l@{}}
          公共衛生學科暨公共衛生研究所\\
          Department of Public Health
        \end{tabular} \\ \hline

      \end{tabular}
    \end{table}

    \newpage
    \begin{table}[h]
      \caption{系所名字 Part 5}
      \begin{tabular}{|l|l|}
        \hline
        寫法 & 系所名字 \\ \hline

        \verb|\SetDeptNeuro| &
        \begin{tabular}[c]{@{}l@{}}
          神經學科\\
          Department of Neurology
        \end{tabular} \\ \hline

        \verb|\SetDeptPsy| &
        \begin{tabular}[c]{@{}l@{}}
          精神學科\\
          Department of Psychiatry
        \end{tabular} \\ \hline

        \verb|\SetDeptParasite| &
        \begin{tabular}[c]{@{}l@{}}
          寄生蟲學科\\
          Department of Parasitology
        \end{tabular} \\ \hline

        \verb|\SetDeptOphth| &
        \begin{tabular}[c]{@{}l@{}}
          眼科學科\\
          Department of Ophthalmology
        \end{tabular} \\ \hline

        \verb|\SetDeptOtolaryngo| &
        \begin{tabular}[c]{@{}l@{}}
          耳鼻喉學科\\
          Department of Otolaryngology
        \end{tabular} \\ \hline

        \verb|\SetDeptDEOH| &
        \begin{tabular}[c]{@{}l@{}}
          工業衛生學科暨環境醫學研究所\\
          Department of Environmental and Occupational Health
        \end{tabular} \\ \hline

        \verb|\SetDeptDerm| &
        \begin{tabular}[c]{@{}l@{}}
          皮膚學科\\
          Department of Dermatology
        \end{tabular} \\ \hline

        \verb|\SetDeptUro| &
        \begin{tabular}[c]{@{}l@{}}
          泌尿學科\\
          Department of Urology
        \end{tabular} \\ \hline

        \verb|\SetDeptPharmaco| &
        \begin{tabular}[c]{@{}l@{}}
          藥理學科暨藥理學研究所\\
          Department of Pharmacology
        \end{tabular} \\ \hline

        \verb|\SetDeptAnesth| &
        \begin{tabular}[c]{@{}l@{}}
          麻醉學科\\
          Department of Anesthesiology
        \end{tabular} \\ \hline

        \verb|\SetDeptRehab| &
        \begin{tabular}[c]{@{}l@{}}
          復健學科\\
          Department of Physical Medicine and Rehabilitation
        \end{tabular} \\ \hline

        \verb|\SetDeptMicrobio| &
        \begin{tabular}[c]{@{}l@{}}
          微生物學及免疫研究所\\
          Department of Microbiology and Immunology
        \end{tabular} \\ \hline

        \verb|\SetDeptRad| &
        \begin{tabular}[c]{@{}l@{}}
          放射線學科\\
          Department of Diagnostic Radiology
        \end{tabular} \\ \hline

        \verb|\SetDeptNM| &
        \begin{tabular}[c]{@{}l@{}}
          核子醫學科\\
          Department of Nuclear Medicine
        \end{tabular} \\ \hline

        \verb|\SetDeptFamily| &
        \begin{tabular}[c]{@{}l@{}}
          家庭醫學科\\
          Department of Family Medicine
        \end{tabular} \\ \hline

        \verb|\SetDeptEmergency| &
        \begin{tabular}[c]{@{}l@{}}
          急診學科\\
          Department of Emergency Medicine
        \end{tabular} \\ \hline

        \verb|\SetDeptDentistry| &
        \begin{tabular}[c]{@{}l@{}}
          牙科學科\\
          Department of Dentistry
        \end{tabular} \\ \hline

        \verb|\SetDeptOEM| &
        \begin{tabular}[c]{@{}l@{}}
          職業及環境醫學科\\
          Department of Occupational and Environmental Medicine
        \end{tabular} \\ \hline

        \verb|\SetDeptForensic| &
        \begin{tabular}[c]{@{}l@{}}
          法醫學科\\
          Department of Forensic Medicine
        \end{tabular} \\ \hline

        \verb|\SetDeptNursing| &
        \begin{tabular}[c]{@{}l@{}}
          護理學系\\
          Department of Nursing
        \end{tabular} \\ \hline

      \end{tabular}
    \end{table}

    \newpage
    \begin{table}[h]
      \caption{系所名字 Part 6}
      \begin{tabular}{|l|l|}
        \hline
        寫法 & 系所名字 \\ \hline

        \verb|\SetDeptMT| &
        \begin{tabular}[c]{@{}l@{}}
          醫學檢驗生物技術學系\\
          Department of Medical Laboratory Science and Biotechnology
        \end{tabular} \\ \hline

        \verb|\SetDeptPT| &
        \begin{tabular}[c]{@{}l@{}}
          物理治療學系\\
          Department of Physical Therapy
        \end{tabular} \\ \hline

        \verb|\SetDeptOT| &
        \begin{tabular}[c]{@{}l@{}}
          職能治療學系\\
          Department of Occupational Therapy
        \end{tabular} \\ \hline

        \verb|\SetDeptPharmacy| &
        \begin{tabular}[c]{@{}l@{}}
          藥學系\\
          School of Pharmacy
        \end{tabular} \\ \hline

        \verb|\SetDeptBasicMed| &
        \begin{tabular}[c]{@{}l@{}}
          基礎醫學研究所\\
          Institute of Basic Medical Sciences
        \end{tabular} \\ \hline

        \verb|\SetDeptBehMed| &
        \begin{tabular}[c]{@{}l@{}}
          行為醫學研究所\\
          Institute of Behavioral Medicine
        \end{tabular} \\ \hline

        \verb|\SetDeptCLPARM| &
        \begin{tabular}[c]{@{}l@{}}
          臨床藥學與藥物科技研究所\\
          Institute of Clinical Pharmacy and Pharmaceutical Sciences
        \end{tabular} \\ \hline

        \verb|\SetDeptIMM| &
        \begin{tabular}[c]{@{}l@{}}
          分子醫學研究所\\
          Institute of Molecular Medicine
        \end{tabular} \\ \hline

        \verb|\SetDeptIOM| &
        \begin{tabular}[c]{@{}l@{}}
          口腔醫學研究所\\
          Institute of Oral Medicine
        \end{tabular} \\ \hline

        \verb|\SetDeptICMMed| &
        \begin{tabular}[c]{@{}l@{}}
          臨床醫學研究所\\
          Institute of Clinical Medicine
        \end{tabular} \\ \hline

        \verb|\SetDeptAlliedHealth| &
        \begin{tabular}[c]{@{}l@{}}
          健康照護科學研究所\\
          Institute of Allied Health Sciences
        \end{tabular} \\ \hline

        \verb|\SetDeptIOG| &
        \begin{tabular}[c]{@{}l@{}}
          老年學研究所\\
          Institute of Gerontology
        \end{tabular} \\ \hline

      \end{tabular}
    \end{table}
  } % End of \item{}
\end{enumerate}



% --------------------------

% 學校排版 Arrangement style
\usepackage{./ncku/style/ncku}

% --------------------------

% 設定學校浮水印 Watermark
\UseSchoolWatermark

% --------------------------

% 在 pdf 簡介欄裡填入相關資料
\FillInPDFData

% ------------------------------------------------

% 一些會受到conf.tex中設定而影響的package

% For reference
\ifthenelse{\equal{\GetBibStyleTypeVar}{\BibStyleTypeApacite}}%
{\usepackage[notocbib]{apacite}}{}

% ------------------------------------------------

% 其他排版的設定

% 設定段落之間的距離
%\setlength{\parskip}{0.3cm}

% 當所有的package都include完後, 才真正設定我們要的字型,
% 以清掉所有由package影響到的設定.
\InitDefaultFontType

% Makes all pages the height of the text on that page.
% No extra vertical space is added. 
\raggedbottom

% ------------------------------------------------
