
% ----------------------------------------------------------------------------
%   Document of oral presentation (Chinese version)
%                   中文版口試証明文件
% ----------------------------------------------------------------------------

% Set the line spacing to single for the titles (to compress the lines)
\renewcommand{\baselinestretch}{1}   %行距 1 倍

% 由於中文版的watermark是用文字, 所以先把Logo版關掉
\ClearWatermark

% ------------------------------------------------

% Page start
% Add to "Table of Contents"
% 設定使用 無頁碼, 有浮水印
\newpage
\phantomsection
\addcontentsline{toc}{section}{Chinese version 2}
\thispagestyle{empty}

% 使用文字版watermark
\UseSchoolTextWatermark

% Aligned to the center of the page
\begin{center}

% ------------------------------------------------

% 顯示 校名, 論文種類
\makebox[\textwidth][c]{\Huge \GetSchoolChiName}\\
\vspace{0.5cm}
\makebox[\textwidth][c]{\Huge \GetChiDegree 論文}\\

% ------------------------------------------------

\vspace{2.0cm}

% ------------------------------------------------

% Chinese and English title 中英文題目
\begin{minipage}[c][3cm][t]{\textwidth}
  \begin{center}
    \Large \GetChiTitle \\
    \Large \GetEngTitle \\
    (口試文件版本2 - 等待認可)
  \end{center}
\end{minipage}

% ------------------------------------------------

\vspace{2.2cm}

% ------------------------------------------------

% 顯示 學生 的名字
\hspace{2.4em}
\makebox[4.8em][r]{\Large 研究生:}
\makebox[7.2em][l]{\Large \GetAuthorChiName} \\

% --------------------------
\vspace{1.0cm}
\makebox[\textwidth][c]{\Large 本論文業經審查及口試合格特此證明} \\
% --------------------------
\vspace{1.0cm}
\makebox[11.6cm][l]{\Large 論文考試委員:} \\
% --------------------------

\vspace{4.5cm}
\hspace{2.4em}
\makebox[4.8em][r]{\Large 指導教授:}
\makebox[7.2em][l]{}

\vspace{0.5cm}
\hspace{2.4em}
\makebox[4.8em][r]{\Large 系(所)主管:}
\makebox[7.2em][l]{}

% --------------------------

% Date 日期
\vspace{0.5cm}
\makebox[\textwidth][s]{\Large 中華民國 \GetOralChiYear 年 \GetOralChiMonth 月 \GetOralChiDay 日}

% ------------------------------------------------

% End of alignment
\end{center}

% End of page
\clearpage

% 重新使用學校浮水印 Watermark
\UseSchoolWatermark

% ------------------------------------------------

