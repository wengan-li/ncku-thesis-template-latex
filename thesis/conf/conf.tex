
% This file is need to encoded in utf-8
%
% Choose or fill in some needed information for this thesis or dissertation
% 選擇或填入你的論文一些需要使用的資料

% ----------------------------------------------------------------------------

% --- 論文的編寫語言 / 用途 ---
% 只能選擇其中一個

% \ChiMode:  編寫的為中英混合版, 有提供基本的所需的檔案
% \EngMode:  編寫的為全英文版, 有提供基本的所需的檔案
% \DemoMode: 完整教學 或 樣板測試用

\DemoMode

% ----------------------------------------------------------------------------

% --- Title 論文題目 ---
% 填寫中文和(或)英文
% 如果題目內有必須以數學模式表示的符號,請用 \mbox{} 包住數學模式
% 如果覺得自動產生出來的題目斷行位置不適合, 可以手動加'\\'來強制斷行
% (圖書館說不管你是編寫中英混合或全英文版, 都必須同時存在中英題目)
%
% 有3種可使用, 可獨立使用, 但只有最後設定的一方有效
% \SetTitle{你的題目}{Your Title}   % 同時設定中英文題目
% \SetChiTitle{你的題目}            % 只設定中文題目
% \SetEngTitle{Your Title}         % 只設定英文題目
%
% e.g:
%
% \SetTitle %
% {國立成功大學碩博士用畢業論文XeLaTex模板} %
% {National Cheng Kung University (NCKU) Thesis/Dissertation Template in XeLaTex}
%
% or
%
% \SetChiTitle{國立成功大學碩博士用畢業論文XeLaTex模板}
% \SetEngTitle{National Cheng Kung University (NCKU)\\ Thesis/Dissertation Template in XeLaTex}

\SetTitle %
{國立成功大學碩博士用畢業論文XeLaTex模板} %
{National Cheng Kung University (NCKU) \\ Thesis/Dissertation Template in XeLaTex}


% ----------------------------------------------------------------------------

% --- Degree name 學位 ---
%
% 有2種可選擇, 但只有最後設定的一方有效
% \PhdDegree    % 博士學位
% \MasterDegree % 碩士學位

\PhdDegree

% ----------------------------------------------------------------------------

% --- Your name 你的名字 ---
% 填寫你的中文和(或)英文

% 有3種可使用, 可獨立使用, 但只有最後設定的一方有效
% \SetMyName{你的名字}{Your name}   % 同時設定你的中英文名字
% \SetMyChiName{你的名字}           % 只設定你的中文名字
% \SetMyEngName{Your name}         % 只設定你的英文名字

\SetMyName{你的名字}{Your name}

% ----------------------------------------------------------------------------

% --- Date 日期 ---

% 論文封面年月日因教務處仍無最後之確定版本, 所以目前仍以口試合格日為基準點.
% (若口試日期為 2014/10/01, 則封面日期需為2014/10, 2014/10/01或之後日期)

% --- 論文封面上的日期 ---
% 設定西元的年月, 會自動計算出民國的年份, 和英文的月份轉換
% 次序: {年份}{月份}
% \SetThesisDate{2014}{12}

\SetThesisDate{2014}{12}

%--------------------------------------------------

% --- 口試的日期 ---
% 設定西元的年月日, 會自動計算出民國的年份, 和英文的月份轉換
% 次序: {年份}{月份}{日}
% \SetOralDate{2014}{12}{31}

\SetOralDate{2014}{12}{31}

% ----------------------------------------------------------------------------

% --- 系所 Department or Institute ---
%
% 設定你的系所名字, e.g:
% \SetDeptMath 數學系
% \SetDeptCSIE 資訊工程學系

\SetDeptCSIE

% ----------------------------------------------------------------------------

% --- 指導老師 Advisor(s) ---
% 在封面上預算了最多3位的空間
% 中文名字固定以 教授  為結尾
% 英文名字固定以 Prof. 為開頭

% 有3種可使用, 用來設定3位老師的名字
% \SetAdvisorNameX{老師的名字}{Professor's name} % 同時設定中英文名字
% \SetAdvisorChiNameX{老師的名字}                % 只設定中文名字
% \SetAdvisorEngNameX{Professor's name}         % 只設定英文名字
% (NameX為NameA, NameB, NameC)

% 使用\SetAdvisorNameA是必須的, 而如果你的指導教授有2或3位,
% 那只要增加\SetAdvisorNameB和\SetAdvisorNameC則可

\SetAdvisorNameA{A}{A}
%\SetAdvisorNameB{B}{B}
%\SetAdvisorNameC{C}{C}

% ----------------------------------------------------------------------------

% --- 口試証明文件 Oral presentation document ---
% 範例版本 或 使用檔案 只能選擇其中一方

% 顯示 / 使用 口試範例版本
\DisplayOralTemplete
% --- 口試委員 Committee member(s) ---
% 口試委員數量 (至少2位, 最多8位, 預設為8位) 
% (Only work with \DisplayOralTemplete)
% 博士學位考試委員會置委員五人至九人
% 碩士學位考試委員會置委員三人至五人
% 口試委員人數含指導教授
\SetCommitteeSize{1}

%--------------------------------------------------

% 使用口試圖片檔案
% 把你的圖片放在'context/oral'下
% 之後設定中英文版所對應是哪一個檔案
% (例子用的'oral-chi.pdf'和'oral-eng.pdf'已放在'context/oral'中)
%
%\DisplayOralImage                % 顯示圖片
%\SetOralImageChi{oral-chi.pdf}   % 中文口試檔案
%\SetOralImageEng{oral-eng.pdf}   % 英文口試檔案

% ----------------------------------------------------------------------------

% 關鍵字 Keyword
% 最多5個關鍵字

%\SetKeywords{Keyword A}{Keyword B}{Keyword C}{Keyword D}{Keyword E}

\SetKeywords{NCKU Thesis/Dissertation templete}{Graduate}{Latex/XeLaTex}

% ----------------------------------------------------------------------------

% 書脊 Spine

% 預設書脊上會使用英文題目
% 把'%'拿掉以設定改使用中文題目
% Default using the english title in Spine
% Remove '%' for change to use chinese title

%\SpineTitleChi

% ----------------------------------------------------------------------------

