\documentclass[12pt, a4paper]{report}

% ------------------------------------------------

% XeLaTex 的設定

% --------------------------
%加這個就可以設定字體
\usepackage{fontspec}

%使用xeCJK,其他的還有CJK或是xCJK
\usepackage{xeCJK}

% Set english font
\setmainfont{Times New Roman} % In Windows
%\usepackage{times} % In *nix 

%設定中英文的字型
%字型的設定可以使用系統內的字型,而不用像以前一樣另外安裝
\setCJKmainfont{標楷體}

%中文自動換行
\XeTeXlinebreaklocale "zh"

%文字的彈性間距
\XeTeXlinebreakskip = 0pt plus 1pt

%設定段落之間的距離
\setlength{\parskip}{0.3cm}

%設定行距
\linespread{1.5}\selectfont

% ------------------------------------------------

% LaTex package

% --------------------------
\usepackage{graphicx}
%\usepackage{ncku/pdf}
%\usepackage{style/rotating}
\usepackage{./ncku/style/macros}
%\usepackage{style/psfig}
%\usepackage{epsfig}
\usepackage{latexsym}
\usepackage{./ncku/style/utdiss}
\usepackage{hyperref}
%\usepackage{hyperxmp}
\usepackage{pdfpages}

%preamble
%\usepackage[nottoc]{tocbibind}

% ------------------------------------------------

% 有關學校對論文要求的設定

% --------------------------

% 學校浮水印 Watermark
%\usepackage{draftwatermark}
%\SetWatermarkText{}

\AddToShipoutPicture{
\put(0,0){
\parbox[b][\paperheight]{\paperwidth}{
\vfill
\centering
\includegraphics[]{./ncku/watersymbol.jpg}
\vfill
}}}


% --------------------------


% ----------------------------------------------------------------------------
%                           Command for cover page
% ----------------------------------------------------------------------------

% default variables definitions
% 此處只是預設值,不需更改此處
% 請更改 names.tex 內容
\newcommand\cTitle{}
\newcommand\eTitle{}
\newcommand\myCname{}
\newcommand\myEname{}
\newcommand\advisorCnameA{}
\newcommand\advisorEnameA{}
\newcommand\advisorCnameB{}
\newcommand\advisorEnameB{}
\newcommand\advisorCnameC{}
\newcommand\advisorEnameC{}
\newcommand\univCname{}
\newcommand\univEname{}
\newcommand\deptCname{}
\newcommand\fulldeptEname{}
\newcommand\deptEname{}
\newcommand\collEname{}
\newcommand\degreeCname{}
\newcommand\degreeEname{}
\newcommand\cYear{}
\newcommand\cMonth{}
\newcommand\eYear{}
\newcommand\eMonth{}
\newcommand\ePlace{}


% --------------------------

% 論文有關資料
\input{./names.tex}

% --------------------------

% 使用 hyperref 在 pdf 簡介欄裡填入相關資料
\ifx \hypersetup \undefined
	\relax  % do nothing
\else
	\hypersetup
    {
        unicode     = true,
	    pdftitle    = {\eTitle (\cTitle)},
	    pdfauthor   = {\myEname (\myCname)},
        pdfcreator  = {\univEname},
%        pdfproducer = {\univEname},
        pdfsubject  = {},
        pdfkeywords = {},
    }
\fi

% ------------------------------------------------
