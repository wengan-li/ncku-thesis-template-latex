%
% this file is encoded in utf-8
% v2.02 (Sep. 12, 2012)
% do not change the content of this file
% unless the thesis layout rule is changed
% 無須修改本檔內容,除非校方修改了
% 封面、書名頁、中文摘要、英文摘要、誌謝、目錄、表目錄、圖目錄、符號說明
% 等頁之格式

% make the line spacing in effect
\renewcommand{\baselinestretch}{\mybaselinestretch}
\large % it needs a font size changing command to be effective

% default variables definitions
% 此處只是預設值,不需更改此處
% 請更改 my_names.tex 內容
\newcommand\cTitle{論文題目}
\newcommand\eTitle{MY THESIS TITLE}
\newcommand\myCname{胡庭瑜}
\newcommand\myEname{Ting-Yu Hu}
\newcommand\advisorCnameA{陳朝鈞博士}
\newcommand\advisorEnameA{Dr.~Chao-Chun~Chen}
\newcommand\advisorCnameB{莊坤達博士}
\newcommand\advisorEnameB{Dr.~Kun-Da~Chuang}
\newcommand\advisorCnameC{王鼎超博士}
\newcommand\advisorEnameC{Dr.~Ding-Chau~Wang}
\newcommand\univCname{成功大學}
\newcommand\univEname{National Cheng Kung University}
\newcommand\deptCname{製造資訊與系統研究所}
\newcommand\fulldeptEname{Institue of Manufacturing Information and Systems}
\newcommand\deptEname{Inst. of Manuf. Info. \& Sys.}
\newcommand\collEname{College of Electrical Engineering and Computer Science}
\newcommand\degreeCname{碩士}
\newcommand\degreeEname{Master of Engineering}
\newcommand\cYear{一○一}
\newcommand\cMonth{七}
\newcommand\eYear{2012}
\newcommand\eMonth{June}
\newcommand\ePlace{Tainan, Taiwan}


 % user's names; to replace those default variable definitions
%
% this file is encoded in big5
% v2.02 (Sep. 12, 2012)
% ��J�A���פ��D�ءB�m�W�����
% �p�G�D�ؤ��������H�ƾǼҦ����ܪ��Ÿ��A�Х� \mbox{} �]���ƾǼҦ��A�p�U�d��
% �p�G����W�r�O��W�A�P�m�󤧶���ij�H���Ϊťն�J�A�p�U�d��
% �^��W�r�����ٿסA�p Prof. �H�� Dr.�A��y�I����ХH���_��ť�~�N���@��ťաA�p�U�d��
% �p�G�A�����ɱб¨S���p�w�]���T��o��h�A�h�Ч�۹������h�l�бª�����B�^��W
%    ���w�q�H�Ū��j�A������
%    �p�A\renewcommand\advisorCnameB{}
%          \renewcommand\advisorEnameB{}
%          \renewcommand\advisorCnameC{}
%          \renewcommand\advisorEnameC{}

% �פ��D�� (����)
\renewcommand\cTitle{%
�ڪ��Ӥh�פ��D�� \mbox{$\cal{H}_\infty$} �P \mbox{Al$_x$Ga$_{1-x}$As}
}

% �פ��D�� (�^��)
\renewcommand\eTitle{%
My Thesis Title  \mbox{$\cal{H}_\infty$} and \mbox{Al$_x$Ga$_{1-x}$As}
}

% �ڪ��m�W (����)
\renewcommand\myCname{���K��}

% �ڪ��m�W (�^��)
\renewcommand\myEname{Aron Wang}

% ���ɱб�A���m�W (����)
\renewcommand\advisorCnameA{�n�c���դh}

% ���ɱб�A���m�W (�^��)
\renewcommand\advisorEnameA{Dr.~Ming Nangong}

% ���ɱб�B���m�W (����)
\renewcommand\advisorCnameB{�����Z�դh}

% ���ɱб�B���m�W (�^��)
\renewcommand\advisorEnameB{Dr.~Stein Lee}

% ���ɱб�C���m�W (����)
\renewcommand\advisorCnameC{�}�@�۳դh}

% ���ɱб�C���m�W (�^��)
\renewcommand\advisorEnameC{Dr.~Sean Hsu}

% �զW (����)
\renewcommand\univCname{�����j��}

% �զW (�^��)
\renewcommand\univEname{Yuan Ze University}

% �t�ҦW (����)
\renewcommand\deptCname{���q�u�{��s��}

% �t�ҥ��W (�^��)
\renewcommand\fulldeptEname{Department of Photonics Engineering}

% �t�ҵu�W (�^��, �Ω�ѦW���Ǧ�W���)
\renewcommand\deptEname{Photonics Engineering}

% �ǰ|�^��W (�p�L�A�h�H�Ū��j�A������)
\renewcommand\collEname{College of Electrical and Communication Engineering}

% �Ǧ�W (����)
\renewcommand\degreeCname{�Ӥh}

% �Ǧ�W (�^��)
\renewcommand\degreeEname{Master of Science}

% �f�զ~�� (����B����)
\renewcommand\cYear{�@���@}

% �f�դ�� (����)
\renewcommand\cMonth{�C} 

% �f�զ~�� (���ԧB�Ʀr�B�褸)
\renewcommand\eYear{2012} 

% �f�դ�� (�^��)
\renewcommand\eMonth{July}

% �ǮթҦb�a (�^��)
\renewcommand\ePlace{Chungli, Taoyuan, Taiwan}

%���~�ŧO�F�Ω�ѭI�C�L�F�Y�L���ݭn�i����
\newcommand\GraduationClass{100}

%%%%%%%%%%%%%%%%%%%%%%


% 使用 hyperref 在 pdf 簡介欄裡填入相關資料
\ifx\hypersetup\undefined
	\relax  % do nothing
\else
	\hypersetup{
	pdftitle=\cTitle,
	pdfauthor=\myCname}
\fi
	

\newcommand\itsempty{}
%%%%%%%%%%%%%%%%%%%%%%%%%%%%%%%
%       NCKU cover 封面
%%%%%%%%%%%%%%%%%%%%%%%%%%%%%%%
%
\begin{titlepage}
% no page number
% next page will be page 1

% aligned to the center of the page
\begin{center}
% font size (relative to 12 pt):
% \large (14pt) < \Large (18pt) < \LARGE (20pt) < \huge (24pt)< \Huge (24 pt)
%
\makebox[6cm][s]{\Huge\univCname}\\  %顯示中文校名
\vspace{0.5cm}
\makebox[12cm][s]{\Huge\deptCname}\\ %顯示中文系所名
\vspace{1.5cm}
\makebox[6cm][s]{\Huge\degreeCname 論文}\\ %顯示論文種類 (中文)
\vspace{1.5cm}
%
% Set the line spacing to single for the titles (to compress the lines)
\renewcommand{\baselinestretch}{1}   %行距 1 倍
%\large % it needs a font size changing command to be effective
\Large\cTitle\\  % 中文題目
%
\vspace{1cm}
%
\Large\eTitle\\ %英文題目
\vspace{2cm}
% \makebox is a text box with specified width;
% option s: stretch; option l: left aligned
% use \makebox to make sure
% 「研究生」 與「指導教授」occupy the same width
% Names are filled in a box with pre-defined width
% the left and right sides of 「:」occupy the same width (use \hspace{} to fill the short)
% to guarantee 「:」is at the center
% assume the width of a Chinese character is 1.2em
% 4.8em is determined by the length of the longest string "指導教授"
% 7.2em is determined by the length of the possibly longest name + title "歐陽明志博士"
\hspace{2.4em}%
\makebox[4.8em][s]{\Large 研究生}%
\makebox[1em][c]{\Large :}%
\makebox[7.2em][l]{\Large\myCname}\\  % 顯示作者中文名
%
\hspace{2.4em}%
\makebox[4.8em][s]{\Large 指導教授}%
\makebox[1em][c]{\Large :}%
\makebox[7.2em][l]{\Large\advisorCnameA}\\  %顯示指導教授A中文名
%
% 判斷是否有共同指導的教授 B
\ifx \advisorCnameB  \itsempty
\relax % 沒有 B 教授,所以不佔版面,不印任何空白
\else
% 共同指導的教授 B
\hspace{2.4em}%
\makebox[4.8em][s]{}%
\makebox[1em][c]{}%
\makebox[7.2em][l]{\Large\advisorCnameB}\\%顯示指導教授B中文名
\fi
%
% 判斷是否有共同指導的教授 C
\ifx \advisorCnameC  \itsempty
\relax % 沒有 C 教授,所以不佔版面,不印任何空白
\else
% 共同指導的教授 C
\hspace{2.4em}%
\makebox[4.8em][s]{}%
\makebox[1em][c]{}%
\makebox[7.2em][l]{\Large\advisorCnameC}\\%顯示指導教授B中文名
\fi
%
\vfill
\makebox[10cm][s]{\Large 中華民國\cYear 年\cMonth 月}%
%
\end{center}
% Resume the line spacing to the desired setting
\renewcommand{\baselinestretch}{\mybaselinestretch}   %恢復原設定
% it needs a font size changing command to be effective
% restore the font size to normal
\normalsize
\end{titlepage}
%%%%%%%%%%%%%%

%%---------------------------------------------------------------------------
%% 英文封面
%%---------------------------------------------------------------------------
    \thispagestyle{empty}%
    \null\vskip0.5in
    \begin{center}
        \hyphenpenalty=10000\Large\uppercase\expandafter{\eTitle}
    \end{center}
    \vfill
    \begin{center}
        \large\rm By \\
        {\large\rm \myEname}
    \end{center}
    \vfill
    \begin{center}
        \footnotesize SUBMITTED IN PARTIAL FULFILLMENT OF THE \\
        REQUIREMENTS FOR THE DEGREE OF \\
        MASTER OF SCIENCE IN \\
        DEPARTMENT OF COMPUTER SCIENCE AND INFORMATION ENGINEERING \\
        NATIONAL CHENG-KUNG UNIVERSITY \\
        TAINAN, TAIWAN, R.O.C. \\
        {\eMonth \ \eYear}
    \end{center}
    \vskip0.75in
    \begin{center}
        \rm \copyright\ Copyright by {\myEname}, \eYear
    \end{center}
%\newpage


























%% 從摘要到本文之前的部份以小寫羅馬數字印頁碼
% 但是從「書名頁」(但不印頁碼) 就開始計算
\setcounter{page}{1}
\pagenumbering{roman}
%%%%%%%%%%%%%%%%%%%%%%%%%%%%%%%
%       書名頁 
%%%%%%%%%%%%%%%%%%%%%%%%%%%%%%%
%
\newpage

% 判斷是否要浮水印?
\ifx\mywatermark\undefined 
  \thispagestyle{empty}  % 無頁碼、無 header (無浮水印)
\else
  \thispagestyle{EmptyWaterMarkPage} % 無頁碼、有浮水印
\fi

%no page number
% create an entry in table of contents for 書名頁
\phantomsection % for hyperref to register this
\addcontentsline{toc}{chapter}{\nameInnerCover}


% aligned to the center of the page
\begin{center}
% font size (relative to 12 pt):
% \large (14pt) < \Large (18pt) < \LARGE (20pt) < \huge (24pt)< \Huge (24 pt)
% Set the line spacing to single for the titles (to compress the lines)
\renewcommand{\baselinestretch}{1}   %行距 1 倍
% it needs a font size changing command to be effective
%中文題目
\Large\cTitle\\ %%%%%
\vspace{1cm}
% 英文題目
\Large\eTitle\\ %%%%%
%\vspace{1cm}
\vfill
% \makebox is a text box with specified width;
% option s: stretch
% use \makebox to make sure
% 「研究生:」 與「指導教授:」occupy the same width
\large %to have correct em value
\makebox[4.8em][s]{研究生}%
\makebox[1em][c]{:}%
\makebox[7.2em][l]{\myCname}%%%%%
\hfill%
\makebox[2cm][l]{Student:}%
\makebox[5cm][l]{\myEname}\\ %%%%%
%
%\vspace{1cm}
%
\makebox[4.8em][s]{指導教授}%
\makebox[1em][c]{:}%
\makebox[7.2em][l]{\advisorCnameA}%%%%%
\hfill%
\makebox[2cm][l]{Advisor:}%
\makebox[5cm][l]{\advisorEnameA}\\ %%%%%
%
% 判斷是否有共同指導的教授 B
\ifx \advisorCnameB  \itsempty
\relax % 沒有 B 教授,所以不佔版面,不印任何空白
\else
%共同指導的教授B
\makebox[4.8em][s]{}%
\makebox[1em][c]{}%
\makebox[7.2em][l]{\advisorCnameB}%%%%%
\hfill%
\makebox[2cm][l]{}%
\makebox[5cm][l]{\advisorEnameB}\\ %%%%%
\fi
%
% 判斷是否有共同指導的教授 C
\ifx \advisorCnameC  \itsempty
\relax % 沒有 C 教授,所以不佔版面,不印任何空白
\else
%共同指導的教授C
\makebox[4.8em][s]{}%
\makebox[1em][c]{}%
\makebox[7.2em][l]{\advisorCnameC}%%%%%
\hfill%
\makebox[2cm][s]{}%
\makebox[5cm][l]{\advisorEnameC}\\ %%%%%
\fi
%
% Resume the line spacing to the desired setting
\renewcommand{\baselinestretch}{\mybaselinestretch}   %恢復原設定
\normalsize %it needs a font size changing command to be effective
\large
%
\vfill
\makebox[4cm][s]{\univCname}\\% 校名
\makebox[6cm][s]{\deptCname}\\% 系所名
\makebox[3cm][s]{\degreeCname 論文}\\% 學位名
%
%\vspace{1cm}
\vfill
\large
A Thesis\\%
Submitted to %
%
\fulldeptEname\\%系所全名 (英文)
%
%
\ifx \collEname  \itsempty
\relax % 沒有學院名 (英文),所以不佔版面,不印任何空白
\else
% 有學院名 (英文)
\collEname\\% 學院名 (英文)
\fi
%
\univEname\\%校名 (英文)
%
in Partial Fulfillment of the Requirements\\
%
for the Degree of\\
%
\degreeEname\\%學位名(英文)
%
in\\
%
\deptEname\\%系所短名(英文;表明學位領域)
%
\eMonth\ \eYear\\%月、年 (英文)
%
\ePlace% 學校所在地 (英文)
\vfill
中華民國%
\cYear% %%%%%
年%
\cMonth% %%%%%
月\\
\end{center}
% restore the font size to normal
\normalsize
\clearpage
%%%%%%%%%%%%%%%%%%%%%%%%%%%%%%%
%       論文口試委員審定書 (計頁碼,但不印頁碼) 
%%%%%%%%%%%%%%%%%%%%%%%%%%%%%%%
%
% insert the printed standard form when the thesis is ready to bind
% 在口試完成後,再將已簽名的審定書放入以便裝訂
% create an entry in table of contents for 審定書
% 目前送出空白頁
\newpage%
{\thispagestyle{empty}%
\phantomsection % for hyperref to register this
\addcontentsline{toc}{chapter}{\nameCommitteeForm}%
\mbox{}\clearpage}

%%%%%%%%%%%%%%%%%%%%%%%%%%%%%%%
%       授權書 (計頁碼,但不印頁碼) 
%%%%%%%%%%%%%%%%%%%%%%%%%%%%%%%
%
% insert the printed standard form when the thesis is ready to bind
% 在口試完成後,再將已簽名的授權書放入以便裝訂
% create an entry in table of contents for 授權書
% 目前送出空白頁
\newpage%
{\thispagestyle{empty}%
\phantomsection % for hyperref to register this
\addcontentsline{toc}{chapter}{\nameCopyrightForm}%
\mbox{}\clearpage}

%%%%%%%%%%%%%%%%%%%%%%%%%%%%%%%
%       中文摘要 
%%%%%%%%%%%%%%%%%%%%%%%%%%%%%%%
%
\newpage
\thispagestyle{plain}  % 無 header,但在浮水印模式下會有浮水印
% create an entry in table of contents for 中文摘要
\phantomsection % for hyperref to register this
\addcontentsline{toc}{chapter}{\nameCabstract}

% aligned to the center of the page
\begin{center}
% font size (relative to 12 pt):
% \large (14pt) < \Large (18pt) < \LARGE (20pt) < \huge (24pt)< \Huge (24 pt)
% Set the line spacing to single for the names (to compress the lines)
\renewcommand{\baselinestretch}{1}   %行距 1 倍
% it needs a font size changing command to be effective
\large\cTitle\\  %中文題目
\vspace{0.83cm}
% \makebox is a text box with specified width;
% option s: stretch
% use \makebox to make sure
% each text field occupies the same width
\makebox[3em][l]{學生:}%
\makebox[4.8em][l]{\myCname}%學生中文姓名
\hfill%
%
\makebox[5em][l]{指導教授:}%
\makebox[7.2em][l]{\advisorCnameA}\\ %教授A中文姓名
%
% 判斷是否有共同指導的教授 B
\ifx \advisorCnameB  \itsempty
\relax % 沒有 B 教授,所以不佔版面,不印任何空白
\else
%共同指導的教授B
\makebox[3em][l]{}%
\makebox[4.8em][l]{}%%%%%
\hfill%
\makebox[5em][l]{}%
\makebox[7.2em][l]{\advisorCnameB}\\ %教授B中文姓名
\fi
%
% 判斷是否有共同指導的教授 C
\ifx \advisorCnameC  \itsempty
\relax % 沒有 C 教授,所以不佔版面,不印任何空白
\else
%共同指導的教授C
\makebox[3em][l]{}%
\makebox[4.8em][l]{}%%%%%
\hfill%
\makebox[5em][l]{}%
\makebox[7.2em][l]{\advisorCnameC}\\ %教授B中文姓名
\fi
%
\vspace{0.42cm}
%
\univCname\deptCname\\ %校名系所名
\vspace{0.83cm}
%\vfill
\makebox[2.5cm][s]{摘要}\\
\end{center}
% Resume the line spacing to the desired setting
\renewcommand{\baselinestretch}{\mybaselinestretch}   %恢復原設定
%it needs a font size changing command to be effective
% restore the font size to normal
\normalsize
%%%%%%%%%%%%%
摘要內容應說明研究目的,資料來源,研究方法及結果等,約 500--1000 字,並以一頁為限。好的摘要應該形如沙漏,上下皆寬,唯腰纖細。對於研究的動機、目的之描述,與研究結果可能的影響與應用之描述,應多於中間的研究執行描述。讓讀者一眼即能確定此論文是否為其尋找的標的。一二三四五六七八九,一二三四五六七八九,一二三四五六七八九,一二三四五六七八九,一二三四五六七八九,一二三四五六七八九,一二三四五六七八九,一二三四五六七八九,一二三四五六七八九,一二三四五六七八九。一二三四五六七八九,一二三四五六七八九,一二三四五六七八九,一二三四五六七八九,一二三四五六七八九,一二三四五六七八九,一二三四五六七八九,一二三四五六七八九,一二三四五六七八九,一二三四五六七八九。一二三四五六七八九,一二三四五六七八九,一二三四五六七八九,一二三四五六七八九,一二三四五六七八九,一二三四五六七八九,一二三四五六七八九,一二三四五六七八九,一二三四五六七八九,一二三四五六七八九。一二三四五六七八九,一二三四五六七八九,一二三四五六七八九,一二三四五六七八九,一二三四五六七八九,一二三四五六七八九,一二三四五六七八九,一二三四五六七八九,一二三四五六七八九,一二三四五六七八九。一二三四五六七八九,一二三四五六七八九,一二三四五六七八九,一二三四五六七八九,一二三四五六七八九,一二三四五六七八九,一二三四五六七八九,一二三四五六七八九,一二三四五六七八九,一二三四五六七八九。一二三四五六七八九,一二三四五六七八九,一二三四五六七八九,一二三四五六七八九,一二三四五六七八九,一二三四五六七八九,一二三四五六七八九,一二三四五六七八九,一二三四五六七八九,一二三四五六七八九。

%%%%%%%%%%%%%%%%%%%%%%%%%%%%%%%
%       英文摘要 
%%%%%%%%%%%%%%%%%%%%%%%%%%%%%%%
%
\newpage
\thispagestyle{plain}  % 無 header,但在浮水印模式下會有浮水印

% create an entry in table of contents for 英文摘要
\phantomsection % for hyperref to register this
\addcontentsline{toc}{chapter}{\nameEabstract}

% aligned to the center of the page
\begin{center}
% font size:
% \large (14pt) < \Large (18pt) < \LARGE (20pt) < \huge (24pt)< \Huge (24 pt)
% Set the line spacing to single for the names (to compress the lines)
\renewcommand{\baselinestretch}{1}   %行距 1 倍
%\large % it needs a font size changing command to be effective
\large\eTitle\\  %英文題目
\vspace{0.83cm}
% \makebox is a text box with specified width;
% option s: stretch
% use \makebox to make sure
% each text field occupies the same width
\makebox[2cm][l]{Student:}%
\makebox[5cm][l]{\myEname}%學生英文姓名
\hfill%
%
\makebox[2cm][l]{Advisor:}%
\makebox[5cm][l]{\advisorEnameA}\\ %教授A英文姓名
%
% 判斷是否有共同指導的教授 B
\ifx \advisorCnameB  \itsempty
\relax % 沒有 B 教授,所以不佔版面,不印任何空白
\else
%共同指導的教授B
\makebox[2cm][l]{}%
\makebox[5cm][l]{}%%%%%
\hfill%
\makebox[2cm][l]{}%
\makebox[5cm][l]{\advisorEnameB}\\ %教授B英文姓名
\fi
%
% 判斷是否有共同指導的教授 C
\ifx \advisorCnameC  \itsempty
\relax % 沒有 C 教授,所以不佔版面,不印任何空白
\else
%共同指導的教授C
\makebox[2cm][l]{}%
\makebox[5cm][l]{}%%%%%
\hfill%
\makebox[2cm][l]{}%
\makebox[5cm][l]{\advisorEnameC}\\ %教授C英文姓名
\fi
%
\vspace{0.42cm}
Submitted to \fulldeptEname\\  %英文系所全名
%
\ifx \collEname  \itsempty
\relax % 如果沒有學院名 (英文),則不佔版面,不印任何空白
\else
% 有學院名 (英文)
\collEname\\% 學院名 (英文)
\fi
%
\univEname\\  %英文校名
\vspace{0.83cm}
%\vfill
%
ABSTRACT\\
%\vspace{0.5cm}
\end{center}
% Resume the line spacing the desired setting
\renewcommand{\baselinestretch}{\mybaselinestretch}   %恢復原設定
%\large %it needs a font size changing command to be effective
% restore the font size to normal
\normalsize
%%%%%%%%%%%%%
The abstract written in English is placed here.  Please keep it in 250--500 words and limited in one page. A good abstract is analog to the shape of the hour glass, with thick top and bottom but a thin waist. The descriptions of your motivation and goal are as much as that of the effect and the application of your research result.  However, keep the description of the execution of your research in  minimal detail.  This enables the readers to quickly identify if this thesis answers their questions.  If it does, then the interested readers can continue to read the rest of the thesis. THE QUICK BROWN FOX JUMPS OVER THE LAZY DOG 01.  The quick brown fox jumps over the lazy dog 02.  The Quick Brown Fox Jumps Over the Lazy Dog 03.  The quick brown fox jumps over the lazy dog 04.  The Quick Brown Fox Jumps Over the Lazy Dog 05.  The quick brown fox jumps over the lazy dog 06.  The Quick Brown Fox Jumps Over the Lazy Dog 07.  The quick brown fox jumps over the lazy dog 08.  The Quick Brown Fox Jumps Over the Lazy Dog 09.  The quick brown fox jumps over the lazy dog 10.  The Quick Brown Fox Jumps Over the Lazy Dog 11.  The quick brown fox jumps over the lazy dog 12.  The Quick Brown Fox Jumps Over the Lazy Dog 13.  The quick brown fox jumps over the lazy dog 14.  The Quick Brown Fox Jumps Over the Lazy Dog 15.  The quick brown fox jumps over the lazy dog 16.  The Quick Brown Fox Jumps Over the Lazy Dog 17.  The quick brown fox jumps over the lazy dog 18.  The Quick Brown Fox Jumps Over the Lazy Dog 19.  The quick brown fox jumps over the lazy dog 20.  The quick brown fox jumps over the lazy dog 21.  The quick brown fox jumps over the lazy dog 22.


%%%%%%%%%%%%%%%%%%%%%%%%%%%%%%%
%       誌謝 
%%%%%%%%%%%%%%%%%%%%%%%%%%%%%%%
%
% Acknowledgment
\newpage
\chapter*{\protect\makebox[5cm][s]{\nameAckn}} %\makebox{} is fragile; need protect
\phantomsection % for hyperref to register this
\addcontentsline{toc}{chapter}{\nameAckn}
首先我要感謝我的指導老師\CJKunderline{南宮明}博士、\CJKunderline{李斯坦}博士、以及\CJKunderline{徐石}博士。我也要感謝各位口試委員。我更要感謝我的父母。也要感謝實驗室的學長、學姊、學弟妹的鼓勵與幫助。\emph{May force be with you!} 

%%%%%%%%%%%%%%%%%%%%%%%%%%%%%%%
%       目錄 
%%%%%%%%%%%%%%%%%%%%%%%%%%%%%%%
%
% Table of contents
\newpage
\renewcommand{\contentsname}{\protect\makebox[5cm][s]{\nameToc}}
%\makebox{} is fragile; need protect
\phantomsection % for hyperref to register this
\addcontentsline{toc}{chapter}{\nameToc}
\tableofcontents

%%%%%%%%%%%%%%%%%%%%%%%%%%%%%%%
%       表目錄 
%%%%%%%%%%%%%%%%%%%%%%%%%%%%%%%
%
% List of Tables
\newpage
\renewcommand{\listtablename}{\protect\makebox[5cm][s]{\nameLot}}
%\makebox{} is fragile; need protect
\phantomsection % for hyperref to register this
\addcontentsline{toc}{chapter}{\nameLot}
\listoftables

%%%%%%%%%%%%%%%%%%%%%%%%%%%%%%%
%       圖目錄 
%%%%%%%%%%%%%%%%%%%%%%%%%%%%%%%
%
% List of Figures
\newpage
\renewcommand{\listfigurename}{\protect\makebox[5cm][s]{\nameTof}}
%\makebox{} is fragile; need protect
\phantomsection % for hyperref to register this
\addcontentsline{toc}{chapter}{\nameTof}
\listoffigures

%%%%%%%%%%%%%%%%%%%%%%%%%%%%%%%
%       程式列表目錄 
%%%%%%%%%%%%%%%%%%%%%%%%%%%%%%%
%
% List of Listings
% 如果需要程式列表目錄,則以下四行每一行的行首的註釋符號 % 刪掉以解除封印
%\newpage
%\phantomsection % for hyperref to register this
%\addcontentsline{toc}{chapter}{\lstlistlistingname}
%\lstlistoflistings




\newpage
%% 論文本體頁碼回復為阿拉伯數字計頁,並從頭起算
\pagenumbering{arabic}
%%%%%%%%%%%%%%%%%%%%%%%%%%%%%%%%